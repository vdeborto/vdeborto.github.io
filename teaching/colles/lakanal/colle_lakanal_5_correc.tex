\documentclass[10pt,a4paper]{article} 
\usepackage[utf8]{inputenc} 
\usepackage[T1]{fontenc} 
\usepackage[english]{babel} 
\usepackage{supertabular} %Nécessaire pour les longs tableaux
\usepackage[top=2.5cm, bottom=2.5cm, right=1.5cm, left=1.5cm]{geometry} %Mise en page 
\usepackage{amsmath} %Nécessaire pour les maths 
\usepackage{amssymb} %Nécessaire pour les maths 
\usepackage{stmaryrd} %Utilisation des double crochets 
\usepackage{pifont} %Utilisation des chiffres entourés 
\usepackage{graphicx} %Introduction d images 
\usepackage{epstopdf} %Utilisation des images .eps 
\usepackage{amsthm} %Nécessaire pour créer des théorèmes 
\usepackage{algorithmic} %Nécessaire pour écrire des algorithmes 
\usepackage{algorithm} %Idem 
\usepackage{bbold} %Nécessaire pour pouvoir écrire des indicatrices 
\usepackage{hyperref} %Nécessaire pour écrire des liens externes 
\usepackage{array} %Nécessaire pour faire des tableaux 
\usepackage{tabularx} %Nécessaire pour faire de longs tableaux 
\usepackage{caption} %Nécesaire pour mettre des titres aux tableaux (tabular) 
\usepackage{color} %nécessaire pour écrire en couleur 
\newtheorem{thm}{Théorème} 
\newtheorem{mydef}{Définition} 
\newtheorem{prop}{Proposition} 
\newtheorem{lemma}{Lemme}
\newcommand{\al}[1]{\begin{aligned} #1 \end{aligned}}

\newcommand{\seq}[2]{\left( #1_{#2} \right)_{#2 \in \mathbb{N}} }
\newcommand{\intt}[4]{\int_{#1}^{#2} #3 \mathop{}\!\mathrm{d} #4}
\newcommand{\summ}[2]{\underset{#1}{\overset{#2}{\sum}}}

\newcommand{\vertt}[1]{\vert #1 \vert}

\title{Semaine 5 - Uniforme continuité, lipschitzianité, comparaison de fonctions}
\author{Valentin De Bortoli \\ email : \ \href{mailto:valentin.debortoli@gmail.com}{valentin.debortoli@gmail.com}}
\date{}
\begin{document}
\maketitle

\section{Uniforme continuité et borne affine}
\subparagraph{1}Soit $\epsilon \in \mathbb{R}_+^*$. Par définition de l'uniforme continuité, $\exists \eta \in \mathbb{R}_+^*, \ \forall (x,y) \in I^2 \ \Longrightarrow \ \vert x-y \vert \le \eta, \ \vert f(x) - f(y) \vert \le \epsilon$. Soit $z_0 \in I$. On définit $Z = \lbrace z_k, k \in \mathbb{Z} \rbrace$, où $z_k = z_0 + k \eta$. Soit $x \in I$ on note $z_x = z_0 + k\eta \in Z$ l'élément de $Z \cap [\min(z_0,x),\max(z_0,x)] \subset I$ le plus proche de $x$. On a $x = z_0 + (k+k_x) \eta$ avec $\vert k_x \vert \le 1$. On suppose $k$ positif la démonstration est identique si $k <0$. On a alors,
\begin{equation*}
\begin{aligned}
\vert f(x) - f(z_0) \vert &\le \vert f(x) - f(z_x) \vert + \vert f(z_x) - f(z_0) \vert \\
&\le \epsilon +  k \epsilon  \\
&\le \epsilon \left( (1-k_x) + (k+k_x) \right) \\
&\le 2 \epsilon + \frac{x-z_0}{\eta} \\
&\le \epsilon (2 - \frac{z_0}{\eta}) + \frac{\epsilon}{\eta} x
\end{aligned}
\end{equation*}
Donc en posant $\alpha = \frac{\epsilon}{\eta}$ et $\beta = \epsilon (2 - \frac{z_0}{\eta}) + \vert f(z_0) \vert$ on obtient la propriété voulue.
\subparagraph{2}Il est évident que $x \mapsto \alpha \vert x \vert + \beta$ est bornée sur $I$ borné. La propriété découle sur $f$. 

\section{Ensemble de k-lipschitzianité}
Soit $(x,y) \in I$.
\subparagraph{1}
$A$ est non vide car $f$ est lipschitzienne.

Soit $k_1$ et $k_2$ dans $A$ et $k \in [k_1,k_2]$. $\exists \delta >0, \ k = \delta k_1 + (1-\delta)k_2$,
\begin{equation*}
\begin{aligned}
\vert f(x) - f(y) \vert &= \delta \vert f(x) - f(y) \vert + (1-\delta) \vert f(x) - f(y) \vert \\
&\le \delta k_1 \vert x-y \vert + (1-\delta)k_2 \vert x-y \vert \\
&\le k \vert x-y \vert
\end{aligned}
\end{equation*}
Donc $A$ est un intervalle.

De plus $A$ est borné inférieurement par $0$ donc par la propriété de la borne inférieure, $\exists B \in \mathbb{R}_+, \ A = [B,+\infty[$ ou $A =  ]B,+\infty[$. La question est maintenant de savoir si $B \in A$. Soit $(a_n)_{n \in \mathbb{N}}$ une suite de $A$ qui tend vers $B$. On a,

\begin{equation*}
\begin{aligned}
&\forall n \in \mathbb{N}, \ \vert f(x) - f(y) \vert \le a_n \vert x-y \vert \\
&\underset{n \rightarrow +\infty}{\lim} \vert f(x) - f(y) \vert \le \underset{n \rightarrow +\infty }{\lim} a_n \vert x-y \vert \\
&\vert f(x) - f(y) \vert \le B \vert x-y \vert
\end{aligned}
\end{equation*}
Donc $B \in A$ et $A = [B,+\infty[$.

\section{Théorème de Picard}
\subparagraph{1} Soit $(n,p) \in \mathbb{N}$
\begin{equation*}
\begin{aligned}
\vert x_{n+p} -x_n \vert &\le \overset{p}{\underset{j=1}{\sum}} \vert x_{n+j} - x_{n+j-1} \vert \\
&\le \overset{p}{\underset{j=1}{\sum}} k^{n+j-1} \vert x_1 - x_0 \vert \\
&\le k^n \vert x_1 -x_0 \vert \overset{p-1}{\underset{j=0}{\sum}} k^j \\
&\le k^n \vert x_1 -x_0 \vert \frac{1 - k^{p}}{1-k} \\
&\le k^n \vert x_1-x_0 \vert \frac{1}{1-k}
\end{aligned}
\end{equation*}
En prenant $n$ assez grand le terme de droite est aussi petit que l'on veut et donc la suite est de Cauchy.
\subparagraph{2}$(x_n)_{n \in \mathbb{N}}$ est de Cauchy dans $\mathbb{R}$ et donc admet une limite, $l$. Par continuité de $f$, $f(x_n) \ \rightarrow \ f(l)$. Mais $f(x_n) = x_{n+1} \ \rightarrow \ l$. Donc par unicité de la limite, $l = f(l)$ et $f$ admet un point fixe. 

\section{Limite et uniforme continuité}
\subparagraph{1} Soit $\epsilon>0$. Par définition de l'uniforme continuité, $\exists \eta>0, \forall (x,y) \in \mathbb{R}^2, \ \vertt{x-y} \le \eta \ \Longrightarrow \ \vertt{f(x)-f(y)} \le \frac{\epsilon}{2}$. Soit $\seq{x}{n}$ telle que $\forall n \in \mathbb{N}, \ x_n = n \eta$. Il existe $N \in \mathbb{N}, \ \forall n \ge N, \vertt{f(x_n)} \le \frac{\epsilon}{2}$. Soit $x \in [N,+\infty[$. ${\vertt{f(x)} \le \vertt{f(x) - f(x_n)} + \vertt{f(x_n} \le \frac{\epsilon}{2} + \frac{\epsilon}{2} \le \epsilon}$. Donc on peut conclure que $f$ tend vers $0$ en $+\infty$.


\section{Produit et équivalent}
\subparagraph{1}Un changement de variable en $y = 1-x$ nous permet de nous ramener à la limite de $\ln(1-y) \ln(y)$ en $0^+$. Or $\ln(1-y) \ \underset{0^+}{\sim} -y$ donc ${\ln(1-y) \ln(y) \underset{0^+}{\sim} -y\ln(y) \ \rightarrow \ 0}$.

\section{Fonction décroissante et équivalent}
\subparagraph{1} $2f(x+1) \le f(x) + f(x+1) \le 2f(x)$ donc $\frac{f(x)+f(x+1)}{2} \le f(x) \le \frac{f(x) + f(x+1)}{2}$. Par le théorème d'encadrement puisque les termes de gauche et de droite tendent vers $0$ on en déduit que $f$ tend vers $0$ en $+\infty$.
\subparagraph{2}On conjecture que l'équivalent est $x \ \mapsto \ \frac{1}{2x}$. On c	alcule ${r(x) = x(2f(x) - \frac{1}{x})}$ qui doit tendre vers $0$ en $+\infty$. On a donc,
\[
x(f(x) + f(x+1)) - 1 \le x(2f(x) - \frac{1}{x}) = 2xf(x) - 1 \le x(f(x) + f(x-1)) - 1
\]
$x(f(x) + f(x+1)) \ \rightarrow \ 1$ et $x(f(x) + f(x-1)) \ \sim \ \frac{x}{x-1} \ \rightarrow \ 1$. On a donc prouvé que $f(x) \sim \frac{1}{2x}$.
\section{Calcul de limites (1)}
\subparagraph{1}Pour $x \ge e$ on a $\ln(x) \ge 1$ donc $x^{\ln(x)} \ge x$ puisque $e \ge 1$. Ainsi, ${\frac{x^{\ln(x)}}{\ln(x)}} \ge \frac{x}{\ln(x)} \rightarrow +\infty$. 
\subparagraph{2}On a ${ \left( \frac{x}{\ln(x)} \right)^{\frac{\ln(x)}{x}} = \exp\left( \frac{\ln(x)^2}{x} - \frac{\ln(\ln(x)) }{x}\right)} \rightarrow 1$.
\subparagraph{3} ${\ln(x + \sqrt{1+x^2}) - \ln(2x) = \ln(\frac{1}{2}+ \frac{1}{2} \sqrt{1 + \frac{1}{x^2}}) = \ln \left( 1+ \frac{1}{2}(\sqrt{1 + \frac{1}{x^2}}-1) \right) \sim \frac{1}{2}(\sqrt{1 + \frac{1}{x^2}}-1) }$. Le dernier terme est borné donc quand on divise par $\ln(2x)$ on trouve bien que $\ln(2x) \sim \ln(x + \sqrt{1+x^2})$. Il est facile de montrer que $\ln(2x) \sim \ln(x)$ et on conclut que la limite de l'expression originale est $1$.

\section{Calcul de limites (2)}
\subparagraph{1}$(x+1)e^x-xe^{x+1} = xe^x(1-e) - e^x$. $e^x$ est négigeable devant $xe^x$ en $+\infty$ et donc puisque $e\ge 1$ on obtient que la limite est $-\infty$.
\subparagraph{2}$x \mapsto (x+1)\ln(x)-x\ln(x+1) = x \ln(\frac{x}{x+1}) + \ln(x) = -x\ln(1 + \frac{1}{x}) + \ln(x) \rightarrow +\infty$.
\section{Quelques considérations sur l'exponentielle}
\subparagraph{1}Une étude de fonction permet de montrer que $\ln(1+x) \ge x$ sur $\mathbb{R}_+$. En appliquant cette inégalité en $\frac{x}{n}$ et en passant à l'exponentielle (qui est une fonction croissante...), on peut conclure.
\subparagraph{2}$(1+ \frac{x}{n})^{n} = \exp\left( n \ln (1+ \frac{x}{n}) \right)$, mais $\ln(1+ \frac{x}{n}) \underset{n \rightarrow +\infty}{\sim} \frac{x}{n}$ et donc $n \ln (1+ \frac{x}{n}) \rightarrow x$ et on conclut.

\section{Logarithme, exponentielle et équivalent}
\subparagraph{1} ${e^f \sim e^g \ \Leftrightarrow \ \frac{e^f -e^g}{e^g} \rightarrow 0 \ \Leftrightarrow \ e^{f-g} - 1 \rightarrow 0 \ \Leftrightarrow \ f-g \rightarrow 0}$. On a donc $e^f \sim e^g$ si et seulement si $f-g \rightarrow 0$. De la même manière $\ln(f) \sim \ln(g)$ si et seulement si $\frac{\ln \left(\frac{f}{g} \right)}{\ln(g)} \rightarrow 0$. On ne peut pas vraiment aller plus loin.
\subparagraph{2}$f(x) = x^2 +x $, $g(x) = x^2$ et $a =+\infty$ convient pour le premier contre-exemple. Pour le second, $f(x) = e^{\frac{2}{x}}$, $g(x) = e^{\frac{1}{x}}$ et $a = +\infty$ convient. 
\subparagraph{3}
On opère le changement de variable $y = \frac{\pi}{2}-x$ et on est ramené à étudier $\ln(\sin(y))$ autour de $0$. $\sin(y) \sim y$ et en vertu de la première question on peut passer au logarithme. Donc $\ln(\cos(x)) \underset{\frac{\pi}{2}}{\sim} \ln(\frac{\pi}{2}-x)$.

\subparagraph{Remarque :} la morale de cet exercice est la suivante, \textbf{les équivalents se comportent mal avec la composition sauf cas très particuliers.}
\end{document}