\documentclass[10pt,a4paper]{article} 
\usepackage[utf8]{inputenc} 
\usepackage[T1]{fontenc} 
\usepackage[english]{babel} 
\usepackage{supertabular} %Nécessaire pour les longs tableaux
\usepackage[top=2.5cm, bottom=2.5cm, right=1.5cm, left=1.5cm]{geometry} %Mise en page 
\usepackage{amsmath} %Nécessaire pour les maths 
\usepackage{amssymb} %Nécessaire pour les maths 
\usepackage{stmaryrd} %Utilisation des double crochets 
\usepackage{pifont} %Utilisation des chiffres entourés 
\usepackage{graphicx} %Introduction d images 
\usepackage{epstopdf} %Utilisation des images .eps 
\usepackage{amsthm} %Nécessaire pour créer des théorèmes 
\usepackage{algorithmic} %Nécessaire pour écrire des algorithmes 
\usepackage{algorithm} %Idem 
\usepackage{bbold} %Nécessaire pour pouvoir écrire des indicatrices 
\usepackage{hyperref} %Nécessaire pour écrire des liens externes 
\usepackage{array} %Nécessaire pour faire des tableaux 
\usepackage{tabularx} %Nécessaire pour faire de longs tableaux 
\usepackage{caption} %Nécesaire pour mettre des titres aux tableaux (tabular) 
\usepackage{color} %nécessaire pour écrire en couleur 
\newtheorem{thm}{Théorème} 
\newtheorem{mydef}{Définition} 
\newtheorem{prop}{Proposition} 
\newtheorem{lemma}{Lemme}
\title{Semaine 14 - Espaces vectoriels}
\author{Valentin De Bortoli \\ email : \ \href{mailto:valentin.debortoli@gmail.com}{valentin.debortoli@gmail.com}}
\date{}
\begin{document}
\maketitle
Dans la suite $k$ est un corps (on se limite à $\mathbb{R}$ et $\mathbb{C}$) et $E$ un $k$-espace vectoriel.
\section{Opérations ensemblistes et espaces vectoriels}
Soit $F$ et $G$ deux sous-espaces vectoriels de $E$.
\subparagraph{1}A quelle condition $F \cup G$ est un sous-espace vectoriel ?
\subparagraph{2}A quelle condition $F \cap G= F+G$ ?
\subparagraph{3}Soit $A$ et $B$ deux ensembles de $E$. Que peut-on dire de $\text{Vect}(A\cap B)$ et $\text{Vect}(A) \cap \text{Vect}(B)$ ?

\section{Une égalité d'espaces}
Soit $F,F',G,G'$ quatre sous-espaces vectoriels de $E$ tels que $F' \cap G' = F \cap G$.
\subparagraph{1}Montrer que $F=(F+(G \cap F')) \cap (F+(G \cap G'))$.

\section{Quelques sous-espaces vectoriels}
\subparagraph{1} Soit $E=\mathcal{F}([0,1])$ le $\mathbb{C}$-espace vectoriel des fonctions à valeurs complexes de $[0,1]$. Montrer que $F=\lbrace f \in E, f(0)=-f(1) \rbrace$ est un espace vectoriel. Trouver un supplémentaire de cet espace.
\subparagraph{2} Soit $E=\mathcal{C}([0,1])$ le $\mathbb{C}$-espace vectoriel des fonctions continues à valeurs complexes de $[0,1]$. Montrer que $F=\lbrace f \in E, \int_0^1f(t) \text{d}t=0 \rbrace$ est un espace vectoriel. Trouver un supplémentaire de cet espace.
\subparagraph{3} Soit $E=\mathcal{C}([0,\pi])$ le $\mathbb{C}$-espace vectoriel des fonctions continues à valeurs complexes de $[0,\pi]$. Trouver un supplémentaire de $G=\text{Vect}(\sin,\cos)$.

\section{Espace vectoriel et fonctions affines}
\subparagraph{1}Soit $F$ l'ensemble des fonctions continues de $[-1,1]$ affines sur $[-1,0]$ et affines sur $[0,1]$. Montrer que $F$ est un sous-espace vectoriel (de quel espace vectoriel ?).
\subparagraph{2}Trouver une base de $F$. 

\section{Une base de polynômes}
\subparagraph{1}Montrer que $(P_k)_{k \in \llbracket 0,n \rrbracket}$ avec $P_k=X^k(1-X)^{n-k}$ est une base de $\mathbb{R}_n[X]$. 

\section{Nombres réels et espace vectoriel}
Le but de cet exercice est d'étudier $\mathbb{R}$ comme $\mathbb{Q}$-espace vectoriel. On note $(p_n)_{n \in \mathbb{N}}$ l'ensemble des nombres premiers rangés par ordre croissant.
\subparagraph{1} Montrer que $\forall N \in \mathbb{N}, \ (p_n)_{n \in \llbracket 1,N \rrbracket}$ est une famille libre de $\mathbb{R}$. En déduire qu'il n'existe pas de base finie de $\mathbb{R}$ comme $\mathbb{Q}$-espace vectoriel.
\subparagraph{2} Autre démonstration : si $(x_n)_{n \in \llbracket 1,N \rrbracket}$ est une base de $\mathbb{R}$ comme $\mathbb{Q}$-espace vectoriel en déduire que tout $x\in \mathbb{R}$ est racine d'un polynôme de degré $N-1$.
\subparagraph{3}En considérant $2^{1/N}$ en déduire une contradiction (on admettra que $X^n-2$ est un polynôme irréductible de $\mathbb{Q}[X]$).

\subparagraph{Remarque :} en fait on peut même montrer en considérant la famille $(\underset{n \ge 1}{\sum} \frac{1}{10^{\lfloor a^n \rfloor}})_{a>1}$ que toute base de $\mathbb{R}$ comme $\mathbb{Q}$-espace vectoriel est de cardinal celui de $\mathbb{R}$. L'existence d'une base de $\mathbb{R}$ est assurée par l'axiome du choix (bases de Hamel) mais n'est pas constructible.

\section{Polynômes à valeurs entières}
\subparagraph{1}Montrer que $(P_k)_{k \in \llbracket 0,n\rrbracket}$ avec $\forall k \in \llbracket 1,n \rrbracket, \ P_k = \frac{X (X-1) \dots (X-k+1)}{k !}$ et $P_0=1$. Montrer que $(P_k)_{k \in \llbracket 0,n\rrbracket}$ base de $\mathbb{R}_n[X]$.
\subparagraph{2}Montrer que $\forall m \in \mathbb{Z}, \forall k \in \llbracket 0,n\rrbracket, \ P_k(m) \in \mathbb{Z}$.
\subparagraph{3}En déduire la forme des polynômes de $\mathbb{R}_n[X]$ qui prennent des valeurs entières sur les entiers.

\section{Divisibilité et sous-espace vectoriel}
Soit $A$ polynôme de $\mathbb{R}_n[X]$.
\subparagraph{1}Montrer que $F=\lbrace P \in \mathbb{R}_n[X], A \vert P \rbrace$ est un sous-espace vectoriel.
\subparagraph{2}Exhiber une base et un supplémentaire de cet espace.

\section{Une équation polynômiale}
\subparagraph{1}Déterminer les polynômes de $\mathbb{R}_n[X]$ qui vérifient $P(X+1)-P(X)=0$.
\subparagraph{2}Montrer qu'il existe un unique polynôme $P\in \mathbb{R}_{n+1}[X]$ tel que $P(0)=0$ et $P(X+1)-P(X)=X^n$.

\section{Une somme directe}
\subparagraph{1}Soit $i \in \llbracket 0,n \rrbracket$ et  $F_i=\lbrace P \in \mathbb{R}_n[X], \forall j \in \llbracket 0,n \rrbracket \backslash \lbrace i \rbrace, \ P(j)=0, \ P(i) \neq 0 \rbrace$. Montrer que $F_i \cup \lbrace 0 \rbrace$ est un espace vectoriel.
\subparagraph{2}Montrer que $\mathbb{R}_n[X]=F_0 \oplus \dots \oplus F_n$.
\end{document}