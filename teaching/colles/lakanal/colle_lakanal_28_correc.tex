\documentclass[10pt,a4paper]{article} 
\usepackage[utf8]{inputenc} 
\usepackage[T1]{fontenc} 
\usepackage[english]{babel} 
\usepackage{supertabular} %Nécessaire pour les longs tableaux
\usepackage[top=2.5cm, bottom=2.5cm, right=1.5cm, left=1.5cm]{geometry} %Mise en page 
\usepackage{amsmath} %Nécessaire pour les maths 
\usepackage{amssymb} %Nécessaire pour les maths 
\usepackage{stmaryrd} %Utilisation des double crochets 
\usepackage{pifont} %Utilisation des chiffres entourés 
\usepackage{graphicx} %Introduction d images 
\usepackage{epstopdf} %Utilisation des images .eps 
\usepackage{amsthm} %Nécessaire pour créer des théorèmes 
\usepackage{algorithmic} %Nécessaire pour écrire des algorithmes 
\usepackage{algorithm} %Idem 
\usepackage{bbold} %Nécessaire pour pouvoir écrire des indicatrices 
\usepackage{hyperref} %Nécessaire pour écrire des liens externes 
\usepackage{array} %Nécessaire pour faire des tableaux 
\usepackage{tabularx} %Nécessaire pour faire de longs tableaux 
\usepackage{caption} %Nécesaire pour mettre des titres aux tableaux (tabular) 
\usepackage{color} %nécessaire pour écrire en couleur 
\newtheorem{thm}{Théorème} 
\newtheorem{mydef}{Définition} 
\newtheorem{prop}{Proposition} 
\newtheorem{lemma}{Lemme}
\title{Semaine 28 - Séries}
\author{Valentin De Bortoli \\ email : \ \href{mailto:valentin.debortoli@gmail.com}{valentin.debortoli@gmail.com}}
\date{}
\begin{document}
\maketitle

\section{Sommes de Riemann et équivalent}
\subparagraph{1} Soit $\alpha \in ]0,1[$. Donner un équivalent de $\underset{k=1}{\overset{n}{\sum}} \frac{1}{k^{\alpha}}$ lorsque $n$ tend vers l'infini.
\subparagraph{2} Soit $\alpha \in ]1,+\infty[$. Donner un équivalent de $\underset{k=n}{\overset{+\infty}{\sum}} \frac{1}{k^{\alpha}}$ lorsque $n$ tend vers l'infini.

\section{Série et intégrale}
Soit $(u_n)_{n \in \mathbb{N}}$ définie par $u_n = \int_n^{2n} \frac{\text{d}x}{1+x^{\frac{3}{2}}}$. 
\subparagraph{1}Donner la nature de la série de terme général $u_n$.

\section{Une série convergente ?}
\subparagraph{1}Que peut-on dire de la convergence de la série de terme général : $u_n = \frac{1+ \frac{1}{2}+ \dots + \frac{1}{n}}{\ln(n!)}$ ?

\section{Un produit convergent ?}
\subparagraph{1}Que peut-on dire de la convergence du produit de terme général $u_n = 1+ \frac{(-1)^n}{\sqrt{n}}$ ?

\section{Séries de Bertrand}
On note $u_{n,\alpha,\beta} = \frac{1}{n^{\alpha} \ln(n) ^{\beta}}$.
\subparagraph{1}Que peut-on dire de la convergence de la série de terme générale $u_{n, \alpha, \beta}$ pour $\alpha>1$ ? Pour $\alpha <1$ ? Pour $\alpha=1$ et $\beta>1$ ? Pour $\alpha=1$ et $\beta \le 1$ ?
\subparagraph{2}Que peut-on dire de la convergence de la série de terme général $u_n = \frac{1}{\underset{k=1}{\overset{n}{\sum}}\ln(k)^2}$ ?

\section{Calcul de limite (1)}
\subparagraph{1}Montrer que la série de terme général $u_n = \arctan \left( \frac{1}{n^2+n+1} \right)$ converge et déterminer cette limite.

\section{Calcul de limite (2)}
\subparagraph{1}Montrer que $S(a)=\underset{n=0}{\overset{+\infty}{\sum}} \frac{a}{a^2+n^2}$ est bien définie sur $\mathbb{R}_+^*$ et calculer la limite de $S(a)$ en $+\infty$.

\section{Somme des inverses des nombres premiers}
On énumère les nombres premiers dans l'ordre croissant : $(p_n)_{n \in \mathbb{N}}$.
\subparagraph{1}Montrer que la convergence de la série de terme général $p_n$ est équivalente à la convergence de la suite $(v_n)_{n \in \mathbb{N}}$ définie par $v_n = \underset{k=1}{\overset{n}{\prod}}\frac{1}{1- \frac{1}{p_k}}$.
\subparagraph{2}Montrer que $v_n \ge \underset{k=1}{\overset{n}{\sum}} \frac{1}{k}$. Conclure.
\subparagraph{3}Discuter en fonction de $\alpha \in \mathbb{R}_+^*$ la convergence de la série de terme général $\frac{1}{p_n^{\alpha}}$.

\section{Transformation d'Abel et application}
Soit $(u_n)_{n \in \mathbb{N}}$ et $(v_n)_{n \in \mathbb{N}}$ deux suites de nombres complexes. On note $s_n = \underset{k=0}{\overset{n}{\sum}}v_k$
\subparagraph{1}On considère $S_n = \underset{k=0}{\overset{n}{\sum}}u_k v_k$. Montrer que :
\begin{equation*}
S_n = u_n s_n - u_0s_0 - \underset{k=1}{\overset{n-1}{\sum}}(u_{k}-u_{k-1}) s_k
\end{equation*}
\subparagraph{2}En déduire la convergence de la série de terme général $u_n = \sin \left( \frac{\sin(n)}{\sqrt[3]{n}} \right)$.
\subparagraph{Remarque :} la transformation d'Abel n'est rien de moins qu'une intégration par partie en discret. Il est peut être utile de garder cette transformation en tête lorsque l'on ne parvient pas à démontrer la convergence de la série avec des règles type Cauchy mais que la forme du terme général nous invite à poursuivre dans cette direction.

\section{Valeur absolue et sinus}
\subparagraph{1}Que peut-on dire de la convergence de la série de terme général $u_n = \frac{\vert \sin(n) \vert}{n}$ ?

\section{Théorème de réarrangement de Riemann}
On dit que la série de terme général $(u_n)_{n \in \mathbb{N}}$ est commutativement convergente si pour toute permutation $\sigma$ de $\mathbb{N}$, la série de terme général $(u_{\sigma(n)})_{n \in \mathbb{N}}$ est convergente.
\subparagraph{1}Montrer que si la série de terme général $(u_n)_{n \in \mathbb{N}}$ est absolument convergente alors elle est commutativement convergente et pour toute permutation $\sigma$ de $\mathbb{N}$ on a $\underset{k=1}{\overset{+\infty}{\sum}}u_{\sigma(k)} = \underset{k=1}{\overset{+\infty}{\sum}}u_k$.
\subparagraph{2}Réciproquement, montrer que si la série de terme général $u_n$ est semi-convergente alors pour tout $ a \in \overline{\mathbb{R}}$ il existe $\sigma$ une permutation de $\mathbb{N}$ tel que $\underset{k=1}{\overset{+\infty}{\sum}}u_{\sigma(k)} = a$.

\section{Série positive et décroissance}
Soit $(u_n)_{n \in \mathbb{N}}$ une suite décroissante positive. On suppose que la série de terme général $(u_n)_{n \in \mathbb{N}}$ converge
\subparagraph{1}Montrer que $u_n = o(\frac{1}{n})$.
\subparagraph{2}Que se passe-t-il si on ne suppose plus la décroissance ?

\section{Critère de Raabe et Duhamel}
\subparagraph{1}Rappeler le critère de d'Alembert. Le démontrer.
\subparagraph{2}Soit $(u_n)_{n \in \mathbb{N}}$ une suite de termes strictement positifs. On suppose que $\frac{u_{n+1}}{u_n} = 1- \frac{\alpha}{n}+ o \left(\frac{1}{n}\right)$. Montrer que :
\begin{itemize}
\item si $\alpha>1$ la série converge.
\item si $\alpha<1$ la série diverge.
\end{itemize}
\subparagraph{3}Que peut-on dire de la convergence de la série de terme général $u_n = \frac{2 \times 4 \times \dots \times (2n-2)}{3 \times 5 \times \dots \times (2n-1)}$ ?
\subparagraph{Remarque :} il s'agit d'une spécification du critère de d'Alembert qui permet de sortir de certains cas douteux où $\frac{u_{n+1}}{u_n}$. D'autres critères sont à votre disposition (pour l'étude des suites à termes \textbf{positifs} seulement) : règle de Cauchy et règle d'Hadamard. Il faut noter que la règle de d'Alembert est moins générale que la règle de Cauchy elle même moins générale que la règle d'Hadamard.

\section{Suites récurrentes et équivalent}
Soit $(u_n)_{n \in \mathbb{N}}$ définie de la manière suivante :
\begin{equation*}
\left\lbrace
\begin{aligned}
& u_0 \in ]0,\frac{\pi}{2}[\\
& u_{n+1}=\sin(u_n)
\end{aligned}
\right.
\end{equation*}
\subparagraph{1}Montrer que $u_n \rightarrow 0$.
\subparagraph{2}Montrer que $\frac{1}{u_{n+1}^2}-\frac{1}{u_n^2}$ admet une limite en $+\infty$ et la calculer.
\subparagraph{3}Déterminer un équivalent de $u_n$ lorsque $n \rightarrow +\infty$.

\section{Développement asymptotique de la série harmonique}
On appelle $H_n= \underset{k=1}{\overset{n}{\sum}}\frac{1}{k}$.
\subparagraph{1}Donner un équivalent de $H_n$.
\subparagraph{2}Poursuivre en donnant un développement à l'ordre 2 de $H_n$.

\section{Presque la série harmonique}
Soit $(k_n)_{n \in \mathbb{N}}$ une énumération croissante des nombres entiers ne contenant pas de $5$ dans leur représentation décimale.
\subparagraph{1}Que peut-on dire de la convergence de la série de terme général $k_n$ ?

\section{Une série toujours convergente}
Soit $(a_n)_{n \in \mathbb{N}} \in \left(\mathbb{R}_+^* \right)^{\mathbb{N}}$ et $(v_n)_{n \in \mathbb{N}}$ définie par ${ \forall n \in \mathbb{N}, \ v_n = \frac{a_n}{\underset{k=0}{\overset{n}{\prod}}(1+a_k)}}$.
\subparagraph{1}Montrer que la série de terme général $v_n$ converge.
\subparagraph{2}Montrer que la série de terme général $v_n$ converge vers $1$ si et seulement si la série de terme général $a_n$ diverge.

\section{Une équation fonctionnelle}
\subparagraph{1}$f$ est continue sur un ensemble compact donc elle atteint son maximum $M$, respectivement son minimum $m$, en $x_M$, respectivement en $x_m$. Dans un premier temps on va supposer que $(x_m,x_M) \in ]0,1[^2$. Dans ce cas on a,
\begin{equation}
\begin{aligned}
M &= \underset{n \in \mathbb{N}^*}{\sum} \frac{f(x_M^n)}{2^n} \\
&\le \underset{n \in \mathbb{N}^*}{\sum} \frac{M}{2^n} \\
&\le M
\end{aligned}
\end{equation}
On est dans le cas d'égalité donc $\forall n \in \mathbb{N}, \ f(x_M^n) = M$. Donc par continuité puisque $x_M^n$ tend vers $0$ on a $f(0)=M$. De la même manière on montre que $f(0)=m$. Ainsi, $M=m$ et $f$ est constante. Si $x_m = x_M$ alors la fonction est constante. Supposons maintenant que $x_m \in [0,1[$ et $x_M = 1$. On a $f(0) = m$. Soit $x_{\epsilon} = \inf A_{\epsilon}$ avec $A_{\epsilon} = \lbrace x \in [0,1], M-f(x) \le \epsilon \rbrace$. $x_{\epsilon} <1$ car $x_M = 1$. Supposons que $x_{\epsilon} >0$, alors $f(x_{\epsilon}^2) < M-\epsilon$. On a,
\begin{equation}
\begin{aligned}
f(x_{\epsilon} &= \underset{n \in \mathbb{N}^*}{\sum} \frac{f(x_{\epsilon}^n)}{2^n}
\end{aligned}
\end{equation}
Le terme de gauche est plus grand que $M-\epsilon$ tandis que celui de droite est strictement plus petit que $M-\epsilon$. C'est absurde donc $x_{\epsilon} = 0$. Donc pour tout $\epsilon$, $x_0 \in A_{\epsilon}$, c'est-à-dire, $M -f(x_0) = M -m \le \epsilon$. Donc $M=m$ et $f$ est constante.
\end{document}