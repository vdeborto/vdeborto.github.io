\documentclass[10pt,a4paper]{article} 
\usepackage[utf8]{inputenc} 
\usepackage[T1]{fontenc} 
\usepackage[english]{babel} 
\usepackage{supertabular} %Nécessaire pour les longs tableaux
\usepackage[top=2.5cm, bottom=2.5cm, right=1.5cm, left=1.5cm]{geometry} %Mise en page 
\usepackage{amsmath} %Nécessaire pour les maths 
\usepackage{amssymb} %Nécessaire pour les maths 
\usepackage{stmaryrd} %Utilisation des double crochets 
\usepackage{pifont} %Utilisation des chiffres entourés 
\usepackage{graphicx} %Introduction d images 
\usepackage{epstopdf} %Utilisation des images .eps 
\usepackage{amsthm} %Nécessaire pour créer des théorèmes 
\usepackage{algorithmic} %Nécessaire pour écrire des algorithmes 
\usepackage{algorithm} %Idem 
\usepackage{bbold} %Nécessaire pour pouvoir écrire des indicatrices 
\usepackage{hyperref} %Nécessaire pour écrire des liens externes 
\usepackage{array} %Nécessaire pour faire des tableaux 
\usepackage{tabularx} %Nécessaire pour faire de longs tableaux 
\usepackage{caption} %Nécesaire pour mettre des titres aux tableaux (tabular) 
\usepackage{color} %nécessaire pour écrire en couleur 
\newtheorem{thm}{Théorème} 
\newtheorem{mydef}{Définition} 
\newtheorem{prop}{Proposition} 
\newtheorem{lemma}{Lemme}


\newcommand{\al}[1]{\begin{aligned} #1 \end{aligned}}

\newcommand{\seq}[2]{\left( #1_{#2} \right)_{#2 \in \mathbb{N}} }
\newcommand{\intt}[4]{\int_{#1}^{#2} #3 \mathop{}\!\mathrm{d} #4}
\newcommand{\summ}[2]{\underset{#1}{\overset{#2}{\sum}}}

\newcommand{\vertt}[1]{\vert #1 \vert}

\title{Semaine 11 - Arithmétique dans $\mathbb{Z}$}
\author{Valentin De Bortoli \\ email : \ \href{mailto:valentin.debortoli@gmail.com}{valentin.debortoli@gmail.com}}
\date{}
\begin{document}
\maketitle
\section{Carrés parfaits}
\subparagraph{1}Le tableau suivant récapitule les valeurs de $m$ modulo $8$ et celles de $m^2$ qui découlent :
\begin{tabular}{|c|c|c|c|c|c|c|c|}
\hline
0 & 1 & 2 & 3 & 4 & 5 & 6 & 7 \\
\hline
0 & 1 & 4 & 1 & 0 & 1 & 4 & 1 \\ \hline
\end{tabular}
Donc aucun nombre de la forme $8n + k$ avec $k \in \lbrace 2,3,5,6,7 \rbrace$ ne peut être un carré parfait.
$N = (n-2)^2 + (n-1)^2 + n + (n+1)^2 + (n+2)^2 = 5n^2 + 10 = 5(n^2+2)$. Donc $5$ divise $N$ et donc $5^2$ aussi. Ce qui signifie que $n^2 = 5m +3$ avec $m \in \mathbb{N}$.
\begin{tabular}{|c|c|c|c|c|}
\hline
0 & 1 & 2 & 3 & 4 \\
\hline
0 & 1 & 4 & 4 & 1 \\ \hline
\end{tabular}
\section{Implication et primalité}
Soit $p$ un nombre premier.
\subparagraph{1}On écarte le cas $p=3$. Dans les autres cas, $p$ est congru à un ou deux modulo trois. Si il est congru à un alors $8p^2+1$ est divisible par trois et donc non premier. Si il est congru à deux alors $8p^2+1$ est encore divisible par trois donc non premier. Donc le seul cas qu'il reste est $p=3$. Dans ce cas $8\times 9 +1 = 73$ qui est premier tout comme $71$.
\section{Puissance et nombres premiers entre eux}
\subparagraph{1}On raisonne par récurrence. Au rang un la propriété est triviale.
Pour l'hérédité on considère 
\[
\al{
(1+\sqrt{2})^{n+1} &= (1+ \sqrt{2})(a_n +b_n\sqrt{2}) \\
&= a_n + 2b_n + (a_n + b_n)\sqrt{2}
}
\]
Ainsi $(a_n,b_n) \in \mathbb{N}^2$.
\subparagraph{2}Encore une fois procédons par récurrence. C'est vrai au rang un.
Ensuite on considère $(a_n+2b_n) \wedge (a_n+ b_n) = b_n \wedge (a_n+b_n) = \wedge b_n \wedge a_n = 1$.
\section{Équations et arithmétique}
\subparagraph{1}

Résoudre dans $\mathbb{Z}^2$ les équations suivantes :
\subparagraph{1} \begin{equation*}
\left\lbrace
\begin{aligned}
&x+y=56 \\
& x \vee y =105
\end{aligned}
\right.
\end{equation*}
\subparagraph{2}
\begin{equation*}
\left\lbrace
\begin{aligned}
&x \wedge y = x-y \\
&x \vee y =72
\end{aligned}
\right.
\end{equation*}
\subparagraph{3} $x \vee y - x \wedge y=243$
\section{Nombres de Fermat}
\subparagraph{1}Si $m = qr$ avec $q$ impair alors $-2^m = \left((-2^r)\right)^q$ et donc $1 + 2^m = 1 - \left((-2^r)\right)^q$. Ainsi $N$ est divisible par $1 + 2^r$. 
\subparagraph{2}Soit $m>n$ et $p$ un diviseur premier de $F_n$. On a $2^{2^n} = -1[p]$. Supposons que $p$ est aussi un diviseur premier de $F_m$ alors on a aussi $2^{2^m} = -1[p]$. Mais $2^{2^m} = \left(2^{2^n}\right)^{2^{m-n}}$ avec $2^{m-n}$ qui est pair et donc $2^{2^m} = 1[p]$. Ainsi $A = -1[p]$ ce qui est valide si et seulement si $p=2$ or il est évident que $2$ ne divise pas $F_n$.
\subparagraph{3}$F_0 = 3$, $F_1 = 5$, $F_2 = 17$, $F_3 = 257$. Ces nombres sont effectivement premiers. Fermat avait conjecturé que tous les $F_n$ étaient premiers. Euler a réfuté cette conjecture. En effet $F_5 = 65537$ est divisible par $641$. On a depuis réussi à montrer que tous les nombres de Fermat de $F_5$ à $F_32$ sont composés. On ignore tout de $F_{33}$ et des suivants (il faut bien se rendre compte que ce sont là des nombres prodigieusement grands et qu'il est inenvisageable de les stocker sur ordinateurs. Par exemple $F_{33}$ s'écrit en base décimale avec plus de mille milliards de chiffres...). On pense que c'est cette conjecture qui a poussé Fermat à penser qu'il possédait une démonstration de son grand théorème (voir remarque de l'exercice 7). Malgré l'échec de cette conjecture les nombres de Fermat sont très utiles. Par exemple les seuls polygones réguliers à $n$ côtés constructibles à la règle et au compas sont ceux tels que $n$ est une puissance de deux ou une puissance de deux multipliée par des nombres de Fermat distincts (théorème de Gauss-Wantzel).
\subparagraph{Remarque :} revenons sur Terre et remarquons simplement que la question deux de cet exercice implique qu'il existe une infinité de nombre premiers. En effet il existe une infinité de nombres de Fermat distincts et donc de facteurs premiers distincts.

\section{Nombres de Mersenne}
\subparagraph{1}Si $a>2$ alors $a^m - 1$ divisible par $a-1 >1$. Si $a = 2$ et $m=nq$ avec $n>1$ et $q>1$ alors $2^m - 1$ divisible par $2^n - 1 > 1$. Donc si $a^m - 1$ $a=2$ et $m$ premier.
\subparagraph{2}Soit $(m,n) \in \mathbb{N}^2$ avec $m>n$. Soit $r$ reste de la division euclidienne de $m$ par $n$, $m=nq+r$. $2^n-1$ divise $2^{nq}-1$. Soit $u$ qui divise $2^m-1$ et $2^n-1$ alors $u$ divise aussi $2^{nq}(2^r - 1)$. Mais $u$ ne peut-être une puissance de deux car tous les nombres de Mersenne sont premiers avec $2$. Donc $u$ divise $2^r-1$ (lemme de Gauss). Ainsi le pgcd de $2^m-1$ et de $2^n-1$ divise celui de $2^n-1$ et $2^r-1$. En procédant à l'algorithme d'Euclide on en déduit que le pgcd de $2^m-1$ et de $2^n-1$ divise $2^{n \wedge m}-1$. Dans le cas des nombres de Mersenne $n = p_{n_1}$ et $m= p_{n_2}$ premiers entre eux car premiers distincts. Ainsi, $M_{n_1} \wedge M_{n_2} = 1$.
\subparagraph{Remarque :} Marin Mersenne était un contemporain de Fermat et tous deux ont beaucoup communiqué. Fermat avait fait part à Mersenne de sa conjecture sur les nombres de Fermat.
\section{Triplets pythagoriciens}
On appelle triplet pythagoricien tout triplet $(x,y,z)\in \mathbb{Z}^3$ tel que $x^2+y^2=z^2$.
\subparagraph{1}Exhiber un tel triplet.
\subparagraph{2}Montrer que l'on peut se restreindre au cas où $x,y$ et $z$ sont premiers entre eux dans leur ensemble.
\subparagraph{3}Montrer qu'alors ils sont premiers entre eux deux à deux.
\subparagraph{4}Dans ce cas, montrer que deux sont impairs et que $z$ est impair. On suppose alors que $y=2y'$ et $x$ et $z$ impairs.
\subparagraph{5}On pose $X=\frac{x+z}{2}$ et $Z=\frac{z-x}{2}$. Montrer que $X \wedge Z=1$ et que $X$ et $Z$ sont des carrés parfaits.
\subparagraph{6}En déduire l'ensemble des triplets pythagoriciens.
\subparagraph{Remarque :} il n'existe pas de solutions si l'exposant est strictement supérieur à $2$. Il s'agit du grand théorème de Fermat que celui-ci pensait avoir montré. Une démonstration rigoureuse a été donnée par Wiles en 1995 après de nombreuses années de recherche.
\section{Factorielle et arithmétique}
\subparagraph{1}Soit $k \in \llbracket 0,n \rrbracket$, montrer que $\left( \begin{matrix}n \\k \end{matrix} \right)$ est un entier.
\subparagraph{2}Soit $k \wedge n=1$ montrer que $n$ divise $\left( \begin{matrix}n \\k \end{matrix} \right)$.
\subparagraph{3}Soit $p$ un nombre premier. Montrer que $\forall k \in \llbracket 0,p-1\rrbracket$, $p$ divise $\left( \begin{matrix}p \\k \end{matrix} \right)$.
\section{Suite de Fibonacci et arithmétique}
On considère la suite de Fibonacci définie par $u_0=0$, $u_1=1$ et $u_{n+2}=u_{n+1}+u_n$.
\subparagraph{1}Montrer que $u_{n+1}u_{n-1}-u_n^2=(-1)^n$. En déduire que $u_{n+1}  \wedge u_n=1$ si $n\in \mathbb{N}^*$.
\subparagraph{2}Montrer que $\forall (m,n)\in \mathbb{N}^* \times \mathbb{N}, \ u_{m+n}=u_{m-1}u_n+u_mu_{n+1}$. On pourra commencer par le cas $m=1$ et le cas $m=2$ puis raisonner par récurrence.
\subparagraph{3}En déduire que $u_n \wedge u_m=u_{n \wedge m}$.

\section{Exo Ulm }
\end{document}