\documentclass[10pt,a4paper]{article} 
\usepackage[utf8]{inputenc} 
\usepackage[T1]{fontenc} 
\usepackage[english]{babel} 
\usepackage{supertabular} %Nécessaire pour les longs tableaux
\usepackage[top=2.5cm, bottom=2.5cm, right=1.5cm, left=1.5cm]{geometry} %Mise en page 
\usepackage{amsmath} %Nécessaire pour les maths 
\usepackage{amssymb} %Nécessaire pour les maths 
\usepackage{stmaryrd} %Utilisation des double crochets 
\usepackage{pifont} %Utilisation des chiffres entourés 
\usepackage{graphicx} %Introduction d images 
\usepackage{epstopdf} %Utilisation des images .eps 
\usepackage{amsthm} %Nécessaire pour créer des théorèmes 
\usepackage{algorithmic} %Nécessaire pour écrire des algorithmes 
\usepackage{algorithm} %Idem 
\usepackage{bbold} %Nécessaire pour pouvoir écrire des indicatrices 
\usepackage{hyperref} %Nécessaire pour écrire des liens externes 
\usepackage{array} %Nécessaire pour faire des tableaux 
\usepackage{tabularx} %Nécessaire pour faire de longs tableaux 
\usepackage{caption} %Nécesaire pour mettre des titres aux tableaux (tabular) 
\usepackage{color} %nécessaire pour écrire en couleur 
\newtheorem{thm}{Théorème} 
\newtheorem{mydef}{Définition} 
\newtheorem{prop}{Proposition} 
\newtheorem{lemma}{Lemme}
\title{Semaine 9 - Intégration de fonctions continues}
\author{Valentin De Bortoli \\ email : \ \href{mailto:valentin.debortoli@gmail.com}{valentin.debortoli@gmail.com}}
\date{}
\begin{document}
\maketitle
\section{Une convergence de norme}
Soit $[ a,b ]$ un intervalle de $\mathbb{R}$. Soit $f \in \mathcal{C}([a,b], \mathbb{R}_+)$. Soit $M=\underset{x \in [a,b]}{\sup} \vert f(t) \vert$.
\subparagraph{1}Montrer que $\underset{n \rightarrow +\infty}{\lim} \left( \int_a^b f(t)^n \text{d}t\right)^{\frac{1}{n}}=M$.
\subparagraph{Indication :} on pourra penser à démontrer que pour tout $\epsilon \in \mathbb{R}_+^*$, il existe $[\alpha, \beta] \subset [a,b]$ tel que $\forall x \in [\alpha,\beta], \ f(x) \ge M-\epsilon$.
\section{Inégalité et intégrale}
Soit $[ a,b ]$ un intervalle de $\mathbb{R}$. Soit $f \in \mathcal{C}^1([a,b])$ telle que $f(a)=0$.
\subparagraph{1}Montrer que $\int_a^b \vert f(t) \vert^2  \text{d}t \le \frac{(b-a)^2}{2} \int_a^b \vert f'(t) \vert^2 \text{d}t$.
\section{Module et cas d'égalité}
Soit $[ a,b ]$ un intervalle de $\mathbb{R}$. Soit $f \in \mathcal{C}([a,b],\mathbb{C})$.
\subparagraph{1}On suppose que $\vert \int_a^b f(t) \text{d}t \vert=\int_a^b \vert f(t) \vert \text{d}t$. Montrer que $\forall t \in [a,b], \ f(t)=\vert f(t) \vert  e^{i \alpha}$ avec $\alpha \in \mathbb{R}$.
\section{Inégalité de Young}
Soit $[ a,b ]$ un intervalle de $\mathbb{R}_+$. Soit $f \in \mathcal{C}^1(\mathbb{R}_+)$, strictement croissante telle que $f(0)=0$.
\subparagraph{1} Soit $x \in \mathbb{R}_+$. Montrer que $\int_0^x f(t) \text{d}t + \int_0^{f(x)} f^{-1}(t) \text{d}t = xf(x)$.
\subparagraph{2} En déduire que $\int_0^a f(t) \text{dt} + \int_0^b f^{-1}(t) \text{d}t \ge ab$ avec égalité si et seulement si $b=f(a)$.
\section{Suite et intégrale (1)}
Soit $n \in \mathbb{N}$. On définit $J_n=\int_{0}^{\frac{\pi}{4}} \tan(x)^n \text{dx}$.
\subparagraph{1}Donner une formule liant $J_{n+2}$ et $J_{n}$. On commencera par calculer $J_{n+2}+J_n$.
\subparagraph{2}Après avoir calculé $J_0$ et $J_1$ exprimer $J_n$ en fonction de la parité de $n$.

\section{Suite et intégrale (2)}
Soit $n \in \mathbb{N}$. On définit $K_n=\int_{0}^{\frac{\pi}{4}} \frac{1}{\cos(x)^n} \text{dx}$.
\subparagraph{1}Calculer $K_0$ et $K_1$.
\subparagraph{2}Donner une formule liant $K_{n+2}$ et $K_{n}$. On pourra intégrer par partie $K_{n+2}$.

\section{Suite et intégrale (3)}
Soit $n \in \mathbb{N}$. On définit $L_n=\int_{1}^{e} \ln(x)^n \text{dx}$.
\subparagraph{1}Donner une formule liant $L_{n+1}$ et $L_{n}$.
\section{Condition suffisante et point fixe}
Soit $f \in \mathcal{C}([0,1],[0,1])$ telle que $\int_0^1f(t)=\frac{1}{2}$.
\subparagraph{1}Montrer que $f$ admet un point fixe.
\section{Inégalité et maximum}
Soit $[ a,b ]$ un intervalle de $\mathbb{R}$. Soit $f \in \mathcal{C}([a,b])$.
\subparagraph{1}Montrer que $\forall c \in ]a,b[, \ \frac{1}{b-a} \int_a^b f(t) \text{d}t \le \max \left( \frac{1}{c-a} \int_a^c f(t) \text{d}t, \frac{1}{b-c} \int_c^b f(t) \text{d}t \right)$.
\subparagraph{2}Donner une interprétation géométrique.
\section{Annulation et intégration (1)}
Soit $f \in \mathcal{C}([0, \pi], \mathbb{R})$. 
\subparagraph{1}On suppose que $\int_0^{\pi} f(t) \sin(t) \text{d}t=0$. Montrer que $f$ s'annule au moins une fois sur $]0,\pi[$. On note $a$ un élément de $]0,\pi[$ tel que $f(a)=0$.
\subparagraph{2}On suppose que $\int_0^{\pi} f(t) \sin(t) \text{d}t=\int_0^{\pi} f(t) \cos(t) \text{d}t=0$. Montrer que $f$ s'annule au moins deux fois sur $]0,\pi[$.
\subparagraph{Indication :} Que peut-on dire de $\int_0^{\pi} f(t) \sin(t-a) \text{d}t$ ?
\section{Annulation et intégration (2)}
Soit $[ a,b ]$ un intervalle de $\mathbb{R}$. Soit $f \in \mathcal{C}([a,b],\mathbb{R})$. Soit $n \in \mathbb{N}$.
\subparagraph{1}On suppose que $\forall k \in \llbracket 0,n \rrbracket, \ \int_a^b f(t) t^k \text{d}t=0$. Montrer que $f$ s'annule au moins $n+1$ fois.
\subparagraph{Indication :} On raisonnera par l'absurde et on posera $P(x)= \prod_{i=1}^{n_0} (x-x_i)$ avec $(x_i)_{i \in \llbracket 1,n_0 \rrbracket}$ les points d'annulation de $f$ en lesquels $f$ change de signe.
\subparagraph{Remarque :} on pourrait même aller plus loin et montrer que $f$ change de signe $n+1$ fois. On montre en utilisant le théorème d'approximation de Weierstrass (que vous verrez l'année prochaine) que si $\forall k \in \mathbb{N}, \ \int_a^b f(t)t^k=0$ alors $f=0$.
\section{Formule de la moyenne}
Soit $[ a,b ]$ un intervalle de $\mathbb{R}$. Soit $f \in \mathcal{C}([a,b])$. Soit $g \in \mathcal{C}([a,b])$ positive.
\subparagraph{1}Montrer qu'il existe $c \in [a,b]$ tel que $\int_a^bf(t)g(t) \text{d}t=f(c) \int_a^b g(t) \text{d}t$.
\subparagraph{2}Soit $I$ définie sur $\mathbb{R}_+^*$ par $I(x)=\int_{\sqrt{x}}^{\sqrt{2x}} \ln(t^2) \sin(\frac{1}{t}) \text{d}t$. Montrer que $\underset{x \rightarrow 0}{\lim}I(x)=0$.
\end{document}