\documentclass[10pt,a4paper]{article} 
\usepackage[utf8]{inputenc} 
\usepackage[T1]{fontenc} 
\usepackage[english]{babel} 
\usepackage{supertabular} %Nécessaire pour les longs tableaux
\usepackage[top=2.5cm, bottom=2.5cm, right=1.5cm, left=1.5cm]{geometry} %Mise en page 
\usepackage{amsmath} %Nécessaire pour les maths 
\usepackage{amssymb} %Nécessaire pour les maths 
\usepackage{stmaryrd} %Utilisation des double crochets 
\usepackage{pifont} %Utilisation des chiffres entourés 
\usepackage{graphicx} %Introduction d images 
\usepackage{epstopdf} %Utilisation des images .eps 
\usepackage{amsthm} %Nécessaire pour créer des théorèmes 
\usepackage{algorithmic} %Nécessaire pour écrire des algorithmes 
\usepackage{algorithm} %Idem 
\usepackage{bbold} %Nécessaire pour pouvoir écrire des indicatrices 
\usepackage{hyperref} %Nécessaire pour écrire des liens externes 
\usepackage{array} %Nécessaire pour faire des tableaux 
\usepackage{tabularx} %Nécessaire pour faire de longs tableaux 
\usepackage{caption} %Nécesaire pour mettre des titres aux tableaux (tabular) 
\usepackage{color} %nécessaire pour écrire en couleur 
\newtheorem{thm}{Théorème} 
\newtheorem{mydef}{Définition} 
\newtheorem{prop}{Proposition} 
\newtheorem{lemma}{Lemme}
\title{Semaine 2 - Généralités sur les fonctions à une variable complexe ou réelle}
\author{Valentin De Bortoli \\ email : \ \href{mailto:valentin.debortoli@gmail.com}{valentin.debortoli@gmail.com}}
\date{}
\begin{document}
\maketitle

\section{Inégalité et dérivée}
Soit $f$ une fonction de $\mathbb{R}$ dans $\mathbb{R}$ dérivable qui vérifie l'inégalité suivante : $\forall x \in \mathbb{R}, \ f(x)^2+(1+f'(x))^2 \leq 1$.
\subparagraph{1}Montrer que $f$ admet une limite en $+\infty$ et $-\infty$
\subparagraph{2}Montrer que $\lim_{x \rightarrow \infty}f(x)=0.$ Que peut-on en déduire sur $f$ ?
\subparagraph{3}Trouver un exemple de fonction non-nulle, dérivable sur $\mathbb{R}_+^*$, satisfaisant l'hypothèse,
\begin{equation}
\forall x \in \mathbb{R}_{+}^*, \ f(x)^2+(1+f'(x))^2 \leq 1.
\end{equation}

\section{Propriété de Darboux et croissance}
Soit $\left[a,b\right]$, un intervalle de $\mathbb{R}$ avec $(a,b) \in \mathbb{R}^2$ et $f$ une fonction croissante de $[a,b]$ dans $f([a,b])$. Supposons de plus que $f([a,b])=[f(a),f(b)]$.
\subparagraph{1}Montrer que $f$ est continue.
\subparagraph{Remarque :} la propriété $f([a,b])=[f(a),f(b)]$ peut se reformuler en : l'image d'un intervalle est un intervalle, la fonction étant croissante. La propriété de Darboux est plus forte et demande à ce que l'image de \textbf{tout} intervalle soit un intervalle. Malgré tout, cette propriété (bien que graphiquement très proche de la continuité) ne suffit pas à assurer que $f$ est continue sans supposer d'hypothèses supplémentaires (ici la croissance). On pourra vérifier que $f(x)=\sin(\frac{1}{x})$ sur $\mathbb{R}_{+}^{*}$ et $f(0)=0$ vérifie cette propriété mais n'est pas continue. Le théorème de Darboux assure que si $f$ est dérivable alors sa dérivée vérifie la propriété de Darboux. Le théorème de Darboux permet par exemple de prouver que le produit de deux fonctions dérivées n'est pas nécessairement une fonction dérivée. L'exemple suivant est dû à Jean-Marie Exbrayat.
\subparagraph{Exemple :} Soit $f_1$ définie sur $\mathbb{R}^*$ par $f_1(x) = x\sin \left(\frac{1}{x}\right)$ et $f_1(0)=0$. Cette fonction est continue donc c'est une fonction dérivée. Soit $f_2$ définie sur $\mathbb{R}^*$ par $f_2(x) = x^2 \sin \left( \frac{1}{x} \right)$ et $f_2(0)=0$. Cette fonction est dérivable et sa dérivée vaut $2x\sin \left(\frac{1}{x} \right) - \sin \left( \frac{1}{x} \right)$ sur $\mathbb{R}^*$ et $f_2'(0)=0$. Soit $u_1$ définie sur $\mathbb{R}^*$ par $u_1=\sin \left( \frac{1}{x} \right)$. $u_1$ est une fonction dérivée comme somme de fonctions dérivées. De même si on définit $u_2$ sur $\mathbb{R}^*$ par $u_2(x)= \cos \left( \frac{1}{x} \right)$ et $u_2(0)=0$ alors $u_2$ est une fonction dérivée. On définit $u = u_1^2+u_2^2$ qui ne vérifie pas la propriété de Darboux (pourquoi ?). Donc $u_1^2$ ou $u_2^2$ n'est pas une fonction dérivée. Avec un peu de travail supplémentaire on peut montrer que ni $u_1^2$, ni $u_2^2$ ne sont des fonctions dérivées.

\section{Une équation fonctionnelle (1)}
Soit $f$ une fonction continue qui va de $\mathbb{R}_{+}$ dans $\mathbb{R}$ telle que $\forall x \in \mathbb{R}_{+}$, $f(x^2)=f(x)$.
\subparagraph{1}Montrer que $f$ est constante.

\section{Une équation fonctionnelle (2)}
\subparagraph{1}Trouver toutes les fonctions continues de $\mathbb{R}$ dans $\mathbb{R}$ qui vérifient : $\forall (x,y) \in \mathbb{R}^2, \ f(x+y)=f(x)+f(y)$.
\subparagraph{Remarque :} on admettra que pour tout réel, il existe une suite de rationnels qui tend vers ce réel. Cette propriété s'appelle la densité de $\mathbb{Q}$ dans $\mathbb{R}$.

\section{Une équation fonctionnelle (3)}
Soit $f$ une fonction dérivable, de dérivée continue, de $\mathbb{R}$ dans $\mathbb{R}$ qui vérifie : $\forall x \in \mathbb{R}, \ f(f(x))=\frac{x}{2}+3$.
\subparagraph{1}Montrer que $f(\frac{x}{2}+3)=\frac{f(x)}{2}+3$.
\subparagraph{2}Montrer que $f'$ est constante.
\subparagraph{3}Déterminer $f$.

\section{Composition, injectivité et surjectivité}
Soit $f$ une fonction de $\mathbb{R}$ dans $\mathbb{R}$ qui vérifie $\forall x \in \mathbb{R}, \ f(f(f(x)))=f(x)$.
\subparagraph{1} Montrer que on a $f$ injective $\Leftrightarrow$ $f$ surjective $\Leftrightarrow$ $f$ bijective.
\subparagraph{2} Exprimer alors $f^{-1}$ en fonction de $f$.

\section{Théorème de Cantor-Bernstein}
Soit $A$ et $B$ deux ensembles. Le but de cet exercice est de montrer que si il existe une injection ($f_1$) de $A$ dans $B$ et une injection ($f_2$) de $B$ dans $A$ alors il existe une bijection entre $A$ et $B$. L'exercice se déroule en deux parties. Premièrement on va montrer que si $C$ est une partie de $A$ et $f$ une injection de $A$ dans $C$, alors $A$ et $C$ sont en bijection. Ensuite on montrera le théorème. On pose :
\begin{equation*}
\left\{
\begin{aligned}
&D_0={}^c C \\
&D_{n+1}=f(D_n) \ \text{pour} \ n \in \mathbb{N}^{*}
\end{aligned}\right.
\end{equation*}
\begin{equation*}
D=\cup_{n=0}^{+\infty} D_n
\end{equation*}
\subparagraph{1}Montrer que $f(D) \subset C \cap D$.
\subparagraph{2}On pose $g$ de $A$ dans $C$ telle que :
\begin{equation*}
\left\{
\begin{aligned}
g(x)=f(x) \ \text{si} \ x \in D \\
g(x)=x \ \text{si} \ x \in {}^cD
\end{aligned}\right.
\end{equation*}
Montrer que $g$ est injective.
\subparagraph{3}Montrer que $g$ est bijective et conclure pour la première partie.
\subparagraph{4}En considérant $f_1 \circ f_2$, montrer le théorème de Cantor-Bernstein.

\section{Union, intersection, image et image réciproque}
Soit $f$ une fonction de $\mathbb{R}$ dans $\mathbb{R}$ et $A$ et $B$ deux ensembles de $\mathbb{R}$.
\subparagraph{1} Parmi les assertions suivantes (donner les quatre assertions) seules trois sont vraies, lesquelles ? Lorsqu'une assertion est vraie, la démontrer, lorsqu'elle est fausse, donner un contre-exemple et établir une condition pour qu'elle devienne vraie.
\begin{itemize}
\item $f(A \cup B) \subset f(A) \cup f(B)$
\item $f(A \cup B) = f(A) \cup f(B)$
\item $f(A \cap B) \subset f(A) \cap f(B)$
\item $f(A \cap B) = f(A) \cap f(B)$
\end{itemize}
\subparagraph{2}En déduire que $ A \cup B\subset f^{-1}(f(A) \cup f(B))$. Trouver un contre-exemple à l'égalité.

\section{Un théorème de point fixe (1)}
Soit $f$ une application croissante de $[0,1]$ dans $[0,1]$.
\subparagraph{1}Soit $A=\{ x \in [0,1], \ f(x) \geq x \}$. Montrer que $A$ admet une borne supérieure.
\subparagraph{2}Montrer que $f$ admet un point fixe, c'est-à-dire, $\exists \ x_0 \in [0,1] \ | \ f(x_0)=x_0$.

\section{Un théorème de point fixe (2)}
Soit $f$ une application continue de $[0,1]$ dans $[0,1]$.
\subparagraph{1}Soit $A=\{ x \in [0,1], \ f(x) \geq x \}$. Montrer que $A$ admet une borne supérieure.
\subparagraph{2}Montrer que $f$ admet un point fixe, c'est-à-dire, $\exists \ x_0 \in [0,1] \ | \ f(x_0)=x_0$.

\section{Itérées et continuité}
Soit $f$ une fonction continue sur $[0,1]$ telle que,
\begin{equation}
\forall x \in [0,1], \ \exists n_x \in \mathbb{N}, \ f^{n_x}(x) = x
\end{equation}
\subparagraph{1}Montrer que $f$ est injective, surjective.
\subparagraph{2}Montrer que $f$ est monotone.
\subparagraph{3}En déduire que $\forall x \in [0,1], \ f(x) = x$ ou $f(f(x)) = x$.
\end{document}