\documentclass[10pt,a4paper]{article} 
\usepackage[utf8]{inputenc} 
\usepackage[T1]{fontenc} 
\usepackage[english]{babel} 
\usepackage{supertabular} %Nécessaire pour les longs tableaux
\usepackage[top=2.5cm, bottom=2.5cm, right=1.5cm, left=1.5cm]{geometry} %Mise en page 
\usepackage{amsmath} %Nécessaire pour les maths 
\usepackage{amssymb} %Nécessaire pour les maths 
\usepackage{stmaryrd} %Utilisation des double crochets 
\usepackage{pifont} %Utilisation des chiffres entourés 
\usepackage{graphicx} %Introduction d images 
\usepackage{epstopdf} %Utilisation des images .eps 
\usepackage{amsthm} %Nécessaire pour créer des théorèmes 
\usepackage{algorithmic} %Nécessaire pour écrire des algorithmes 
\usepackage{algorithm} %Idem 
\usepackage{bbold} %Nécessaire pour pouvoir écrire des indicatrices 
\usepackage{hyperref} %Nécessaire pour écrire des liens externes 
\usepackage{array} %Nécessaire pour faire des tableaux 
\usepackage{tabularx} %Nécessaire pour faire de longs tableaux 
\usepackage{caption} %Nécesaire pour mettre des titres aux tableaux (tabular) 
\usepackage{color} %nécessaire pour écrire en couleur 
\newtheorem{thm}{Théorème} 
\newtheorem{mydef}{Définition} 
\newtheorem{prop}{Proposition} 
\newtheorem{lemma}{Lemme}
\title{Semaine 20 - Matrices et applications linéaires}
\author{Valentin De Bortoli \\ email : \ \href{mailto:valentin.debortoli@gmail.com}{valentin.debortoli@gmail.com}}
\date{}
\begin{document}
\maketitle
Dans tous les exercices $n \in \mathbb{N}$.

\section{Trace et forme linéaire}
Soit $f$ une forme linéaire de $\mathcal{M}_n \left( \mathbb{R} \right)$.
\subparagraph{1}Soit $(i,j,k,l) \in \llbracket 1,n\rrbracket^4$. Que vaut $E_{i,j}E_{k,l}$ ?
\subparagraph{2}Montrer qu'il existe $A \in \mathcal{M}_n\left( \mathbb{R} \right)$ tel que $\forall M \in\mathcal{M}_n \left( \mathbb{R} \right), \ f(M) = \text{tr} (AM)$.
\subparagraph{Remarque :} l'année prochaine vous verrez qu'on peut utiliser un théorème plus fort pour conclure directement. Il s'agit d'une simple application du théorème de représentation de Riesz (il convient de remarquer que la trace est un produit scalaire sur l'espace des matrices). Néanmoins, l'utilisation de ce théorème nous empêche de choisir des matrices sur des corps finis...

\section{Une condition de commutativité}
Soit $(A,B) \in \mathcal{M}_n\left( \mathbb{R} \right)^2$. On suppose que $A+B=AB$
\subparagraph{1} Montrer que $I_n-A$ est inversible et donner son inverse.
\subparagraph{2} En déduire que $A$ et $B$ commutent.

\section{Rang et vecteur}
Soit $x_0 \in \mathbb{R}^n$ (vecteur colonne). On note $M = x_0 x_0^T$.
\subparagraph{1}A quel espace appartient $M$ ?
\subparagraph{2}Quel est le rang de $M$ ?

\section{Un endomorphisme sur l'espace des matrices}
Soit $D$ une matrice diagonale de $\mathcal{M}_n \left( \mathbb{R} \right)$. On note $\phi$ l'endomorphisme de $\mathcal{M}_n \left( \mathbb{R} \right)$ défini par $\phi(M) =DM-MD$.
\subparagraph{1} Déterminer $\text{ker} \phi$ et $\text{Im} \phi$. Donner des bases de ces espaces.
\subparagraph{2} Préciser lorsque tous les coefficients de la matrice $D$ sont distincts.

\section{Somme de matrices inversibles}
\subparagraph{1}Montrer que tout matrice $M \in \mathcal{M}_n \left( \mathbb{R} \right)$ peut s'écrire comme la somme de deux matrices inversibles.

\section{Une égalité matricielle (1)}
Soit $A \in \mathcal{M}_n \left( \mathbb{R} \right)$ avec $n \in \mathbb{N}$.
\subparagraph{1}Quelles sont les matrices $M \in \mathcal{M}_n \left( \mathbb{R} \right)$ qui vérifient $M+M^T = \text{tr}(M) A$ ?

\section{Une égalité matricielle (2)}
Soit $(A,B) \in \left( \mathcal{M}_n \left( \mathbb{R} \right) \right)^2$ avec $n \in \mathbb{N}$.
\subparagraph{1}Quelles sont les matrices $M \in \mathcal{M}_n \left( \mathbb{R} \right)$ qui vérifient $M + \text{tr}(M)A = B$ ?

\section{Endomorphisme de carré nul}
Soit $u$ un endomorphisme de $E$ (espace vectoriel de dimension finie). On suppose que $u^2=0$
\subparagraph{1}Montrer qu'il existe une base de $E$ tel que la représentation matricielle de $u$ soit la suivante :
\[
\left( \begin{matrix} 0 & I_r \\ 0 & 0 \end{matrix} \right)
\]
où $I_r$ est une matrice identité de taille $r \times r$ (que vaut $r$ ?).

\section{Une égalité d'endomorphismes}
Soit $(u,v)$ deux endomorphismes de $\mathbb{R}^2$ qui vérifient :
\begin{itemize}
\item $u^2=0$
\item $v^2=0$
\item $u \circ v = v \circ u$
\end{itemize}
\subparagraph{1}Que peut-on dire de $u \circ v$ ?

\section{Matrice et changement de base}
Soit $\mathcal{B}$ une base de $\mathbb{R}^n$. On note $M_{\mathcal{B}}(u)$ la matrice de $u$ (endomorphisme de $\mathbb{R}^n$) exprimé dans la base $\mathcal{B}$. 
\subparagraph{1}Montrer que si $(e_1,\dots,e_n)$ base de $\mathbb{R}^n$ alors $(e_1,\dots,e_i+e_j,\dots,e_n)$ avec $(i,j) \in \llbracket 1,n \rrbracket$ également.
\subparagraph{2}Soit $u$ un endomorphisme tel que $\forall i \in \llbracket 1,n \rrbracket, \ u(e_i) = \lambda_i e_i$ (avec $(\lambda_i)_{i \in \llbracket 1,n \rrbracket} \in \mathbb{R}^n$) quelle que soit la base $(e_1,\dots,e_n)$. Montrer que $u$ est une homothétie.
\subparagraph{3}En déduire les endomorphismes $u$ tels que $\exists M \in \mathcal{M}_n \left( \mathbb{R} \right), \ \forall \mathcal{B} \ \text{base de } \ \mathbb{R}^n, \ M_{\mathcal{B}}(u) = M$.

\subparagraph{Remarque :} cet exercice est à mettre en lien avec un autre exercice qui s'appuie sur les mêmes propriétés des endomorphismes et qui donne un résultat similaire : quels sont les endomorphismes qui stabilisent toutes les droites vectorielles ?


\section{Un polynôme annulateur}
Soit $u$ un endomorphisme de $\mathbb{R}^3$. On suppose que $u^3+u=0$.
\subparagraph{1}Montrer que $\text{ker}u \bigoplus \text{Im}u = \mathbb{R}^3$.
\subparagraph{2}En déduire qu'il existe une base de $\mathbb{R}^3$ tel que $u$ a pour matrice :
\[
M = \left( \begin{matrix}
0 & 0 & 0 \\ 0 & 0 & 1 \\ 0 & -1 & 0 
\end{matrix}\right)
\]

\section{Transformée de Fourier discrète}
Soit $\omega$ une racine n-ième de l'unité et $F_{\omega}(P) = \underset{k=0}{\overset{n-1}{\sum}}P(\omega^k)X^k$.
\subparagraph{1}Montrer que $F_{\omega}$ est un automorphisme de $\mathbb{R}_{n-1}[X]$.
\end{document}