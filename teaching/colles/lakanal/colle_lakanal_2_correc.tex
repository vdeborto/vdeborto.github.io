\documentclass[10pt,a4paper]{article} 
\usepackage[utf8]{inputenc} 
\usepackage[T1]{fontenc} 
\usepackage[english]{babel} 
\usepackage{supertabular} %Nécessaire pour les longs tableaux
\usepackage[top=2.5cm, bottom=2.5cm, right=1.5cm, left=1.5cm]{geometry} %Mise en page 
\usepackage{amsmath} %Nécessaire pour les maths 
\usepackage{amssymb} %Nécessaire pour les maths 
\usepackage{stmaryrd} %Utilisation des double crochets 
\usepackage{pifont} %Utilisation des chiffres entourés 
\usepackage{graphicx} %Introduction d images 
\usepackage{epstopdf} %Utilisation des images .eps 
\usepackage{amsthm} %Nécessaire pour créer des théorèmes 
\usepackage{algorithmic} %Nécessaire pour écrire des algorithmes 
\usepackage{algorithm} %Idem 
\usepackage{bbold} %Nécessaire pour pouvoir écrire des indicatrices 
\usepackage{hyperref} %Nécessaire pour écrire des liens externes 
\usepackage{array} %Nécessaire pour faire des tableaux 
\usepackage{tabularx} %Nécessaire pour faire de longs tableaux 
\usepackage{caption} %Nécesaire pour mettre des titres aux tableaux (tabular) 
\usepackage{color} %nécessaire pour écrire en couleur 
\newtheorem{thm}{Théorème} 
\newtheorem{mydef}{Définition} 
\newtheorem{prop}{Proposition} 
\newtheorem{lemma}{Lemme}
\title{Semaine 2 - Généralités sur les fonctions à une variable complexe ou réelle}
\author{Valentin De Bortoli \\ email : \ \href{mailto:valentin.debortoli@gmail.com}{valentin.debortoli@gmail.com}}
\date{}
\begin{document}
\maketitle

\section{Inégalité et dérivée}

\subparagraph{1}Il convient tout d'abord de remarquer que $f$ est bornée. En effet, $\forall x \in \mathbb{R}, \ f(x) \in [-1,1]$. Il s'agit ensuite de montrer que $f$ est décroissante. En effet,
\begin{equation}
\forall x \in \mathbb{R}, \ \vert 1+f'(x) \vert \le 1
\end{equation}
On remarque que $f'(x) \le 0$. Donc $f$ est décroissante et bornée donc elle admet une limite en $\pm \infty$.

\subparagraph{2} Supposons que $\exists A \in \mathbb{R}_+, \exists B \in \mathbb{R}_+, \ \forall x>B, \ f'(x)<-A$. Si cette condition est vérifiée alors on a $f(x) \le -Ax +C$ et cela contredit l'hypothèse de bornitude. Donc $\forall \epsilon \in \mathbb{R}_+, \forall B \in \mathbb{R}_+, \ \exists x>B, \ 0 \ge f'(x)>-\epsilon$. On peut donc construire une suite $(x_n)_{n \in \mathbb{N}} \in \mathbb{R}^{\mathbb{N}}$ telle que $x_n \ \rightarrow \ +\infty$ et $f'(x_n) \ \rightarrow \ 0$. On note $l$ la limite de $f$ en $+\infty$. On obtient $l^2 + (1+0)^2 \le 1$, c'est-à-dire $l=0$. Donc $f$ tend vers $0$ en $\pm \infty$. L'hypothèse de décroissance permet de conclure que $f=0$.

\subparagraph{3}Si on se place sur $\mathbb{R}_+$ la nullité de la fonction n'est plus vraie. On peut par exemple choisir $f(x) = e^{-x}$. Cette fonction est bien dérivable sur $\mathbb{R}_+^*$. On a,
\begin{equation}
\begin{aligned}
&\forall x \in \mathbb{R}_+^*, \ e^{-2x} (1-e^{-x})^2 \le 1 \\
&\Leftrightarrow \forall x \in \mathbb{R}_+^*, \ 2e^{-2x} -2e^{-x} \le 0 \\
&\Leftrightarrow \forall x \in \mathbb{R}_+^*, \ e^{-x} \le 1
\end{aligned}
\end{equation}

\section{Propriété de Darboux et croissance}

\subparagraph{1}Soit $x_0$ un point de discontinuité dans $]a,b[$. $\exists \epsilon \in \mathbb{R}_+^*, \ \forall x<x_0, \ f(x) + \epsilon \le f(x_0)$. Donc $f([a,b]) \subset [f(a), f(x_0)-\epsilon] \cup [f(x_0), f(b)]$ qui n'est pas un intervalle donc $f([a,b]) \neq [f(a) f(b)]$. On a démontré la propriété par contraposée.

\section{Une équation fonctionnelle (1)}

\subparagraph{1}Soit $(x_n)_{n \in \mathbb{N}}$ définie par,
\begin{equation}
\left\lbrace
\begin{aligned}
&x_0 \in ]0,1] \\
&\forall n \in \mathbb{N}, \ x_{n+1} = \sqrt{x_n}
\end{aligned}
\right.
\end{equation}
Alors on montre par récurrence que $(x_n)_{n \in \mathbb{N}}$ est croissante et majorée par $1$. Elle admet donc une limite et par continuité de la racine, celle-ci vaut $1$. Mais de plus $\forall n \in \mathbb{N}, \ f(x_{n+1}) = f(x_n)$. Donc par continuité de $f$, $f(x_0) = f(1)$. Donc $f$ constante sur $]0,1]$. La même construction (cette fois-ci on montre que la suite est décroissante et minorée) permet de conclure pour les $x_0>1$. Ainsi, $f$ est constante. Il est trivial de vérifier que les fonctions constantes vérifient la condition.

\section{Une équation fonctionnelle (2)}

\subparagraph{1}C'est un immense classique. Il s'agit de trouver les morphismes continus de $\mathbb{R}$ dans lui-même. On constate que $f(0) = 0$. On nomme $a=f(1)$. On démontre facilement par récurrence,
\begin{equation}
\forall n \in \mathbb{N}, \ f(n) = an
\end{equation}
Puisque $f(n+(-n)) = 0 = f(n) + f(-n)$ on a,
\begin{equation}
\forall n \in \mathbb{Z}, \ f(n) = an
\end{equation}
Soit $r = \frac{p}{q} \in \mathbb{Q}$. $f(rq) = qf(r) = f(p) = pa$ donc $f(r) = ar$. Soit $(r_n)_{n \in \mathbb{Q}}$ qui tend vers $x$ dans $\mathbb{R}$ alors,
\begin{equation}
f(x) = \underset{n \rightarrow +\infty}{\lim} f(r_n) = \underset{n \rightarrow +\infty}{\lim} ar_n = ax
\end{equation}. Donc on a,
\begin{equation}
\forall x \in \mathbb{R}, \ f(x) = ax
\end{equation}
Réciproquement, toute fonction linéaire satisfait la propriété $f(x+y) = f(x) +f(y)$ et est continue.

\section{Une équation fonctionnelle (3)}

\subparagraph{1}Soit $x \in \mathbb{R}$. On a,
\begin{equation}
\begin{aligned}
&f(f(f(x))) = f \left( \frac{x}{2}+3 \right) \\
&\frac{f(x)}{2} + 3 = f \left( \frac{x}{2}+3 \right).
\end{aligned}
\end{equation}
\subparagraph{2}On dérive la relation trouvée et on a,
\begin{equation}
\forall x \in \mathbb{R}, \ f'(x) = f'\left( \frac{x}{2}+3 \right)
\end{equation}
Soit $x_0 <6$ alors la suite $(x_n)_{n \in \mathbb{N}}$ définie par $\forall n \in \mathbb{N}, \ x_{n+1} = \frac{x_n}{2}+3$ est croissante et majorée (par récurrence). Elle converge vers $6$. Mais $f'(x_0) = f'(x_n) = f'(6)$ (par récurrence pour la première égalité, par passage à la limite pour la seconde). On aboutit à la même conclusion si $x_0 \ge 6$ (la même suite est considérée, on montre par récurrence qu'elle est décroissante et minorée). Donc, $\exists a \in \mathbb{R}, \ \forall x \in \mathbb{R}, \ f'(x) = a$.
\subparagraph{3}Ainsi $\exists (a,b) \in \mathbb{R}^2, \ \forall x \in \mathbb{R}, \ f(x) = ax+b$. Il s'agit de déterminer $a$ et $b$. En utilisant l'hypothèse,
\begin{equation}
\begin{aligned}
&\forall x \in \mathbb{R}, \ a(ax+b) +b = \frac{x}{2} +3\\
&\forall x \in \mathbb{R}, \ a^2x+b(a+1) = \frac{x}{2}+3
\end{aligned}
\end{equation}
Ainsi, $a = \frac{\sqrt{2}}{2}$ et $b = \frac{6}{2 + \sqrt{2}} = 6 - 3\sqrt{2}$

\section{Composition, injectivité et surjectivité}

\subparagraph{1}Il est évident que si $f$ est bijective alors $f$ est injective et surjective. De même si $f$ injective implique $f$ surjective alors cela implique que $f$ est bijective. La même remarque s'applique si $f$ surjective. Il suffit donc de montrer que $f$ injective est équivalent à $f$ surjective. 
\subparagraph{}Supposons que $f$ est injective alors soit $z \in \mathbb{R}$. $f(f(f(z))) = f(z)$ donc $z=f(f(z))$ par injectivité et donc $z \in \operatorname{Im}(f)$.
\subparagraph{}Supposons que $f$ est surjective. Soit $x$ et $y$ tels que $f(x)=f(y)$. Il existe $z_x$ et $z_y$ tels que $f(z_x) = x$ et $f(z_y) = y$. On a alors $x = f(z_x) = f(f(f(z_x))) = f(f(x)) = f(f(y)) = f(f(f(z_y))) = f(z_y) = y$.

\subparagraph{2}Il suffit de composer par $f^{-1}$ à gauche. On a $\forall x \in \mathbb{R}, \ f(f(x)) = x$. Donc $f^{-1}=f$. On dit que $f$ est une involution.
\section{Théorème de Cantor-Bernstein}
\subparagraph{1}Soit $y \in f(D)$, $\exists n \in \mathbb{N}, \ y \in f(D_n)$. Donc $y \in D_{n+1} \subset D$ et $y \in f(D) \subset C$. Donc $y \in C \cap D$. D'où $f(D) \subset C \cap D$.

\subparagraph{2} Soit $(x,y)\in A^2$ tels que $g(x) = g(y)$. On distingue trois cas :
\begin{itemize}
\item $(x,y) \in D^2$ alors $f(x) = f(y)$ et puisque $f$ est une injection $x=y$,
\item $(x,y) \in {}^cD$ et alors $x=y$,
\item $x \in D$ et $y \in D^c$ alors $f(x) = y$ donc $f(x) \in {}^cD$. Ce n'est pas possible car $f(D) \subset D$.
\end{itemize}
Ainsi dans tous les cas $g(x) = g(y)$ implique $x=y$ et donc $g$ injective.

\subparagraph{3}Montrons que $g$ est bijective. Soit $z \in C$. Si $z \in {}^cD$ alors $z = g(z)$ et donc $z \in \operatorname{Im}(g)$. Si $z \in D$ alors $\exists n \in \mathbb{N}$ tel que $z \in D_n$. $n \neq 0$ sinon $z \in {}^c C$ et c'est absurde. Donc $\exists n \ge 1$ tel que $z \in D_n$. Ainsi, $z \in f(D_{n-1})$. Donc dans tous les cas $z \in \operatorname{Im}(g)$ et $g$ surjective. On a donc $g$ bijective.

\subparagraph{4}$f_1 \circ f_2$ est une fonction injective de $B$ dans $\operatorname{Im}(f_1)$. Ainsi $B$ est en bijection avec $\operatorname{Im}(f_1)$ via $g_1$. Considérons $g_1 \circ f_1$. Cette fonction est injective car $f_1$ et $g_1$ le sont. De plus $f_1$ surjective sur $\operatorname{Im}(f_1)$ et $g_1$ surjective de $\operatorname{Im}(f_1)$ dans $B$. Ainsi $g_1 \circ f_1$ est une bijection entre $A$ et $B$.


\section{Union, intersection, image et image réciproque}
Soit $f$ une fonction de $\mathbb{R}$ dans $\mathbb{R}$ et $A$ et $B$ deux ensembles de $\mathbb{R}$.

\subparagraph{1} On a :
\begin{itemize}
\item \textbf{Vrai} : Soit $x \in A \cup B$ alors $x \in A$ par exemple et donc $f(x) \in f(A) \subset f(A) \cup f(B)$.
\item \textbf{Vrai} : Soit $y \in f(A) \cup f(B)$ alors $y \in f(A)$ par exemple et donc $y \in f(A \cup B)$. On conclut par double inclusion.
\item \textbf{Vrai} : Soit $x \in A \cap B$ alors $x \in A$ donc $f(x) \in f(A)$ mais aussi $x \in B$ donc $f(x) \in f(B)$. Donc $f(A\cap B) \subset f(A) \cap f(B)$.
\item \textbf{Faux} : $A = ]-1,0[$ et $B=]0,1[$. On pose $f:x \ \mapsto \ x^2$. $f(A \cap B) = \emptyset$ mais $f(A) = f(B) = f(A) \cap f(B) = ]0,1[$.
\end{itemize}

\subparagraph{2}Soit $x \in A \cup B$. Alors $f(x) \in f(A) \cup f(B)$ et donc $x \in f^{-1}(f(A) \cup f(B))$. Pour le contre-exemple on peut poser $A = [-\pi,\pi[$, $B = \lbrace 5 \rbrace$ et $f = \sin$. Dans ce cas $f(A) \cup f(B) = [-1,1]$. Donc $f^{-1}(f(A) \cup f(B)) = \mathbb{R}$. 

\section{Un théorème de point fixe (1)}
Soit $f$ une application croissante de $[0,1]$ dans $[0,1]$.
\subparagraph{1}Sous ensemble non-vide de $\mathbb{R}$ (contient $0$) et borné par $1$ donc admet une borne supérieure.

\subparagraph{2}Soit $x_0$ cette borne supérieure. Si $x=1$ alors $f(1)=1$ et $f$ admet un point fixe. Supposons que $x_0 <1$. Soit $x>x_0$ alors $f(x) > f(x_0)$ par croissance. Mais $x>f(x)$ sinon $x \in A$. Donc $\forall x\ge x_0 x \ge f(x_0)$. En considérant une suite qui tend vers $x_0$ on obtient que $f(x_0) \le x_0$. Si $x_0=0$ alors $f(x_0)=0$ et $f$ admet un point fixe. Supposons désormais que $x_0>0$. Alors par la propriété de la borne supérieure $\forall \epsilon>0, \ \exists x<x_0, \ x\in A \ \text{et} \ x_0 - \epsilon \le x \le x_0$. D'où par croissance $f(x) \le f(x_0)$ et par appartenance à $A$ $x\le f(x_0)$. On peut donc choisir une suite $(x_n)_{n \in \mathbb{N}} \in A^{\mathbb{N}} $ qui tend vers $x_0$ d'où $f(x_0) \ge x_0$ par passage à la limite. On peut conclure $f(x_0) = x_0$ et $f$ admet un point fixe.

\section{Un théorème de point fixe (2)}
Soit $f$ une application continue de $[0,1]$ dans $[0,1]$.
\subparagraph{1}Sous ensemble non-vide de $\mathbb{R}$ (contient $0$) et borné par $1$ donc admet une borne supérieure.

\subparagraph{2}Soit $x_0$ cette borne supérieure. Si $x=1$ alors $f(1)=1$ et $f$ admet un point fixe. Supposons que $x_0 <1$. Soit $x>x_0$ alors $x>f(x)$ sinon $x \in A$. En considérant une suite qui tend vers $x_0$ et telle que tout élément de la suite est plius grand que $x_0$ on obtient que $f(x_0) \le x_0$ par continuité de $f$. $A = g^{-1}(\mathbb{R}_+)$ avec $g = f- \operatorname{Id}$. Donc, puisque $g$ est continue, $A$ est fermé car $\mathbb{R}_+$ l'est. Ainsi la borne supérieure de $A$ est un élément de $A$ et donc $f(x_0) \ge x_0$. Ainsi $x_0$ est un point fixe.

\subparagraph{Remarque :} on aurait pu conclure en introduisant le $\epsilon$ comme pour l'exercice précédent et conclure via la continuité mais il revient au même de faire une remarque sur l'image réciproque d'un fermé.

\section{Itérées et continuité}
\subparagraph{1}La surjectivité est triviale car $f(f^{n-1}(x)) = x$. Soit $x$ et $y$ tels que $f(x) = f(y)$. Soit $N = n_x n_y$, alors ${x = f^{n_x}\left( ... \left((f^{n_x}(x))\right) \right) = f^N(x) = f^N(y) = f^{n_y}\left( ... \left((f^{n_y}(x))\right) \right) = y} $. Donc $f$ est injective.
\subparagraph{2}Il s'agit de prouver que si une fonction est injective et continue alors elle est monotone. Un moyen assez élégant d'y parvenir et de considérer les trois ensembles suivants :
\begin{itemize}
\item $A = \lbrace (x,y) \in [0,1]^2, \ x<y\rbrace$
\item $B = \lbrace (x,y) \in A, \ f(x) - f(y) <0 \rbrace$
\item $C = \lbrace (x,y) \in A, \ f(x) - f(y) <0 \rbrace$
\end{itemize}
$A$ est connexe.
Par continuité, $B$ et $C$ sont complémentaires dans $A$. $B$ et $C$ sont ouverts dans $A$. Par continuité de $f$, $B$ et $C$ sont connexes. Donc $A$ est la réunion de $B$ et $C$, deux ouverts disjoints. Par connexité, cela implique que $B = \emptyset$ ou $C = \emptyset$, c'est-à-dire $f$ croissante ou décroissante.
\subparagraph{Remarque :} il n'est pas nécessaire de passer par des notions un peu subtiles de connexité mais les autres démonstrations de cette propriété sont un peu lourdes avec des outils de début de MPSI.
\subparagraph{3}Soit $x \in [0,1]$. Supposons que $f$ est croissante. Si $x \le f(x)$ alors ${x \le f(x) \le \dots \le f^{n_x-1}(x) \le x}$. Donc $x = f(x)$. De même si $x \ge f(x)$. Donc $\forall x \in [0,1], \ f(x) =x$. Supposons que $f$ est décroissante. Alors $f \circ f$ est croissante et vérifie les hypothèses de l'énoncé. Donc $\forall x \in [0,1], \ f(f(x)) = x$.

\end{document}