\documentclass[10pt,a4paper]{article} 
\usepackage[utf8]{inputenc} 
\usepackage[T1]{fontenc} 
\usepackage[english]{babel} 
\usepackage{supertabular} %Nécessaire pour les longs tableaux
\usepackage[top=2.5cm, bottom=2.5cm, right=1.5cm, left=1.5cm]{geometry} %Mise en page 
\usepackage{amsmath} %Nécessaire pour les maths 
\usepackage{amssymb} %Nécessaire pour les maths 
\usepackage{stmaryrd} %Utilisation des double crochets 
\usepackage{pifont} %Utilisation des chiffres entourés 
\usepackage{graphicx} %Introduction d images 
\usepackage{epstopdf} %Utilisation des images .eps 
\usepackage{amsthm} %Nécessaire pour créer des théorèmes 
\usepackage{algorithmic} %Nécessaire pour écrire des algorithmes 
\usepackage{algorithm} %Idem 
\usepackage{bbold} %Nécessaire pour pouvoir écrire des indicatrices 
\usepackage{hyperref} %Nécessaire pour écrire des liens externes 
\usepackage{array} %Nécessaire pour faire des tableaux 
\usepackage{tabularx} %Nécessaire pour faire de longs tableaux 
\usepackage{caption} %Nécesaire pour mettre des titres aux tableaux (tabular) 
\usepackage{color} %nécessaire pour écrire en couleur 
\newtheorem{thm}{Théorème} 
\newtheorem{mydef}{Définition} 
\newtheorem{prop}{Proposition} 
\newtheorem{lemma}{Lemme}

\newcommand{\al}[1]{\begin{aligned} #1 \end{aligned}}

\newcommand{\seq}[2]{\left( #1_{#2} \right)_{#2 \in \mathbb{N}} }
\newcommand{\intt}[4]{\int_{#1}^{#2} #3 \mathop{}\!\mathrm{d} #4}
\newcommand{\summ}[2]{\underset{#1}{\overset{#2}{\sum}}}

\newcommand{\vertt}[1]{\vert #1 \vert}

\title{Semaine 7 - Développements limités}
\author{Valentin De Bortoli \\ email : \ \href{mailto:valentin.debortoli@gmail.com}{valentin.debortoli@gmail.com}}
\date{}
\begin{document}
\maketitle
\section{Développements limités (1)}
\subparagraph{1} Justifier l'existence et calculer le développement limité en 0 à l'ordre 4 de $x \mapsto \frac{e^x}{(1+x)^3}$.
\subparagraph{2} Justifier l'existence et calculer le développement limité en 0 à l'ordre 6 de $x \mapsto\sin(x^2)$
\subparagraph{3} Justifier l'existence et calculer le développement limité en 0 à l'ordre 6 de $x \mapsto \ln(1+x)\sin(x)$

\section{Développements limités et asymptotiques (1)}
Soit $f : x \mapsto \sqrt{1+x+x^2}$.
\subparagraph{1}Justifier l'existence et calculer un développement limité de $f$ en 0 à l'ordre 2.
\subparagraph{2}Le graphe de $f$ admet-il une tangente en $0$ ? Si oui, donner la position du graphe de $f$ par rapport à cette tangente autour de $0$.
\subparagraph{3}Déterminer une asymptote en $+\infty$ au graphe de $f$.

\section{Développements limités et asymptotiques (2)}
Soit $f \mapsto (x^2-1) \ln\left(\vert \frac{1+x}{1-x} \vert\right)$.
\subparagraph{1}Justifier l'existence et calculer un développement limité de $f$ en 0 à l'ordre 3.
\subparagraph{2}Déterminer une asymptote en $+\infty$ au graphe de $f$ et donner la position de la courbe par rapport à cette asymptote lorsque $x$ est grand.

\section{Développements limités (2)}
\subparagraph{1}Justifier l'existence et calculer un développement limité en 0 à l'ordre 5 de $x \mapsto \ln\left( \sqrt{\frac{1+x}{1-x}} \right)$.
\subparagraph{2}Justifier l'existence et calculer un développement limité en 0 à l'ordre 2 de $x \mapsto \frac{\ln\left( \sqrt{\frac{1+x}{1-x}} \right)-x}{\sin(x)-x}$.

\section{Développements limités et dérivabilité}
\subparagraph{1}Montrer que $f$ est continue en 0 si et seulement si $f$ admet un développement limité d'ordre 0 en 0.
\subparagraph{2}Montrer que $f$ est dérivable en 0 si et seulement si $f$ admet un développement limité d'ordre 1 en 0.
\subparagraph{3}Montrer que si $f$ est deux fois dérivable en 0 alors $f$ admet un développement limité d'ordre 2 en 0.
\subparagraph{4}Montrer que $x \mapsto x^3 \sin(\frac{1}{x})$ définie sur $\mathbb{R}^*$ et prolongée par continuité en 0 admet un développement limité à l'ordre 2 mais n'est pas 2 fois dérivable en 0.

\section{Calcul de développements limités}
\subparagraph{1}Justifier l'existence et calculer un développement limité en $\frac{\pi}{2}$ à l'ordre 2 de $x \mapsto \ln(\sin(x))$.
\subparagraph{2}Justifier l'existence et calculer un développement limité en $\frac{\pi}{2}$ à l'ordre 2 de $x \mapsto (1+\cos(x))^{\frac{1}{x}}$.

\section{Développement limité et approximation par une fraction rationnelle d'ordre 2}
\subparagraph{1}Déterminer $(a,b)\in \mathbb{R}^2$ telle que la partie principal de $x \mapsto \cos(x) -\frac{1+ax^2}{1+bx^2}$ en $0$ soit la plus petite possible.
\subparagraph{2}Donner un équivalent de $x \mapsto \cos(x) -\frac{1+ax^2}{1+bx^2}$ en $0$ pour les valeurs de $(a,b)$ trouvées.

\section{Suite et équivalent (1)}
\subparagraph{1}Montrer que $\forall n \in \mathbb{N}, \exists x_n \in ]n\pi-\frac{\pi}{2},n\pi+\frac{\pi}{2}[ \ \vert \ \tan(x_n)=x_n$.
\subparagraph{2}Montrer $x_n \underset{+\infty}{\sim} n\pi$.
\subparagraph{3}Montrer $x_n-n\pi-\frac{\pi}{2} \underset{+\infty}{\rightarrow} 0$.
\subparagraph{4}Montrer que $x_n=n\pi+\frac{\pi}{2}-\frac{1}{n \pi}+\frac{1}{2 \pi n^2}+o(\frac{1}{n^2})$.

\section{Suite et équivalent (2)}
\subparagraph{1}Montrer que $e^x+x-n=0$ admet une unique solution sur $\mathbb{R}$. On la note $u_n$.
\subparagraph{2}En posant $v_n=u_n-\ln(n)$, montrer que $v_n \underset{+\infty}{\rightarrow} 0$.
\subparagraph{3}Trouver un équivalent de $v_n$ en $+\infty$.
\subparagraph{4}En déduire un développement asymptotique de $u_n$.

\section{Réciproque et développement limité (1)}
\subparagraph{1}Montrer que $f \ : \ x \mapsto 2 \tan(x) -x$ est une bijection $\mathcal{C}^{\infty}$ de $]-\frac{\pi}{2},\frac{\pi}{2}[$ dans $\mathbb{R}$. Montrer que $f^{-1}$ est impaire.
\subparagraph{2}Donner un développement limité à l'ordre $3$ de $f^{-1}$.

\section{Réciproque et développement limité (2)}
\subparagraph{1}Montrer que $x \mapsto x\exp(x^2)$ est une bijection de $\mathbb{R}$ dans $\mathbb{R}$.
\subparagraph{2}On admet que $f^{-1}$ est infiniment dérivable sur $\mathbb{R}$. Donner un développement limité à l'ordre $4$ de $f^{-1}$.

\section{Développement limité et grand ordre}
\subparagraph{1}Donner un développement à l'ordre $1000$ de $x \mapsto \ln(\summ{k=0}{999} \frac{x^k}{k!})$.

\section{Dérivée de grand ordre}
\subparagraph{1}Donner la dérivée en 0 d'ordre $1000$ de $x \mapsto \frac{x^4}{1+x^6}$.


\end{document}