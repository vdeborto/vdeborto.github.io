\documentclass[10pt,a4paper]{article} 
\usepackage[utf8]{inputenc} 
\usepackage[T1]{fontenc} 
\usepackage[english]{babel} 
\usepackage{supertabular} %Nécessaire pour les longs tableaux
\usepackage[top=2.5cm, bottom=2.5cm, right=1.5cm, left=1.5cm]{geometry} %Mise en page 
\usepackage{amsmath} %Nécessaire pour les maths 
\usepackage{amssymb} %Nécessaire pour les maths 
\usepackage{stmaryrd} %Utilisation des double crochets 
\usepackage{pifont} %Utilisation des chiffres entourés 
\usepackage{graphicx} %Introduction d images 
\usepackage{epstopdf} %Utilisation des images .eps 
\usepackage{amsthm} %Nécessaire pour créer des théorèmes 
\usepackage{algorithmic} %Nécessaire pour écrire des algorithmes 
\usepackage{algorithm} %Idem 
\usepackage{bbold} %Nécessaire pour pouvoir écrire des indicatrices 
\usepackage{hyperref} %Nécessaire pour écrire des liens externes 
\usepackage{array} %Nécessaire pour faire des tableaux 
\usepackage{tabularx} %Nécessaire pour faire de longs tableaux 
\usepackage{caption} %Nécesaire pour mettre des titres aux tableaux (tabular) 
\usepackage{color} %nécessaire pour écrire en couleur 
\newtheorem{thm}{Théorème} 
\newtheorem{mydef}{Définition} 
\newtheorem{prop}{Proposition} 
\newtheorem{lemma}{Lemme}
\title{Semaine 6 - Dérivabilité et équations différentielles du second ordre à coefficients constants}
\author{Valentin De Bortoli \\ email : \ \href{mailto:valentin.debortoli@gmail.com}{valentin.debortoli@gmail.com}}
\date{}
\begin{document}
\maketitle
\section{Résolution d'équation différentielle du second ordre (1)}
\subparagraph{1}Déterminer les solutions réelles de $y''(x)-3y'(x)+2y(x)=\sin(2x)$.

\section{Résolution d'équation différentielle du second ordre (2)}
\subparagraph{1}Déterminer les solutions réelles de $y''(x)+y(x)=\sinh(x)$.
\subparagraph{2}Déterminer les solutions réelles de $y''(x)-2y'(x)+y(x)=2\cosh(x)$.

\section{Résolution d'équation différentielle du second ordre (3)}
\subparagraph{1}Déterminer les solutions réelles de $y''(x)+2y'(x)+y(x)=\sin(x)^3$.
\subparagraph{2}Déterminer les solutions réelles de $y''(x)+y(x)=2\cos(x)^2$.

\section{Résolution d'équation différentielles du second ordre (4)}
\subparagraph{1}Déterminer les solutions réelles de $ax^2y''(x)+bxy'(x)+cy(x)=0$ sur $\mathbb{R}_+^*$.
\subparagraph{Remarque :} on pourra penser à poser $z(t)=y(e^t)$ et remarquer que $z$ vérifie une équation différentielle.

\section{Solutions bornées et équations différentielles du second ordre}
\subparagraph{1}Déterminer les couples $(a,b)\in \mathbb{R}^2$ tels que toute solution de $y''(x)+ay'(x)+by(x)=0$ soit bornée sur $\mathbb{R}_+$.

\section{Des équations presque différentielles}
\subparagraph{1}Soit $\lambda \in \mathbb{R}$. Trouver les fonctions de classe $\mathcal{C}^1(\mathbb{R})$ qui satisfont sur $\mathbb{R}$, $f'(x) = f(\lambda - x)$.
\subparagraph{2}Trouver les fonctions de classe $\mathcal{C}^1(\mathbb{R})$ qui satisfont sur $\mathbb{R}$, $f'(x) + f(-x) = e^x$.

\section{Racines réelles de polynôme (1)}
Soit $(a,b) \in \mathbb{R}^2$, $n \in \mathbb{N}$.
\subparagraph{1}Montrer que le polynôme $X^n+aX+b$ admet au plus trois racines réelles.

\section{Racines réelles de polynômes (2)}
\subparagraph{1}Montrer que $P_n=((1-X^2)^n)^{(n)}$ est un polynôme de degré $n$ dont les racines sont réelles, simples et appartiennent à $[-1,1]$.

\section{Théorème des accroissements finis (1)}
Soit $(\alpha,\beta,\gamma) \in \mathbb{R}^3$. De même, soit $(a,b) \in \mathbb{R}^2$
\subparagraph{1}Rappeler et démontrer le théorème des accroissements finis.
\subparagraph{2}Soit $f \ : \ \mathbb{R} \ \rightarrow \ \mathbb{R}$, définie par $f(x)=\alpha x^2+\beta x +\gamma$. Déterminer le point $"c"$ du théorème des accroissements finis.
\subparagraph{3}Interpréter géométriquement ce résultat.

\section{Théorème des accroissements finis (2)}
Soit $f: \ [a,b] \ \rightarrow \ \mathbb{R}$ et $g: \ [a,b] \ \rightarrow \ \mathbb{R}$ dérivables sur $]a,b[$, continues sur $[a,b]$, et telles que $g'$ ne s'annule pas sur $]a,b[$.
\subparagraph{1}Montrer que $g(b) \neq g(a)$.
\subparagraph{2}En s'inspirant de la preuve du théorème des accroissements finis, montrer que : $\exists c \in ]a,b[, \ \frac{f(b)-f(a)}{g(b)-g(a)}=\frac{f'(c)}{g'(c)}$.
\subparagraph{3}On suppose que $\underset{x \rightarrow b}{\lim}\frac{f'(x)}{g'(x)}$ existe. Montrer que $\underset{x \rightarrow b}{\lim}\frac{f(x)-f(a)}{g(x)-g(a)}=\underset{x \rightarrow b}{\lim}\frac{f'(x)}{g'(x)}$.
\subparagraph{4}En déduire $\underset{x \rightarrow 1^-}{\lim} \frac{\arccos(x)}{\sqrt{1-x^2}}$.
\subparagraph{Remarque :} ce théorème est une généralisation du théorème des accroissements finis. On peut l'interpréter de la même manière que le théorème des accroissements finis classique mais pour une courbe paramétrée du plan.
 
\section{Théorème de Rolle généralisé}
Le but est de montrer le théorème de Rolle généralisé. Soit $f: \ [a,+\infty[ \ \rightarrow \ \mathbb{R}$, continue sur $[a,+\infty[$, dérivable sur $]a,+\infty[$. On suppose de plus que $f$ possède une limite en $+\infty$ et que celle-ci vaut $f(a)$. On peut alors dire : $\exists c \in ]a,+\infty[, \ f'(c)=0$.
\subparagraph{1}On définit $g : [a,a+\frac{\pi}{2}[ \ \rightarrow \ \mathbb{R}$ par $g(x)=f(\tan(x-a))$. Montrer que l'on peut prolonger $g$ par continuité en $a+\frac{\pi}{2}$.
\subparagraph{2}Appliquer le théorème de Rolle à $g$ et conclure sur le théorème de Rolle généralisé.
\subparagraph{3}Montrer via le théorème de Rolle généralisé que $x \mapsto \frac{\ln(x)}{x}$ voit sa dérivée s'annuler au moins une fois sur $]1,+\infty[$.

\section{Une propriété du logarithme}
Soit $(x,y)\in \mathbb{R}^2$, avec $0<x<y$.
\subparagraph{1}Montrer que $x < \frac{y-x}{\ln(y)-\ln(x)} <y$.
\subparagraph{2}Considérer la fonction $f:[0,1] \ \rightarrow \ \mathbb{R}$, telle que $f(\alpha)=\ln(\alpha y+(1-\alpha)x))-\alpha \ln(y) -(1-\alpha) \ln(x)$. Montrer que $f$ est positive.
\subparagraph{3}Interpréter géométriquement cette inégalité.
\end{document}