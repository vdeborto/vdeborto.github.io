\documentclass[10pt,a4paper]{article} 
\usepackage[utf8]{inputenc} 
\usepackage[T1]{fontenc} 
\usepackage[english]{babel} 
\usepackage{supertabular} %Nécessaire pour les longs tableaux
\usepackage[top=2.5cm, bottom=2.5cm, right=1.5cm, left=1.5cm]{geometry} %Mise en page 
\usepackage{amsmath} %Nécessaire pour les maths 
\usepackage{amssymb} %Nécessaire pour les maths 
\usepackage{stmaryrd} %Utilisation des double crochets 
\usepackage{pifont} %Utilisation des chiffres entourés 
\usepackage{graphicx} %Introduction d images 
\usepackage{epstopdf} %Utilisation des images .eps 
\usepackage{amsthm} %Nécessaire pour créer des théorèmes 
\usepackage{algorithmic} %Nécessaire pour écrire des algorithmes 
\usepackage{algorithm} %Idem 
\usepackage{bbold} %Nécessaire pour pouvoir écrire des indicatrices 
\usepackage{hyperref} %Nécessaire pour écrire des liens externes 
\usepackage{array} %Nécessaire pour faire des tableaux 
\usepackage{tabularx} %Nécessaire pour faire de longs tableaux 
\usepackage{caption} %Nécesaire pour mettre des titres aux tableaux (tabular) 
\usepackage{color} %nécessaire pour écrire en couleur 
\newtheorem{thm}{Théorème} 
\newtheorem{mydef}{Définition} 
\newtheorem{prop}{Proposition} 
\newtheorem{lemma}{Lemme}
\title{Semaine 13 - Arithmétique des polynômes}
\author{Valentin De Bortoli \\ email : \ \href{mailto:valentin.debortoli@gmail.com}{valentin.debortoli@gmail.com}}
\date{}
\begin{document}
\maketitle
\section{Écriture binaire et polynôme}
Soit $P_n(x)=(1+X)(1+X^2) \dots (1+X^{2^n})$ avec $n \in \mathbb{N}$.
\subparagraph{1}Donner la forme développée de $P_n$.
\subparagraph{2}Montrer que tout entier $p \in \mathbb{N}$ s'écrit de manière unique comme la somme de puissance de deux.
\subparagraph{Remarque :} ce résultat permet de montrer de manière élégante, l'existence et l'unicité de l'écriture binaire des entiers.
\section{Équations polynomiale(s) (1)}
Résoudre dans $k[X]$ les équations suivantes.
\subparagraph{1} $Q^2=XP^2$ en $(P,Q)$.
\subparagraph{2} $P \circ P =P$ en $P$.
\subparagraph{3} $P(X^2)=(X^2+1)P(X)$ en $P$
\section{Équations polynomiale(s) (2)}
Soit $P \in \mathbb{C}[X]$ tel que $P(X^2)=P(X)P(X-1)$ et $P$ non nul.
\subparagraph{1}Montrer que les racines de $P$ sont de module 1.
\subparagraph{2}Déduire $P$.
\section{Intégration et polynômes (1)}
Soit $[ a,b ]$ un intervalle non vide de $\mathbb{R}$. Soit $f \in \mathcal{C}([a,b])$. Soit $n \in \mathbb{N}$.
\subparagraph{1}On suppose que $\forall k \in \llbracket 0,n \rrbracket, \ \int_a^b f(t) t^k \text{d}t=0$. Montrer que $f$ s'annule au moins $n+1$ fois.
\subparagraph{Remarque :} le théorème de Weierstrass permet de montrer une version limite de ce théorème à savoir : si $\forall P \in \mathbb{R}[X], \ \int_a^b f(x)P(x) \text{d}x=0$, alors $f=0$.
\section{Intégration et polynômes (2)}
\subparagraph{1}Trouver tous les polynômes de $\mathbb{R}[X]$ qui vérifient : $\forall k \in \mathbb{N}, \ \int_k^{k+1}P(x) \text{d}x= k+1$.
\section{Localisation des racines}
Soit $P=X^n+a_{n-1}X^{n-1}+ \dots +a_0$ un polynôme de $\mathbb{C}[X]$. Soit $z$ une racine complexe de $P$.
\subparagraph{1}Montrer que $\vert z \vert \le 1 + \underset{j \in \llbracket 0,n-1 \rrbracket}{\max} \vert a_j \vert$.
\subparagraph{Remarque :} cette majoration permet de réduire l'ensemble de recherche des racines du polynômes. D'autres techniques permettent d'affiner le domaine : règle de changement des signes de Descartes, suites de Sturm, disques de Gershgörin.
\section{Le théorème de Gauss-Lucas}
Soit $P \in \mathbb{C}[X]$.
\subparagraph{1}Montrer que toute racine de $P'$ est barycentre des racines de $P$.
\section{Majoration des coefficients}
Soit $P=a_nX^n+a_{n-1}X^{n-1}+ \dots +a_0$.
\subparagraph{1}Calculer $P(1)+P(\omega)+\dots+P(\omega^n)$ avec $\omega$ une racine $n+1$-ème de l'unité.
\subparagraph{2}En déduire que $\forall k \in \llbracket 0,n\rrbracket, \ \vert a_k \vert \le M$ avec $M=\underset{z \in \mathbb{U}}{\sup}(\vert P(z) \vert)$.
\section{Localité et polynômes}
Soit $f$ une fonction sur $\mathbb{R}$ localement polynômiale :
\begin{equation*}
\forall x_0 \in \mathbb{R}, \ \exists (\epsilon,P_{x_0}) \in \mathbb{R}_+^* \times \mathbb{R}[X], \ \forall x \in ]x_0-\epsilon,x_0+\epsilon[, \ f(x)=P(x)
\end{equation*}
\subparagraph{1}Montrer que $f$ est un polynôme.
\subparagraph{Remarque :} on peut encore affaiblir les hypothèses (théorème de Balaguer-Corominas) :
\begin{equation*}
\forall x \in \mathbb{R}, \ \exists n_x, \ f^{(n_x)}(x)=0 \ \Leftrightarrow \ f \ \text{est polynômiale.}
\end{equation*}
\section{Trigonométrie et polynômes}
\subparagraph{1}Peut-on écrire la fonction $\cos$ comme un polynôme ?
\section{Racines réelles de polynôme (1)}
Soit $(a,b) \in \mathbb{R}^2$, $n \in \mathbb{N}$.
\subparagraph{1}Montrer que le polynôme $X^n+aX+b$ admet au plus trois racines réelles.

\section{Racines réelles de polynômes (2)}
\subparagraph{1}Montrer que $P_n=((1-X^2)^n)^{(n)}$ est un polynôme de degré $n$ dont les racines sont réelles, simples et appartiennent à $[-1,1]$.

\section{Condition nécessaire et suffisante de primalité}
Soit $A$ et $B$ deux polynômes de $k[X]$ non constants.
\subparagraph{1} Montrer que les propositions suivantes sont équivalentes :
\begin{itemize}
\item $A$ et $B$ sont premiers entre eux.
\item $\exists (U,V) \in k[X]^2, \ AU+BV = 1$ et $\text{deg}(U)<\text{deg}(B)$ et $\text{deg}(V)<\text{deg}(A)$.
\end{itemize}
\subparagraph{2} Montrer que les propositions suivantes sont équivalentes :
\begin{itemize}
\item $A$ et $B$ ne sont pas premiers entre eux.
\item $\exists (U,V) \in k[X]^2, \ AU+BV = 0$ et $\text{deg}(U)<\text{deg}(B)$ et $\text{deg}(V)<\text{deg}(A)$.
\end{itemize}

\section{Factorisation (1)}
\subparagraph{1}Factoriser dans $\mathbb{C}[X], \ (X+i)^n-(X-i)^n$.
\subparagraph{Remarque : } on rappelle que $\frac{1-x^2}{1+x^2}=\cos(\theta)$ et $\frac{2x}{1+x^2}=\sin(\theta) \ \Leftrightarrow \ x=\tan(\frac{theta}{2})$.

\section{Factorisation (2)}
\subparagraph{1}Factoriser dans $\mathbb{C}[X], \ X^n+2\cos(na)X+1$.
\subparagraph{2}Factoriser ce polynôme dans $\mathbb{R}[X]$.

\section{Division euclidienne et polynôme}
Soit $(n,m)\in \mathbb{N}^2$. Soit $r$ le reste de la division euclidienne de $n$ par $m$.
\subparagraph{1} Donner le reste de la division euclidienne de $X^n-1$ par $X^m-1$.
\subparagraph{2} Montrer que $X^n-1 \wedge X^m-1 = X^{n \wedge m}-1$.
\end{document}