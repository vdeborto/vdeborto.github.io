\documentclass[10pt,a4paper]{article} 
\usepackage[utf8]{inputenc} 
\usepackage[T1]{fontenc} 
\usepackage[english]{babel} 
\usepackage{supertabular} %Nécessaire pour les longs tableaux
\usepackage[top=2.5cm, bottom=2.5cm, right=1.5cm, left=1.5cm]{geometry} %Mise en page 
\usepackage{amsmath} %Nécessaire pour les maths 
\usepackage{amssymb} %Nécessaire pour les maths 
\usepackage{stmaryrd} %Utilisation des double crochets 
\usepackage{pifont} %Utilisation des chiffres entourés 
\usepackage{graphicx} %Introduction d images 
\usepackage{epstopdf} %Utilisation des images .eps 
\usepackage{amsthm} %Nécessaire pour créer des théorèmes 
\usepackage{algorithmic} %Nécessaire pour écrire des algorithmes 
\usepackage{algorithm} %Idem 
\usepackage{bbold} %Nécessaire pour pouvoir écrire des indicatrices 
\usepackage{hyperref} %Nécessaire pour écrire des liens externes 
\usepackage{array} %Nécessaire pour faire des tableaux 
\usepackage{tabularx} %Nécessaire pour faire de longs tableaux 
\usepackage{caption} %Nécesaire pour mettre des titres aux tableaux (tabular) 
\usepackage{color} %nécessaire pour écrire en couleur 
\newtheorem{thm}{Théorème} 
\newtheorem{mydef}{Définition} 
\newtheorem{prop}{Proposition} 
\newtheorem{lemma}{Lemme}
\title{Semaine 19 - Théorie de la dimension et applications linéaires}
\author{Valentin De Bortoli \\ email : \ \href{mailto:valentin.debortoli@gmail.com}{valentin.debortoli@gmail.com}}
\date{}
\begin{document}
\maketitle
Dans la suite $k$ est un corps (on se limite à $\mathbb{R}$ et $\mathbb{C}$) et $E$ un $k$-espace vectoriel.

\section{Matrices circulantes et polygones}
Soit $(a_i)_{i \in \llbracket 1,n\rrbracket} \in \mathbb{C}^n$ ($n \in \mathbb{N}$).
\subparagraph{1}Peut-on trouver $(z_i)_{i \in \llbracket 1,n \rrbracket}$ un polygone du plan complexe tel que $a_i$ soit le milieu de $[z_i,z_{i+1}]$ si $i<n$ et $a_n$ milieu de $[z_n,z_1]$ ?

\section{Espace vectoriel et fonctions affines}
\subparagraph{1}Soit $F$ l'ensemble des fonctions continues de $[-1,1]$ affines sur $[-1,0]$ et affines sur $[0,1]$. Montrer que $F$ est un sous-espace vectoriel (de quel espace vectoriel ?).
\subparagraph{2}Trouver une base de $F$. 

\section{Une base de polynômes}
\subparagraph{1}Montrer que $(P_k)_{k \in \llbracket 0,n \rrbracket}$ avec $P_k=X^k(1-X)^{n-k}$ est une base de $\mathbb{R}_n[X]$. 

\section{Nombres réels et espace vectoriel}
Le but de cet exercice est d'étudier $\mathbb{R}$ comme $\mathbb{Q}$-espace vectoriel. On note $(p_n)_{n \in \mathbb{N}}$ l'ensemble des nombres premiers rangés par ordre croissant.
\subparagraph{1} Montrer que $\forall N \in \mathbb{N}, \ (p_n)_{n \in \llbracket 1,N \rrbracket}$ est une famille libre de $\mathbb{R}$. En déduire qu'il n'existe pas de base finie de $\mathbb{R}$ comme $\mathbb{Q}$-espace vectoriel.
\subparagraph{2} Autre démonstration : si $(x_n)_{n \in \llbracket 1,N \rrbracket}$ est une base de $\mathbb{R}$ comme $\mathbb{Q}$-espace vectoriel en déduire que tout $x\in \mathbb{R}$ est racine d'un polynôme de degré $N-1$.
\subparagraph{3}En considérant $2^{1/N}$ en déduire une contradiction (on admettra que $X^n-2$ est un polynôme irréductible de $\mathbb{Q}[X]$).

\subparagraph{Remarque :} en fait on peut même montrer en considérant la famille $(\underset{n \ge 1}{\sum} \frac{1}{10^{\lfloor a^n \rfloor}})_{a>1}$ que toute base de $\mathbb{R}$ comme $\mathbb{Q}$-espace vectoriel est de cardinal celui de $\mathbb{R}$. L'existence d'une base de $\mathbb{R}$ est assurée par l'axiome du choix (bases de Hamel) mais n'est pas constructible.

\section{Polynômes à valeurs entières}
\subparagraph{1}Montrer que $(P_k)_{k \in \llbracket 0,n\rrbracket}$ avec $\forall k \in \llbracket 1,n \rrbracket, \ P_k = \frac{X (X-1) \dots (X-k+1)}{k !}$ et $P_0=1$. Montrer que $(P_k)_{k \in \llbracket 0,n\rrbracket}$ base de $\mathbb{R}_n[X]$.
\subparagraph{2}Montrer que $\forall m \in \mathbb{Z}, \forall k \in \llbracket 0,n\rrbracket, \ P_k(m) \in \mathbb{Z}$.
\subparagraph{3}En déduire la forme des polynômes de $\mathbb{R}_n[X]$ qui prennent des valeurs entières sur les entiers.

\section{Divisibilité et sous-espace vectoriel}
Soit $A$ polynôme de $\mathbb{R}_n[X]$.
\subparagraph{1}Montrer que $F=\lbrace P \in \mathbb{R}_n[X], A \vert P \rbrace$ est un sous-espace vectoriel.
\subparagraph{2}Exhiber une base et un supplémentaire de cet espace.

\section{Une équation polynômiale}
\subparagraph{1}Montrer qu'il existe un unique polynôme $P\in \mathbb{R}_{n+1}[X]$ tel que $P(0)=0$ et $P(X+1)-P(X)=X^n$.

\section{Une somme directe}
\subparagraph{1}Soit $i \in \llbracket 0,n \rrbracket$ et  $F_i=\lbrace P \in \mathbb{R}_n[X], \forall j \in \llbracket 0,n \rrbracket \backslash \lbrace i \rbrace, \ P(j)=0, \ P(i) \neq 0 \rbrace$. Montrer que $F_i \cup \lbrace 0 \rbrace$ est un espace vectoriel.
\subparagraph{2}Montrer que $\mathbb{R}_n[X]=F_0 \oplus \dots \oplus F_n$.

\section{Drapeaux}
Soit $u$ un endomorphisme de $E$. 
\subparagraph{1}Montrer que $\forall k \in  \mathbb{N}, \ \text{ker}u^k \subset \text{ker}u^{k+1}$. Conjecturer et prouver une propriété similaire sur $\text{Im}u^k$.
\subparagraph{2}On suppose qu'il existe $p \in \mathbb{N}$ tel que $\text{ker}u^n=\text{ker}u^{n+1}$. Montrer que pour tout $p \in \mathbb{N}, \ p \ge n \ \Rightarrow \ \text{ker}u^p=\text{ker}u^{p+1}$.
\subparagraph{3}En déduire que pour ce $n$, $\text{ker}u^n$ et $\text{Im}u^n$ sont en somme directe. Que peut-on dire dans le cas de la dimension finie ?

\section{Stabilisation et endomorphismes}
Soit $u$ un endomorphisme de $E$.
\subparagraph{1} On suppose que $u$ stabilise toutes les droites (sous espaces vectoriels de dimension 1), c'est-à-dire que pour toute droite $D$, $u(D) \subset D$. Que peut-on dire de $u$ ?
\subparagraph{2}On suppose maintenant que $u$ stabilise tous les sous-espaces vectoriels de dimension k. Que peut-on dire de $u$ ?

\section{Polynômes annulateurs}
Soit $u$ un endomorphisme de $E$. On suppose que $E$ est de dimension finie.
\subparagraph{1}Montrer qu'il existe $P \in k[X]$ tel que $P(u)= \underset{k=0}{\overset{\deg P}{\sum}}a_k u^k=0.$
\subparagraph{2}Montrer que $u$ est bijectif si et seulement un de ses polynômes annulateurs vérifie $a_0 \neq 0$.
\subparagraph{3}Montrer que $\text{ker}u$ et $\text{Im}u$ sont en somme directe si et seulement il existe un polynôme annulateur dont 0 est racine d'ordre au plus 1.
\subparagraph{4}Que se passe-t-il en dimension infinie ?

\section{Rang et endomorphisme}
Soit $u $ et $v$ deux endomorphismes de $E$ ($k-$espace vectoriel de dimension finie) tels que $u\circ v=0$ et $u+v$ bijectif.
\subparagraph{1}Montrer que $\text{rg} u +\text{rg}v= \text{dim}E$.

\section{Rang et composition}
Soit $(f,g)$ deux endomorphismes de $E$ un $k-$espace vectoriel de dimension $n \in \mathbb{N}$. 
\subparagraph{1}Montrer que $\text{rg} f+\text{rg} f -n \le \text{rg} f \circ g$.
\subparagraph{2} En déduire les endomorphismes $u$ de $\mathbb{R}^3$ tels que $u^2=0$.

\section{Rang et sous-espace vectoriel}
Soit $f$ un endomorphisme de $E$ ($k-$espace vectoriel de dimension $n \in \mathbb{N}$). Soit $F$ un sous-espace vectoriel de $E$.
\subparagraph{1}Montrer que $\text{dim} (\text{ker}f \cap F) \ge \text{dim}F - \text{rg}f$

\section{Endomorphismes et polynômes}
Soit $n \in \mathbb{N}$ on note $\phi : \ \mathbb{R}_n[X] \ \rightarrow  \ \mathbb{R}_n[X]$ défini par $\phi(P)=P(X+1)+P(X)$.
\subparagraph{1}Montrer que $\phi$ est un isomorphisme. On note $\forall k \in \llbracket 0,n \rrbracket, \ P_k=\phi^{-1}(2X^k)$.
\subparagraph{2}Montrer que $P_n(X+1)$ est combinaison linéaire des $(P_k)_{k \in \llbracket 0,n \rrbracket}$. En considérant $P_n(X+2)-P_n(X+1)$ expliciter cette combinaison linéaire.
\subparagraph{3} Donner une relation liant $P_n$ aux $(P_k)_{k \in \llbracket 0,n-1 \rrbracket}$.
\end{document}