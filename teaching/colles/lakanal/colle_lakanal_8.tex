\documentclass[10pt,a4paper]{article} 
\usepackage[utf8]{inputenc} 
\usepackage[T1]{fontenc} 
\usepackage[english]{babel} 
\usepackage{supertabular} %Nécessaire pour les longs tableaux
\usepackage[top=2.5cm, bottom=2.5cm, right=1.5cm, left=1.5cm]{geometry} %Mise en page 
\usepackage{amsmath} %Nécessaire pour les maths 
\usepackage{amssymb} %Nécessaire pour les maths 
\usepackage{stmaryrd} %Utilisation des double crochets 
\usepackage{pifont} %Utilisation des chiffres entourés 
\usepackage{graphicx} %Introduction d images 
\usepackage{epstopdf} %Utilisation des images .eps 
\usepackage{amsthm} %Nécessaire pour créer des théorèmes 
\usepackage{algorithmic} %Nécessaire pour écrire des algorithmes 
\usepackage{algorithm} %Idem 
\usepackage{bbold} %Nécessaire pour pouvoir écrire des indicatrices 
\usepackage{hyperref} %Nécessaire pour écrire des liens externes 
\usepackage{array} %Nécessaire pour faire des tableaux 
\usepackage{tabularx} %Nécessaire pour faire de longs tableaux 
\usepackage{caption} %Nécesaire pour mettre des titres aux tableaux (tabular) 
\usepackage{color} %nécessaire pour écrire en couleur 
\newtheorem{thm}{Théorème} 
\newtheorem{mydef}{Définition} 
\newtheorem{prop}{Proposition} 
\newtheorem{lemma}{Lemme}
\title{Semaine 8 - Suites numériques}
\author{Valentin De Bortoli \\ email : \ \href{mailto:valentin.debortoli@gmail.com}{valentin.debortoli@gmail.com}}
\date{}
\begin{document}
\maketitle
\section{Divergence et suite extraites}
\subparagraph{1}Montrer que $(\cos(n))_{n \in \mathbb{N}}$ est divergente.
\subparagraph{2}Montrer de même que  $(\sin(n))_{n \in \mathbb{N}}$ est divergente.
\section{Une suite complexe}
Soit la suite $(z_n)_{n \in \mathbb{N}} \in \mathbb{C}^{\mathbb{N}}$ définie par ($\theta\in [-\pi,\pi[$, $\rho \in \mathbb{R}_+$) :
\begin{equation*}
\left\lbrace\begin{aligned}
&z_0=\rho e^{i\theta} \in \mathbb{C} \\
&z_{n+1}=\frac{z_n+\vert z_n \vert}{2}, \ \forall n \in \mathbb{N}
\end{aligned} \right.
\end{equation*}
\subparagraph{1}Exprimer $z_n$ sous la forme d'un produit.
\subparagraph{2}Montrer que $\forall n \in \mathbb{N}^*, \ \underset{k=1}{\overset{n}{\prod}}\cos(\frac{\theta}{2^k})=\frac{\sin(\theta)}{2^n\sin(\frac{\theta}{2^n})}$.
\subparagraph{3}Montrer que $z_n$ admet une limite et la calculer.
\section{Irrationalité de e}
Soit $(u_n)_{n \in \mathbb{N}} \in \mathbb{R}^{\mathbb{N}}$ et $(v_n)_{n \in \mathbb{N}} \in \mathbb{R}^{\mathbb{N}}$ définies par $\forall n \in \mathbb{N}, \ u_n=\underset{k=0}{\overset{n}{\sum}}\frac{1}{k!}$ et $v_n=u_n+\frac{1}{n! n}$
\subparagraph{1}Montrer que les suites $(u_n)_{n \in \mathbb{N}}$ et $(v_n)_{n \in \mathbb{N}}$ sont adjacentes.
\subparagraph{2}On admet l'inégalité de Taylor-Lagrange. Celle-ci assure que pour toute fonction $f \ : \ [a,b] \rightarrow \mathbb{R} \ \in \mathcal{C}^n([a,b])$ et dérivable $n+1$ fois sur $]a,b[$ et telle que $f^{(n+1)}$ est bornée, on a :
\begin{equation*}
\vert f(b)-f(a)-\frac{f'(a)}{1!}(b-a)-\frac{f''(a)}{2!}(b-a)^2-\dots-\frac{f^{(n)}(a)}{n!}(b-a)^n \vert \le \frac{(b-a)^{n+1}}{(n+1)!} \underset{x \in ]a,b[}{\sup}(\vert f^{(n+1)} \vert)
\end{equation*}
En l'appliquant à la fonction $x \mapsto e^x$ montrer que $u_n \rightarrow e$.
\subparagraph{3}On suppose que $e=\frac{p}{q}$ avec $(p,q)\in \mathbb{N}\times \mathbb{N}^*$. En considérant $u_q q! q$ et $v_q q! q$ aboutir à une contradiction.
\section{Moyenne arithmético-géométrique}
Soit $(a,b)\in \mathbb{R}_+^2$ avec $a \le b$. On définit $(u_n)_{n \in \mathbb{N}} \in \mathbb{R}^{\mathbb{N}}$ et $(v_n)_{n \in \mathbb{N}} \in \mathbb{R}^{\mathbb{N}}$ de la manière suivante :
\begin{equation*}
\left\lbrace 
\begin{aligned}
&u_0=a, \ v_0=b \\
&u_{n+1}=\sqrt{u_n v_n}, \ v_{n+1}=\frac{v_n+u_n}{2}, \ \forall n \in \mathbb{N}
\end{aligned}
\right.
\end{equation*}
\subparagraph{1}Montrer que $\forall n \in \mathbb{N}, \ u_n \le v_n$ ainsi que $u_n \le u_{n+1}$ et $v_{n+1} \le v_n$. En déduire que $(u_n)_{n \in \mathbb{N}}$ et $(v_n)_{n \in \mathbb{N}}$ admettent des limites.
\subparagraph{2}Montrer que ces limites sont égales. On la note $M(a,b)$ et on l'appelle moyenne arithmético-géométrique de $a$ et $b$.
\subparagraph{3}Calculer $M(a,a)$, $M(0,a)$ ainsi que $M( \lambda a, \lambda b)$  pour $\lambda \in \mathbb{R}_+$.
\section{Critère spécial des séries alternées}
Soit $(u_n)_{n \in \mathbb{N}} \in \mathbb{R}^{\mathbb{N}}$ une suite positive, décroissante qui tend vers $0$ lorsque $n \rightarrow +\infty$. On pose $S_n=\underset{k=0}{\overset{n}{\sum}}(-1)^ku_k$.
\subparagraph{1}Montrer que $(S_{2n})_{n \in \mathbb{N}}$ et $(S_{2n+1})_{n \in \mathbb{N}}$ sont deux suites adjacentes.
\subparagraph{2}En déduire que $(S_n)_{n \in \mathbb{N}}$ converge. On note $S$ sa limite.
\subparagraph{3}Montrer que $\forall n \in \mathbb{N}, \ S_{2n+1}\le S \le S_{2n}$.
\subparagraph{4}Que peut-on dire de la convergence de la suite $\underset{k=1}{\overset{n}{\sum}} \frac{\cos(k\pi)}{\sqrt{k}}$ ?
\section{Suite sous-additive}
Soit $(u_n)_{n \in \mathbb{N}} \in \mathbb{R}_+^{\mathbb{N}}$ une suite sous-additive au sens où :
\begin{equation*}
\forall (p,q) \in \mathbb{N}^2, \ u_{p+q}\le u_p+u_q
\end{equation*}
\subparagraph{1}Rappeler la définition de $\inf \left\lbrace  \frac{u_n}{n}, \ n \in \mathbb{N}^*\right\rbrace$.
\subparagraph{2}Soit $n=qm+r$, $r \in \llbracket 0,q-1 \rrbracket$, la division euclidienne de $n$ par $q$. Établir une inégalité faisant intervenir $u_n$, $u_q$ et $u_1$.
\subparagraph{3}Montrer que la suite $(\frac{u_n}{n})_{n \in \mathbb{N}^*}$ tend vers $\inf \left\lbrace  \frac{u_n}{n}, \ n \in \mathbb{N}^*\right\rbrace$.
\subparagraph{2}Soit $(v_n)_{n \in \mathbb{N}} \in \mathbb{R_+^*}^{\mathbb{N}}$ qui vérifie :
\begin{equation*}
\forall (p,q) \in \mathbb{N}^2, \ v_{p+q}\le v_p v_q
\end{equation*}
Que peut-on dire de la suite $(v_n)_{n \in \mathbb{N}}$ ?
\section{Récurrence et nombre d'or}
\subparagraph{1}Montrer que $\sqrt{1+\sqrt{1+\sqrt{\dots}}}=1+\frac{1}{1+\frac{1}{1+\dots}}=\frac{1+\sqrt{5}}{2}$.
\section{Suite et équivalent}
Soit $(u_n)_{n \in \mathbb{N}}$ définie de la manière suivante :
\begin{equation*}
\left\lbrace
\begin{aligned}
& u_0 \in ]0,\frac{\pi}{2}[\\
& u_{n+1}=\sin(u_n)
\end{aligned}
\right.
\end{equation*}
\subparagraph{1}Montrer que $u_n \rightarrow 0$.
\subparagraph{2}Montrer que $\frac{1}{u_{n+1}^2}-\frac{1}{u_n^2}$ admet une limite en $+\infty$ et la calculer.
\subparagraph{3}Déterminer un équivalent de $u_n$ lorsque $n \rightarrow +\infty$.
\section{Méthode de Newton}
Soit $f \in \mathcal{C}^2([a,b])$ à valeurs réelles avec $f'>0$ sur $[a,b]$. On définit $(x_n)_{n \in \mathbb{R}} \in \mathbb{R}^{\mathbb{N}}$ de la manière suivante :
\begin{equation*}
x_{n+1}=x_n-\frac{f(x_n)}{f'(x_{n})}
\end{equation*}
On suppose également que $f$ s'annule en $c \in ]a,b[$. Enfin on suppose que $\forall n \in \mathbb{N}, \ x_n \in [a,b]$.
\subparagraph{1}Montrer que $\forall n \in \mathbb{N}, \ x_{n+1}$ correspond à l'abscisse du point d'intersection entre la tangente à $f$ en $x_n$ et l'axe des abscisses. Faire un dessin.
\subparagraph{2}Montrer que $\forall n \in \mathbb{N}, \ \vert x_{n+1}-c \vert \le C \vert x_n-c \vert^2$ avec $C \in \mathbb{R}_+$.
\subparagraph{Indication :} on utilisera l'inégalité de Taylor-Lagrange (voir l'exercice 3 question 2).
\subparagraph{3}Donner la formule liant $x_{n+1}$ et $x_n$ dans le cas où $f : \ x \mapsto x^2-a$ et $[0,2a], \ (a\in\mathbb{R}_+)$ comme intervalle d'étude.
\subparagraph{Remarque :} cette méthode peut être utilisée pour trouver le minimum de fonctionnelle en considérant $f'$ plutôt que $f$. Néanmoins cette méthode bien que très rapide fait intervenir des dérivées secondes et demande donc de gros calculs (si on est en grande dimension). Le plus souvent on privilégie des algorithmes d'ordre de dérivation inférieur. Les garanties de convergence sont moindre mais l'implémentation est bien plus aisée.
\subparagraph{Remarque :} la dernière remarque de cet exercice présente la méthode de Héron, popularisée par Héron d'Alexandrie au premier siècle après Jésus-Christ, qui donne un moyen pratique et rapide de calculer des racines.
\end{document}