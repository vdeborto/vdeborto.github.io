\documentclass[10pt,a4paper]{article} 
\usepackage[utf8]{inputenc} 
\usepackage[T1]{fontenc} 
\usepackage[english]{babel} 
\usepackage{supertabular} %Nécessaire pour les longs tableaux
\usepackage[top=2.5cm, bottom=2.5cm, right=1.5cm, left=1.5cm]{geometry} %Mise en page 
\usepackage{amsmath} %Nécessaire pour les maths 
\usepackage{amssymb} %Nécessaire pour les maths 
\usepackage{stmaryrd} %Utilisation des double crochets 
\usepackage{pifont} %Utilisation des chiffres entourés 
\usepackage{graphicx} %Introduction d images 
\usepackage{epstopdf} %Utilisation des images .eps 
\usepackage{amsthm} %Nécessaire pour créer des théorèmes 
\usepackage{algorithmic} %Nécessaire pour écrire des algorithmes 
\usepackage{algorithm} %Idem 
\usepackage{bbold} %Nécessaire pour pouvoir écrire des indicatrices 
\usepackage{hyperref} %Nécessaire pour écrire des liens externes 
\usepackage{array} %Nécessaire pour faire des tableaux 
\usepackage{tabularx} %Nécessaire pour faire de longs tableaux 
\usepackage{caption} %Nécesaire pour mettre des titres aux tableaux (tabular) 
\usepackage{color} %nécessaire pour écrire en couleur 
\newtheorem{thm}{Théorème} 
\newtheorem{mydef}{Définition} 
\newtheorem{prop}{Proposition} 
\newtheorem{lemma}{Lemme}



\newcommand{\al}[1]{\begin{aligned} #1 \end{aligned}}

\newcommand{\seq}[2]{\left( #1_{#2} \right)_{#2 \in \mathbb{N}} }
\newcommand{\intt}[4]{\int_{#1}^{#2} #3 \mathop{}\!\mathrm{d} #4}
\newcommand{\summ}[2]{\underset{#1}{\overset{#2}{\sum}}}
\newcommand{\prodd}[2]{\underset{#1}{\overset{#2}{\prod}}}

\newcommand{\vertt}[1]{\vert #1 \vert}

\title{Semaine 9 - Intégration de fonctions continues}
\author{Valentin De Bortoli \\ email : \ \href{mailto:valentin.debortoli@gmail.com}{valentin.debortoli@gmail.com}}
\date{}
\begin{document}
\maketitle
\section{Une convergence de norme}

\subparagraph{1}Notons $I_n = \left( \intt{a}{b}{f(t)^n}{t} \right)^{\frac{1}{n}}$.
On a tout d'abord que $\forall x \in [a,b], \ \frac{f(x)}{M} \le 1$. Ainsi,  $\intt{a}{b}{\left(\frac{f(t)}{M}\right)^n}{t} \le b-a$, et donc $\frac{1}{M} I_n \le (b-a)^{\frac{1}{n}}$. Donc $I_n \le M+\epsilon$ pour $n$ assez grand. D'un autre côté $M$ est atteint car $f$ est continue sur $[a,b]$ (intervalle borné et fermé). Il existe donc un intervalle $I_{x_0,\eta} = [x_0-\frac{\eta}{2},x_0+\frac{\eta}{2}]$ tel que si $x \in I_{a,\eta}, \ f(x) \ge M - \frac{\epsilon}{2}$. $\intt{a}{b}{f(t)^n}{t} \ge \intt{a-\frac{\eta}{2}}{a+\frac{\eta}{2}}{f(t)^n}{t} \ge \eta (M-\frac{\epsilon}{2})^n$ (si $x_0$ atteint sur $]a,b[$, le raisonnement est le même aux bords, il faut seulement modifier $I_{x_0,\eta}$. Donc $I_n \ge \eta^{\frac{1}{n}}(M-\frac{\epsilon}{2})$ et ce quelque soit $n \in \mathbb{N}$. Donc pour $n$ assez grand $\eta^{\frac{1}{n}}$ est plus grand que $\frac{M-\epsilon}{M-\frac{\epsilon}{2}}<1$. Donc pour $n$ assez grand $M-\epsilon \le I_n \le M+\epsilon$ et donc $\seq{I}{n}$ tend vers $M$.

\subparagraph{Remarque :} l'ensemble des fonctions à valeurs réelles pour lesquelles l'intégrales $\int \vert f \vert^{p} $ est définie est appelé fonctions $L^p(\mathbb{R})$ (la définition est volontairement assez informelle, il faudrait des outils de théorie de la mesure pour définir de manière rigoureuse ces fonctions). Sur les espaces $L^p(\mathbb{R})$ une norme est donnée par $\|f\|_p = \left(\int \vert f \vert^p \right)^{\frac{1}{p}}$ (seule l'inégalité triangulaire est vraiment dure à démontrer et fait appel à l'inégalité de Hölder). On appelle souvent $L^{\infty}(\mathbb{R})$ l'espace des fonctions bornées (dans un sens légèrement différent de celui que vous connaissez) et cet espace peut aussi être muni d'une norme appelée la norme infinie, $\| f \|_{\infty} = \sup \vert f \vert$ (encore une fois ce n'est pas vraiment une borne supérieure mais une borne supérieure essentielle, il se trouve que toutes les subtilités précisées ici n'entrainent pas de modifications par rapport à ce que vous connaissez si on se restreint aux fonctions continues). 
Cet exercice montre que la norme infinie peut effectivement se voir comme la norme limite des normes des différents $L^p$. Ces espaces vectoriels possèdent de nombreuses propriétés et constitue un objet d'étude privilégié de l'analyse fonctionnelle.
\section{Inégalité et intégrale}
Il s'agit d'écrire $f(x) = \intt{a}{x}{f'(t)}{t}$. De cette manière, $\vert f(x) \vert^2 \le \left( \intt{a}{x}{\vert f'(t) \vert^2}{t} \right) (t-a) \le \left( \intt{a}{b}{\vert f'(t) \vert^2}{t} \right) (t-a)$ par l'inégalité de Cauchy-Schwarz. Il s'agit ensuite d'intégrer des deux côtés de l'inégalité en notant que $\intt{a}{b}{t-a}{t} = \frac{(b-a)^2}{2}$.
\section{Module et cas d'égalité}
On pose $I =\intt{a}{b}{f(t)}{t} = \vert I \vert e^{i\alpha}$ avec $\alpha \in \mathbb{R}$. On pose également $\phi$ une fonction de $[a,b]$ dans $\mathbb{R}$ telle que $f(t) = \vert f(t) \vert e^{i \phi(t)}$. $I = \vert I \vert e^{i\alpha} = \intt{a}{b}{\vert f(t) \vert}{t} e^{i \alpha} = \intt{a}{b}{\vert f(t) \vert e^{i\alpha}}{t}$. Mais aussi $I = \intt{a}{b}{\vert f(t) \vert e^{i \phi(t)}}{t}$. On obtient en faisant la différence, $\intt{a}{b}{\vert f(t) \vert (1 - e^{i (\phi(t) - \alpha)})}{t} = 0$. En passant à la partie réelle et en remarquant que $\vert f(\cdot) \vert$ et $1 - \cos \left( \phi(\cdot) - \alpha\right)$ sont deux fonctions positives on obtient que $f = 0$ ou $\phi(\cdot) = \alpha$. Ainsi dans tous les cas il existe $\alpha \in \mathbb{R}, \ \forall t \in [a,b], \ f(t) = \vert f(t) \vert e^{i\alpha}$.

\section{Inégalité de Young}
\subparagraph{1}La forme de la seconde intégrale invite au changement de variable $u=f(t)$. Ce changement de variable est valable car la fonction $f$ est une bijection continûment dérivable. Ainsi on obtient $\intt{0}{x}{uf'(u)}{u}$. En effet $\text{d}u f'(u) = \text{d}t$. Une intégration par partie permet d'écrire $\intt{0}{x}{uf'(u)}{u} = xf(x) - \intt{0}{x}{f(t)}{t}$. On obtient donc l'égalité voulue.
\subparagraph{2}Supposons que $b \ge f(a)$ (l'autre cas ce traite exactement de la même manière en considérant $f^{-1}$ plutôt que $f$). \[
\al{
\intt{0}{a}{f(t)}{t} + \intt{0}{b}{f(t)}{t} &= \intt{0}{a}{f(t)}{t} + \intt{0}{f(a)}{f(t)}{t} + \intt{f(a)}{b}{f(t)}{t}\\
&= af(a) + \intt{f(a)}{b}{f(t)}{t}\\
&= ab + \intt{f(a)}{b}{f^{-1}(t)}{t} -a(b-f(a))\\
&= \intt{f(a)}{b}{f^{-1}(t) - a}{t}\\
& \ge 0
}
\] car $f^{-1}$ est strictement croissante et $f^{-1}(f(a)) = a$. On a égalité si et seulement si $f(a) = b$.

\section{Suite et intégrale (1)}
\subparagraph{1} $J_{n+2} + J_n = \underset{0}{\overset{\frac{\pi}{4}}{\int}}(1+ \tan(x)^2) \tan(x)^n \text{dx} = \frac{1}{n+1} \underset{0}{\overset{\frac{\pi}{4}}{\int}} (\tan(.)^{n+1})'(x) \text{dx} = \frac{1}{n+1}$.
\subparagraph{2} $J_0 = \frac{\pi}{4}$, $J_1 = \underset{0}{\overset{\frac{\pi}{4}}{\int}} \frac{\sin(x)}{\cos(x)} \text{dx} = -\ln(\cos(\frac{\pi}{4})) = \frac{1}{2}\ln(2)$.
On a donc les formules suivantes selon la parité de $n$ :
\begin{itemize}
\item $J_{2n} = (-1)^n \left( \underset{k=1}{\overset{n}{\sum}}\frac{(-1)^k}{2k-1} + \frac{\pi}{4} \right)$
\item $J_{2n+1} = (-1)^n \left(\frac{1}{2}\underset{k=1}{\overset{n}{\sum}}\frac{(-1)^k}{k} + \frac{1}{2}\ln(2) \right)$
\end{itemize}

\section{Suite et intégrale (2)}
\subparagraph{1}$K_0 = \frac{\pi}{4}$ et $K_1= \int_0^{\frac{\pi}{4}} \frac{1}{\cos(x)} \text{dx}$. En utilisant les règles de Bioche qui sont rappelées à la fin de cet exercice, on trouve que le changement de variable en sinus est adapté ici. On obtient, $K_1= \int_0^{\frac{\pi}{4}} \frac{1}{\cos(x)} \text{dx} = \int_0^{\frac{\sqrt{2}}{2}} \frac{1}{1-x^2} \text{dx} = \frac{1}{2} \left( \int_0^{\frac{\sqrt{2}}{2}} \frac{1}{1+x} + \frac{1}{1-x} \text{dx} \right)$. En intégrant, on trouve, $K_1 = \ln \left( \sqrt{\frac{2+ \sqrt{2}}{2 - \sqrt{2}}}\right)= \ln(1 + \sqrt{2})$.

\subparagraph{2} On a,
\begin{equation}
\begin{aligned}
K_{n+2} &= \int_0^{\frac{\pi}{4}} \frac{1}{\cos(x)^2} \frac{1}{\cos(x)^n} \text{dx} \\
&= \int_0^{\frac{\pi}{4}} \tan(x)' \frac{1}{\cos(x)^n} \text{dx} \\
&= 2^{\frac{n}{2}} - n \int_0^{\frac{\pi}{4}} \tan(x) \sin(x) \frac{1}{\cos(x)^{n+1}} \text{dx} \\
&= 2^{\frac{n}{2}} - n \int_0^{\frac{\pi}{4}} \frac{1-\cos(x)^2}{\cos(x)^{n+2}} \text{dx} \\
&=2^{\frac{n}{2}} - n K_{n+2} +nK_n \\
&=\frac{2^{\frac{n}{2}}}{n+1} + \frac{n}{n+1} K_n
\end{aligned}
\end{equation}.
Ainsi on a entièrement déterminé la suite $(K_n)_{n \in \mathbb{N}}$.

\subparagraph{Remarque :} on rappelle ici les règles de Bioche qui sont très utiles pour calculer des intégrales de fonctions trigonométriques.
On pose $f(x) = g(\cos(x), \sin(x), \tan(x))$ et $F(x) = f(x) \text{dx}$. Si :
\begin{itemize}
\item $F(x) = F(-x)$ alors on effectue le changement de variable $x \ \mapsto \ \cos(x)$
\item $F(x) = F(\pi-x)$ alors on effectue le changement de variable $x \ \mapsto \ \sin(x)$
\item $F(x) = F(\pi+x)$ alors on effectue le changement de variable $x \ \mapsto \ \tan(x)$
\end{itemize}.
\section{Suite et intégrale (3)}

\subparagraph{1}$L_{n+1} = \int_1^e \log(x)^{n+1} \text{dx} = [x \log(x)^{n+1}]_1^e - (n+1)L_n = e - (n+1)L_n$. De plus, $L_0 = e-1$. On détermine donc de manière unique la suite $(L_n)_{n \in \mathbb{N}}$.
\subparagraph{2}Le changement de variable $u = \log(x)$ permet d'écrire $L_n = \int_0^1 u^n e^{-nu} \text{du} \le \int_0^1 u^n \text{du} \ \longrightarrow \ 0$. Donc ${e - (n+1)L_n \ \longrightarrow \ 0}$. D'où $L_n \sim \frac{e}{n}$.
\section{Condition suffisante et point fixe}
\subparagraph{1} Il suffit de remarquer que $\intt{0}{1}{(f(t)-t)}{t}$. Posons $g(t) = f(t) -t$. $g(0) = f(0) \ge 0$ et $g(1) = f(1) - 1 \le 0$. Donc par le théorème des valeurs intermédiaires il existe $t \in [0,1], \ g(t) = t$. Donc $f(t) = t$ et $f$ admet un point fixe
\section{Inégalité et maximum}
\subparagraph{1}
Notons $\alpha = \frac{c-a}{b-a}$. On remarque que $1 - \alpha = \frac{b-c}{b-a}$. Donc $\alpha \frac{1}{c-a} \intt{a}{c}{f(t)}{t} + (1- \alpha) \frac{1}{b-c}\intt{c}{b}{f(t)}{t} = \frac{1}{b-a}\left(\intt{a}{c}{f(t)}{t} + \intt{b}{c}{f(t)}{t} \right) = \frac{1}{b-a} \intt{a}{b}{f(t)}{t}$. On note $M = \max \left( \frac{1}{c-a} \intt{a}{c}{f(t)}{t},  \frac{1}{b-c}\intt{c}{b}{f(t)}{t} \right)$. On a $\alpha \frac{1}{c-a} \intt{a}{c}{f(t)}{t} + (1- \alpha) \frac{1}{b-c}\intt{c}{b}{f(t)}{t} \le \alpha M + (1-\alpha) M \le M$. Ainsi, $\frac{1}{b-a} \intt{a}{b}{f(t)}{t} \le M$.
\subparagraph{2}L'interprétation géométrique est la suivante. Il s'agit de remarquer que $\frac{1}{b-a} \intt{a}{b}{f(t)}{t}$ est la moyenne de $f$ sur $[a,b]$. L'inégalité précédente assure que la moyenne sur $[a,b]$ est plus petite que le maximum entre la moyenne sur $[a,c]$ et celle sur $[c,b]$.
\section{Annulation et intégration (1)}
\subparagraph{1} Si il existe $(x_0,x_1) \in [0,\pi]^2$ tels que $f(x_0)f(x_1) \le 0$ alors par le théorème des valeurs intermédiaires ($f$ est continue) $f$ admet un point d'annulation. 
Il s'agit de remarquer que $x \ \mapsto \ \sin(x)$ est positive sur $[0, \pi]$. Supposons donc que $f$ est positive (le cas négatif se traite de la même manière). Alors l'intégrale $\intt{0}{\pi}{\sin(t) f(t)}{t} \ge 0$ et l'inégalité est une égalité si et seulement si $f=0$. Ainsi $f$ admet un point d'annulation.

\subparagraph{1}Supposons que $f$ ne s'annule pas sur $I_1 = [0,a[$ et sur $I_2 = ]a,\pi]$ alors $f$ est de signe constant sur chacun de ces intervalles. En effet sinon le théorème des valeurs intermédiaires permet de conclure. Supposons que $f$ est positive sur $I_1$ et $I_2$ (le cas négatif se traite de la même façon) alors $f$ est positive sur $[0,  \pi]$ et comme pour la question précédente cela implique que $f=0$ et donc $f$ possède deux points d'annulation. 

Supposons désormais que $f$ change de signe en $a$. Par exemple $f$ négative sur $I_1$ et positive sur $I_2$. On a donc $x \ \mapsto \ f(x) \sin(x-a)$ qui est positive sur $[0,\pi]$ et donc $\intt{0}{\pi}{f(t) \sin(t-a)}{t} \ge 0$. Mais $\sin(t-a) = \sin(a) \cos(t) - \cos(a) \sin(t)$ et donc $\intt{0}{\pi}{f(t) \sin(t-a)}{t} = 0$ ce qui n'est possible que si $f = 0$. Dans tous les cas $f$ possède deux points d'annulation.

\section{Annulation et intégration (2)}
\subparagraph{1} On montre que pour tout polynôme $P = \summ{k=0}{n}{a_kX^k}$ (avec $\seq{a}{n} \in \mathbb{R}^{\mathbb{N}}$) de degré inférieur ou égal à $n$, $\intt{a}{b}{f(t)P(t)}{t} = \summ{k=0}{n}{a_k \intt{a}{b}{f(t)t^k}{t}} = 0$. On raisonne par l'absurde en considérant une fonction qui vérifie l'hypothèse précédente et qui possède moins de $n-1$ points d'annulation. Posons $P$ comme proposé dans l'indication. Il convient de remarquer que $x \ \mapsto \ f(x)P(x)$ est de signe constant. Mais puisque le nombre de points de changement de signe est plus petit que le nombre de points d'annulation on obtient que $\intt{a}{b}{f(t)P(t)}{t} = 0$. Cela n'est possible que si $x \ \mapsto \ f(x)P(x)$ est la fonction nulle. Ainsi $f$ admet une infinité non dénombrable de points d'annulation donc c'est absurde.

\subparagraph{Remarque :} en analyse numérique on cherche souvent des polynômes qui ressemblent le plus possible à une fonction $f$. Pour définir la "ressemblance" on utilise les différentes normes sur les espaces de fonctions à notre disposition (voir exercice 1 pour une explication à ce sujet). Parmi les normes populaires on trouve $\| \cdot \|_2$ et $\| \cdot \|_{\infty}$. Par exemple pour la norme $\| \cdot \|_2$ on cherche le minimum $\| f - P \|_2$ sur l'espace des polynômes de degré $n$ ($f$ est fixée). Ce polynôme existe et est unique (pour $\| \cdot \|_{\infty}$ on a seulement l'existence). On le note $P_{n,2}(f)$ et $h_{n,2} = f-P_{n,2}$. Alors on peut montrer que $f - P_{n,2}$ vérifie l'hypothèse de l'énoncé. En somme toute interpolation polynômiale pour la norme $\| \cdot \|_2$ (c'est le nom donné à $P_{n,2}(f)$) oscille au moins $n+1$ fois. Ce résultat d'oscillation reste vrai pour la norme infinie mais la preuve est beaucoup plus compliquée (théorème d'équioscillation de Tchebychev). 

\subparagraph{Remarque :} à la suite de cet exercice on peut se demander ce qu'il advient si on change $k \in \llbracket 0,n\rrbracket$ en $\mathbb{N}$. Dans ce cas, on montre que $f=0$. Il faut cependant déployer des outils d'analyse de Fourier et d'analyse complexe pour conclure correctement.
\section{Formule de la moyenne}
\subparagraph{1} On commence par supposer que $\intt{a}{b}{g(t}{t} = 1$. Soit $x_0$ tel que $f(x_0) = \min \ f$ et $x_1$ tel que $f(x_1) = \max \ f$ (possible car $f$ est continue sur un intervalle fermé borné). Il convient de remarque que
\[
f(x_0) = \intt{a}{b}{f(x_0)g(t)}{t} \le \intt{a}{b}{f(t)g(t)}{t} \intt{a}{b}{f(x_1)g(t)}{t} \le f(x_1).
\]
Donc si on pose $g(x) = f(x) - \intt{a}{b}{f(t)g(t)}{t}$, $g(x_0) \le 0 \le g(x_1)$. Il existe donc $c$ dans $[a,b]$ tel que $g(c) = 0$, c'est-à-dire $\intt{a}{b}{f(t)g(t)}{t} = f(c)$ ce qui conclut le cas  $\intt{a}{b}{g(t}{t} = 1$. Dans le cas général on se ramène au particulier en considérant $\tilde{g} = \frac{g}{\intt{a}{b}{g(t}{t}}$.
\subparagraph{2} A FAIRE
\end{document}