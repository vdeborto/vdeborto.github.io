\documentclass[10pt,a4paper]{article} 
\usepackage[utf8]{inputenc} 
\usepackage[T1]{fontenc} 
\usepackage[english]{babel} 
\usepackage{supertabular} %Nécessaire pour les longs tableaux
\usepackage[top=2.5cm, bottom=2.5cm, right=1.5cm, left=1.5cm]{geometry} %Mise en page 
\usepackage{amsmath} %Nécessaire pour les maths 
\usepackage{amssymb} %Nécessaire pour les maths 
\usepackage{stmaryrd} %Utilisation des double crochets 
\usepackage{pifont} %Utilisation des chiffres entourés 
\usepackage{graphicx} %Introduction d images 
\usepackage{epstopdf} %Utilisation des images .eps 
\usepackage{amsthm} %Nécessaire pour créer des théorèmes 
\usepackage{algorithmic} %Nécessaire pour écrire des algorithmes 
\usepackage{algorithm} %Idem 
\usepackage{bbold} %Nécessaire pour pouvoir écrire des indicatrices 
\usepackage{hyperref} %Nécessaire pour écrire des liens externes 
\usepackage{array} %Nécessaire pour faire des tableaux 
\usepackage{tabularx} %Nécessaire pour faire de longs tableaux 
\usepackage{caption} %Nécesaire pour mettre des titres aux tableaux (tabular) 
\usepackage{color} %nécessaire pour écrire en couleur 
\newtheorem{thm}{Théorème} 
\newtheorem{mydef}{Définition} 
\newtheorem{prop}{Proposition} 
\newtheorem{lemma}{Lemme}
\title{Semaine 17 - Théorie de la dimension et applications linéaires}
\author{Valentin De Bortoli \\ email : \ \href{mailto:valentin.debortoli@gmail.com}{valentin.debortoli@gmail.com}}
\date{}
\begin{document}
\maketitle
La plupart des corrections ne sont pas très détaillées et ne comportent que les grandes lignes pour vous guider dans la résolution. Si vous avez des questions n'hésitez pas.

\section{Une famille libre ?}
\subparagraph{1} Pour montrer que cette famille est libre on choisit arbitrairement une sous famille \textbf{finie}. Il s'agit ensuite de montrer que $(x \mapsto \cos(n_k x))_{k \in \llbracket 1,N \rrbracket}$ est une famille libre (on a fixé $N$ et les $n_k$). Pour cela deux méthodes :
\begin{itemize}
\item on peut raisonner par récurrence sur $N$ et on raisonne comme suit pour l'hérédité. On a $(\lambda_i)_{i \llbracket 1,N \rrbracket}$ tels que $\forall x \in \mathbb{R},\ \underset{i=1}{\overset{N}{\sum}} \lambda_i \cos(n_i x) =0$. On veut montrer que les $\lambda_i$ sont nuls. On dérive deux fois la relation et je vous laisse conclure.
\item on peut aussi remarquer que $\int_0^{2 \pi} \cos(nx)\cos(mx) \text{dx}=0$ sauf si $n \neq m$ (que vaut l'intégrale dans ce cas ?). Conclure à partir de cette remarque.
\end{itemize}

\section{Espace vectoriel et fonctions affines}
\subparagraph{1}Il est facile de vérifier que l'on a bien un sous-espace vectoriel de $\mathcal{F}(\mathbb{R},\mathbb{R})$.
\subparagraph{2}On peut utiliser la base suivante à trois éléments :
\begin{itemize}
\item la fonction qui vaut $0$ sur $[-1,0]$ et l'identité sinon. On la note $f_1$
\item la fonction qui vaut $0$ sur $[0,1]$ et l'identité sinon. On la note $f_2$
\item la constante égale à 1. On la note $f_3$
\end{itemize} 
C'est une famille libre. En effet si on a une combinaison linéaire nulle alors on a une combinaison linéaire nulle de $f_1$ et $f_3$ sur $[0,1]$ (car $f_2$ est nulle). Il est facile de montrer que sur $[0,1]$ $f_1$ et $f_3$ sont libres. Donc $f_1=f_3=0$. On conclut $f_2=0$.

C'est une famille génératrice. En effet on a que $f-f(0)$ est linéaire sur $[0,1]$ et sur $[-1,0]$. Il s'agit simplement de déterminer les coefficients linéaires sur ces intervalles (on les note $\alpha$ et $\beta$). On a alors $f = \alpha f_1 + \beta f_2 +f_3$.

On peut conclure.  

\section{Une base de polynômes}

On montre seulement que c'est une famille libre. On peut conclure par dimension pour le caractère base. Soit $\lambda_i$ une famille de $n+1$ scalaires tels que $\underset{i=1}{\overset{n}{\sum}} \lambda_i X^i(1-X)^{n-i}=0$. On dérive n fois et évalue en $0$. On trouve que $n! \lambda_n=0$. Donc $\lambda_n=0$. On reprend l'opération en dérivant $n-1$ fois et en évaluant en $0$. On trouve que $(n-1)! \lambda_{n-1}=0$. On peut conclure par récurrence.

\section{Nombres réels et espace vectoriel}
Attention : erreur typographique on considère $\log(p_n)$ ici et non $p_n$.

On suppose qu'on dispose de $\lambda_i= \frac{r_i}{q_i} \in \mathbb{Q}$ une famille de $N$ rationnels tels que $\underset{i=1}{\overset{N}{\sum}} \frac{r_i}{q_i} \log( p_i)=0$. Donc $\underset{i=1}{\overset{n}{\prod}} p_i^{\frac{r_i}{q_i}}=1$. En mettant tout cela à la puissance $q_1 q_2 \dots q_n$ on trouve que $\underset{i=1}{\overset{n}{\prod}} p_i^{r_i \underset{j \neq i}{\prod} q_j}=1$. Il s'agit ensuite de séparer les cas où $r_i$ est positif des cas où il est négatif. $\underset{r_i >0}{\overset{n}{\prod}} p_i^{r_i \underset{j \neq i}{\prod} q_j}=\underset{r_i  \le 0}{\overset{n}{\prod}} p_i^{r_i \underset{j \neq i}{\prod} q_j}$. Par unicité de la décomposition en produit de facteurs premiers on trouve que $r_i  \underset{j \neq i}{\prod} q_j=0$ et donc $r_i=0$. D'où $\lambda_i=0$.

\subparagraph{2} Autre erreur typographique : on va trouver un polynôme de degré $N$ qui annule x et non $N-1$.

Il s'agit simplement de constater que $(1,x,\dots,x^n)$ est une famille de $n+1$ éléments de $\mathbb{R}$ donc une famille liée. On peut donc conclure sur l'existence d'un polynôme annulateur de degré n.
\subparagraph{3} L'erreur typographique se poursuit ici. On considère $2^{n+1}$. $P=X^{n+1}-2$ annule ce réel. On admet son irréductibilité dans $\mathbb{Q}$. Soit un autre polynôme annulateur de degré strictement inférieur à $n+1$. On le note $Q$. On remarque que $Q \wedge P$ est toujours annulateur de $x$. Si il existe un polynôme annulateur de degré plus petit ou égal à $n$ alors on a un polynôme annulateur de degré plus petit ou égal à n qui divise $P$. Mais $P$ est irréductible donc le polynôme annulateur $Q$ n'existe pas et on a une contradiction avec notre hypothèse de départ.
\section{Polynômes à valeurs entières}
On suppose $\lambda_i$, $n+1$ scalaires tels que $$ \sum_{i=0}^n \lambda_i P_i=0$$. En évaluant en $0$ on trouve que $\lambda_0=0$. En évaluant en $1$ on trouve ensuite que $\lambda_1=0$. On poursuit par récurrence jusqu'à $\lambda_n=0$. On conclut sur le caractère de base via la dimension.
\subparagraph{2}Trivial pour $P_0$. On pose $k>1$. On a plusieurs cas :
\begin{itemize}
\item si $m \ge k$ on trouve $P_k(m) = \binom{k}{m}$.
\item si $m \in \llbracket 0,k-1 \rrbracket$ alors $P_k(m)=0$.
\item si $m<0$ on trouve $P_k(m)= \binom{k}{m+k} (-1)^k$.
\end{itemize}
\subparagraph{3}On a que $$P = \sum_{k=0}^n \lambda_k P_k$$. En évaluant en $0$ on trouve que $\lambda_0$ est un entier. Ainsi de suite on trouve que tous les $\lambda_k$ sont entiers. Réciproquement si un polynôme s'écrit comme $$\sum_{k=0}^n \lambda_k P_k$$ avec $\lambda_k \in \mathbb{Z}$ il est trivial qu'il prend des valeurs entières sur les entiers.

\section{Divisibilité et sous-espace vectoriel}
\subparagraph{1}Simple vérification. 
\subparagraph{2}Une base de cet espace peut être exhibé de la manière suivante. On suppose que le degré de $A$ est $p\le n$ : $(A,AX,\dots AX^{n-p})$. Cette famille est libre car de degré échelonnée. Elle est génératrice car puisque tout polynôme de notre sous-espace vectoriel est divisible par $A$ et dans $\mathbb{R}_n[X]$ il s'écrit comme combinaison linéaire des polynômes exprimés. 

Pour trouver un supplémentaire on pose $\mathbb{R}_p[X]$. Tout polynôme s'écrit de manière unique $P=AQ+R$ avec $R \in \mathbb{R}_p[X]$ (division euclidienne). Cela suffit à montrer la proposition.

\section{Une équation polynômiale}
\subparagraph{1}Montrer qu'il existe un unique polynôme $P\in \mathbb{R}_{n+1}[X]$ tel que $P(0)=0$ et $P(X+1)-P(X)=X^n$.

On considère $\phi : \ \mathbb{R}_{n+1}[X] \ \rightarrow \ \mathbb{R}_n[X]$ et tel que $\phi(P)=P(X+1)-P(X)$. $\phi$ endomorphisme de noyau les polynômes constants. En appliquant le théorème du rang on montre que cet endomorphisme est surjectif. Donc l'ensemble des polynômes qui vérifie $P(X+1)-P(X)=X^n$ est un sous-espace vectoriel de dimension 1. En fixant la constante on démontre l'unicité.

\section{Une somme directe}
\subparagraph{1}Soit $P \in F_j$. On a que $$Q=\prod_{i=0, i\neq j}^n (X-i)$$ divise $P$. Mais pour des raisons de degré $P = \lambda Q$ avec $\lambda \in k^*$. On conclut que $F_j \cup \lbrace 0 \rbrace$ est un sous-espace vectoriel de dimension 1.
\subparagraph{2}Il est facile de montrer que les $F_j$ sont en somme directe. On conclut sur l'égalité avec $\mathbb{R}_n[X]$ par dimension.

\section{Drapeaux}
Soit $u$ un endomorphisme de $E$. 
\subparagraph{1}Il est facile de démontrer que l'on a des inclusions croissantes sur le noyau et décroissantes sur les images.
\subparagraph{2}On suppose qu'il existe $p \in \mathbb{N}$ tel que $\text{ker}u^n=\text{ker}u^{n+1}$. Montrer que pour tout $p \in \mathbb{N}, \ p \ge n \ \Rightarrow \ \text{ker}u^p=\text{ker}u^{p+1}$.

On démontre la proposition par récurrence. Elle est vraie pour $p=n$ on suppose que $p>n$
Soit $x \in \text{ker} u^{p+1}$. $u^p(u(x))=0$. Donc $u(x) \in \text{ker} u^p =\text{ker} u^{p-1}$. Donc $u^{p-1}(u(x))= u^p(x)=0$. Donc on peut conclure sur l'égalité des noyaux.
\subparagraph{3}La propriété de somme directe n'est pas compliquée à établir. Il suffit de montrer que l'intersection est vide. On conclut ensuite grâce au théorème du rang que $E = \text{ker} u^n \bigoplus \text{Im} u^n$.

\section{Stabilisation et endomorphismes}
\subparagraph{1} Les homothéties vérifient cette propriété. Soit un endomorphisme qui stabilise les droites. Il convient de remarquer que si la dimension de $E$ est égale à $1$ alors les seules endomorphismes sont les homothéties et que ceux-ci stabilisent les droites... On suppose donc que $\text{dim} E >2$. Soit $x \in E$. On choisit $x'$ tel que $(x,x')$ est une famille libre. Puisque $u$ stabilise les droites, $\exists \lambda_1 \in k, \ u(x) = \lambda_1 x$, $\exists \lambda_2 \in k, \ u(x') = \lambda_2 x'$, $\exists \lambda_3 \in k, \ u(x+x') = \lambda_3 (x+x')$. Par linéarité on trouve que $(\lambda_1-\lambda_3)x+(\lambda_2-\lambda_3)x'=0$. On peut donc conclure que $\lambda_1=\lambda_2=\lambda_3$. Donc on a bien une homothétie.
\subparagraph{2}Il suffit de remarquer que tout sous-espace vectoriel de dimension $k-1$ peut s'écrire comme intersection de sous-espaces vectoriels de dimension $k$. En raisonnant par récurrence on montre que à nouveau les homothéties sont les seuls endomorphismes qui ont cette propriété.

\section{Polynômes annulateurs}
Soit $u$ un endomorphisme de $E$. On suppose que $E$ est de dimension finie.
\subparagraph{1} On pose $n=\text{dim}E$. $(Id, u, \dots, u^{n^2} )$ famille liée. Donc il existe un polynôme annulateur.
\subparagraph{2}Si $u$ bijectif et $P(u)=0$ alors on compose par $u^-1$ autant de fois que nécessaire pour que $P(u)u^{-k}$ ait un coefficient constant non-nul et on a toujours $P(u)u^{-k}$ annulateur. Réciproquement, $a_0 Id + Q(u)u=0$ (en regroupant les termes sans coefficient d'ordre 0). On trouve donc que $u$ est inversible d'inverse $-\frac{1}{a_0}Q(u)$.
\subparagraph{3}$uQ(u)=0$ avec $Q \wedge X =1$. Donc $\text{ker}Q(u)$ et $\text{ker}u$ sont en somme directe. De plus si $x \in \text{ker} Q(u)$, $a_0x+u(\tilde{Q}(u(x)))=0$. Donc $x \in \text{Im}(u)$. On peut conclure via un argument de dimension. La réciproque est plus compliquée. On considère $X$ qui est annulateur de la restriction de $u$ à $ker{u}$ et $Q$ qui est annulateur de la restriction de $u$ à $\text{Im}(u)$. Puisque la restriction de $u$ à son image est injective on peut considérer que $Q$ n'est pas divisible par $X$ (voir question 1). Ensuite il convient de vérifier que $XQ$ est bien annulateur de $u$. Une fois cela vérifié on peut conclure.
\subparagraph{4}On n'a pas forcément de polynôme annulateur. Considérer l'endomorphisme dérivation sur les polynômes...

\section{Rang et endomorphisme}
\subparagraph{1} L'image de $u+v$ est incluse dans la somme de l'image de $u$ et de l'image de $v$. En utilisant la formule de Grassmann on a que $n=\text{rg}(u+v) \le \text{rg}(u) +\text{rg}(v)$. 

De plus, $u \circ v=0$ donc $\text{Im}(v) \subset \text{ker}u$. Le théorème du rang donne donc $\text{rg}(u) +\text{rg}(v) \le n$. On peut donc conclure.

\section{Rang et composition}
\subparagraph{1}On applique le théorème du rang à la restriction de $f$ à $\text{Im}g$. On a alors $\text{rg}g= \text{rg}(f \circ g) + \text{dim} \left( \text{ker} f \cap \text{Im}g \right)$. Mais ce dernier ensemble est bien inclus dans le noyau de $f$ et on peut conclure en utilisant le théorème du rang sur $f$.
\subparagraph{2} On trouve les endomorphismes de rang $1$ dont l'image est incluse dans le noyau.

\section{Rang et sous-espace vectoriel}
\subparagraph{1}Il s'agit simplement de considérer la restriction de l'endomorphisme à $F$ et d'appliquer le théorème du rang.

\section{Endomorphismes et polynômes}
Il y a une erreur dans l'exercice. A la dernière question il faut remplacer $P_n(X+2)-P_n(X+1)$ par $P_n(X+2)+P_n(X+1)$
\subparagraph{1}Le noyau est vide donc on a isomorphisme. On peut donc construire une base comme image réciproque de la base proposée.
\subparagraph{2}$P_n(X+2)+P_n(X+1)=\phi(P_n(X+1))$ mais on a aussi que $P_n(X+2)+P_n(X+1)=2(X+1)^n$. Donc $$\phi(P_n(X+1))=\sum_{k=0}^n \binom{k}{n} 2X^k$$. D'où $$P_n(X+1)=\sum_{k=0}^n \binom{k}{n} P_k$$. Donc puisque $P_n(X+1)+P_n(X)=2X^n$ on trouve $$P_n = X^n - \frac{1}{2} \sum_{k=0}^{n-1} \binom{k}{n} P_k$$.
\end{document}