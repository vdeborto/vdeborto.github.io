\documentclass[10pt,a4paper]{article} 
\usepackage[utf8]{inputenc} 
\usepackage[T1]{fontenc} 
\usepackage[english]{babel} 
\usepackage{supertabular} %Nécessaire pour les longs tableaux
\usepackage[top=2.5cm, bottom=2.5cm, right=1.5cm, left=1.5cm]{geometry} %Mise en page 
\usepackage{amsmath} %Nécessaire pour les maths 
\usepackage{amssymb} %Nécessaire pour les maths 
\usepackage{stmaryrd} %Utilisation des double crochets 
\usepackage{pifont} %Utilisation des chiffres entourés 
\usepackage{graphicx} %Introduction d images 
\usepackage{epstopdf} %Utilisation des images .eps 
\usepackage{amsthm} %Nécessaire pour créer des théorèmes 
\usepackage{algorithmic} %Nécessaire pour écrire des algorithmes 
\usepackage{algorithm} %Idem 
\usepackage{bbold} %Nécessaire pour pouvoir écrire des indicatrices 
\usepackage{hyperref} %Nécessaire pour écrire des liens externes 
\usepackage{array} %Nécessaire pour faire des tableaux 
\usepackage{tabularx} %Nécessaire pour faire de longs tableaux 
\usepackage{caption} %Nécesaire pour mettre des titres aux tableaux (tabular) 
\usepackage{color} %nécessaire pour écrire en couleur 
\newtheorem{thm}{Théorème} 
\newtheorem{mydef}{Définition} 
\newtheorem{prop}{Proposition} 
\newtheorem{lemma}{Lemme}
\title{Semaine 7 - Bornes supérieures, équations différentielles linéaires}
\author{Valentin De Bortoli \\ email : \ \href{mailto:valentin.debortoli@gmail.com}{valentin.debortoli@gmail.com}}
\date{}
\begin{document}
\maketitle
\section{Résolution d'une équation différentielle (1)}
\subparagraph{1}Résoudre en $y$ sur $]-\infty,-1[$, sur $]-1,1[$ et sur $]1,+\infty[$ l'équation suivante : $(1-x^2)y'(x)-2xy(x)=x^2$.

\section{Résolution d'une équation différentielle (2)}
\subparagraph{1}Résoudre en $y$ sur $]-\infty,0[$ et sur $]0,+\infty[$ l'équation suivante : $\lvert x \rvert y'(x) +(x-1)y(x)=x^3$.

\section{Résolution d'une équation différentielle (3)}
\subparagraph{1}Résoudre en $y$ sur $\mathbb{R}$ l'équation suivante : $x^2y'(x)-y(x)=0$

\section{Fonctions trigonométriques et équation différentielle}
\subparagraph{1}Calculer $\cos(\arctan(x))$ pour $x \in]-1,1[$.
\subparagraph{2}Calculer $\sin(\arctan(x))$ pour $x \in]-1,1[$.

\section{Résolution d'équation différentielle du second ordre (1)}
\subparagraph{1}Déterminer les solutions réelles de $y''(x)-3y'(x)+2y(x)=\sin(2x)$.

\section{Résolution d'équation différentielle du second ordre (2)}
\subparagraph{1}Déterminer les solutions réelles de $y''(x)+y(x)=\sinh(x)$.
\subparagraph{2}Déterminer les solutions réelles de $y''(x)-2y'(x)+y(x)=2\cosh(x)$.

\section{Résolution d'équation différentielle du second ordre (3)}
\subparagraph{1}Déterminer les solutions réelles de $y''(x)+2y'(x)+y(x)=\sin(x)^3$.
\subparagraph{2}Déterminer les solutions réelles de $y''(x)+y(x)=2\cos(\frac{x}{2})^2$.

\section{Résolution d'équation différentielles du second ordre (4)}
\subparagraph{1}Déterminer les solutions réelles de $ax^2y''(x)+bxy'(x)+cy(x)=0$ sur $\mathbb{R}_+^*$.
\subparagraph{Remarque :} on pourra penser à poser $z(t)=y(e^t)$ et remarquer que $z$ vérifie une équation différentielle.

\section{Solutions bornées et équations différentielles du second ordre}
\subparagraph{1}Déterminer les couples $(a,b)\in \mathbb{R}^2$ tels que toute solution de $y''(x)+ay'(x)+by(x)=0$ soit bornée sur $\mathbb{R}_+$.

\section{Un théorème de point fixe (1)}
Soit $f$ une application croissante de $[0,1]$ dans $[0,1]$.
\subparagraph{1}Soit $A=\{ x \in [0,1], \ f(x) \geq x \}$. Montrer que $A$ admet une borne supérieure.
\subparagraph{2}Montrer que $f$ admet un point fixe, c'est-à-dire, $\exists \ x_0 \in [0,1] \ | \ f(x_0)=x_0$.

\section{Un théorème de point fixe (2)}
Soit $f$ une application continue de $[0,1]$ dans $[0,1]$.
\subparagraph{1}Soit $A=\{ x \in [0,1], \ f(x) \geq x \}$. Montrer que $A$ admet une borne supérieure.
\subparagraph{2}Montrer que $f$ admet un point fixe, c'est-à-dire, $\exists \ x_0 \in [0,1] \ | \ f(x_0)=x_0$.

\section{Inégalité(s) de Shapiro}
\subparagraph{1}Montrer que $\forall (a,b,c) \in {\mathbb{R}_+^*}^3$ on  $\frac{b+c}{a}+\frac{c+a}{b}+\frac{a+b}{c} \ge 6$.
\subparagraph{2}Soit $(x_1,x_2,x_3)\in {\mathbb{R}_+^*}^3$. On pose $y_1=x_2+x_3$, $y_2=x_1+x_3$ et $y_3=x_1+x_2$. Montrer que $\frac{x_1}{y_1}+\frac{x_2}{y_2}+\frac{x_2}{y_2} \ge \frac{3}{2}$.
\subparagraph{3} Soit $(x_1,x_2,x_3,x_4)\in {\mathbb{R}_+^*}^3$. On pose $y_1=x_2+x_3$, $y_2=x_3+x_4$, $y_3=x_4+x_1$ et $y_4=x_1+x_2$. Montrer que $(x_1+x_2+x_3+x_4)^2 \ge 2(x_1 y_1+x_2 y_2 +x_3 y_3+x_4 y_4)$.
\subparagraph{4}Montrer que $\sum_{i=1}^4 \frac{x_i}{y_i} \ge 2$.
\subparagraph{Remarque :} ces inégalités sont appelées les inégalités de Shapiro et on a $\sum_{i=1}^n \frac{x_i}{x_{i+1}+x_{i+2}} \ge \frac{n}{2}$ (où l'addition est à prendre modulo $n$) pour $n \le 12 $ dans le cas pair et $n \le 23$ dans le cas impair. On remarquera qu'ici on a montré les cas $n=3$ et $n=4$. Un contre-exemple pour le cas $n=14$ a été trouvé en 1985 par Troesch, le voici : $(0, 42, 2, 42, 4, 41, 5, 39, 4, 38, 2, 38, 0, 40)$.

\section{Une borne inférieure}
Soit $n \in \mathbb{N}^*$.
\subparagraph{1}Déterminer $\inf \left\lbrace \left( x_1+ \dots + x_n \right) \left( \frac{1}{x_1}+ \dots + \frac{1}{x_n} \right), \ (x_1, \dots x_n) \in \mathbb{R}_+^n\right\rbrace$.
\subparagraph{2}Cet infimum est-il atteint ?

\section{Borne inférieure et borne supérieure}
Soit $A= \left\lbrace \frac{mn}{(m+n)^2}, \ (m,n) \in \mathbb{N}^* \right\rbrace$.
\subparagraph{1}Montrer que $\forall x \in  A, 0<x \le \frac{1}{4}$.
\subparagraph{2}Montrer que $A$ admet une borne supérieure et une borne inférieure et les déterminer.
\subparagraph{3}L'infimum est-il atteint ? Même question concernant le supremum.
\end{document}