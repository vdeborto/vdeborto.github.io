\documentclass[10pt,a4paper]{article} 
\usepackage[utf8]{inputenc} 
\usepackage[T1]{fontenc} 
\usepackage[english]{babel} 
\usepackage{supertabular} %Nécessaire pour les longs tableaux
\usepackage[top=2.5cm, bottom=2.5cm, right=1.5cm, left=1.5cm]{geometry} %Mise en page 
\usepackage{amsmath} %Nécessaire pour les maths 
\usepackage{amssymb} %Nécessaire pour les maths 
\usepackage{stmaryrd} %Utilisation des double crochets 
\usepackage{pifont} %Utilisation des chiffres entourés 
\usepackage{graphicx} %Introduction d images 
\usepackage{epstopdf} %Utilisation des images .eps 
\usepackage{amsthm} %Nécessaire pour créer des théorèmes 
\usepackage{algorithmic} %Nécessaire pour écrire des algorithmes 
\usepackage{algorithm} %Idem 
\usepackage{bbold} %Nécessaire pour pouvoir écrire des indicatrices 
\usepackage{hyperref} %Nécessaire pour écrire des liens externes 
\usepackage{array} %Nécessaire pour faire des tableaux 
\usepackage{tabularx} %Nécessaire pour faire de longs tableaux 
\usepackage{caption} %Nécesaire pour mettre des titres aux tableaux (tabular) 
\usepackage{color} %nécessaire pour écrire en couleur 
\newtheorem{thm}{Théorème} 
\newtheorem{mydef}{Définition} 
\newtheorem{prop}{Proposition} 
\newtheorem{lemma}{Lemme}
\title{Semaine 5 - Calcul intégral}
\author{Valentin De Bortoli \\ email : \ \href{mailto:valentin.debortoli@gmail.com}{valentin.debortoli@gmail.com}}
\date{}
\begin{document}
\maketitle

\section{Intégrales de Wallis}
Soit $n \in \mathbb{N}$ on définit $I_n$ (l'intégrale de Wallis) par $I_n=\int_0^{\frac{\pi}{2}} \sin(x)^n \text{dx}$.
\subparagraph{1}Trouver une relation entre $I_{n+2}$ et $I_n$. En déduire la valeur de $I_n$ en fonction de la parité de $n$.
\subparagraph{2}Montrer que $\lim_{n \rightarrow +\infty}\frac{I_{2n}}{I_{2n+1}}=1$.
\subparagraph{3}En déduire que $\lim_{n \rightarrow +\infty} \sqrt{p}\frac{(2p-1)(2p-3)\dots 1}{2p(2p-2)\dots 2}=\frac{1}{\sqrt{\pi}}$
\subparagraph{Remarque :} ce calcul est un grand classique. Il sert notamment pour obtenir une formule très utile : la formule de Stirling. Celle-ci donne un équivalent en l'infini de la factorielle. Plus précisément : $n! \underset{+\infty}{\sim} (\frac{n}{e})^n\sqrt{2\pi n}$.

\section{Suite et intégrale (1)}
Soit $n \in \mathbb{N}$. On définit $J_n=\int_{0}^{\frac{\pi}{4}} \tan(x)^n \ \text{dx}$.
\subparagraph{1}Donner une formule liant $J_{n+2}$ et $J_{n}$. On commencera par calculer $J_{n+2}+J_n$.
\subparagraph{2}Après avoir calculé $J_0$ et $J_1$ exprimer $J_n$ en fonction de la parité de $n$.

\section{Suite et intégrale (2)}
Soit $n \in \mathbb{N}$. On définit $K_n=\int_{0}^{\frac{\pi}{4}} \frac{1}{\cos(x)^n}\ \text{dx}$.
\subparagraph{1}Calculer $K_0$ et $K_1$.
\subparagraph{2}Donner une formule liant $K_{n+2}$ et $K_{n}$. On pourra intégrer par partie $K_{n+2}$.

\section{Suite et intégrale (3)}
Soit $n \in \mathbb{N}$. On définit $L_n=\int_{1}^{e} \log(x)^n \ \text{dx}$.
\subparagraph{1}Donner une formule liant $L_{n+1}$ et $L_{n}$.

\section{Suite et intégrale (4)}
Soit $(f_n)_{n \in \mathbb{N}}$ une suite de fonctions définie par $\forall x \in \mathbb{R}, \ f_n(x)=\frac{1}{(1+x^2)^n}$.
\subparagraph{1}Montrer que $\forall n \in \mathbb{N}, \ f_n \ $ admet une primitive qui s'annule en $0$ notée $F_n$.
\subparagraph{2}Déterminer $F_1$.
\subparagraph{3}Déterminer $F_2$. On pourra penser à faire un changement de variable en tangente.
\subparagraph{4}Trouver une formule liant $F_{n+1}$ et $F_n$.
\subparagraph{5}Déterminer $F_3$.

\section{Primitive et fonction circulaire}
\subparagraph{1}Donner une primitive de $x \mapsto \arccos(x)$.
\subparagraph{2}Donner une primitive de $x \mapsto \frac{1}{\cos(x)}$.
\subparagraph{3}Donner une primitive de $x \mapsto \frac{1}{\sin(x)}$.
\subparagraph{4}Donner une primitive de $x \mapsto \frac{1}{2+\sin(x)^2}$.

\section{Primitive et fonction hyperbolique}
\subparagraph{1}Donner une primitive de $x \mapsto \frac{1}{\cosh(x)}$.
\subparagraph{2}Donner une primitive de $x \mapsto \frac{1}{\sinh(x)}$.
\subparagraph{3}Donner une primitive de $x \mapsto \frac{1}{\tanh(x)}$.
\subparagraph{4}Donner une primitive de $x \mapsto \frac{1}{1-\cosh(x)}$.

\section{Primitive et fonction rationnelle en un radical (1)}
\subparagraph{1}Donner une primitive de $x \mapsto \frac{1}{\sqrt{x+1}(x+4)}$.
\subparagraph{2}Donner une primitive de $x \mapsto \frac{1}{\sqrt{x}(x+3)}$.

\section{Primitive et fonction rationnelle en un radical (2)}
\subparagraph{1} Donner une primitive de $x \mapsto \frac{x+1}{\sqrt{2-x^2}}$.
\subparagraph{2}Donner une primitive de $x \mapsto \frac{1}{x+\sqrt{1+x^2}}$.

\section{Calcul d'intégrales}
\subparagraph{1}Calculer $\int_0^{\frac{\pi}{4}}\ln(1+\tan(x)) \text{d}x$.
\subparagraph{2}Calculer $\int_0^{\frac{\pi}{2}}\frac{\sqrt{\sin(x)}}{\sqrt{\sin(x)}+\sqrt{\cos(x)}}\ \text{d}x$.

\section{Fonction définie par une intégrale}
\subparagraph{1}Montrer que $F$, d'expression $F(x)=\int_x^{2x}\frac{1}{\sqrt{1+t^4}}\ \text{d}t$ est bien définie sur un intervalle à préciser.
\subparagraph{2}Montrer la continuité et l'imparité de $F$.
\subparagraph{3}Trouver une formule liant $F(x)$ et $F(\frac{1}{2x})$ sur $\mathbb{R}^*$ et déterminer la limite de $F$ en $+\infty$.
\end{document}