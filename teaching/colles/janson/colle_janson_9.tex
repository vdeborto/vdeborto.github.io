\documentclass[10pt,a4paper]{article} 
\usepackage[utf8]{inputenc} 
\usepackage[T1]{fontenc} 
\usepackage[english]{babel} 
\usepackage{supertabular} %Nécessaire pour les longs tableaux
\usepackage[top=2.5cm, bottom=2.5cm, right=1.5cm, left=1.5cm]{geometry} %Mise en page 
\usepackage{amsmath} %Nécessaire pour les maths 
\usepackage{amssymb} %Nécessaire pour les maths 
\usepackage{stmaryrd} %Utilisation des double crochets 
\usepackage{pifont} %Utilisation des chiffres entourés 
\usepackage{graphicx} %Introduction d images 
\usepackage{epstopdf} %Utilisation des images .eps 
\usepackage{amsthm} %Nécessaire pour créer des théorèmes 
\usepackage{algorithmic} %Nécessaire pour écrire des algorithmes 
\usepackage{algorithm} %Idem 
\usepackage{bbold} %Nécessaire pour pouvoir écrire des indicatrices 
\usepackage{hyperref} %Nécessaire pour écrire des liens externes 
\usepackage{array} %Nécessaire pour faire des tableaux 
\usepackage{tabularx} %Nécessaire pour faire de longs tableaux 
\usepackage{caption} %Nécesaire pour mettre des titres aux tableaux (tabular) 
\usepackage{color} %nécessaire pour écrire en couleur 
\newtheorem{thm}{Théorème} 
\newtheorem{mydef}{Définition} 
\newtheorem{prop}{Proposition} 
\newtheorem{lemma}{Lemme}
\title{Semaine 9 - Nombres réels, suites réelles}
\author{Valentin De Bortoli \\ email : \ \href{mailto:valentin.debortoli@gmail.com}{valentin.debortoli@gmail.com}}
\date{}
\begin{document}
\maketitle
\section{Un théorème de point fixe (1)}
Soit $f$ une application croissante de $[0,1]$ dans $[0,1]$.
\subparagraph{1}Soit $A=\{ x \in [0,1], \ f(x) \geq x \}$. Montrer que $A$ admet une borne supérieure.
\subparagraph{2}Montrer que $f$ admet un point fixe, c'est-à-dire, $\exists \ x_0 \in [0,1] \ | \ f(x_0)=x_0$.

\section{Un théorème de point fixe (2)}
Soit $f$ une application continue de $[0,1]$ dans $[0,1]$.
\subparagraph{1}Soit $A=\{ x \in [0,1], \ f(x) \geq x \}$. Montrer que $A$ admet une borne supérieure.
\subparagraph{2}Montrer que $f$ admet un point fixe, c'est-à-dire, $\exists \ x_0 \in [0,1] \ | \ f(x_0)=x_0$.

\section{Inégalité(s) de Shapiro}
\subparagraph{1}Montrer que $\forall (a,b,c) \in {\mathbb{R}_+^*}^3$ on  $\frac{b+c}{a}+\frac{c+a}{b}+\frac{a+b}{c} \ge 6$.
\subparagraph{2}Soit $(x_1,x_2,x_3)\in {\mathbb{R}_+^*}^3$. On pose $y_1=x_2+x_3$, $y_2=x_1+x_3$ et $y_3=x_1+x_2$. Montrer que $\frac{x_1}{y_1}+\frac{x_2}{y_2}+\frac{x_2}{y_2} \ge \frac{3}{2}$.
\subparagraph{3} Soit $(x_1,x_2,x_3,x_4)\in {\mathbb{R}_+^*}^3$. On pose $y_1=x_2+x_3$, $y_2=x_3+x_4$, $y_3=x_4+x_1$ et $y_4=x_1+x_2$. Montrer que $(x_1+x_2+x_3+x_4)^2 \ge 2(x_1 y_1+x_2 y_2 +x_3 y_3+x_4 y_4)$.
\subparagraph{4}Montrer que $\sum_{i=1}^4 \frac{x_i}{y_i} \ge 2$.
\subparagraph{Remarque :} ces inégalités sont appelées les inégalités de Shapiro et on a $\sum_{i=1}^n \frac{x_i}{x_{i+1}+x_{i+2}} \ge \frac{n}{2}$ (où l'addition est à prendre modulo $n$) pour $n \le 12 $ dans le cas pair et $n \le 23$ dans le cas impair. On remarquera qu'ici on a montré les cas $n=3$ et $n=4$. Un contre-exemple pour le cas $n=14$ a été trouvé en 1985 par Troesch, le voici : $(0, 42, 2, 42, 4, 41, 5, 39, 4, 38, 2, 38, 0, 40)$.

\section{Une borne inférieure}
Soit $n \in \mathbb{N}^*$.
\subparagraph{1}Déterminer $\inf \left\lbrace \left( x_1+ \dots + x_n \right) \left( \frac{1}{x_1}+ \dots + \frac{1}{x_n} \right), \ (x_1, \dots x_n) \in \mathbb{R}_+^n\right\rbrace$.
\subparagraph{2}Cet infimum est-il atteint ?

\section{Borne inférieure et borne supérieure}
Soit $A= \left\lbrace \frac{mn}{(m+n)^2}, \ (m,n) \in \mathbb{N}^* \right\rbrace$.
\subparagraph{1}Montrer que $\forall x \in  A, 0<x \le \frac{1}{4}$.
\subparagraph{2}Montrer que $A$ admet une borne supérieure et une borne inférieure et les déterminer.
\subparagraph{3}L'infimum est-il atteint ? Même question concernant le supremum.

\section{Convergence au sens de Césaro}
Soit $(u_n)_{n \in \mathbb{N}^*} \in \mathbb{C}^{\mathbb{N}^*}$. On définit $(v_n)_{n \in \mathbb{N}^*}$ de la manière suivante :
\begin{equation*}
\forall n \in \mathbb{N}^*, \ v_n=\frac{\underset{k=1}{\overset{n}{\sum}}u_k}{n}
\end{equation*}
\subparagraph{1}Montrer que  $ u_n \rightarrow l \in \mathbb{C} \ \Rightarrow \ v_n \rightarrow l \in \mathbb{C}$.
\subparagraph{2}Trouver un contrexemple à la réciproque.
\subparagraph{3}Supposons que $(\omega_n)_{n \in \mathbb{N}} \in \mathbb{C}^{\mathbb{N}}$ et $\omega_{n+1}-\omega_n \rightarrow l \in \mathbb{C}^*$. Montrer que $\omega_n \underset{+\infty}{\sim} ln$.
\subparagraph{4}On définit $(w_n)_{n \in \mathbb{N}^*}$ de la manière suivante :
\begin{equation*}
\forall n \in \mathbb{N}^*, \ w_n=\frac{\underset{k=1}{\overset{n}{\sum}}ku_k}{n^2}
\end{equation*}
Montrer que $w_n \rightarrow \frac{l}{2}$.

\section{Suite sous-additive}
Soit $(u_n)_{n \in \mathbb{N}} \in \mathbb{R}^{\mathbb{N}}$ une suite sous-additive au sens où :
\begin{equation*}
\forall (p,q) \in \mathbb{N}^2, \ u_{p+q}\le u_p+u_q
\end{equation*}
\subparagraph{1}Rappeler la définition de $\inf \left\lbrace  \frac{u_n}{n}, \ n \in \mathbb{N}^*\right\rbrace$.
\subparagraph{2}Soit $n=qm+r$, $r \in \llbracket 0,q-1 \rrbracket$, la division euclidienne de $n$ par $q$. Établir une inégalité faisant intervenir $u_n$, $u_q$ et $u_1$.
\subparagraph{3}Montrer que la suite $(\frac{u_n}{n})_{n \in \mathbb{N}^*}$ tend vers $\inf \left\lbrace  \frac{u_n}{n}, \ n \in \mathbb{N}^*\right\rbrace$.
\subparagraph{2}Soit $(v_n)_{n \in \mathbb{N}} \in \mathbb{R_+^*}^{\mathbb{N}}$ qui vérifie :
\begin{equation*}
\forall (p,q) \in \mathbb{N}^2, \ v_{p+q}\le v_p v_q
\end{equation*}
Que peut-on dire de la suite $(v_n)_{n \in \mathbb{N}}$ ?

\section{Rationnels et irrationnels}
Soit $\left( r_n=\frac{p_n}{q_n} \right)_{n \in \mathbb{N}}$ une suite de rationnels ($(p_n)_{n\in \mathbb{N}} \in \mathbb{Z}^{\mathbb{N}}$ et $(q_n)_{n \in \mathbb{N}} \in \left(\mathbb{N} \backslash \lbrace 0 \rbrace \right) ^{\mathbb{N}}$). On suppose que $r_n \rightarrow x \in \mathbb{R} \backslash \mathbb{Q}$.
\subparagraph{1} Montrer que $q_n \rightarrow +\infty$.

\section{Une équation et des parties entières}
\subparagraph{1}Pour $x=8$ on a $\frac{x}{2}-\sqrt{x}=2(2-\sqrt{2})>1$. Donc à partir de $x=8$ les parties entières de $\frac{x}{2}$ et $\sqrt{x}$ sont différentes. Il s'agit maintenant de trouver les solutions pour $x<8$ (une autre solution aurait de résoudre une équation du second degré $-\sqrt{x}^2+(\frac{x}{2}-1)^2=0$. On aurait trouvé une borne similaire) :
\begin{itemize}
\item $\lfloor \frac{x}{2} \rfloor =0$ si et seulement si $x \in [0,2[$. $\lfloor \sqrt{x} \rfloor=0$ si et seulement si $x \in [0,1[$. Donc $[0,1[$ dans l'ensemble des solutions.
\item $\lfloor \frac{x}{2} \rfloor =1$ si et seulement si $x \in [2,4[$. $\lfloor \sqrt{x} \rfloor=1$ si et seulement si $x \in [1,4[$. Donc $[2,4[$  dans l'ensemble des solutions.
\item $\lfloor \frac{x}{2} \rfloor =2$ si et seulement si $x \in [4,6[$. $\lfloor \sqrt{x} \rfloor=2$ si et seulement si $x \in [4,8[$. Donc $[4,6[$  dans l'ensemble des solutions.
\end{itemize}
On s'arrête là car on sait que les prochaines solutions se trouvent après $x=9$. On sait donc qu'il n'y en aura plus. On a l'ensemble de solutions suivant :
\begin{equation}
S=[0,1[ \cup [4,6[
\end{equation}
\section{Une propriété de la partie entière}
Soit $n \in \mathbb{N}^*$. Soit $x \in \mathbb{R}_+$.
\subparagraph{1}Montrer que $\lfloor \frac{\lfloor  nx \rfloor}{n} \rfloor =\lfloor x \rfloor$.

\section{Somme et partie entière}
Soit $n \in \mathbb{N}^*$. Soit $x \in \mathbb{R}_+$.
\subparagraph{1}Montrer que $\lfloor nx \rfloor - n \lfloor x \rfloor = n \lbrace x \rbrace - \lbrace nx \rbrace$. En déduire que $n \lbrace x \rbrace - \lbrace nx \rbrace \in \mathbb{Z}$.
\subparagraph{2}Montrer que $\underset{k=0}{\overset{n-1}{\sum}} \lfloor x +\frac{k}{n}\rfloor= \lfloor nx \rfloor$.
\subparagraph{Remarque :} on pourra commencer par considérer $x$ tel que $\lbrace x \rbrace < \frac{1}{n}$.

\section{Nombre de zéros et factorielle}
\subparagraph{1}Donner le nombre de zéros à la fin de $10!$.
\subparagraph{2}Même question mais pour $25!$
\subparagraph{3}Même question mais pour $n!$ avec $n\in \mathbb{N}$. On exprimera le résultat en utilisant la fonction partie entière.
\end{document}