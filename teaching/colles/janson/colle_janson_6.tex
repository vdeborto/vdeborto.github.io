\documentclass[10pt,a4paper]{article} 
\usepackage[utf8]{inputenc} 
\usepackage[T1]{fontenc} 
\usepackage[english]{babel} 
\usepackage{supertabular} %Nécessaire pour les longs tableaux
\usepackage[top=2.5cm, bottom=2.5cm, right=1.5cm, left=1.5cm]{geometry} %Mise en page 
\usepackage{amsmath} %Nécessaire pour les maths 
\usepackage{amssymb} %Nécessaire pour les maths 
\usepackage{stmaryrd} %Utilisation des double crochets 
\usepackage{pifont} %Utilisation des chiffres entourés 
\usepackage{graphicx} %Introduction d images 
\usepackage{epstopdf} %Utilisation des images .eps 
\usepackage{amsthm} %Nécessaire pour créer des théorèmes 
\usepackage{algorithmic} %Nécessaire pour écrire des algorithmes 
\usepackage{algorithm} %Idem 
\usepackage{bbold} %Nécessaire pour pouvoir écrire des indicatrices 
\usepackage{hyperref} %Nécessaire pour écrire des liens externes 
\usepackage{array} %Nécessaire pour faire des tableaux 
\usepackage{tabularx} %Nécessaire pour faire de longs tableaux 
\usepackage{caption} %Nécesaire pour mettre des titres aux tableaux (tabular) 
\usepackage{color} %nécessaire pour écrire en couleur 
\newtheorem{thm}{Théorème} 
\newtheorem{mydef}{Définition} 
\newtheorem{prop}{Proposition} 
\newtheorem{lemma}{Lemme}
\title{Semaine 6 - Déceloppements limités et équivalents}
\author{Valentin De Bortoli \\ email : \ \href{mailto:valentin.debortoli@gmail.com}{valentin.debortoli@gmail.com}}
\date{}
\begin{document}
\maketitle
\section{Développements limités (1)}
\subparagraph{1} Justifier l'existence et calculer le développement limité en 0 à l'ordre 4 de $x \mapsto \frac{e^x}{(1+x)^3}$.
\subparagraph{2} Justifier l'existence et calculer le développement limité en 0 à l'ordre 6 de $x \mapsto\sin(x^2)$
\subparagraph{3} Justifier l'existence et calculer le développement limité en 0 à l'ordre 6 de $x \mapsto \ln(1+x)\sin(x)$

\section{Développements limités et asymptotiques (1)}
Soit $f : x \mapsto \sqrt{1+x+x^2}$.
\subparagraph{1}Justifier l'existence et calculer un développement limité de $f$ en 0 à l'ordre 2.
\subparagraph{2}Le graphe de $f$ admet-il une tangente en $0$ ? Si oui, donner la position du graphe de $f$ par rapport à cette tangente autour de $0$.
\subparagraph{3}Déterminer une asymptote en $+\infty$ au graphe de $f$.

\section{Développements limités et asymptotiques (2)}
Soit $f \mapsto (x^2-1) \ln\left(\vert \frac{1+x}{1-x} \vert\right)$.
\subparagraph{1}Justifier l'existence et calculer un développement limité de $f$ en 0 à l'ordre 3.
\subparagraph{2}Déterminer une asymptote en $+\infty$ au graphe de $f$ et donner la position de la courbe par rapport à cette asymptote lorsque $x$ est grand.

\section{Développements limités (2)}
\subparagraph{1}Justifier l'existence et calculer un développement limité en 0 à l'ordre 5 de $x \mapsto \ln\left( \sqrt{\frac{1+x}{1-x}} \right)$.
\subparagraph{2}Justifier l'existence et calculer un développement limité en 0 à l'ordre 2 de $x \mapsto \frac{\ln\left( \sqrt{\frac{1+x}{1-x}} \right)-x}{\sin(x)-x}$.

\section{Développements limités (3)}
\subparagraph{1}Justifier l'existence et calculer un développement limité en 0 à l'ordre 3 de $x \mapsto \frac{\cosh(x) \ln(1+x)}{\cos(x)}$.

\section{Développements limités et dérivabilité}
\subparagraph{1}Montrer que $f$ est continue en 0 si et seulement si $f$ admet un développement limité d'ordre 0 en 0.
\subparagraph{2}Montrer que $f$ est dérivable en 0 si et seulement si $f$ admet un développement limité d'ordre 1 en 0.
\subparagraph{3}Montrer que si $f$ est deux fois dérivable en 0 alors $f$ admet un développement limité d'ordre 2 en 0.
\subparagraph{3}Montrer que $x \mapsto x^3 \sin(\frac{1}{x})$ définie sur $\mathbb{R}^*$ et prolongée par continuité en 0 admet un développement limité à l'ordre 2 mais n'est pas 2 fois dérivable en 0.

\section{Développements limités (4)}
\subparagraph{1}Justifier l'existence et calculer un développement limité en $\frac{\pi}{2}$ à l'ordre 2 de $x \mapsto \ln(\sin(x))$.
\subparagraph{2}Justifier l'existence et calculer un développement limité en $\frac{\pi}{2}$ à l'ordre 2 de $x \mapsto (1+\cos(x))^{\frac{1}{x}}$.

\section{Développements limités (5)}
\subparagraph{1}Calculer un développement limité en 0 à l'ordre 2 de $x \mapsto \frac{1}{\sin(x)^2}-\frac{1}{\sinh(x)^2}$ (existence admise).
\subparagraph{2}Justifier l'existence et calculer un développement limité en 0 à l'ordre 4 de $x \mapsto \cos(2x)^{\frac{3}{x^2}}$.
\subparagraph{3}Justifier l'existence et calculer un développement limité en 0 à l'ordre 5 de $x \mapsto \sin(x)^3(e^{x^2}-1)$.

\section{Fonction décroissante et équivalent}
Soit $f$ une fonction décroissante qui de $\mathbb{R}$ dans $\mathbb{R}$. On suppose que $f(x)+f(x+1) \underset{x \rightarrow +\infty}{\sim} \frac{1}{x}$.
\subparagraph{1}Montrer que $f$ admet une limite et la calculer.
\subparagraph{2}Donner un équivalent de $f$.

\section{Calcul de limites (1)}
\subparagraph{1}Montrer que $x \mapsto \frac{x^{\ln(x)}}{\ln(x)}$ admet une limite en $+\infty$ et la calculer.
\subparagraph{2}Montrer que $x \mapsto (\frac{x}{\ln(x)})^{\frac{\ln(x)}{x}}$ admet une limite en $+\infty$ et la calculer.
\subparagraph{3}Montrer que $x \mapsto \frac{\ln(x+\sqrt{x^2+1})}{\ln(x)}$ admet une limite en $+\infty$ et la calculer.

\section{Calcul de limites (2)}
\subparagraph{1}Montrer que $x \mapsto (x+1)e^x-xe^{x+1}$ admet une limite en $+\infty$ et la calculer.
\subparagraph{2}Montrer que $x \mapsto (x+1)\ln(x)-x\ln(x+1)$ admet une limite en $+\infty$ et la calculer.
\end{document}