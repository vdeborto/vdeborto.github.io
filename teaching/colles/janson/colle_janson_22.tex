\documentclass[10pt,a4paper]{article} 
\usepackage[utf8]{inputenc} 
\usepackage[T1]{fontenc} 
\usepackage[english]{babel} 
\usepackage{supertabular} %Nécessaire pour les longs tableaux
\usepackage[top=2.5cm, bottom=2.5cm, right=1.5cm, left=1.5cm]{geometry} %Mise en page 
\usepackage{amsmath} %Nécessaire pour les maths 
\usepackage{amssymb} %Nécessaire pour les maths 
\usepackage{stmaryrd} %Utilisation des double crochets 
\usepackage{pifont} %Utilisation des chiffres entourés 
\usepackage{graphicx} %Introduction d images 
\usepackage{epstopdf} %Utilisation des images .eps 
\usepackage{amsthm} %Nécessaire pour créer des théorèmes 
\usepackage{algorithmic} %Nécessaire pour écrire des algorithmes 
\usepackage{algorithm} %Idem 
\usepackage{bbold} %Nécessaire pour pouvoir écrire des indicatrices 
\usepackage{hyperref} %Nécessaire pour écrire des liens externes 
\usepackage{array} %Nécessaire pour faire des tableaux 
\usepackage{tabularx} %Nécessaire pour faire de longs tableaux 
\usepackage{caption} %Nécesaire pour mettre des titres aux tableaux (tabular) 
\usepackage{color} %nécessaire pour écrire en couleur 
\newtheorem{thm}{Théorème} 
\newtheorem{mydef}{Définition} 
\newtheorem{prop}{Proposition} 
\newtheorem{lemma}{Lemme}
\title{Semaine 22 - Systèmes linéaires et suites récurrentes}
\author{Valentin De Bortoli \\ email : \ \href{mailto:valentin.debortoli@gmail.com}{valentin.debortoli@gmail.com}}
\date{}
\begin{document}
\maketitle
Dans la suite $k$ est un corps (on se limite à $\mathbb{R}$ et $\mathbb{C}$) et $E$ un $k$-espace vectoriel.

\section{Matrices circulantes et polygones}
Soit $(a_i)_{i \in \llbracket 1,n\rrbracket} \in \mathbb{C}^n$ ($n \in \mathbb{N}$).
\subparagraph{1}Peut-on trouver $(z_i)_{i \in \llbracket 1,n \rrbracket}$ un polygone du plan complexe tel que $a_i$ soit le milieu de $[z_i,z_{i+1}]$ si $i<n$ et $a_n$ milieu de $[z_n,z_1]$ ?


\section{Un calcul d'inverse}
Soit $A \in \mathcal{G}l_n\left( \mathbb{R} \right)$. On suppose de plus que $A+A^{-1}=I_n$.
\subparagraph{1}Montrer que $\forall k \in \mathbb{N}, \ A^k+A^{-k}$ est scalaire.
\subparagraph{2}En déduire $A^k+A^{-k}$.


\section{Matrices de permutation}
Soit $\sigma$ une bijection de $\llbracket 1,n \rrbracket$ dans $\llbracket 1,n \rrbracket$. On note $M_{\sigma} \in \mathcal{M}_n\left( \mathbb{R} \right)$ la matrice définie par :
\begin{equation*}
\forall (i,j) \in \llbracket 1,n \rrbracket, \ M_{\sigma}(i,j) = \left\lbrace \begin{matrix} 0 \ \text{si } \ j \neq \sigma(i) \\ 
1 \ \text{sinon}\end{matrix} \right.
\end{equation*}
\subparagraph{1}Quel est l'effet d'une multiplication à droite par une matrice de permutation ? A gauche ?
\subparagraph{2}On suppose que la permutation considérée est une transposition, c'est-à-dire $\exists! \ (i,j) \in \llbracket 1,n \rrbracket, \ \sigma(i) \neq i \wedge \sigma(j) \neq j \wedge \sigma(i)=j$. Montrer que dans ce cas $M_{\sigma}$ peut s'écrire comme produit de matrices de type $I_n+E_{i,j}$ (matrices de transvection) et d'une matrice de dilatation (matrice diagonale inversible dont un seul des coefficients est différent de $1$).
\subparagraph{3}En supposant que toute permutation peut s'écrire comme un composition de transpositions, conclure que toute matrice de permutation peut s'écrire comme produit de matrices de transvection et de matrices de dilatation.

\section{Matrices de transvection, matrice de dilatation}
On appelle matrices de transvection les matrices de la forme $I_n+E_{i,j}$. On appelle matrice de dilatation toute matrice diagonale inversible dont un seul des coefficients est différent de $1$. Ces deux ensembles jouent un rôle fondamental pour la description du groupe linéaire.
\subparagraph{1}Reprendre la question 1 de l'exercice précédent. A partir de maintenant on admettra la dernière question de l'exercice précédent.
\subparagraph{2}Montrer que tout élément du groupe linéaire peut s'écrire comme produit de matrices de transvection et de dilatation.

\section{Matrices échelonnées et nombres entiers}
\subparagraph{1}Donner la forme échelonnée selon les colonnes de $\left( \begin{matrix} 3 & 2 \\1 & 1\end{matrix}\right)$.
\subparagraph{2} Soit $M=\left( \begin{matrix} a & b \\ c & d \end{matrix} \right) \in \mathcal{M}_2 \left( \mathbb{Z} \right)$. Montrer qu'il existe $P=\left( \begin{matrix} s & t \\ u & v \end{matrix} \right) \in \mathcal{M}_2 \left( \mathbb{Z} \right)$ inversible et d'inverse dans $\mathcal{M}_2 \left( \mathbb{Z} \right)$ tel que $MP = \left( \begin{matrix} 1 & 0 \\ c' & d' \end{matrix} \right)$ avec $(c',d') \in \mathbb{Z}^2$.
\subparagraph{3}Comment obtenir une matrice échelonnée selon les colonnes dans $\mathbb{Z}$ ?
\subparagraph{4}Appliquer les conclusions de la question précédente à l'exemple de la première question.


\section{Somme de matrices inversibles}
\subparagraph{1}Montrer que tout matrice $M \in \mathcal{M}_n \left( \mathbb{R} \right)$ peut s'écrire comme la somme de deux matrices inversibles.

\section{Dimension d'un espace matriciel}
Soit $A \in \mathcal{M}_n \left( \mathbb{R} \right)$ de rang $r$.
\subparagraph{1}Montrer que $\mathcal{A} = \lbrace B \in \mathcal{M}_n\left( \mathbb{R} \right), ABA=0 \rbrace$ est un espace-vectoriel et donner sa dimension.

\section{Somme de matrice inversible}
Soit $(A,B) \in \left( \mathcal{M}_n\left( \mathbb{R} \right) \right)^2$.
\subparagraph{1} On suppose que $\text{rg}(A)+\text{rg}(B) \ge n$. Montrer qu'il existe $U$ et $V$ deux matrices de $\mathcal{G}l_n\left(\mathbb{R} \right)$ telles que $UA+BV$ inversible.

\section{Suite et équivalent}
Soit $(u_n)_{n \in \mathbb{N}}$ définie de la manière suivante :
\begin{equation*}
\left\lbrace
\begin{aligned}
& u_0 \in ]0,\frac{\pi}{2}[\\
& u_{n+1}=\sin(u_n)
\end{aligned}
\right.
\end{equation*}
\subparagraph{1}Montrer que $u_n \rightarrow 0$.
\subparagraph{2}Montrer que $\frac{1}{u_{n+1}^2}-\frac{1}{u_n^2}$ admet une limite en $+\infty$ et la calculer.
\subparagraph{3}En utilisant la question 3 de l'exercice précédent déterminer un équivalent de $u_n$ lorsque $n \rightarrow +\infty$.
\section{Méthode de Newton}
Soit $f \in \mathcal{C}^2([a,b])$ à valeurs réelles avec $f'>0$ sur $[a,b]$. On définit $(x_n)_{n \in \mathbb{R}} \in \mathbb{R}^{\mathbb{N}}$ de la manière suivante :
\begin{equation*}
x_{n+1}=x_n-\frac{f(x_n)}{f'(x_{n})}
\end{equation*}
On suppose également que $f$ s'annule en $c \in ]a,b[$. Enfin on suppose que $\forall n \in \mathbb{N}, \ x_n \in [a,b]$.
\subparagraph{1}Montrer que $\forall n \in \mathbb{N}, \ x_{n+1}$ correspond à l'abscisse du point d'intersection entre la tangente à $f$ en $x_n$ et l'axe des abscisses. Faire un dessin.
\subparagraph{2}Montrer que $\forall n \in \mathbb{N}, \ \vert x_{n+1}-c \vert \le C \vert x_n-c \vert^2$ avec $C \in \mathbb{R}_+$.
\subparagraph{3}Donner la formule liant $x_{n+1}$ et $x_n$ dans le cas où $f : \ x \mapsto x^2-a$ et $f $ définie sur $[0,2a], \ (a\in\mathbb{R}_+)$.
\subparagraph{Remarque :} cette méthode est très utilisée pour trouver le minimum de fonctionnelle. Néanmoins $f$ n'est pas toujours dérivable ou peut être très compliquée à dériver. On utilise alors d'autres algorithmes (méthode de la corde, méthode de la sécante, dichotomie et bien d'autres...). 

\section{Récurrence et nombre d'or}
\subparagraph{1}Montrer que $\sqrt{1+\sqrt{1+\sqrt{\dots}}}=1+\frac{1}{1+\frac{1}{1+\dots}}=\frac{1+\sqrt{5}}{2}$.

\section{Une suite complexe}
Soit la suite $(z_n)_{n \in \mathbb{N}} \in \mathbb{C}^{\mathbb{N}}$ définie par :
\begin{equation*}
\left\lbrace\begin{aligned}
&z_0=\rho e^{i\theta} \in \mathbb{C}, \ \rho \in \mathbb{R}_+, \theta \in [-\pi,\pi] \\
&z_{n+1}=\frac{z_n+\vert z_n \vert}{2}, \ \forall n \in \mathbb{N}
\end{aligned} \right.
\end{equation*}
\subparagraph{1}Exprimer $z_n$ sous la forme d'un produit.
\subparagraph{2}Montrer que $\forall n \in \mathbb{N}^*, \ \underset{k=1}{\overset{n}{\prod}}\cos(\frac{\theta}{2^k})=\frac{\sin(\theta)}{2^n\sin(\frac{\theta}{2^n})}$.
\subparagraph{3}Montrer que $z_n$ admet une limite et la calculer.
\end{document}