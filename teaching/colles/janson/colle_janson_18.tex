\documentclass[10pt,a4paper]{article} 
\usepackage[utf8]{inputenc} 
\usepackage[T1]{fontenc} 
\usepackage[english]{babel} 
\usepackage{supertabular} %Nécessaire pour les longs tableaux
\usepackage[top=2.5cm, bottom=2.5cm, right=1.5cm, left=1.5cm]{geometry} %Mise en page 
\usepackage{amsmath} %Nécessaire pour les maths 
\usepackage{amssymb} %Nécessaire pour les maths 
\usepackage{stmaryrd} %Utilisation des double crochets 
\usepackage{pifont} %Utilisation des chiffres entourés 
\usepackage{graphicx} %Introduction d images 
\usepackage{epstopdf} %Utilisation des images .eps 
\usepackage{amsthm} %Nécessaire pour créer des théorèmes 
\usepackage{algorithmic} %Nécessaire pour écrire des algorithmes 
\usepackage{algorithm} %Idem 
\usepackage{bbold} %Nécessaire pour pouvoir écrire des indicatrices 
\usepackage{hyperref} %Nécessaire pour écrire des liens externes 
\usepackage{array} %Nécessaire pour faire des tableaux 
\usepackage{tabularx} %Nécessaire pour faire de longs tableaux 
\usepackage{caption} %Nécesaire pour mettre des titres aux tableaux (tabular) 
\usepackage{color} %nécessaire pour écrire en couleur 
\newtheorem{thm}{Théorème} 
\newtheorem{mydef}{Définition} 
\newtheorem{prop}{Proposition} 
\newtheorem{lemma}{Lemme}
\title{Semaine 18 - Calcul de rang et Dérivation}
\author{Valentin De Bortoli \\ email : \ \href{mailto:valentin.debortoli@gmail.com}{valentin.debortoli@gmail.com}}
\date{}
\begin{document}
\maketitle
\section{Un calcul d'inverse}
Soit $A \in \mathcal{G}l_n\left( \mathbb{R} \right)$. On suppose de plus que $A+A^{-1}=I_n$.
\subparagraph{1}Montrer que $\forall k \in \mathbb{N}, \ A^k+A^{-k}$ est scalaire.
\subparagraph{2}En déduire $A^k+A^{-k}$.

\section{Rang et vecteur}
Soit $x_0 \in \mathbb{R}^n$ (vecteur colonne). On note $M = x_0 x_0^T$.
\subparagraph{1}A quel espace appartient $M$ ?
\subparagraph{2}Quel est le rang de $M$ ?

\section{Matrices échelonnées et nombres entiers}
\subparagraph{1}Donner la forme échelonnée selon les colonnes de $\left( \begin{matrix} 3 & 2 \\1 & 1\end{matrix}\right)$.
\subparagraph{2} Soit $M=\left( \begin{matrix} a & b \\ c & d \end{matrix} \right) \in \mathcal{M}_2 \left( \mathbb{Z} \right)$. Montrer qu'il existe $P=\left( \begin{matrix} s & t \\ u & v \end{matrix} \right) \in \mathcal{M}_2 \left( \mathbb{Z} \right)$ inversible et d'inverse dans $\mathcal{M}_2 \left( \mathbb{Z} \right)$ tel que $MP = \left( \begin{matrix} 1 & 0 \\ c' & d' \end{matrix} \right)$ avec $(c',d') \in \mathbb{Z}^2$.
\subparagraph{3}Comment obtenir une matrice échelonnée selon les colonnes dans $\mathbb{Z}$ ?
\subparagraph{4}Appliquer les conclusions de la question précédente à l'exemple de la première question.

\section{Calcul d'un inverse}
Soit $A = \left( \begin{matrix} 1 & 1 & \dots & 1 \\ 0 & 1 & \dots & 1 \\ \vdots & \ddots & \ddots & \vdots \\ 0 & \dots & \dots & 1 \end{matrix} \right) \in \mathcal{M}_n \left( \mathbb{R} \right)$.
\subparagraph{1}Montrer que $A$ est inversible et donner $A^{-1}$.
\subparagraph{2}Soit $k \in \mathbb{N}^*$. Calculer $A^{-k}$.
\subparagraph{3}Conjecturer $A^k$ et démontrer cette conjecture par récurrence.

\section{Trace et forme linéaire}
Soit $f$ une forme linéaire de $\mathcal{M}_n \left( \mathbb{R} \right)$.
\subparagraph{1}Soit $(i,j,k,l) \in \llbracket 1,n\rrbracket^4$. Que vaut $E_{i,j}E_{k,l}$ ?
\subparagraph{2}Montrer qu'il existe $A \in \mathcal{M}_n\left( \mathbb{R} \right)$ tel que $\forall M \in\mathcal{M}_n \left( \mathbb{R} \right), \ f(M) = \text{tr} (AM)$.
\subparagraph{Remarque :} l'année prochaine vous verrez qu'on peut utiliser un théorème plus fort pour conclure directement. Il s'agit d'une simple application du théorème de représentation de Riesz (il convient de remarquer que la trace est un produit scalaire sur l'espace des matrices). Néanmoins, l'utilisation de ce théorème nous empêche de choisir des matrices sur des corps finis...

\section{Racines réelles de polynôme (1)}
Soit $(a,b) \in \mathbb{R}^2$, $n \in \mathbb{N}$.
\subparagraph{1}Montrer que le polynôme $X^n+aX+b$ admet au plus trois racines réelles.

\section{Racines réelles de polynômes (2)}
\subparagraph{1}Montrer que $P_n=((1-X^2)^n)^{(n)}$ est un polynôme de degré $n$ dont les racines sont réelles, simples et appartiennent à $[-1,1]$.

\section{Théorème des accroissements finis (1)}
Soit $(\alpha,\beta,\gamma) \in \mathbb{R}^3$. De même, soit $(a,b) \in \mathbb{R}^2$
\subparagraph{1}Rappeler et démontrer le théorème des accroissements finis.
\subparagraph{2}Soit $f \ : \ \mathbb{R} \ \rightarrow \ \mathbb{R}$, définie par $f(x)=\alpha x^2+\beta x +\gamma$. Déterminer le point $"c"$ du théorème des accroissements finis.
\subparagraph{3}Interpréter géométriquement ce résultat.

\section{Théorème des accroissements finis (2)}
Soit $f: \ [a,b] \ \rightarrow \ \mathbb{R}$ et $g: \ [a,b] \ \rightarrow \ \mathbb{R}$ dérivables sur $]a,b[$, continues sur $[a,b]$, et telles que $g'$ ne s'annule pas sur $]a,b[$.
\subparagraph{1}Montrer que $g(b) \neq g(a)$.
\subparagraph{2}En s'inspirant de la preuve du théorème des accroissements finis, montrer que : $\exists c \in ]a,b[, \ \frac{f(b)-f(a)}{g(b)-g(a)}=\frac{f'(c)}{g'(c)}$.
\subparagraph{3}On suppose que $\underset{x \rightarrow b}{\lim}\frac{f'(x)}{g'(x)}$ existe. Montrer que $\underset{x \rightarrow a}{\lim}\frac{f(x)-f(a)}{g(x)-g(a)}=\underset{x \rightarrow b}{\lim}\frac{f'(x)}{g'(x)}$.
\subparagraph{4}En déduire $\underset{x \rightarrow 1^-}{\lim} \frac{\arccos(x)}{\sqrt{1-x^2}}$.
\subparagraph{Remarque :} ce théorème est une généralisation du théorème des accroissements finis. On peut l'interpréter de la même manière que le théorème des accroissements finis classique mais pour une courbe paramétrée du plan.
 
\section{Théorème de Rolle généralisé}
Le but est de montrer le théorème de Rolle généralisé. Soit $f: \ [a,+\infty[ \ \rightarrow \ \mathbb{R}$, continue sur $[a,+\infty[$, dérivable sur $]a,+\infty[$. On suppose de plus que $f$ possède une limite en $+\infty$ et que celle-ci vaut $f(a)$. On peut alors dire : $\exists c \in ]a,+\infty[, \ f'(c)=0$.
\subparagraph{1}On définit $g : [a,a+\frac{\pi}{2}[ \ \rightarrow \ \mathbb{R}$ par $g(x)=f(\tan(x-a))$. Montrer que l'on peut prolonger $g$ par continuité en $a+\frac{\pi}{2}$.
\subparagraph{2}Appliquer le théorème de Rolle à $g$ et conclure sur le théorème de Rolle généralisé.
\subparagraph{3}Montrer via le théorème de Rolle généralisé que $x \mapsto \frac{\ln(x)}{x}$ voit sa dérivée s'annuler au moins une fois sur $]1,+\infty[$.

\section{Une propriété du logarithme}
Soit $(x,y)\in \mathbb{R}^2$, avec $0<x<y$.
\subparagraph{1}Montrer que $x < \frac{y-x}{\ln(y)-\ln(x)} <y$.
\subparagraph{2}Considérer la fonction $f:[0,1] \ \rightarrow \ \mathbb{R}$, telle que $f(\alpha)=\ln(\alpha y+(1-\alpha)x))-\alpha \ln(y) -(1-\alpha) \ln(x)$. Montrer que $f$ est positive.
\subparagraph{3}Interpréter géométriquement cette inégalité.

\section{Une égalité polynômiale}
Soit $(a,b,c) \in \mathbb{R}^3$. 
\subparagraph{1}Montrer qu'il existe $x \in ]0,1[$ tel que $4ax^3+3bx^2+2cx=a+b+c$. 

\section{Variante du théorème de Rolle}
Soit $I$ un intervalle de $\mathbb{R}$ tel que $[a,b] \subset I$ et $f$ dérivable sur $I$. On suppose que $f'(a)>0$ et $f'(b)<0$.
\subparagraph{1}Montrer qu'il existe $c \in ]a,b[$ tel que $f'(c) =0$.

\section{Dérivabilité et annulation}
Soit $n \in \mathbb{N}^*$ et $f \in \mathcal{C}\left( \mathbb{R}^n \right)$. On suppose que la fonction s'annule en $n+1$ points distincts. Soit $\alpha \in \mathbb{R}$.
\subparagraph{1}Montrer que la dérivée n-ième s'annule au moins une fois.
\subparagraph{2}Montrer que la dérivée $n-1$-ème de $f+\alpha f'$ s'annule au moins une fois.

\section{Trois points, deux dérivées}
Soit $(a,b,c) \in \mathbb{R}$ trois points distincts. Soit $f$ deux fois dérivable sur $\mathbb{R}$. \subparagraph{1} Montrer qu'il existe $d \in \mathbb{R}$ tel que :
$$ \frac{f(a)}{(a-b)(a-c)}+\frac{f(b)}{(b-a)(b-c)}+\frac{f(c)}{(c-a)(c-b)}=\frac{1}{2}f''(d)$$
\section{Cas d'égalité avec dérivée}
Soit $I$ un intervalle de $\mathbb{R}$ tel que $[a,b] \subset I$ et $f$ dérivable sur $I$. On suppose que $f'(a)=f(a)$ et $f'(b)=f(b)$.
\subparagraph{1}Montrer qu'il existe $c \in ]a,b[$ tel que $f''(c)=f(c)$.

\section{Développements limités et dérivabilité}
\subparagraph{1}Montrer que $f$ est continue en 0 si et seulement si $f$ admet un développement limité d'ordre 0 en 0.
\subparagraph{2}Montrer que $f$ est dérivable en 0 si et seulement si $f$ admet un développement limité d'ordre 1 en 0.
\subparagraph{3}Montrer que si $f$ est deux fois dérivable en 0 alors $f$ admet un développement limité d'ordre 2 en 0.
\subparagraph{4}Montrer que $x \mapsto x^3 \sin(\frac{1}{x})$ définie sur $\mathbb{R}^*$ et prolongée par continuité en 0 admet un développement limité à l'ordre 2 mais n'est pas 2 fois dérivable en 0.

\end{document}