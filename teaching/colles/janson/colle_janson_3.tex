\documentclass[10pt,a4paper]{article} 
\usepackage[utf8]{inputenc} 
\usepackage[T1]{fontenc} 
\usepackage[english]{babel} 
\usepackage{supertabular} %Nécessaire pour les longs tableaux
\usepackage[top=2.5cm, bottom=2.5cm, right=1.5cm, left=1.5cm]{geometry} %Mise en page 
\usepackage{amsmath} %Nécessaire pour les maths 
\usepackage{amssymb} %Nécessaire pour les maths 
\usepackage{stmaryrd} %Utilisation des double crochets 
\usepackage{pifont} %Utilisation des chiffres entourés 
\usepackage{graphicx} %Introduction d images 
\usepackage{epstopdf} %Utilisation des images .eps 
\usepackage{amsthm} %Nécessaire pour créer des théorèmes 
\usepackage{algorithmic} %Nécessaire pour écrire des algorithmes 
\usepackage{algorithm} %Idem 
\usepackage{bbold} %Nécessaire pour pouvoir écrire des indicatrices 
\usepackage{hyperref} %Nécessaire pour écrire des liens externes 
\usepackage{array} %Nécessaire pour faire des tableaux 
\usepackage{tabularx} %Nécessaire pour faire de longs tableaux 
\usepackage{caption} %Nécesaire pour mettre des titres aux tableaux (tabular) 
\usepackage{color} %nécessaire pour écrire en couleur 
\newtheorem{thm}{Théorème} 
\newtheorem{mydef}{Définition} 
\newtheorem{prop}{Proposition} 
\newtheorem{lemma}{Lemme}
\title{Semaine 3 - Généralités sur les fonctions à une variable complexe ou réelle}
\author{Valentin De Bortoli \\ email : \ \href{mailto:valentin.debortoli@gmail.com}{valentin.debortoli@gmail.com}}
\date{}
\begin{document}
\maketitle

\section{Une équation fonctionnelle (1)}
Soit $f$ une fonction continue qui va de $\mathbb{R}_{+}^{*}$ dans $\mathbb{R}$ telle que $\forall x \in \mathbb{R}_{+}^{*}$, $f(x^2)=f(x)$.
\subparagraph{1}Montrer que $f$ est constante.

\section{Une équation fonctionnelle (2)}
\subparagraph{1}Trouver toutes les fonctions continues de $\mathbb{R}$ dans $\mathbb{R}$ qui vérifient : $\forall (x,y) \in \mathbb{R}^2, \ f(x+y)=f(x)+f(y)$.
\subparagraph{Remarque :} on admettra que pour tout réel, il existe une suite de rationnels qui tend vers ce réel. Cette propriété s'appelle la densité de $\mathbb{Q}$ dans $\mathbb{R}$. On pourra commencer par trouver la forme de $f$ sur $\mathbb{N}$, puis $\mathbb{Z}$, puis $\mathbb{Q}$ et enfin $\mathbb{R}$.

\section{Une équation fonctionnelle (3)}
Soit $f$ une fonction dérivable, de dérivée continue, de $\mathbb{R}$ dans $\mathbb{R}$ qui vérifie : $\forall x \in \mathbb{R}, \ f(f(x))=\frac{x}{2}+3$.
\subparagraph{1}Montrer que $f(\frac{x}{2}+3)=\frac{f(x)}{2}+3$.
\subparagraph{2}Montrer que $f'$ vérifie une équation fonctionnelle.
\subparagraph{3}Montrer que $f'$ est constante.
\subparagraph{4}Déterminer $f$.

\section{Composition, injectivité et surjectivité}
Soit $f$ une fonction de $\mathbb{R}$ dans $\mathbb{R}$ qui vérifie $\forall x \in \mathbb{R}, \ f(f(f(x)))=f(x)$.
\subparagraph{1} Montrer que on a $f$ injective $\Leftrightarrow$ $f$ surjective $\Leftrightarrow$ $f$ bijective.
\subparagraph{2} Exprimer alors $f^{-1}$ en fonction de $f$.

\section{Théorème de Cantor-Bernstein}
Soit $A$ et $B$ deux ensembles. Le but de cet exercice est de montrer que si il existe une injection ($f_1$) de $A$ dans $B$ et une injection ($f_2$) de $B$ dans $A$ alors il existe une bijection entre $A$ et $B$. L'exercice se déroule en deux parties. Premièrement on va montrer que si $C$ est une partie de $A$ et $f$ une injection de $A$ dans $C$, alors $A$ et $C$ sont en bijection. Ensuite on montrera le théorème. On pose :
\begin{equation*}
\left\{
\begin{aligned}
&D_0={}^c C \\
&D_{n+1}=f(D_n) \ \text{pour} \ n \in \mathbb{N}^{*}
\end{aligned}\right.
\end{equation*}
\begin{equation*}
D=\cup_{n=0}^{+\infty} D_n
\end{equation*}
\subparagraph{1}Montrer que $f(D) \subset C \cap D$.
\subparagraph{2}On pose $g$ de $A$ dans $C$ telle que :
\begin{equation*}
\left\{
\begin{aligned}
g(x)=f(x) \ \text{si} \ x \in D \\
g(x)=x \ \text{si} \ x \in {}^cD
\end{aligned}\right.
\end{equation*}
Montrer que $g$ est injective.
\subparagraph{3}Montrer que $g$ est bijective et conclure pour la première partie.
\subparagraph{4}En considérant $f_1 \circ f_2$, montrer le théorème de Cantor-Bernstein.

\section{Union, intersection, image et image réciproque}
Soit $f$ une fonction de $\mathbb{R}$ dans $\mathbb{R}$ et $A$ et $B$ deux ensembles de $\mathbb{R}$.
\subparagraph{1} Parmi les assertions suivantes seules trois sont vraies, lesquelles ? Lorsqu'une assertion est vraie, la démontrer, lorsqu'elle est fausse, donner un contre-exemple et établir une condition pour qu'elle devienne vraie.
\begin{itemize}
\item $f(A \cup B) \subset f(A) \cup f(B)$
\item $f(A \cup B) = f(A) \cup f(B)$
\item $f(A \cap B) \subset f(A) \cap f(B)$
\item $f(A \cap B) = f(A) \cap f(B)$
\end{itemize}
\subparagraph{2}En déduire que $ A \cup B\subset f^{-1}(f(A) \cup f(B))$. Trouver un contre-exemple à l'égalité.

\section{Système et logarithme népérien}
\subparagraph{1}Factoriser le polynôme suivant : $X^2-2\text{ln}(3)X+1$. On précise que $\text{ln}(3) \approx 1,09$.
\subparagraph{2}Résoudre dans $\mathbb{R}^2$ le système suivant :
\begin{equation*}
\left\{
\begin{aligned}
e^xe^y=9 \\
xy=1
\end{aligned}\right.
\end{equation*}
\subparagraph{Remarque :} on pourra considérer $Q$, un polynôme unitaire de degré 2, qui a pour racines $x$ et $y$.

\section{Fonction trigonométrique réciproque}
\subparagraph{1}Montrer que $\sin$ est une bijection de $[-\frac{\pi}{2},\frac{\pi}{2}]$ dans $[-1,1]$.
\subparagraph{2}On appelle arcsinus et on note $\arcsin$ sa réciproque. Est-elle dérivable ? Si oui, déterminer sa dérivée.
\subparagraph{3}Tracer cette fonction.
\subparagraph{4}Reprendre la même étude pour $\cos$.

\section{Fonction trigonométrique hyperbolique et réciproque}
\subparagraph{1}On appelle cosinus hyperbolique et on note $\cosh$ la fonction qui va de $\mathbb{R}$ dans $[1,+\infty[$ et qui a pour expression $\cosh(x)=\frac{e^x+e^{-x}}{2}$. Faire une étude détaillée de cette fonction. Justifier pourquoi on appelle cette fonction "cosinus hyperbolique".
\subparagraph{2}Montrer que $\cosh$ est une bijection de $\mathbb{R}_{+}$ dans $[1,+\infty[$. Donner sa réciproque. On l'appelle argument cosinus hyperbolique et on note $\text{argch}$. Faire une étude détaillée de cette fonction.
\subparagraph{Remarque :} on peut définir de la même manière le cosinus hyperbolique, la tangente hyperbolique et leurs réciproques. On peut, de même que pour la trigonométrie classique, établir un formulaire de trigonométrie hyperbolique.

\section{Quelques considérations sur l'exponentielle}
\subparagraph{1}Montrer que $\forall (n,x)\in \mathbb{N}^{*}\times \mathbb{R}_{+}, \ (1+\frac{x}{n})^n \le e^x$.
\subparagraph{2}Montrer que $\lim_{n \rightarrow +\infty} (1+\frac{x}{n})^n=e^x$

\section{Quelques considérations sur le logarithme}
\subparagraph{1}Faire l'étude de la fonction $f$ définie par l'expression $f(x)=\frac{\text{ln}(x)}{x}$.
\subparagraph{2}Trouver tous les couples d'entiers $(a,b)\in {\mathbb{N}^*}^2$ avec $a \neq b$ qui vérifient : $a^b=b^a$.

\end{document}