\documentclass[12pt]{article}
%\usepackage[frenchb]{babel}
\usepackage[T1]{fontenc}                % Accents codés dans la fonte.
\usepackage[utf8x]{inputenc}           % Accents 8 bits dans le fichier.
\usepackage{ifthen}                     % Faire des tests if/then/else.
\usepackage{amsmath}    % Les symboles les plus fréquents.
\usepackage{amsfonts}   % Des fontes, eg pour \mathbb.
\usepackage{vmargin}                    % Régler la taille de la feuille.
\usepackage{amssymb}
\usepackage{lastpage}
\usepackage{accents}
\usepackage{enumitem}

% La taille de la page
\setpapersize{A4}
\setmarginsrb{20mm}{10mm}{14mm}{15mm}{10mm}{6mm}{0mm}{10mm}

% La mise en page
\newcounter{vcenterstest}
\newlength{\leftlength}
\newlength{\rightlength}
\newlength{\vcentersskip}
\newcommand{\leftcentersright}[4][2]{%
        \settowidth{\leftlength}{#2}%
        \settowidth{\rightlength}{#4}%
        \setcounter{vcenterstest}{#1}%
        \ifthenelse{\value{vcenterstest} = 0}
                {\setlength{\vcentersskip}{0pt}}{}%
        \ifthenelse{\value{vcenterstest} = 1}
                {\setlength{\vcentersskip}{\smallskipamount}}{}%
        \ifthenelse{\value{vcenterstest} = 2}
                {\setlength{\vcentersskip}{\medskipamount}}{}%
        \ifthenelse{\value{vcenterstest} = 3}
                {\setlength{\vcentersskip}{\bigskipamount}}{}%
        \ifthenelse{\value{vcenterstest} = 4}
                {\setlength{\vcentersskip}{1cm}}{}%
                % On laisse un espace vertical défini par l'argument
                % optionnel #1
        \vskip\vcentersskip
                % On place #2 et on recule de sa longueur
        \noindent#2\hskip-\leftlength%
                % On centre #3
        \hfill#3\hfill%
                % On va au bout de la ligne, on recule de la longueur de #4 et
                % on place #4
        \mbox{}\hskip-\rightlength#4%
                % On laisse un espace vertical défini par l'argument
                % optionnel #1
        \vskip\vcentersskip%
    }
\newcommand{\centers}[2][2]{\leftcentersright[#1]{}{#2}{}}
\newcommand{\leftcenters}[3][2]{\leftcentersright[#1]{#2}{#3}{}}
\newcommand{\centersright}[3][2]{\leftcentersright[#1]{}{#2}{#3}}
\newcommand{\leftencadreright}[4][2]{\leftcentersright[#1]{#2}{\fbox{#3}}{#4}}

\newcommand{\numtoi}[1]{
  \ifthenelse{ \equal{#1}{1} }{i}{
  \ifthenelse{ \equal{#1}{2} }{ii}{
  \ifthenelse{ \equal{#1}{3} }{iii}{
  \ifthenelse{ \equal{#1}{4} }{iv}{
  \ifthenelse{ \equal{#1}{5} }{v}{
  \ifthenelse{ \equal{#1}{6} }{vi}{
  \ifthenelse{ \equal{#1}{7} }{vii}{
  \ifthenelse{ \equal{#1}{8} }{viii}{
  \ifthenelse{ \equal{#1}{9} }{ix}{
  \ifthenelse{ \equal{#1}{10} }{x}{
  \ifthenelse{ \equal{#1}{11} }{xi}{
  \ifthenelse{ \equal{#1}{12} }{xii}{
  \ifthenelse{ \equal{#1}{13} }{xiii}{
  \ifthenelse{ \equal{#1}{14} }{xiv}{
  \ifthenelse{ \equal{#1}{15} }{xv}{
  \ifthenelse{ \equal{#1}{16} }{xvi}{
  \ifthenelse{ \equal{#1}{17} }{xvii}{
  \ifthenelse{ \equal{#1}{18} }{xviii}{
    ERREUR,~MODIFIER~LA~MACRO~numtoi
  } } } } } } } } } } } } } } } } } }
}

\newcounter{cretraiti}
\newenvironment{ri}[1]{

\begin{list}{($\numtoi{\thecretraiti}$)}{
  \usecounter{cretraiti}
  \topsep=0.5ex
  \itemsep=0.3ex
  \labelsep=0.3em
  \parsep=0ex
  \listparindent=1em
  \settowidth{\labelwidth}{(#1)}
  \leftmargin=\labelwidth
  %\addtolength{\leftmargin}{\labelsep}
}}{\end{list}
}


% Quelques raccourcis pour taper des maths
\newcommand{\Sfr}{\mathfrak{S}}
\newcommand{\Afr}{\mathfrak{A}}
\newcommand{\QQ}{\mathbb{Q}}
\newcommand{\ZZ}{\mathbb{Z}}
\newcommand{\si}{\sigma}
\newcommand{\ins}[1]{{\ensuremath{\textrm{#1}}}}
\newcommand{\tiret}{\rule[0.6ex]{1.3ex}{0.26ex}}
\newcommand{\accolades}[1]{\ensuremath{\left\{#1\right\}}}
\newcommand{\paa}[1]{\ensuremath{\accolades{#1}}}       % Synonyme
\newcommand{\ds}{\displaystyle}
   \newcommand{\der}{
      \mathop{}\mathopen{}\mathrm{d}
   }

% Numerotation des pages / la derniere
\makeatletter
\renewcommand{\@evenfoot}%
        {\thepage \slash \pageref{LastPage}}
\renewcommand{\@oddfoot}{\@evenfoot}
\makeatother


%%%%% Pour faire les exos
\newcommand{\envfont}{\sf\bfseries}
\newcounter{cexo}
\renewcommand{\thecexo}{\arabic{cexo}}

\newenvironment{exo}[1][toto]
{
  \refstepcounter{cexo}
  \ifthenelse{ \equal{#1}{toto} }{
   \noindent {\envfont Exercice \thecexo}
  }{
   \noindent {\envfont Exercice \thecexo{} \tiret\, #1.}
  }
%\newline
%\indent
}
{}

\newcounter{cquestions}
\newenvironment{questions}{
\begin{list}{{\envfont\alph{cquestions})}}{
  \usecounter{cquestions}
  \topsep=0.5ex
  \partopsep=1em
  \labelsep=0.3em
  \itemsep=0.3ex
  \settowidth{\labelwidth}{{\envfont b)}}\listparindent=1em
  \leftmargin=\labelwidth \addtolength{\leftmargin}{\labelsep}
}}{\end{list} }

\newcommand{\poubelle}[1]{}

%%%%%%%%%%%%%%%%%%%%%%%%%%%%%%%%%%%%%%%%%
\newcommand{\spacebeforeenv}{\vspace{1ex}}
\newcommand{\spaceafterenv}{\vspace{1ex}}
\newcommand{\myqedenv}{}


\newcommand{\metaenvironnement}[2]  % 1er argument : nom de l'env, 2e argument : titre
{
  \ifthenelse{ \equal{#2}{toto} }{
    \spacebeforeenv \noindent {\envfont #1 }
  }{
    \spacebeforeenv \noindent {\envfont #1  \tiret\, #2.}
  }
}

%----------------------------------
% Rappel
%----------------------------------
\newenvironment{rappel}[1][toto]
{ \metaenvironnement{Rappel}{#1} }
{\myqedenv\spaceafterenv}


\newcommand{\fonction}[5]{%
        \ensuremath{#1\colon
        \left\{\hskip -1.5 mm
        \begin{array}{c@{\ }c@{\ }l}
        \medskip #2 & \longrightarrow & #3 \\
        #4 & \longmapsto & #5 \\
        \end{array}
        \right .
        }}
\newcommand{\ph}{\varphi}
\newcommand{\funu}[1]{\mathcal{C}^{#1}}
\newcommand{\RR}{\mathbb{R}}
\newcommand{\NN}{\mathbb{N}}
\newcommand{\CC}{\mathbb{C}}

\newenvironment{syst}[1][c]%
    {\ensuremath{\left \{ \hskip -1.5 mm \begin{array}{#1@{\ =\ }l}}}%
    {\end{array}\right.}

\newcommand{\f}[2]{{\ensuremath{\mathchoice%
    {\dfrac{#1}{#2}}
        {\dfrac{#1}{#2}}
    {\frac{#1}{#2}}
    {\frac{#1}{#2}}
    }}}
\renewcommand{\ni}{\noindent}
\newcommand{\ra}{\rightarrow}
%\newcommand{\item}[1]{{\sf\bfseries #1}}
\newcommand{\norm}[1]{\ensuremath{\Arrowvert #1 \Arrowvert}}
\newcommand{\Ende}[3]{\ensuremath{\mathrm{End}_{#1}^{#2}(#3)}}
\newcommand{\End}[2]{\Ende{#1}{}{#2}}
\newcommand{\id}{\ensuremath{\mathrm{id}}}
\renewcommand{\d}{\ensuremath{\mathrm{d}}}
\renewcommand{\leq}{\leqslant}
\renewcommand{\geq}{\geqslant}


\usepackage{datetime}
\newdateformat{annscol}{\ifthenelse{\THEMONTH > 7}{\THEYEAR--\refstepcounter{YEAR}\THEYEAR}{\addtocounter{YEAR}{-1}\THEYEAR--\refstepcounter{YEAR}\THEYEAR}}

\begin{document}
 

%------------------------------------------------------------------------------------
\leftcentersright[0]{{\bf\sf\large Formation en Mathématiques Commune}}{}
    {{\bf\sf\large Année~\annscol\today}}
\leftcentersright[0]{{\bf\sf\large Calcul différentiel}}{}{{\bf\sf Valentin De Bortoli, Frédéric Pascal}}

\begin{center}
\huge{TD 2.2 \\ Différentielle d'ordre supérieur}
\end{center}
%------------------------------------------------------------------------------------------




%------------------------------------------------------------------------------------------
%------------------------------------------------------------------------------------------


\vskip2ex 
\begin{exo}\label{exo-cdif-exemple-contre-exemple}
\begin{questions}
\item Soit $f: \mathbb{R}^2 \longrightarrow \mathbb{R} $ l'application donnée par
  \begin{displaymath}
f(x,y)=
\left|
  \begin{array}{rc}
  
  
  x y \frac{ x^2 - y^2}{x^2 + y^2} & \mbox{si }  (x,y) \neq (0,0) \\
0 &  \mbox{si }(x,y) = (0,0)\\
  \end{array}
\right.
\end{displaymath}

      Montrer que les dérivées partielles 
      $\frac{\partial^2{f}}{\partial{x}\partial{y}}(0,0)$ et
      $\frac{\partial^2{f}}{\partial{y}\partial{x}}(0,0)$ existent mais ne sont pas égales.

 \end{questions}

\end{exo}

\begin{exo}\label{exo-cdif-diffmultiples}
\begin{questions}
\item  Soient $E,F$ deux espaces vectoriels de dimension finie, $U$ un ouvert de $E$, $f : U \ra F$ une application
différentiable, $a \in U$ et $v \in E$. Montrer que l'application
$$f_{a,v} : t \longmapsto f(a +tv)$$
 est définie sur un intervalle ouvert contenant $0$ et dérivable en tout point où elle est définie. Calculer
la dérivée de $f_{a,v}$ en $t = 0$. On suppose que $f$ est $k$-fois différentiable en $a$.
Calculer les dérivés successives de ${f_{a,v}}^{(k)}(0)$.

\item Soient $E,F$ deux espaces vectoriels de dimension finie, $U$ un ouvert de $E$ et $f : U \ra F$ une application
deux fois différentiable. Montrer que l'application $x \mapsto \der f(x)(x)$ est différentiable et calculer sa différentielle.

\item Soient $E,F$ deux espaces vectoriels de dimension finie, $U$ un ouvert de $E$, $f : U \ra F$ une application
de classe $C^{\infty}$, $a \in U$ et $v \in E$. Montrer que l'application $v \mapsto \der ^k f(a)(v, \ldots, v)$ est
différentiable et calculer sa différentielle.
  \end{questions}
\end{exo}
\vskip2ex

\begin{exo}
On se place dans le cas d'une fonction $f: U\rightarrow \RR$ où $U$ est un ouvert de $\RR^{n}$ et on fixe $a \in U$. Dans la suite, lorsque $f$ est suffisamment régulière, on adoptera la notation simplificatrice suivante :
%\begin{equation*}
$$ D^{k}f(a)(h)^{k}=D^{k}f(a)(h,h,...,h)$$
%\end{equation*}
\begin{questions}
\item  Récrire explicitement la précédente expression en utilisant les dérivées partielles $k^{e}$ de $f$.
\item On suppose pour cette question que $f$ est de classe $C^{k+1}$ sur $U$. Soit $h \in \RR^{n}$ tel que le segment $[a,a+h]$ soit inclus dans $U$. Montrer alors la formule de Taylor avec reste intégral en appliquant la formule correspondante du cas réel à une fonction bien choisie.
\begin{equation*}
 f(a+h)-f(a)-Df(a)(h)-...-\frac{1}{k!}D^{k}f(a)(h)^{k}=\int_{0}^{1} \frac{(1-t)^{k}}{k!} D^{k+1}f(a+th)(h)^{k+1} dt
\end{equation*}
\item  En supposant $f$ de classe $C^{k}$ sur $U$ et $h \in \RR^{n}$ tel que $[a,a+h]$ soit inclus dans $U$, montrer qu'il existe $\theta \in ]0,1[$ tel que :
\begin{equation*}
 f(a+h)-f(a)-Df(a)(h)-...-\frac{1}{(k-1)!}D^{k-1}f(a)(h)^{k-1}= \frac{1}{k!}D^{k}f(a+\theta h)(h)^{k}
\end{equation*}
Démontrer de même la formule de Taylor-Young à l'ordre $k$ pour une fonction de n variables.
\item Expliciter la formule de Taylor-Young à l'ordre 2 pour une fonction de trois variables puis celui à l'ordre 3 d'une fonction de deux variables.
\end{questions}
\end{exo}
\vskip2ex

\begin{exo}\textbf{Dérivée seconde et développement limité}
  Soit $f$ une fonction définie sur $U \subset E$ ($U$ ouvert) à valeurs dans $F$ avec $E$ et $F$ deux espaces vectoriels de dimension finie. Soit $a \in U$.
  \begin{questions}
\item Montrer que $f$ est continue $a$ si et seulement si elle admet un développement limité à l'ordre $0$ en $a$.
  De même, montrer que $f$ est dérivable en $a$ si et seulement si elle admet un développement limité à l'ordre $1$ en $a$.
\item Trouver une fonction $f$ qui soit $\mathcal{C}^1(U)$, qui admette un développement limité à l'ordre 2 et qui ne soit pas deux fois différentiable en $0$.
  \end{questions}
\end{exo}

\begin{exo} \textbf{Caractérisation des fonctions convexes}
  Soient $U$ un ouvert convexe d'un espace vectoriel normé $E$ et $f : U \ra \RR$ une application. On dit que 
$f$ est convexe si pour tous $x,y \in U$ et tout $t \in  [0,1]$, on a
$$f((1-t)x + ty) \leq (1-t)f(x) + tf(y)$$
$ie$ si le graphe de $f$ est sous ses cordes.  
\begin{questions}
\item  On suppose que $f$ est différentiable sur $U$. Montrer que $f$ est convexe si et seulement si 
$$f(y)-f(x) \geq \der f(x)(y-x)$$
 pour tous $x,y \in U$ ($ie$ si le graphe est au-dessus de ses tangentes).

\item On suppose que $f$ est deux fois différentiable sur $U$. Montrer que $f$ est convexe si et seulement si $\der ^2 f(x)$ est positive (en tant que forme quadratique) pour tout $x \in U$. 

\end{questions}
\end{exo}

\vskip2ex
\begin{exo}
\begin{questions}
\item Soit $f : \RR^n \ra \RR$ une fonction de classe $C^{\infty}$. Montrer qu'il existe des fonctions
$g_i : \RR^n \ra \RR$ de classe $C^{\infty}$ ($1 \leq i \leq n$) telles que
$\forall x \in \RR^n$ 
$$f(x) = f(0) + \sum_{i=1}^{n} x_i g_i(x)$$
 On souhaite remplacer $\RR^n$ par un ouvert $U$ de $\RR^n$, donner des hypothèses sur $U$ pour lesquelles
le résultat est toujours valable;
\item Soit $f : \RR^n \ra \RR$ une fonction de classe $C^{\infty}$ telle que $f(0) = 0$ et $\der f(0) = 0$.
Montrer qu'il existe des fonctions $h_{ij} : \RR^n \ra \RR$ de classe $C^{\infty}$ ($1 \leq i,j \leq n$) telles que
$\forall\,x \in \RR^n$, $$ f(x) = \sum_{i,j} x_ix_j h_{i,j}(x)$$
\vspace{-1cm}
\item Montrer que $I = \{f \in C^{\infty}(\RR^{n}, \RR), \quad f(0) = 0 \}$ est un idéal maximal de
$C^{\infty}(\RR^{n}, \RR)$ de type fini, principal pour $n=1$.
\end{questions}
\end{exo}



\begin{exo}\textbf{Théorème de Whitney}
Soit F un fermé de $\RR^{n}$. On va montrer qu'il existe une fonction $f$ de classe $C^{\infty}$ de $\RR^{n}$ dans $\RR^{+}$ dont l'ensemble des zéros est exactement F. On supposera dans la suite que $F \neq \RR^{n}$.
\begin{questions}
\item  Donner l'exemple d'une fonction de classe $C^{\infty}$ strictement positive à l'intérieur de la boule unité et nulle partout en dehors.
\item Justifier l'existence d'une suite de boules ouvertes $(B_{i})_{i \in \NN}$ telles que $F=\RR^{n}\backslash \bigcup_{i \in \NN} B_{i}$ ainsi que d'une suite d'applications $(\phi_{i})_{i \in \NN}$ de $\RR^{n}$ dans $\RR^{+}$ telle que chaque $\phi_{i}$ s'annule uniquement sur $\RR^{n}\backslash B_{i}$.
\item Déterminer alors une fonction $f$ satisfaisant au problème. \\
\end{questions}
\end{exo}






\end{document}
%%% Local Variables:
%%% mode: latex
%%% TeX-master: t
%%% End:
