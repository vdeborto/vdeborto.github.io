\documentclass[10pt,a4paper]{article}
\input{header}
\begin{document}
Soit $E$ et $F$ deux espaces vectoriels normés.
Soit $U$ un ouvert de $E$.
Soit $f$ différentiable de $U$ dans $F$.
\begin{thm}[Accroissements finis]
  Soit $[x,y] \subset U$. On suppose que $df$ est majorée par $M\in\R_+$ sur  $[x,y]$.
  \[\|f(x) -f(y)| \le M \|x-y\|\]
\end{thm}
\remark{
  Il est remarquable qu'une telle propriété persiste même en dehors de la dimension un.
  En effet en dimension un la preuve se base sur le théorème de Rolle et l'égalité des accroissements finis. Pour étendre l'inégalité des accroissements finis à la dimension finie on peut appliquer l'égalité des accroissements finis sur chaque composante et majorer. En dimension infinie cela n'est plus possible.
  }
  \begin{proof}

    Soit $\epsilon \in \R_+$. On définit
    \[F_{\epsilon} =  \lbrace z \in [x,y], \ \| f(z) - f(x) \| \le (M+\epsilon)\|z-x\| \rbrace \]
    On définit également \[F_{\epsilon,+} = \lbrace z \in [x,y], \ \forall z' \in [x,z],  z' \in F_{\epsilon} \rbrace\]
    Par continuité de $f$ , $F_{\epsilon}$ est fermé. De même, $F_{\epsilon,+}$ est fermé. En effet soit $\seq{z}{n} \in F_{\epsilon,+}^{\mathbb{N}}$ qui converge vers $z$. Soit $z' \in [x,y]$ tel que $\|z'-x\| \le \|z-x\|$. Il existe $n_0$ tel que pour tout $n \ge n_0$ on a $\|z_n-z\| \le \|z-z'\|$. On a forcément $\|z-x\| \le \|z-z'\|$. Donc $\|z'-x\| = \|z-x\| - \|z-z'\| \le \|z_n -x \|$. Donc $z' \in F_{\epsilon}$.\\
    On va maintenant mon que $F_{\epsilon,+}$ est ouvert. Soit $z \in F_{\epsilon}$ et on pose $z'\in [z,z+\eta(y-x)]$ avec $\eta \in \R_+$. On a
    \[ \al{\|f(z') - f(x) \| &\le \| f(z) - f(x) \| + \| f(z') - f(z) \|\\
        &\le (M+\epsilon)\|z-x\| + (M+\epsilon)\|z'-z|\\
        &\le (M+\epsilon) \|z'-x|
      }\]
    Donc $F_{\epsilon,+}$ est ouvert. De plus $[x,y]$ est connexe (non vide $x \in F_{\epsilon}$) donc $[x,y] \subset F_{\epsilon}$.
    Ce résultat est valable pour tout $\epsilon$ donc on peut faire tendre $\epsilon$ vers $0$ et on obtient l'inégalité des accroissements finis.
\end{proof}
\end{document}
%%% Local Variables:
%%% mode: latex
%%% TeX-master: t
%%% End:
