\documentclass[12pt]{article}
%\usepackage[frenchb]{babel}
\usepackage[T1]{fontenc}                % Accents codés dans la fonte.
\usepackage[utf8x]{inputenc}           % Accents 8 bits dans le fichier.
\usepackage{ifthen}                     % Faire des tests if/then/else.
\usepackage{amsmath}    % Les symboles les plus fréquents.
\usepackage{amsfonts}   % Des fontes, eg pour \mathbb.
\usepackage{vmargin}                    % Régler la taille de la feuille.
\usepackage{amssymb}
\usepackage{lastpage}
\usepackage{accents}
\usepackage{enumitem}

% La taille de la page
\setpapersize{A4}
\setmarginsrb{20mm}{10mm}{14mm}{15mm}{10mm}{6mm}{0mm}{10mm}

% La mise en page
\newcounter{vcenterstest}
\newlength{\leftlength}
\newlength{\rightlength}
\newlength{\vcentersskip}
\newcommand{\leftcentersright}[4][2]{%
        \settowidth{\leftlength}{#2}%
        \settowidth{\rightlength}{#4}%
        \setcounter{vcenterstest}{#1}%
        \ifthenelse{\value{vcenterstest} = 0}
                {\setlength{\vcentersskip}{0pt}}{}%
        \ifthenelse{\value{vcenterstest} = 1}
                {\setlength{\vcentersskip}{\smallskipamount}}{}%
        \ifthenelse{\value{vcenterstest} = 2}
                {\setlength{\vcentersskip}{\medskipamount}}{}%
        \ifthenelse{\value{vcenterstest} = 3}
                {\setlength{\vcentersskip}{\bigskipamount}}{}%
        \ifthenelse{\value{vcenterstest} = 4}
                {\setlength{\vcentersskip}{1cm}}{}%
                % On laisse un espace vertical défini par l'argument
                % optionnel #1
        \vskip\vcentersskip
                % On place #2 et on recule de sa longueur
        \noindent#2\hskip-\leftlength%
                % On centre #3
        \hfill#3\hfill%
                % On va au bout de la ligne, on recule de la longueur de #4 et
                % on place #4
        \mbox{}\hskip-\rightlength#4%
                % On laisse un espace vertical défini par l'argument
                % optionnel #1
        \vskip\vcentersskip%
    }
\newcommand{\centers}[2][2]{\leftcentersright[#1]{}{#2}{}}
\newcommand{\leftcenters}[3][2]{\leftcentersright[#1]{#2}{#3}{}}
\newcommand{\centersright}[3][2]{\leftcentersright[#1]{}{#2}{#3}}
\newcommand{\leftencadreright}[4][2]{\leftcentersright[#1]{#2}{\fbox{#3}}{#4}}

\newcommand{\numtoi}[1]{
  \ifthenelse{ \equal{#1}{1} }{i}{
  \ifthenelse{ \equal{#1}{2} }{ii}{
  \ifthenelse{ \equal{#1}{3} }{iii}{
  \ifthenelse{ \equal{#1}{4} }{iv}{
  \ifthenelse{ \equal{#1}{5} }{v}{
  \ifthenelse{ \equal{#1}{6} }{vi}{
  \ifthenelse{ \equal{#1}{7} }{vii}{
  \ifthenelse{ \equal{#1}{8} }{viii}{
  \ifthenelse{ \equal{#1}{9} }{ix}{
  \ifthenelse{ \equal{#1}{10} }{x}{
  \ifthenelse{ \equal{#1}{11} }{xi}{
  \ifthenelse{ \equal{#1}{12} }{xii}{
  \ifthenelse{ \equal{#1}{13} }{xiii}{
  \ifthenelse{ \equal{#1}{14} }{xiv}{
  \ifthenelse{ \equal{#1}{15} }{xv}{
  \ifthenelse{ \equal{#1}{16} }{xvi}{
  \ifthenelse{ \equal{#1}{17} }{xvii}{
  \ifthenelse{ \equal{#1}{18} }{xviii}{
    ERREUR,~MODIFIER~LA~MACRO~numtoi
  } } } } } } } } } } } } } } } } } }
}

\newcounter{cretraiti}
\newenvironment{ri}[1]{

\begin{list}{($\numtoi{\thecretraiti}$)}{
  \usecounter{cretraiti}
  \topsep=0.5ex
  \itemsep=0.3ex
  \labelsep=0.3em
  \parsep=0ex
  \listparindent=1em
  \settowidth{\labelwidth}{(#1)}
  \leftmargin=\labelwidth
  %\addtolength{\leftmargin}{\labelsep}
}}{\end{list}
}


% Quelques raccourcis pour taper des maths
\newcommand{\Sfr}{\mathfrak{S}}
\newcommand{\Afr}{\mathfrak{A}}
\newcommand{\QQ}{\mathbb{Q}}
\newcommand{\ZZ}{\mathbb{Z}}
\newcommand{\si}{\sigma}
\newcommand{\ins}[1]{{\ensuremath{\textrm{#1}}}}
\newcommand{\tiret}{\rule[0.6ex]{1.3ex}{0.26ex}}
\newcommand{\accolades}[1]{\ensuremath{\left\{#1\right\}}}
\newcommand{\paa}[1]{\ensuremath{\accolades{#1}}}       % Synonyme
\newcommand{\ds}{\displaystyle}
   \newcommand{\der}{
      \mathop{}\mathopen{}\mathrm{d}
   }

% Numerotation des pages / la derniere
\makeatletter
\renewcommand{\@evenfoot}%
        {\thepage \slash \pageref{LastPage}}
\renewcommand{\@oddfoot}{\@evenfoot}
\makeatother


%%%%% Pour faire les exos
\newcommand{\envfont}{\sf\bfseries}
\newcounter{cexo}
\renewcommand{\thecexo}{\arabic{cexo}}

\newenvironment{exo}[1][toto]
{
  \refstepcounter{cexo}
  \ifthenelse{ \equal{#1}{toto} }{
   \noindent {\envfont Exercice \thecexo}
  }{
   \noindent {\envfont Exercice \thecexo{} \tiret\, #1.}
  }
%\newline
%\indent
}
{}

\newcounter{cquestions}
\newenvironment{questions}{
\begin{list}{{\envfont\alph{cquestions})}}{
  \usecounter{cquestions}
  \topsep=0.5ex
  \partopsep=1em
  \labelsep=0.3em
  \itemsep=0.3ex
  \settowidth{\labelwidth}{{\envfont b)}}\listparindent=1em
  \leftmargin=\labelwidth \addtolength{\leftmargin}{\labelsep}
}}{\end{list} }

\newcommand{\poubelle}[1]{}

%%%%%%%%%%%%%%%%%%%%%%%%%%%%%%%%%%%%%%%%%
\newcommand{\spacebeforeenv}{\vspace{1ex}}
\newcommand{\spaceafterenv}{\vspace{1ex}}
\newcommand{\myqedenv}{}


\newcommand{\metaenvironnement}[2]  % 1er argument : nom de l'env, 2e argument : titre
{
  \ifthenelse{ \equal{#2}{toto} }{
    \spacebeforeenv \noindent {\envfont #1 }
  }{
    \spacebeforeenv \noindent {\envfont #1  \tiret\, #2.}
  }
}

%----------------------------------
% Rappel
%----------------------------------
\newenvironment{rappel}[1][toto]
{ \metaenvironnement{Rappel}{#1} }
{\myqedenv\spaceafterenv}


\newcommand{\fonction}[5]{%
        \ensuremath{#1\colon
        \left\{\hskip -1.5 mm
        \begin{array}{c@{\ }c@{\ }l}
        \medskip #2 & \longrightarrow & #3 \\
        #4 & \longmapsto & #5 \\
        \end{array}
        \right .
        }}
\newcommand{\ph}{\varphi}
\newcommand{\funu}[1]{\mathcal{C}^{#1}}
\newcommand{\RR}{\mathbb{R}}
\newcommand{\NN}{\mathbb{N}}
\newcommand{\CC}{\mathbb{C}}

\newenvironment{syst}[1][c]%
    {\ensuremath{\left \{ \hskip -1.5 mm \begin{array}{#1@{\ =\ }l}}}%
    {\end{array}\right.}

\newcommand{\f}[2]{{\ensuremath{\mathchoice%
    {\dfrac{#1}{#2}}
        {\dfrac{#1}{#2}}
    {\frac{#1}{#2}}
    {\frac{#1}{#2}}
    }}}
\renewcommand{\ni}{\noindent}
\newcommand{\ra}{\rightarrow}
\newcommand{\question}[1]{{\sf\bfseries #1}}
\newcommand{\norm}[1]{\ensuremath{\Arrowvert #1 \Arrowvert}}
\newcommand{\Ende}[3]{\ensuremath{\mathrm{End}_{#1}^{#2}(#3)}}
\newcommand{\End}[2]{\Ende{#1}{}{#2}}
\newcommand{\id}{\ensuremath{\mathrm{id}}}
\renewcommand{\d}{\ensuremath{\mathrm{d}}}
\renewcommand{\leq}{\leqslant}
\renewcommand{\geq}{\geqslant}


\usepackage{datetime}
\newdateformat{annscol}{\ifthenelse{\THEMONTH > 7}{\THEYEAR--\refstepcounter{YEAR}\THEYEAR}{\addtocounter{YEAR}{-1}\THEYEAR--\refstepcounter{YEAR}\THEYEAR}}

\begin{document}
 

%------------------------------------------------------------------------------------
\leftcentersright[0]{{\bf\sf\large Formation en Mathématiques Commune Cachan / P7}}{}
    {{\bf\sf\large Année~\annscol\today}}
\leftcentersright[0]{{\bf\sf\large Calcul différentiel}}{}{{\bf\sf Mariano Rodríguez, Frédéric Pascal}}

%------------------------------------------------------------------------------------------




%------------------------------------------------------------------------------------------
%------------------------------------------------------------------------------------------


\begin{center}
\huge{TD 5 \\ Théorèmes d'inversion locale et des fonctions implicites}
\end{center}

\vspace{0.5cm}
\begin{exo}
Soit $\mathcal{S}_{rs}$ l'ensemble des polynômes de $\mathbb{R}_n\left[X\right]$ scindés à n racines simples. Montrer que $\mathcal{S}_{rs}$ est un ouvert de $\mathbb{R}_n\left[X\right]$.
\end{exo}


\vspace{0.5cm}

\begin{exo}
Soient $A_0 \in Mn (\RR)$ et $\lambda_0 \in \RR$ une valeur propre de multiplicité dans le polynôme caractéristique égale à 1. Montrer qu'il existe une fonction " valeur propre " de classe $C^{1}$ dans un voisinage de $A_0$. De façon précise, montrer qu'il existe un voisinage $V$ de $A_0$ dans $Mn (\RR)$, un voisinage $W$ de $\lambda_0$ et une fonction $vp : V \longrightarrow W$ tel que $vp(A)$ soit une valeur propre de $A$ pour tout $A \in V$.
\end{exo}


\poubelle{
\vspace{0.5cm}
\begin{exo}
Soit $\Omega$ un ouvert de $\RR ^m$ et $f : \Omega \ra \RR ^n $ une application de classe $C^1$ localement injective ($ie$ $\forall a \in \Omega$ il existe un voisinage de $a$ sur lequel $f$ est injective). On pose $X = \{ x \in \Omega | \der f_x \texttt{ est injective } \}$. Montrer que $X$ est ouvert dense de $\Omega$.

\end{exo}
}

\vspace{0.5cm}

\begin{exo}
Soit $\Omega$  un ouvert convexe de $\RR ^n$ et $f : \Omega \ra \RR ^n$ une application de classe $C^1$ telle que pour tout $x \in \Omega$, $\der f_x \in O_n (\RR)$. On va montrer que $f$ est en fait une isométrie affine.

\begin{questions}
\item  Justifier qu'il suffit de prouver que $\der f$ est localement constante sur $\Omega$.
\item Si $x_0 \in \Omega$, montrer que pour $x$ assez proche de $x_0$ on a $$ \Vert  f(x) - f(x_0)  \Vert = \Vert  x - x_0 \Vert  $$
\item Soit $g$ continue de $[0 , 1]$ à valeurs dans la boule unité de $\RR ^n$. Soit $I = \int_0 ^1 g(t) dt$. Montrer que si $\Vert  I \Vert   = 1$ alors $g$ est constante égale à $I$ en tout $t$.
\item En déduire que $\der f$ est localement une isométrie affine. Conclure
\end{questions}
\end{exo}



\vspace{0.5cm}
\begin{exo}
Soit $S_n (\RR) \subset M_n (\RR)$ le sous-espace des matrices symétriques d'ordre $n$. On rappelle qu'une forme quadratique $Q$ sur $\RR ^n$ (identifiée à une matrice de $S_n (\RR)$ dans une base fixée) est dite de signature $(p,q)$ avec $p , q \in \NN$, $p+q \leq n$ s'il existe une matrice inversible $B$ telle que  
$$ \forall x \in \RR ^n , (Bx)^T Q (Bx) = x_1^2 + ... + x_p^2 - x_{p+1}^2 - ... - x_{p+q}^2 $$

\begin{questions}
\item Soit $A_0 \in S_n (\RR)$ inversible. Soit $\Phi$ de $M_n (\RR)$ dans $S_n (\RR)$ telle que $\Phi(M) = M^T A_0 M$. Montrer que $\der \Phi (I_n)$ est surjective et préciser son noyau.
\item En considérant la restriction de $\Phi$ au sous-espace $A_0 ^{-1} S_n (\RR)$  montrer qu'il existe un voisinage $V$ de $A_0$ dans $S_n (\RR)$ et une application de classe $C^1$ de $V$ à valeurs dans $GL_n (\RR)$ qui à $A$ associe $M$ telle que $\forall A \in V$, $A = M^{T} A_0 M$.
\item Soit maintenant $U$ un ouvert de $\RR ^n$ contenant $0$ et $f : U \ra \RR$ une application de classe $C^3$. On suppose que $ \der f (0) = 0$ et que la forme quadratique hessienne $\der ^2 f (0)$ est non dégénérée de signature $(p , n-p)$. Montrer que $f(x) - f(0) = x^T Q(x) x$ où $Q$ est une fonction de classe $C^1$ de $\RR ^n $ dans $S_n (\RR)$. 
\item Montrer qu'au voisinage de $0 \in \RR^n$ il existe une fonction $x \ra M(x)$ de classe $C^1$ à valeurs dans $GL_n (\RR)$ telle que $Q(x) = M(x)^T Q(0) M(x)$. En déduire l'existence d'un difféomorphisme : $ x \ra \Psi (x) = (\Psi _ 1 (x) , ... , \Psi _ n (x) ) $ entre deux voisinages de l'origine dans $\RR ^n$ tel que $\Psi(0) = 0$ et $f(x) - f(0) = \Psi _1 (x) ^2 + ... + \Psi _p (x) ^2 - \Psi _{p+1} (x) ^2 - ... -\Psi _n (x) ^2 $.
\end{questions}
\end{exo}





%exo supplementaire de barbara gris
\poubelle{
\vspace{0.5cm}

\begin{exo}
Soient $f: \RR \ra \RR$ et $g: \RR \ra \RR $ $C^1$, on pose
\begin{equation*}  
F :\left.
\begin{array}{ccc}
\RR ^2 & \ra & \RR ^2\\
(x,y) & \ra & (x+y, f(x) + g(y) )
\end{array}\right.
\end{equation*}
\begin{questions}
\item Montrer que $F$ est un $C^1$ difféomorphisme local $ssi$ il existe $a \in \RR$ tel tel que soit $\forall x \in \RR$ $f'(x) \leq a$ et $g'(x) \geq a$ soit $\forall x \in \RR$ $f'(x) \geq a$ et $g'(x) \leq a$ avec à chaque fois l'une des inégalités stricte.

\item On suppose cette condition vérifiée. Montrer que $F$ est un difféomorphisme global sur son image.

\end{questions}
\end{exo}
}

\vspace{0.5cm}
\begin{exo}
Soit $\mu \in \RR$. On considère le système différentiel
\begin{equation*}  
S_{\mu}\left\lbrace
\begin{array}{c}
\frac{\der x}{\der t} = y^2 - \cos x\\
\frac{\der y}{\der t} = - y \sin x - \mu y e^y
\end{array}\right.
\end{equation*}

\begin{questions}
\item Montrer que pour tout $0 \leq \mu < 1$ le système $S_{\mu}$ admet un unique point d'équilibre $C(\mu) = ( x_c (\mu , y_c (\mu))$ dans la bande $B = \{ (x,y) \in \RR ^2 | - \frac{\pi}{2} < x \leq 0 , y >0 \}$.
\item Montrer que la fonction $ : \mu \ra C(\mu)$ est de classe $C ^1$ au voisinage de $0$ et donner un développement limité à l'ordre 1 de $x_c (\mu)$ et $y_c (\mu)$.
\item On note $A(\mu)$ la matrice du problème linéarisé au point $C( \mu)$. Écrire le développement limité à l'ordre 1 de $A(\mu)$ en $0$.
\end{questions}
\end{exo}




\end{document}