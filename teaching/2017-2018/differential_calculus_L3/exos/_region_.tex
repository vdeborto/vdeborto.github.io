\message{ !name(Feuille02.tex)}\documentclass[12pt]{article}
\usepackage[frenchb]{babel}
\usepackage[T1]{fontenc}                % Accents codés dans la fonte.
\usepackage[utf8x]{inputenc}           % Accents 8 bits dans le fichier.
\usepackage{ifthen}                     % Faire des tests if/then/else.
\usepackage{amsmath}    % Les symboles les plus fréquents.
\usepackage{amsfonts}   % Des fontes, eg pour \mathbb.
\usepackage{vmargin}                    % Régler la taille de la feuille.
\usepackage{amssymb}
\usepackage{lastpage}
\usepackage{accents}


% La taille de la page
\setpapersize{A4}
\setmarginsrb{20mm}{10mm}{14mm}{15mm}{10mm}{6mm}{0mm}{10mm}

% La mise en page
\newcounter{vcenterstest}
\newlength{\leftlength}
\newlength{\rightlength}
\newlength{\vcentersskip}
\newcommand{\leftcentersright}[4][2]{%
        \settowidth{\leftlength}{#2}%
        \settowidth{\rightlength}{#4}%
        \setcounter{vcenterstest}{#1}%
        \ifthenelse{\value{vcenterstest} = 0}
                {\setlength{\vcentersskip}{0pt}}{}%
        \ifthenelse{\value{vcenterstest} = 1}
                {\setlength{\vcentersskip}{\smallskipamount}}{}%
        \ifthenelse{\value{vcenterstest} = 2}
                {\setlength{\vcentersskip}{\medskipamount}}{}%
        \ifthenelse{\value{vcenterstest} = 3}
                {\setlength{\vcentersskip}{\bigskipamount}}{}%
        \ifthenelse{\value{vcenterstest} = 4}
                {\setlength{\vcentersskip}{1cm}}{}%
                % On laisse un espace vertical défini par l'argument
                % optionnel #1
        \vskip\vcentersskip
                % On place #2 et on recule de sa longueur
        \noindent#2\hskip-\leftlength%
                % On centre #3
        \hfill#3\hfill%
                % On va au bout de la ligne, on recule de la longueur de #4 et
                % on place #4
        \mbox{}\hskip-\rightlength#4%
                % On laisse un espace vertical défini par l'argument
                % optionnel #1
        \vskip\vcentersskip%
    }
\newcommand{\centers}[2][2]{\leftcentersright[#1]{}{#2}{}}
\newcommand{\leftcenters}[3][2]{\leftcentersright[#1]{#2}{#3}{}}
\newcommand{\centersright}[3][2]{\leftcentersright[#1]{}{#2}{#3}}
\newcommand{\leftencadreright}[4][2]{\leftcentersright[#1]{#2}{\fbox{#3}}{#4}}

\newcommand{\poubelle}[1]{}
\newcommand{\numtoi}[1]{
  \ifthenelse{ \equal{#1}{1} }{i}{
  \ifthenelse{ \equal{#1}{2} }{ii}{
  \ifthenelse{ \equal{#1}{3} }{iii}{
  \ifthenelse{ \equal{#1}{4} }{iv}{
  \ifthenelse{ \equal{#1}{5} }{v}{
  \ifthenelse{ \equal{#1}{6} }{vi}{
  \ifthenelse{ \equal{#1}{7} }{vii}{
  \ifthenelse{ \equal{#1}{8} }{viii}{
  \ifthenelse{ \equal{#1}{9} }{ix}{
  \ifthenelse{ \equal{#1}{10} }{x}{
  \ifthenelse{ \equal{#1}{11} }{xi}{
  \ifthenelse{ \equal{#1}{12} }{xii}{
  \ifthenelse{ \equal{#1}{13} }{xiii}{
  \ifthenelse{ \equal{#1}{14} }{xiv}{
  \ifthenelse{ \equal{#1}{15} }{xv}{
  \ifthenelse{ \equal{#1}{16} }{xvi}{
  \ifthenelse{ \equal{#1}{17} }{xvii}{
  \ifthenelse{ \equal{#1}{18} }{xviii}{
    ERREUR,~MODIFIER~LA~MACRO~numtoi
  } } } } } } } } } } } } } } } } } }
}

\newcounter{cretraiti}
\newenvironment{ri}[1]{

\begin{list}{($\numtoi{\thecretraiti}$)}{
  \usecounter{cretraiti}
  \topsep=0.5ex
  \itemsep=0.3ex
  \labelsep=0.3em
  \parsep=0ex
  \listparindent=1em
  \settowidth{\labelwidth}{(#1)}
  \leftmargin=\labelwidth
  %\addtolength{\leftmargin}{\labelsep}
}}{\end{list}
}


% Quelques raccourcis pour taper des maths
\newcommand{\Sfr}{\mathfrak{S}}
\newcommand{\Afr}{\mathfrak{A}}
\newcommand{\QQ}{\mathbb{Q}}
\newcommand{\ZZ}{\mathbb{Z}}
\newcommand{\si}{\sigma}
\newcommand{\ins}[1]{{\ensuremath{\textrm{#1}}}}
\newcommand{\tiret}{\rule[0.6ex]{1.3ex}{0.26ex}}
\newcommand{\accolades}[1]{\ensuremath{\left\{#1\right\}}}
\newcommand{\paa}[1]{\ensuremath{\accolades{#1}}}       % Synonyme
\newcommand{\ds}{\displaystyle}


% Numerotation des pages / la derniere
\makeatletter
\renewcommand{\@evenfoot}%
        {\thepage \slash \pageref{LastPage}}
\renewcommand{\@oddfoot}{\@evenfoot}
\makeatother


%%%%% Pour faire les exos
\newcommand{\envfont}{\sf\bfseries}
\newcounter{cexo}
\renewcommand{\thecexo}{\arabic{cexo}}

\newenvironment{exo}[1][toto]
{
  \refstepcounter{cexo}
  \ifthenelse{ \equal{#1}{toto} }{
   \noindent {\envfont Exercice \thecexo}
  }{
   \noindent {\envfont Exercice \thecexo{} \tiret\, #1.}
  }
%\newline
%\indent
}
{}

\newcounter{cquestions}
\newenvironment{questions}{
\begin{list}{{\envfont\alph{cquestions})}}{
  \usecounter{cquestions}
  \topsep=0.5ex
  \partopsep=1em
  \labelsep=0.3em
  \itemsep=0.3ex
  \settowidth{\labelwidth}{{\envfont b)}}\listparindent=1em
  \leftmargin=\labelwidth \addtolength{\leftmargin}{\labelsep}
}}{\end{list} }


%%%%%%%%%%%%%%%%%%%%%%%%%%%%%%%%%%%%%%%%%
\newcommand{\spacebeforeenv}{\vspace{1ex}}
\newcommand{\spaceafterenv}{\vspace{1ex}}
\newcommand{\myqedenv}{}


\newcommand{\metaenvironnement}[2]  % 1er argument : nom de l'env, 2e argument : titre
{
  \ifthenelse{ \equal{#2}{toto} }{
    \spacebeforeenv \noindent {\envfont #1 }
  }{
    \spacebeforeenv \noindent {\envfont #1  \tiret\, #2.}
  }
}

%----------------------------------
% Rappel
%----------------------------------
\newenvironment{rappel}[1][toto]
{ \metaenvironnement{Rappel}{#1} }
{\myqedenv\spaceafterenv}


\newcommand{\fonction}[5]{%
        \ensuremath{#1\colon
        \left\{\hskip -1.5 mm
        \begin{array}{c@{\ }c@{\ }l}
        \medskip #2 & \longrightarrow & #3 \\
        #4 & \longmapsto & #5 \\
        \end{array}
        \right .
        }}
\newcommand{\ph}{\varphi}
\newcommand{\funu}[1]{\mathcal{C}^{#1}}
\newcommand{\RR}{\mathbb{R}}
\newcommand{\NN}{\mathbb{N}}
\newcommand{\CC}{\mathbb{C}}

\newenvironment{syst}[1][c]%
    {\ensuremath{\left \{ \hskip -1.5 mm \begin{array}{#1@{\ =\ }l}}}%
    {\end{array}\right.}

\newcommand{\f}[2]{{\ensuremath{\mathchoice%
    {\dfrac{#1}{#2}}
        {\dfrac{#1}{#2}}
    {\frac{#1}{#2}}
    {\frac{#1}{#2}}
    }}}
\renewcommand{\ni}{\noindent}
\newcommand{\ra}{\rightarrow}
\newcommand{\question}[1]{{\sf\bfseries #1}}
\newcommand{\norm}[1]{\ensuremath{\Arrowvert #1 \Arrowvert}}
\newcommand{\Ende}[3]{\ensuremath{\mathrm{End}_{#1}^{#2}(#3)}}
\newcommand{\End}[2]{\Ende{#1}{}{#2}}
\newcommand{\id}{\ensuremath{\mathrm{id}}}
\renewcommand{\d}{\ensuremath{\mathrm{d}}}
\renewcommand{\leq}{\leqslant}
\renewcommand{\geq}{\geqslant}


\usepackage{datetime}
\newdateformat{annscol}{\ifthenelse{\THEMONTH > 7}{\THEYEAR--\refstepcounter{YEAR}\THEYEAR}{\addtocounter{YEAR}{-1}\THEYEAR--\refstepcounter{YEAR}\THEYEAR}}

\begin{document}

\message{ !name(Feuille02.tex) !offset(-3) }

 

% ------------------------------------------------------------------------------------
\leftcentersright[0]{{\bf\sf\large Formation en Mathématiques Commune Cachan}}{}
{{\bf\sf\large Année~\annscol\today}} \leftcentersright[0]{{\bf\sf\large Calcul
    différentiel}}{}{{\bf\sf Valentin De Bortoli, Frédéric Pascal}}
\vskip3ex\centers{{\bf\sf \Large Feuille 0.2 - Espaces de Banach}}
% ------------------------------------------------------------------------------------------




% ------------------------------------------------------------------------------------------
% ------------------------------------------------------------------------------------------


\vskip2ex
\begin{exo}[Espace vectoriel normé]
  Soit $(E,\|.\|)$ un espace vectoriel normé (e.v.n) défini sur $\RR$ ou
  $\CC$. $E$ est un espace métrique pour la distance associée à la norme.
  \begin{questions}
  \item Montrer que l'application qui à $x \in E$ associe $\|x\|$ est continue
    sur $E$.
  \item Soit $ A \subset \RR^n$ fermé et non borné. Montrer que si
    $f : A \to \RR$ est continue et coercive i.e.
    $\lim_{\|x\|\to + \infty} f(x) = + \infty$ alors $f$ est minorée et atteint
    sa borne inférieure sur $A$.
  \item Montrer que si $E$ est de dimension finie alors sa boule unité est
    compacte.  \textit{Remarque :} Pourquoi ne précise-t-on pas la norme
    utilisée ?
  \item Soit $F$ un sous-espace vectoriel fermé strictement inclus dans
    $E$. Soit $1>r>0$. Montrer qu'il existe $u \in E$, tel que $\| u \| = 1$ et
    $d(u,F) > r$ (lemme de Riesz).
  \item Montrer que $E$ est de dimension finie $ssi$ sa boule unité est compacte
    (théorème de Riesz).
    % \\ \textit{Remarque :} Ce résultat s'étend aux espaces topologiques
    % séparés. Un espace vectoriel topologique réel séparé est de dimension
    % finie $ssi$ il est localement compact.
    % \item Montrer que $\RR$ muni de la norme usuelle, vu comme $\QQ$
    %   espace vectoriel, est de dimension infinie et que sa boule unité
    %   est compacte. Ce résultat est-il en contradiction avec le théorème
    %   de Riesz ?
  \end{questions}
\end{exo}


\vskip2ex
\begin{exo}[Espace de Banach]
  Un e.v.n complet est appelé espace de Banach.
  \begin{questions}
  \item Soit $(a_n)_n$ une suite de $E$ espace de Banach, dont la série est
    absolument convergente.  Montrer que cette série converge dans $E$ et que
    l'on $\|\sum_{n=0}^{+ \infty} a_n \| \leq \sum_{n=0}^{+ \infty} \|a_n\|$.
  \item Soient $E$ un espace de Banach et $A$ un ensemble quelconque. Montrer
    que ${\cal B}=\{ f : A \to E \, / \, f \, \mbox{bornée} \}$ muni de la norme
    sup (norme de la convergence uniforme) est un espace de Banach.
  \item Soient $E$ un espace de Banach et $A$ un espace topologique quelconque
    non vide. Montre que
    ${\cal C}_b=\{ f : A \to E \, / \, f \, \mbox{bornée et continue} \}$ est un
    sous espace de Banach dans ${\cal B}$. On notera que si $A$ est compact, il
    coincide avec
    ${\cal C}(A,E)= \{ f : A \to E \, / \, f \, \mbox{continue} \}$.
  \item Montrer que ${\cal C}([0,1],\RR)$ ensemble des fonctions continues de
    $[0,1]$ vers $\RR$, muni de la norme $\int_0^1 [f(t)|dt$ n'est pas un espace
    de Banach en prenant par exemple $f_n(x)= 0$ sur $[0,\frac{1}{2}]$,
    $f_n(x)=1$ sur $[\frac{1}{2}+\frac{1}{n},1]$ et linéaire sur
    $[\frac{1}{2}, \frac{1}{2}+\frac{1}{n}]$.
  \end{questions}
\end{exo}


\vskip2ex
% \begin{exo}[Espace de Fréchet]
%   Soit $E$ un espace vectoriel défini sur $\mathbb{R}$ ou $\mathbb{C}$ (on
%   note $k$ ce corps), muni d'une famille de semi-normes $(p_n)_{n \in \NN}$,
%   i.e
%   $\forall (x,\lambda) \in E \times k, \ \forall n \in \mathbb{N}, \
%   p_n(\lambda x) = \vert \lambda \vert p_n(x)$,
%   $\forall (x,y) \in E^2, \ \forall n \in \mathbb{N}, \ p_n(x+y) \le p_n(x) +
%   p_n(y)$. On suppose de plus que cette famille est séparante, i.e soit
%   $(x,y) \in E^2$,
%   $x \neq y \Rightarrow \exists n \in \mathbb{N}, \ p_n(x-y) >0$.
%   \begin{questions}
%   \item Montrer que
%     $d(x,y) = \underset{n \in \mathbb{N}}{\sum} 2^{-n}
%     \frac{p_n(x-y)}{1+p_n(x-y)}$ est une distance.  \\ \textit{Remarque :} Un
%     tel espace est dit de Fréchet lorsqu'il est complet pour cette métrique.
%   \item Soit $(x_k)_k$ une suite de $(E,d)$. Montrer que $(x_k)_k$ converge
%     vers $x$ $ssi$ $\forall n \in \mathbb{N}, \ p_n(x_k-x) \rightarrow 0$.
%   \item Montrer que $\mathcal{C}(\RR)$ est un espace de Fréchet.
%   \end{questions}
% \end{exo}

\vskip2ex
\begin{exo}[Fonctions linéaires continues dans les e.v.n]
  Soient $(E,\|.\|_E)$ et $(F,\|.\|_F)$ deux e.v.n.
  \begin{questions}
  \item Soit $f : E \to F$ linéaire. Montrer les assertions suivantes sont
    équivalentes : i) $f$ est continue en un point de $E$, ii) $f$ est continue
    sur $E$, iii) $f$ est bornée sur toute partie bornée de $E$, iv) il existe
    une constante $M>0$ telle que $\|f(x)\|_F \leq M \|x\|_E$ pour tout $x$ dans
    $E$.
  \item Montrer que si $f$ est une forme linéaire non nulle alors $f$ est
    continue si et seulement si $f^{-1}(0)$ est fermé.
  \item Montrer que si $F$ est un espace de Banach, l'ensemble ${\cal L}(E,F)$
    des applications linéaires continues de $E$ dans $F$ est
    un espace de Banach pour la norme subordonnée.  \\
    \textit{Rappel} : Si $f \in {\cal L}(E,F)$, sa norme subordonnée est
    $$|f| = \sup \{ |f(x)|_F \quad | \quad x\in S_E (0,1) \} = \sup \{
    \frac{|f(x)|_F}{|x|_E} \quad | \quad x\in E-\{0\} \}$$
  \item Soit $p: E \to \RR^+$ une semi-norme. Montrer que $p$ est continue sur
    $(E,\|.\|)$ si et seulement si il existe $M>0$ telle que pour tout $x$ de
    $E$, $$p(x) \leq M \|x\|.$$ En déduire que toute semi-norme $p$ sur $\RR^n$
    muni de la norme euclidienne est continue et par suite que toutes les normes
    sur $\RR^n$ sont équivalentes et que toutes les applications linéaires
    définies sur $\RR^n$ et à valeur dans $F$ sont continues.
  \item Soit $E$ un espace vectoriel de dimension finie muni de 2 normes
    $\|.\|_1$ et $\|.\|_2$. Montrer que les boules unités pour ces 2 normes sont
    homéomorphes (on pourra utiliser une fonction qui à $x$ associe
    $\lambda(x) x$ avec $ \lambda(x) \geq 0$).
  \item En déduire en particulier que dans $\RR^2$ le disque unité est le carré
    unité sont homéomorphes et définir l'application homéomorphe.
  \end{questions}
\end{exo}




\vskip2ex
\begin{exo}[Caractérisation des espaces de Banach]
  Soient $(E,\|.\|)$ un espace de Banach et $0 < k < 1$
  \begin{questions}
  \item Montrer que les propriétés suivantes sont équivalentes : i) $E$ est un
    espace de Banach, ii) toute série absolument convergente est convergente,
    iii) pour toute suite $(a_n)_n$ de $E$ vérifiant $\|a_n\| \leq k^n$, la
    série $a_n$ converge.
  \end{questions}
\end{exo}



\vskip2ex
\begin{exo}[Isomorphisme]
  Soient $(E,\|.\|)$ et $(F,\|.\|)$ deux espaces de Banach.
  \begin{questions}
  \item Soit $u \in {\cal L}(E,E)$ telle que $\|u\|< 1$. Montrer que $Id-u$ est
    inversible.
  \item On dit que $f$ est un isomorphisme, si $f$ est linéaire, continue,
    bijective et si la bijection réciproque est continue.  Montrer que
    l'ensemble $ Isom(E,F)$ formé des isomorphisme de $E$ sur $F$ est un ouvert
    dans ${\cal L}(E,F)$.
  \item Montrer que l'application $u \to u^{-1}$ de $ Isom(E,F)$ dans
    $ Isom(F,E)$ est continue.
  \end{questions}
\end{exo}


\message{ !name(Feuille02.tex) !offset(-5) }

\end{document}