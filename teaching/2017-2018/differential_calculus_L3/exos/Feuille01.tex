\documentclass[12pt]{article}
%\usepackage[frenchb]{babel}
\usepackage[T1]{fontenc}                % Accents codés dans la fonte.
\usepackage[utf8x]{inputenc}           % Accents 8 bits dans le fichier.
\usepackage{ifthen}                     % Faire des tests if/then/else.
\usepackage{amsmath}    % Les symboles les plus fréquents.
\usepackage{amsfonts}   % Des fontes, eg pour \mathbb.
\usepackage{vmargin}                    % Régler la taille de la feuille.
\usepackage{amssymb}
\usepackage{lastpage}
\usepackage{accents}


% La taille de la page
\setpapersize{A4}
\setmarginsrb{20mm}{10mm}{14mm}{15mm}{10mm}{6mm}{0mm}{10mm}

% La mise en page
\newcounter{vcenterstest}
\newlength{\leftlength}
\newlength{\rightlength}
\newlength{\vcentersskip}
\newcommand{\leftcentersright}[4][2]{%
        \settowidth{\leftlength}{#2}%
        \settowidth{\rightlength}{#4}%
        \setcounter{vcenterstest}{#1}%
        \ifthenelse{\value{vcenterstest} = 0}
                {\setlength{\vcentersskip}{0pt}}{}%
        \ifthenelse{\value{vcenterstest} = 1}
                {\setlength{\vcentersskip}{\smallskipamount}}{}%
        \ifthenelse{\value{vcenterstest} = 2}
                {\setlength{\vcentersskip}{\medskipamount}}{}%
        \ifthenelse{\value{vcenterstest} = 3}
                {\setlength{\vcentersskip}{\bigskipamount}}{}%
        \ifthenelse{\value{vcenterstest} = 4}
                {\setlength{\vcentersskip}{1cm}}{}%
                % On laisse un espace vertical défini par l'argument
                % optionnel #1
        \vskip\vcentersskip
                % On place #2 et on recule de sa longueur
        \noindent#2\hskip-\leftlength%
                % On centre #3
        \hfill#3\hfill%
                % On va au bout de la ligne, on recule de la longueur de #4 et
                % on place #4
        \mbox{}\hskip-\rightlength#4%
                % On laisse un espace vertical défini par l'argument
                % optionnel #1
        \vskip\vcentersskip%
    }
\newcommand{\centers}[2][2]{\leftcentersright[#1]{}{#2}{}}
\newcommand{\leftcenters}[3][2]{\leftcentersright[#1]{#2}{#3}{}}
\newcommand{\centersright}[3][2]{\leftcentersright[#1]{}{#2}{#3}}
\newcommand{\leftencadreright}[4][2]{\leftcentersright[#1]{#2}{\fbox{#3}}{#4}}

\newcommand{\numtoi}[1]{
  \ifthenelse{ \equal{#1}{1} }{i}{
  \ifthenelse{ \equal{#1}{2} }{ii}{
  \ifthenelse{ \equal{#1}{3} }{iii}{
  \ifthenelse{ \equal{#1}{4} }{iv}{
  \ifthenelse{ \equal{#1}{5} }{v}{
  \ifthenelse{ \equal{#1}{6} }{vi}{
  \ifthenelse{ \equal{#1}{7} }{vii}{
  \ifthenelse{ \equal{#1}{8} }{viii}{
  \ifthenelse{ \equal{#1}{9} }{ix}{
  \ifthenelse{ \equal{#1}{10} }{x}{
  \ifthenelse{ \equal{#1}{11} }{xi}{
  \ifthenelse{ \equal{#1}{12} }{xii}{
  \ifthenelse{ \equal{#1}{13} }{xiii}{
  \ifthenelse{ \equal{#1}{14} }{xiv}{
  \ifthenelse{ \equal{#1}{15} }{xv}{
  \ifthenelse{ \equal{#1}{16} }{xvi}{
  \ifthenelse{ \equal{#1}{17} }{xvii}{
  \ifthenelse{ \equal{#1}{18} }{xviii}{
    ERREUR,~MODIFIER~LA~MACRO~numtoi
  } } } } } } } } } } } } } } } } } }
}

\newcounter{cretraiti}
\newenvironment{ri}[1]{

\begin{list}{($\numtoi{\thecretraiti}$)}{
  \usecounter{cretraiti}
  \topsep=0.5ex
  \itemsep=0.3ex
  \labelsep=0.3em
  \parsep=0ex
  \listparindent=1em
  \settowidth{\labelwidth}{(#1)}
  \leftmargin=\labelwidth
  %\addtolength{\leftmargin}{\labelsep}
}}{\end{list}
}


% Quelques raccourcis pour taper des maths
\newcommand{\Sfr}{\mathfrak{S}}
\newcommand{\Afr}{\mathfrak{A}}
\newcommand{\QQ}{\mathbb{Q}}
\newcommand{\ZZ}{\mathbb{Z}}
\newcommand{\si}{\sigma}
\newcommand{\ins}[1]{{\ensuremath{\textrm{#1}}}}
\newcommand{\tiret}{\rule[0.6ex]{1.3ex}{0.26ex}}
\newcommand{\accolades}[1]{\ensuremath{\left\{#1\right\}}}
\newcommand{\paa}[1]{\ensuremath{\accolades{#1}}}       % Synonyme
\newcommand{\ds}{\displaystyle}


% Numerotation des pages / la derniere
\makeatletter
\renewcommand{\@evenfoot}%
        {\thepage \slash \pageref{LastPage}}
\renewcommand{\@oddfoot}{\@evenfoot}
\makeatother


%%%%% Pour faire les exos
\newcommand{\envfont}{\sf\bfseries}
\newcounter{cexo}
\renewcommand{\thecexo}{\arabic{cexo}}

\newenvironment{exo}[1][toto]
{
  \refstepcounter{cexo}
  \ifthenelse{ \equal{#1}{toto} }{
   \noindent {\envfont Exercice \thecexo}
  }{
   \noindent {\envfont Exercice \thecexo{} \tiret\, #1.}
  }
%\newline
%\indent
}
{}

\newcounter{cquestions}
\newenvironment{questions}{
\begin{list}{{\envfont\alph{cquestions})}}{
  \usecounter{cquestions}
  \topsep=0.5ex
  \partopsep=1em
  \labelsep=0.3em
  \itemsep=0.3ex
  \settowidth{\labelwidth}{{\envfont b)}}\listparindent=1em
  \leftmargin=\labelwidth \addtolength{\leftmargin}{\labelsep}
}}{\end{list} }


%%%%%%%%%%%%%%%%%%%%%%%%%%%%%%%%%%%%%%%%%
\newcommand{\spacebeforeenv}{\vspace{1ex}}
\newcommand{\spaceafterenv}{\vspace{1ex}}
\newcommand{\myqedenv}{}


\newcommand{\metaenvironnement}[2]  % 1er argument : nom de l'env, 2e argument : titre
{
  \ifthenelse{ \equal{#2}{toto} }{
    \spacebeforeenv \noindent {\envfont #1 }
  }{
    \spacebeforeenv \noindent {\envfont #1  \tiret\, #2.}
  }
}

%----------------------------------
% Rappel
%----------------------------------
\newenvironment{rappel}[1][toto]
{ \metaenvironnement{Rappel}{#1} }
{\myqedenv\spaceafterenv}


\newcommand{\fonction}[5]{%
        \ensuremath{#1\colon
        \left\{\hskip -1.5 mm
        \begin{array}{c@{\ }c@{\ }l}
        \medskip #2 & \longrightarrow & #3 \\
        #4 & \longmapsto & #5 \\
        \end{array}
        \right .
        }}
\newcommand{\ph}{\varphi}
\newcommand{\funu}[1]{\mathcal{C}^{#1}}
\newcommand{\RR}{\mathbb{R}}
\newcommand{\NN}{\mathbb{N}}
\newcommand{\CC}{\mathbb{C}}

\newenvironment{syst}[1][c]%
    {\ensuremath{\left \{ \hskip -1.5 mm \begin{array}{#1@{\ =\ }l}}}%
    {\end{array}\right.}

\newcommand{\f}[2]{{\ensuremath{\mathchoice%
    {\dfrac{#1}{#2}}
        {\dfrac{#1}{#2}}
    {\frac{#1}{#2}}
    {\frac{#1}{#2}}
    }}}
\renewcommand{\ni}{\noindent}
\newcommand{\ra}{\rightarrow}
\newcommand{\question}[1]{{\sf\bfseries #1}}
\newcommand{\norm}[1]{\ensuremath{\Arrowvert #1 \Arrowvert}}
\newcommand{\Ende}[3]{\ensuremath{\mathrm{End}_{#1}^{#2}(#3)}}
\newcommand{\End}[2]{\Ende{#1}{}{#2}}
\newcommand{\id}{\ensuremath{\mathrm{id}}}
\renewcommand{\d}{\ensuremath{\mathrm{d}}}
\renewcommand{\leq}{\leqslant}
\renewcommand{\geq}{\geqslant}


\usepackage{datetime}
\newdateformat{annscol}{\ifthenelse{\THEMONTH > 7}{\THEYEAR--\refstepcounter{YEAR}\THEYEAR}{\addtocounter{YEAR}{-1}\THEYEAR--\refstepcounter{YEAR}\THEYEAR}}

\begin{document}
 

%------------------------------------------------------------------------------------
\leftcentersright[0]{{\bf\sf\large Formation en Mathématiques Commune}}{}
    {{\bf\sf\large Année~\annscol\today}}
\leftcentersright[0]{{\bf\sf\large Calcul différentiel}}{}{{\bf\sf
    Valentin De Bortoli , Frédéric Pascal}}
\vskip3ex\centers{{\bf\sf \Large Feuille 0.1 - Généralités sur les
    espaces métriques}}
%------------------------------------------------------------------------------------------




%------------------------------------------------------------------------------------------
%------------------------------------------------------------------------------------------


\vskip2ex 
\begin{exo}[Topologie dans les espaces métriques]
Soit $(E,d)$ un espace métrique. 
\begin{questions}
%\item Rappeler la définition d'une boule ouverte $B(x,r)$ de centre $x \in E$ et de rayon $r>0$.
\item On définit $\mathcal{O}= \{ U\subset E \mid \forall x\in U, \exists r>0 : B(x,r)\subset U \}$. Montrer que $(E, \mathcal{O})$ est un espace topologique. 
\item Soit $d'$ une seconde distance sur $E$ équivalente à $d$,
  c'est-à-dire qu'il existe $C_1, C_2 >0$ tels que pour tout $x,y \in
  E$, $C_1 d(x,y) \leq d'(x,y) \leq C_2 d(x,y)$. Montrer que les
  topologies engendrées par $d$ et $d'$ ont les mêmes ouverts.
  \item Soit $d'(x,y) = \min(d(x,y),1)$, $d''(x,y) =
    \frac{d(x,y)}{1+d(x,y)}$. Montrer que $d$ et $d'$ sont des
    métriques et que les topologies engendrées
    par $d$, $d'$ et $d''$ sont identiques.\\
    \textit{Remarque :} deux métriques qui engendrent la même
    topologie sont-elles équivalentes ?
\item Soient $a$ et $b$ deux points distincts de $E$, montrer qu'il
  existe deux ouverts contenant respectivement $a$ et $b$,
  d'intersection vide, i.e l'espace est séparé.
%Un exemple non-trivial d'espace non-séparé est n'importe quel ensemble
%infini muni de la topologie cofinie, i.e $E$ infini et $\mathcal{O} = \{ A \in
%\mathcal{P}(E), A^c \ \text{est fini} \}$.
\item Montrer que $A$ est un ouvert si et seulement si $A$ est une union de boules ouvertes.
\item Soit $x \in E$. On  note ${\cal V}(x)$ l'ensemble des voisinages de $x$.
Montrer que $A \subset E$ est un ouvert si et seulement si 
$A$ est un voisinage de chacun de ses points. 
\item On note $\accentset{\circ}{A}$ (respectivement $\overline A$) l'ensemble des points intérieurs 
(respectivement adhérents) à $A$. Montrer que 
\[
 (\accentset{\circ}{A})^c = \overline{A^c}.
\]
\item Montrer que
$\overline{A}$ est le plus petit fermé contenant $A$ ou encore l'intersection
de tous les fermés contenant $A$ et que $(\overline{A})^c=\accentset{\circ}{(A^c)}$. En déduire que $A$ est fermé 
si et seulement si $A=\overline{A}$.
\end{questions}
\end{exo}


\vskip2ex 
\begin{exo}[Suites dans les espaces métriques]
Soit $(E,d)$ un espace métrique et soit $\mathcal{O}$ l'ensemble de ses ouverts (comme défini dans l'exercice précédent).
\begin{questions}
\item Soit $(a_n)_{n }$ une suite de $E$. Montrer que la convergence de $a$ au sens topologique est équivalente à la convergence au sens métrique.
\item Pour $A \subset E$, montrer que 
\[
x \in \overline{A} \quad \Leftrightarrow \quad
\exists (a_n)_{n} \mbox{ suite dans $A$ } \, / \, x = \lim_{n\to\infty} a_n \,.
\]
\item Montrer que $A\subset E$ est fermé $ssi$ pour toute suite
  $(u_n)_n$ de $A$, si $(u_n)_n$ admet une limite $l \in E$, alors $l \in A$.
\item On dit que $a$ est une valeur d'adhérence de $(a_n)_{n }$ 
lorsqu'il existe une suite extraite de $(a_n)_{n}$ qui converge vers $a$. 
Montrer que c'est équivalent à 
\[
 \forall \epsilon >0, \, \forall n_0 \in \NN, \, 
 \exists n \geq n_0 \quad / \quad d(a_n,a) \leq \epsilon
\]
\item Soit $(a_n)_{n}$ une suite de $E$ et soit $a\in E$. Montrer que si $a$ est une valeur d'adhérence (au sens défini dans la question précédente) alors $a$ est dans l'adhérence de $\{a_n \mid n \in \NN \}$.
\item Montrer que l'ensemble des valeurs d'adhérence d'une suite $(a_n)_n$ de $E$ est égal à $\cap_{N \in \NN} \overline{\{a_n \mid n \geq N \}}$.
\end{questions}
\end{exo}



\vskip2ex 
\begin{exo}[Ensembles compacts dans les espaces métriques]
Soit $(E,d)$ un espace métrique.
\begin{questions}
\item Soit $A \subset E$ un sous-ensemble de $E$. Montrer que les propriétés suivantes sont équivalentes :
  \begin{itemize}
  \item $A$ vérifie la propriété de Bolzano Weiestrass, $ie$ toute suite dans $A$ a au moins 
une valeur d'adhérence dans $A$,
  \item $A$ est précompact, $ie$ pour tout $\epsilon >0$, $\exists N
    \in \mathbb{N}, \ (x_i)_{i \in \{ 1,\dots, N \} } \in A^N,
    \ \underset{i=1}{\overset{N}{\cup}}B(x_i,\epsilon) \supset A$, et
    $A$ est complet,
  \item $A$ vérifie la propriété de Borel Lebesgue, $ie$ si $(U_i)_{i \in I}$ est une famille d'ouverts de $E$ telle que $\cup_i U_i \cap A = A$ alors il existe $i_1, \cdots i_N \in I$ tels que  $\cup_{k=1}^N U_{i_k} \cap A = A$.
  \end{itemize}
\textit{Remarque :} Dans le cas d'espaces topologiques généraux on ne
peut plus définir un compact par les deux premiers points. Il est
alors défini par la propriété de Borel-Lebesgue. Il faut néanmoins
ajouter la propriété $A$ est séparé.
\item Montrer que toute suite d'un compact $A$ ayant une seule valeur 
d'adhérence converge.
\item Montrer qu'un sous ensemble  $A$ compact de $E$ est fermé et borné. On notera que dans $E=\RR^m$ où $(a_n)_n$ converge vers $a$ si et seulement si les
composantes $(a^i_n)_n$ convergent dans $\RR$ vers les composantes
$a^i$, les fermés bornés sont des compacts.\\
\textit{Remarque :} Le théorème de Riesz permet d'assurer que la boule
unité d'un espace vectoriel normé est compacte si et seulement si
l'espace est de dimension finie.

\end{questions}
\end{exo}


\vskip2ex 
\begin{exo}[Ensembles complets dans les espaces métriques]
Soit $(E,d)$ un espace métrique. 
\textit{Définitions}: \begin{itemize}
\item Une suite $(a_n)_n$ de $E$ est dite de Cauchy si pour tout $\epsilon >0$ il existe $N \in \NN$ tel que pour tout $p,q \geq N$, $d(a_p, a_q)o \leq \epsilon$.
\item Un sous ensemble $A\subset E$ est dit complet si toute suite de Cauchy de $A$ converge dans $A$.
\end{itemize}

\begin{questions}
\item Montrer qu'une suite de $E$ convergente est de Cauchy.
\item Montrer qu'une suite de $E$ ayant deux valeurs d'adhérences distinctes n'est pas de Cauchy.
\item Montrer que si $A$ est compact alors $A$ est complet. 
\item Montrer que si $A$ est complet alors $A$ est fermé.
\item Montrer que si $E$ est complet alors $A \subset E$ est complet si et seulement si $A$ est fermé.
\item Montrer que si $E$ est complet, toute intersection dénombrable
  d'ouverts dense est dense (théorème de Baire).\\
  \textit{Remarque :} Le théorème de Baire est utilisé en analyse
  fonctionnelle pour prouver de nombreux théorèmes fondamentaux
  (théorème de l'application ouverte, théorème de
  Banach-Steinhaus...). Il permet également d'affirmer l'existence de
  fonctions continues partout non-dérivables, de fonctions continues
  dont la série de Fourier ne converge pas... Néanmoins ces preuves ne
  sont pas constructives.
\item Montrer que $\mathbb{R}[X]$ ne peut pas être un espace complet et ce
  quelle que soit la norme.
\end{questions}
\end{exo}





\vskip2ex 
\begin{exo}[Fonctions continues]
Soient $(E,d)$ et $(F,d')$ deux espaces métriques. 
\begin{questions}
\item Soit $a$ fixé, montrer que $f : E \to \RR^+$ définie par $f(x) = d(a,x)$ est continue.
\item Montrer que pour qu'une application $f$ soit continue de $E$ sur $F$, il faut et il suffit
que l'image réciproque de tout ouvert de $F$ soit un ouvert de $E$.
\item Montrer que $f$ est continue en $a$ si et seulement si l'image par $f$ de toute suite de $E$ 
convergeant vers $a$ converge vers $f(a)$. En déduire que $f$ est continue si et seulement si l'image réciproque de tout fermé de $F$ est 	 un fermé de $E$.
\item Montrer qu'étant donnés 2 applications $f$ et $g$ continues sur $E$, le sous ensemble des 
points où $f$ et $g$ coïncident est fermé et qu'en particulier, si $f$ et $g$ prennent les mêmes valeurs
sur un ensemble dense dans $E$ (i.e. dont l'adhérence est égal à $E$) alors $f=g$.
\item Soit $(f_n)_n$ une suite d'applications de $E$ sur $F$ qui converge uniformément vers $f$ c'est-à-dire telle que 
\[
\sup_{x \in E} d'(f_n(x),f(x)) \longrightarrow_{n \to +\infty} 0 \,.
\]
Montrer que si les $f_n$ sont continues alors $f$ est continue.
\end{questions}
\end{exo}




\vskip2ex 
\begin{exo}[Fonctions définies sur un compact]
Soient $(E,d)$ et $(E',d')$ deux espaces métriques, et soit $f : E \longrightarrow E'$.\\
\textit{Définitions} : 
 \begin{itemize}
\item On dit que $f$ est un homéomorphisme de $E$ sur $E'$ si $f$ est inversible continue et $f^{-1}$ est également continue.
\item On dit que $f$ est ouverte si l'image de tout ouvert de $E$ par $f$ est un ouvert de $E'$. 
\item On dit que $f : E \to E'$ est uniformément continue sur $E$
si pour tout $\epsilon >0$, il existe $\eta >0$ tel que $d(x,y) \leq \eta$ entraîne $d'(f(x),f(y))\leq \epsilon$ 
pour tout $x$ et $y$.
\end{itemize}
\begin{questions}
\item Montrer que si $f$ est continue  sur $K$ compact de $E$ alors $f(K)$ est compact et $f$ est uniformément continue sur $K$.
\textit{Remarque:} En particulier une application numérique (à valeur dans $\RR$) définie sur un compact est bornée et atteint ses bornes.
\item Montrer que si $f$ est bijective et continue sur un compact $K$ alors $f$ est une application ouverte de $K$ sur $f(K)$, et donc un homéomorphisme de $K$ sur $f(K)$.
\end{questions}

\end{exo}


\end{document}