%Format feuille d'exercice
%T.R. 03/09/05.
%version: 0.1

\usepackage[latin1]{inputenc}
\usepackage[T1]{fontenc}
\usepackage[french]{babel}
\usepackage{lmodern}
\usepackage{bm}
\usepackage{enumitem}
\usepackage{graphicx}
\usepackage{color}
%\usepackage[proportional]{libertine}


%------------pour ceux qui utilisent pdflatex-----------------------------
%\newif\ifpdf 
%\ifx\pdfoutput\undefined \pdffalse % we are not running PDFLaTeX
%\else \pdfoutput=1 % we are running PDFLaTeX
%\pdftrue \fi

%\ifpdf
%\usepackage[pdftex]{graphicx}
%\usepackage[pdftex]{hyperref}
%\usepackage{pdfsync}
%\else
%\usepackage{graphicx}
%\usepackage{epsfig}
%\fi


%-----------------------aspects de la page-------------------------------

\usepackage{a4wide}

%\setlength{\textheight}{28cm}

\voffset -1cm

\pagestyle{empty}

%\usepackage[parfill]{parskip} %un nouveau paragraphe commence par un saut de ligne; pas d'identation

%\parskip 2pt

%-----------------------symboles mathematiques--------------------------

\usepackage{amsthm,amssymb,amsmath,amsfonts}
%\usepackage[libertine]{newtxmath}
\usepackage{latexsym}
\usepackage{mathrsfs}

\def\N{{\mathbb N}}
\def\Z{{\mathbb Z}}
\def\Q{{\mathbb Q}}
\def\R{{\mathbb R}}
\def\K{{\mathbb K}}
\def\C{{\mathbb C}}
\def\T{{\mathbb T}}
\def\CC{{\mathcal C}}
\def\LL{{\mathcal L}}
\def\BL{{\mathcal{BL}}}
\def\GL{{\mathcal{GL}}}
\def\MM{{\mathcal M}}
\def\SS{{\mathcal S}}
\def\SO{{\mathcal{SO}}}
\def\SL{{\mathcal{SL}}}
\def\BB{{\mathcal B}}
\def\SS{{\mathcal S}}
\def\OO{{\mathcal O}}


\theoremstyle{plain}

\newtheorem{theorem}{Th\'eor\`eme}%[section]
\newtheorem{prop}{Proposition}
\newtheorem{lemma}[theorem]{Lemme}
\newtheorem{cor}{Corollaire}

\theoremstyle{definition}
\newtheorem{definition}{D\'efinition}
%\renewtheorem{def}{Définition}
\newtheorem{ex}{Exemple}
\newtheorem{rem}{Remarque}
\newtheorem{notations}{Notations}

%----------------------abbreviations--------------------------

\newcommand{\ie}{\emph{i.e.\ }}
\newcommand{\ibid}{\emph{ibid.\ }}
\newcommand{\etc}{etc.\ }
\newcommand{\cf}{cf.\ }
\newcommand{\et}{\mbox{ et }}
\newcommand{\ds}{\displaystyle}
%\newcommand{\ssi}{if and only if \relax}
% \newcommand{\cad}{c'est-{\`a}-dire\relax }
% \newcommand{\nb}{\emph{n.b.\ }}
% \newcommand{\via}{\emph{via\relax }}
% \newcommand{\et}{\mbox{ et }}
% \newcommand{\ou}{\mbox{ ou }}
% \newcommand{\si}{\mbox{ si }}
% \newcommand{\sinon}{\mbox{ sinon }}
% \newcommand{\sur}{\mbox{ sur }}
% \newcommand{\avec}{\mbox{ avec }}
\newcommand{\ssi}{si et seulement si \relax}
\newcommand{\dev}{$\blacktriangleright$ DEV $\blacktriangleleft$ }
\renewcommand{\ker}{\mathrm{Ker}}
\newcommand{\im}{\mathrm{Im}}
\newcommand{\id}{\mathrm{Id}}
\newcommand{\tr}{\mathrm{tr}}

\DeclareMathOperator{\re}{Re}
\DeclareMathOperator{\argmin}{argmin}
\DeclareMathOperator{\Div}{div}
\DeclareMathOperator{\Sp}{Sp}
\DeclareMathOperator{\Vect}{Vect}
\DeclareMathOperator{\supp}{supp}
\DeclareMathOperator{\sinc}{sin_c}

%----------------------quelques macros plus ou moins utiles-----------

\def\s{\smallskip}
\def\m{\medskip}
\def\b{\bigskip}

\newcommand{\verif}{\textcolor{green}{\textbf{A v\'erifier.}}}
\newcommand{\reference}{\textcolor{red}{\textbf{R\'ef\'erence?}}}


%-------------------environnements-----------------------------
\newcounter{numexo}
\newcounter{numsubexo}
\newcounter{numsubsubexo}
\setcounter{numexo}{0}
\setcounter{numsubexo}{0}
\setcounter{numsubsubexo}{0}


\newenvironment{exercice}{\noindent\stepcounter{numexo}\b\textbf{\arabic{numexo}.---}}{}

%exo sans num\'ero de feuille
\newenvironment{exerciceexamen}{\noindent\stepcounter{numexo}\b\textbf{Exercice \arabic{numexo}.---}}{}

\newenvironment{exerciceen}[1]{\noindent\stepcounter{numexo}\b\textbf{Exercice \arabic{numexo}.}{--- #1}\begin{enumerate}}{\end{enumerate} }

\newcommand{\etoile}{\noindent$\star$}
\newcommand{\exo}{\refstepcounter{numexo}	\setcounter{numsubexo}{0}
%	\setcounter{numsubsubexo}{0} \s \noindent\textbf{Exercice \numfeuille.\arabic{numexo}. }} 
	\setcounter{numsubsubexo}{0} \s \noindent\textbf{Exercice \arabic{numexo}. }}
\newcommand{\exodur}{\refstepcounter{numexo}	\setcounter{numsubexo}{0}
%	\setcounter{numsubsubexo}{0} \s \noindent\textbf{Exercice \numfeuille.\arabic{numexo}. }} 
	\setcounter{numsubsubexo}{0} \s \noindent\textbf{Exercice \arabic{numexo}*. }}	
\newcommand{\exoprio}{\refstepcounter{numexo}	\setcounter{numsubexo}{0}
%	\setcounter{numsubsubexo}{0} \s \noindent\textbf{Exercice \numfeuille.\arabic{numexo}. }} 
	\setcounter{numsubsubexo}{0} \s \noindent\textbf{Exercice \arabic{numexo}${}^\square$. }}
\newcommand{\subexo}{\stepcounter{numsubexo}
	\setcounter{numsubsubexo}{0}
	\s\noindent\textbf{\arabic{numsubexo}. }}
\newcommand{\subsubexo}{ \stepcounter{numsubsubexo}
	\textbf{\alph{numsubsubexo}. }} 

\newcommand{\qtest}[1]{\stepcounter{numexo}	\setcounter{numsubexo}{0}
	\setcounter{numsubsubexo}{0} \b\noindent\textbf{\arabic{numexo}.---}  } 



%---------------------elements fixes --------------------------------------------


\def\enteteCachan{
\noindent\textbf{ENS Cachan}\hfill
\textbf{Math\'ematiques} 
\\
\noindent\textbf{Pr\'eparation \`a l'agr\'egation}\hfill
\textbf{Ann\'ee 2016/2017}\break
}

\def\feuilleexo{
\hrule\m
\begin{center}
\large \textbf{\titre}\\
\s
\normalsize\auteur
\end{center}
\m
\hrule\b
}

%--------------------------------------------- a laisser la, paske sinon ca marche pas--------------------------

%\ifpdf
%\DeclareGraphicsExtensions{.pdf, .jpg, .tif}
%\else
%\DeclareGraphicsExtensions{.eps, .jpg}
%\fi

%-------------------------Debut de la partie modifiable-------------------------
