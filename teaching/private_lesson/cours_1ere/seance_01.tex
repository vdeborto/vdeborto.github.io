\documentclass[10pt,a4paper]{article} 
\usepackage[utf8]{inputenc} 
\usepackage[T1]{fontenc} 
\usepackage[english]{babel} 
\usepackage{supertabular} %Nécessaire pour les longs tableaux
\usepackage[top=2.5cm, bottom=2.5cm, right=2.5cm, left=2.5cm]{geometry} %Mise en page 
\usepackage{amsmath} %Nécessaire pour les maths 
\usepackage{amssymb} %Nécessaire pour les maths 
\usepackage{stmaryrd} %Utilisation des double crochets 
\usepackage{pifont} %Utilisation des chiffres entourés 
\usepackage{graphicx} %Introduction d images 
\usepackage{epstopdf} %Utilisation des images .eps 
\usepackage{amsthm} %Nécessaire pour créer des théorèmes 
\usepackage{algorithmic} %Nécessaire pour écrire des algorithmes 
\usepackage{algorithm} %Idem 
\usepackage{bbold} %Nécessaire pour pouvoir écrire des indicatrices 
\usepackage{hyperref} %Nécessaire pour écrire des liens externes 
\usepackage{array} %Nécessaire pour faire des tableaux 
\usepackage{tabularx} %Nécessaire pour faire de longs tableaux 
\usepackage{caption} %Nécesaire pour mettre des titres aux tableaux (tabular) 
\usepackage{color} %nécessaire pour écrire en couleur 
\newtheorem{thm}{Théorème} 
\newtheorem{mydef}{Définition}
\newtheorem{prop}{Proposition} 
\newtheorem{lemma}{Lemme}
\title{Séance 01 - 1ere}
\author{Valentin De Bortoli}
\begin{document}
\maketitle
\section{Exercice 1}
Donner les solutions réelles (si elles existent) des équations
suivantes :
\begin{itemize}
\item $x^2 + 3x -6 = 0$
\item $x^4 + 6 -10x^2 = (x^2+1)^2$
\item $\frac{x}{1+x}  + \frac{1-x}{x} = 3$
\item $x^8 = 1$
\item $\frac{1}{x^2} + x^2 =2$
\end{itemize}

\section{Exercice 2}
Prouver que $\sqrt{2}$ n'est pas un nombre rationnel.

  \section{Exercice 3}
  Soit $u_n = (1 + \sqrt{2})^n$. Montrer qu'il existe $(a_n,b_n) \in
  \mathbb{N}^2$ tels que $u_n = a_n + \sqrt{2} b_n$. On pourra
  procéder par récurrence.

  \section{Exercice 4}
  De quelle équation à coefficients rationnels $\sqrt{1 + \sqrt{2}}$
  est solution ?

  \section{Exercice 5}
  Déterminer les solutions du système :
  \begin{equation*}
    \left\lbrace
      \begin{aligned}
        &x+y = 3 \\
        & xy = 5
    \end{aligned}
    \right.
    \end{equation*}
  
\end{document}