% cours réccurence (esomme)
% vecteurs équations droites
% produit scalaire

% Charlotte :
% -écriture scientifique/chiffres significatifs
% -problèmes
% -valeur absolue
% -pourcentage

\documentclass[10pt,a4paper]{article} 
\usepackage[utf8]{inputenc} 
\usepackage[T1]{fontenc} 
\usepackage[english]{babel} 
%\usepackage{supertabular} %Nécessaire pour les longs tableaux
\usepackage[top=2.5cm, bottom=2.5cm, right=2.5cm, left=2.5cm]{geometry} %Mise en page 
\usepackage{amsmath} %Nécessaire pour les maths 
\usepackage{amssymb} %Nécessaire pour les maths 
% %\usepackage{stmaryrd} %Utilisation des double crochets 
% \usepackage{pifont} %Utilisation des chiffres entourés 
% \usepackage{graphicx} %Introduction d images 
% \usepackage{epstopdf} %Utilisation des images .eps 
% \usepackage{amsthm} %Nécessaire pour créer des théorèmes 
% \usepackage{algorithmic} %Nécessaire pour écrire des algorithmes 
% \usepackage{algorithm} %Idem 
% \usepackage{bbold} %Nécessaire pour pouvoir écrire des indicatrices 
% \usepackage{hyperref} %Nécessaire pour écrire des liens externes 
% \usepackage{array} %Nécessaire pour faire des tableaux 
% \usepackage{tabularx} %Nécessaire pour faire de longs tableaux 
% \usepackage{caption} %Nécesaire pour mettre des titres aux tableaux (tabular) 
% \usepackage{color} %nécessaire pour écrire en couleur 
\newtheorem{thm}{Théorème} 
\newtheorem{mydef}{Définition}
\newtheorem{prop}{Proposition} 
\newtheorem{lemma}{Lemme}
\title{Séance 02 - 1ere}
\author{Valentin De Bortoli}
\begin{document}
\maketitle

\section{Exercice 1}
Tracer le cercle trigonométrique.\\
Donner les valeurs du sinus et du cosinus de $45^o$, $30^o$ et $60^o$ (on attend une démonstration).\\
Rappeler les liens entre sinus, cosinus, produit scalaire et déterminant.

\section{Exercice 2}
Soit $A(0,1)$, $B(-\frac{1}{2}, \frac{\sqrt{3}}{2})$,
$C(-\frac{1}{2}, \frac{-\sqrt{3}}{2})$. Montrer que le triangle $ABC$ est équilatéral.\\
On note $c$ le milieu de $[AB]$, $b$ le milieu de $[AC]$ et $a$ le milieu $[BC]$.\\
On note $A'$ l'intersection de $(ab)$ et $(ac)$. On note $B'$ l'intersection de
$(ab)$ et $(bc)$. Enfin, on note $C'$ l'intersection de $(ac)$ et $(bc)$. \\
Montrer que $A'B'C'$ est équilatéral.

\section{Exercice 3}
Montrer que l'intersection de deux plans en dimension trois est une droite.\\
Donner un vecteur directeur de cette droite.

\section{Exercice 4}
\subparagraph{}Montrer par récurrence que $\underset{k=0}{\overset{n}{\sum}}x^k = \frac{1-x^{n+1}}{1-x}$.
\subparagraph{}Que vaut $\underset{k=0}{\overset{n}{\sum}}2^{-k}$ ? Que se passe-t-il lorsque $n$ tend vers l'infini ?

\section{Exercice 5}
Factoriser les expressions suivantes :
\begin{itemize}
\item $A = (2x+1) ^2 - 4x^2 +4x -1$
\item $B = 8xyz + 5zy^2 + 5 yz^2 + 3xy^2 + 3yx^2$
\item $C = (x+3)^3 - x^3 + 9$
\item $D = 11x^2 - 143$
\end{itemize}

\end{document}