\documentclass[10pt,a4paper]{article} 
\usepackage[utf8]{inputenc} 
\usepackage[T1]{fontenc} 
\usepackage[english]{babel} 
\usepackage{supertabular} %Nécessaire pour les longs tableaux
\usepackage[top=2.5cm, bottom=2.5cm, right=2.5cm, left=2.5cm]{geometry} %Mise en page 
\usepackage{amsmath} %Nécessaire pour les maths 
\usepackage{amssymb} %Nécessaire pour les maths 
\usepackage{stmaryrd} %Utilisation des double crochets 
\usepackage{pifont} %Utilisation des chiffres entourés 
\usepackage{graphicx} %Introduction d images 
\usepackage{epstopdf} %Utilisation des images .eps 
\usepackage{amsthm} %Nécessaire pour créer des théorèmes 
\usepackage{algorithmic} %Nécessaire pour écrire des algorithmes 
\usepackage{algorithm} %Idem 
\usepackage{bbold} %Nécessaire pour pouvoir écrire des indicatrices 
\usepackage{hyperref} %Nécessaire pour écrire des liens externes 
\usepackage{array} %Nécessaire pour faire des tableaux 
\usepackage{tabularx} %Nécessaire pour faire de longs tableaux 
\usepackage{caption} %Nécesaire pour mettre des titres aux tableaux (tabular) 
\usepackage{color} %nécessaire pour écrire en couleur 
\newtheorem{thm}{Théorème} 
\newtheorem{mydef}{Définition}
\newtheorem{prop}{Proposition} 
\newtheorem{lemma}{Lemme}
\title{Séance 2 - 3ème}
\author{Valentin De Bortoli}
\begin{document}
\maketitle
\section{Exercice 1}
Calculer le PGCD des nombres suivants.
\begin{equation}
\text{PGCD}(480,2016)
\end{equation}
\begin{equation}
\text{PGCD}(42875,46)
\end{equation}
\begin{equation}
\text{PGCD}(704,1323)
\end{equation}
\begin{equation}
\text{PGCD}(21216,59)
\end{equation}
\begin{equation}
\text{PGCD}(n,n+1) \ \text{avec n un nombre entier naturel}
\end{equation}
\section{Exercice 2}
Donner la décomposition en produit de facteurs premiers des nombres suivants.
\begin{equation}
A=(2 \times 56)^2
\end{equation}
\begin{equation}
B=1644
\end{equation}
\begin{equation}
C=2816
\end{equation}
\section{Exercice 3}
Mettre sous forme irréductible les fractions suivantes.
\begin{equation}
D=\frac{102}{288}
\end{equation}
\begin{equation}
E=\frac{-94}{26}
\end{equation}
\section{Exercice 4}
Réduire les expressions suivantes
\begin{equation}
F=(\sqrt{6}+7)^3
\end{equation}
\begin{equation}
G=(\sqrt{69}-\sqrt{56})(\sqrt{69}+\sqrt{56})
\end{equation}
\begin{equation}
H=\sqrt{5}\times (\sqrt{7}-\sqrt{5})
\end{equation}
\section{Exercice 5}
Donner l'écriture scientifique des nombres suivants.
\begin{equation}
I=\frac{10^4+(2\times 3 \times 10)^2}{10^5}
\end{equation}
\begin{equation}
J=6^2 \times (-7 \times 10)^3
\end{equation}
\end{document}