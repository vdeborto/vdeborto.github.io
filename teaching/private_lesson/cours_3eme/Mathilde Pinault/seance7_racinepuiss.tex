\documentclass[10pt,a4paper]{article} 
\usepackage[utf8]{inputenc} 
\usepackage[T1]{fontenc} 
\usepackage[english]{babel} 
\usepackage{supertabular} %Nécessaire pour les longs tableaux
\usepackage[top=2.5cm, bottom=2.5cm, right=2.5cm, left=2.5cm]{geometry} %Mise en page 
\usepackage{amsmath} %Nécessaire pour les maths 
\usepackage{amssymb} %Nécessaire pour les maths 
\usepackage{stmaryrd} %Utilisation des double crochets 
\usepackage{pifont} %Utilisation des chiffres entourés 
\usepackage{graphicx} %Introduction d images 
\usepackage{epstopdf} %Utilisation des images .eps 
\usepackage{amsthm} %Nécessaire pour créer des théorèmes 
\usepackage{algorithmic} %Nécessaire pour écrire des algorithmes 
\usepackage{algorithm} %Idem 
\usepackage{bbold} %Nécessaire pour pouvoir écrire des indicatrices 
\usepackage{hyperref} %Nécessaire pour écrire des liens externes 
\usepackage{array} %Nécessaire pour faire des tableaux 
\usepackage{tabularx} %Nécessaire pour faire de longs tableaux 
\usepackage{caption} %Nécesaire pour mettre des titres aux tableaux (tabular) 
\usepackage{color} %nécessaire pour écrire en couleur 
\newtheorem{thm}{Théorème} 
\newtheorem{mydef}{Définition}
\newtheorem{prop}{Proposition} 
\newtheorem{lemma}{Lemme}
\title{Séance 7 - 3ème}
\author{Valentin De Bortoli}
\begin{document}
\maketitle
\section{Exercice 1}
Simplifier le plus possible les expressions suivantes :
\begin{equation}
A=\sqrt{10000}(\sqrt{5}+\sqrt{36})^2
\end{equation}
\begin{equation}
B=(\sqrt{81}\sqrt{7} +\sqrt{25})^2
\end{equation}
\begin{equation}
C=\left( \sqrt{\frac{\sqrt{16}}{3}} \right)^2 \frac{1}{\sqrt{8}}
\end{equation}
\section{Exercice 3}
Calculer les expressions suivantes :
\begin{equation}
A=PGCD(763,211)
\end{equation}
\begin{equation}
B=PPCM(1782,143)
\end{equation}
\section{Exercice 3}
Rappeler les règles suivantes et dans chaque cas donner un exemple :
\begin{itemize}
\item distributivité de la puissance sur le produit
\item distributivité de la puissance sur le quotient
\item règle des signes et de la puissance
\item règle  de multiplication des puissances
\end{itemize}
\section{Exercice 4}
Rappeler les règles suivantes et dans chaque cas donner un exemple :
\begin{itemize}
\item la racine du produit est le produit des racines
\item la racine du quotient est le quotient des racines
\item valeur de $\sqrt{x^2}$ ?
\item valeur de $\left( \sqrt{x} \right)^2$ ?
\end{itemize}
\section{Exercice 5}
Exprimer sous la forme d'une puissance les expressions suivantes :
\begin{equation}
G=\left( \frac{-11}{4} \right)^7 \times \left( \frac{4}{11} \right)^7 \times (4^{-3})^{-2}
\end{equation}
\begin{equation}
H=\frac{6^3 \times 7^{-3}}{7^{-6}}\times (-6)^3
\end{equation}
\end{document}
