\documentclass[10pt,a4paper]{article} 
\usepackage[utf8]{inputenc} 
\usepackage[T1]{fontenc} 
\usepackage[english]{babel} 
\usepackage{supertabular} %Nécessaire pour les longs tableaux
\usepackage[top=2.5cm, bottom=2.5cm, right=2.5cm, left=2.5cm]{geometry} %Mise en page 
\usepackage{amsmath} %Nécessaire pour les maths 
\usepackage{amssymb} %Nécessaire pour les maths 
\usepackage{stmaryrd} %Utilisation des double crochets 
\usepackage{pifont} %Utilisation des chiffres entourés 
\usepackage{graphicx} %Introduction d images 
\usepackage{epstopdf} %Utilisation des images .eps 
\usepackage{amsthm} %Nécessaire pour créer des théorèmes 
\usepackage{algorithmic} %Nécessaire pour écrire des algorithmes 
\usepackage{algorithm} %Idem 
\usepackage{bbold} %Nécessaire pour pouvoir écrire des indicatrices 
\usepackage{hyperref} %Nécessaire pour écrire des liens externes 
\usepackage{array} %Nécessaire pour faire des tableaux 
\usepackage{tabularx} %Nécessaire pour faire de longs tableaux 
\usepackage{caption} %Nécesaire pour mettre des titres aux tableaux (tabular) 
\usepackage{color} %nécessaire pour écrire en couleur 
\newtheorem{thm}{Théorème} 
\newtheorem{mydef}{Définition}
\newtheorem{prop}{Proposition} 
\newtheorem{lemma}{Lemme}
\title{Séance 8 - 3ème}
\author{Valentin De Bortoli}
\begin{document}
\maketitle
\section{Exercice 1}
\textbf{Les règles sur les puissances doivent-être impérativement connues.}
Exprimer sous la forme d'une puissance les expressions suivantes.
\begin{equation}
A=\left(\frac{1}{3}\right)^8\left(\frac{1}{3}\right)^{-5}\times (3^{-4})^{-7}
\end{equation}
\begin{equation}
B=\left(\frac{-8}{3}\right)^2\times 8^2 \times (-3^2)^2
\end{equation}
\section{Exercice 2}
De la même manière, les règles de calcul sur les racines doivent être impérativement connues.
Simplifier les expressions suivantes :
\begin{equation}
C=(\sqrt{7}-\sqrt{3})(\sqrt{7}+\sqrt{3})
\end{equation}
\begin{equation}
D=(\sqrt{5}-2\sqrt{4})^2
\end{equation}
\begin{equation}
E=\sqrt{64+36}-\sqrt{64}-\sqrt{36}
\end{equation}
\begin{equation}
F=\sqrt{\frac{2}{3}}\times ((\sqrt{3}+1)^2-4)
\end{equation}
\subparagraph{Remarque :} pas beaucoup d'exercices pour cette semaine. Je préfère que tu revoies le cours, sérieusement et que tu tentes de bien assimiler les propriétés. Si tu as des questions note les sur une feuille qu'on puisse en parler mardi soir. 
\end{document}