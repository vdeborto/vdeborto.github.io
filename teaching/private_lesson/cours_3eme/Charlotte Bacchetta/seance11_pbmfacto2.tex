\documentclass[10pt,a4paper]{article} 
\usepackage[utf8]{inputenc} 
\usepackage[T1]{fontenc} 
\usepackage[english]{babel} 
\usepackage{supertabular} %Nécessaire pour les longs tableaux
\usepackage[top=2.5cm, bottom=2.5cm, right=2.5cm, left=2.5cm]{geometry} %Mise en page 
\usepackage{amsmath} %Nécessaire pour les maths 
\usepackage{amssymb} %Nécessaire pour les maths 
\usepackage{stmaryrd} %Utilisation des double crochets 
\usepackage{pifont} %Utilisation des chiffres entourés 
\usepackage{graphicx} %Introduction d images 
\usepackage{epstopdf} %Utilisation des images .eps 
\usepackage{amsthm} %Nécessaire pour créer des théorèmes 
\usepackage{algorithmic} %Nécessaire pour écrire des algorithmes 
\usepackage{algorithm} %Idem 
\usepackage{bbold} %Nécessaire pour pouvoir écrire des indicatrices 
\usepackage{hyperref} %Nécessaire pour écrire des liens externes 
\usepackage{array} %Nécessaire pour faire des tableaux 
\usepackage{tabularx} %Nécessaire pour faire de longs tableaux 
\usepackage{caption} %Nécesaire pour mettre des titres aux tableaux (tabular) 
\usepackage{color} %nécessaire pour écrire en couleur 
\newtheorem{thm}{Théorème} 
\newtheorem{mydef}{Définition}
\newtheorem{prop}{Proposition} 
\newtheorem{lemma}{Lemme}
\title{Séance 11 - 3ème}
\author{Valentin De Bortoli}
\begin{document}
\maketitle
\section{Exercice 1}
Exprimer sous la forme d'une puissance les expressions suivantes.
\begin{equation}
A=\frac{3^8 \times 3^{-8}}{((-3^5)^3)^7}
\end{equation}
\begin{equation}
B=-\frac{(\frac{5}{7})^{-30} \times 5^{15}}{5^{-15}}
\end{equation}
\section{Exercice 2}
Comment exprimer $\sqrt{3675}$ sous la forme $a\sqrt{b}$ avec $a$ et $b$ deux nombres entiers les plus petits possibles ?
\section{Exercice 3}
Calculer les nombres suivants :
\begin{equation}
C=(\sqrt{64}-\sqrt{100})^2
\end{equation}
\begin{equation}
D=(\sqrt{5}-\sqrt{7})^2
\end{equation}
\begin{equation}
E=(\sqrt{3}-4\sqrt{4})\sqrt{6}
\end{equation}
\section{Exercice 4}
Factoriser le plus possible les expressions suivantes :
\begin{equation}
F=9x^2-18x+9
\end{equation}
\begin{equation}
G=36-64x^2z^2
\end{equation}
\begin{equation}
H=x^2-2x+1-2(x+1)(4x+1)
\end{equation}
\section{Exercice 5}
Développer et simplifier les expressions suivantes :
\begin{equation}
I=(x+2)^3
\end{equation}
\begin{equation}
J=(x-2)(x+3)(x-3)
\end{equation}
\begin{equation}
K=5(x-2x+1)^2
\end{equation}
\end{document}