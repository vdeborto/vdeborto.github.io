\documentclass[10pt,a4paper]{article} 
\usepackage[utf8]{inputenc} 
\usepackage[T1]{fontenc} 
\usepackage[english]{babel} 
\usepackage{supertabular} %Nécessaire pour les longs tableaux
\usepackage[top=2.5cm, bottom=2.5cm, right=2.5cm, left=2.5cm]{geometry} %Mise en page 
\usepackage{amsmath} %Nécessaire pour les maths 
\usepackage{amssymb} %Nécessaire pour les maths 
\usepackage{stmaryrd} %Utilisation des double crochets 
\usepackage{pifont} %Utilisation des chiffres entourés 
\usepackage{graphicx} %Introduction d images 
\usepackage{epstopdf} %Utilisation des images .eps 
\usepackage{amsthm} %Nécessaire pour créer des théorèmes 
\usepackage{algorithmic} %Nécessaire pour écrire des algorithmes 
\usepackage{algorithm} %Idem 
\usepackage{bbold} %Nécessaire pour pouvoir écrire des indicatrices 
\usepackage{hyperref} %Nécessaire pour écrire des liens externes 
\usepackage{array} %Nécessaire pour faire des tableaux 
\usepackage{tabularx} %Nécessaire pour faire de longs tableaux 
\usepackage{caption} %Nécesaire pour mettre des titres aux tableaux (tabular) 
\usepackage{color} %nécessaire pour écrire en couleur 
\newtheorem{thm}{Théorème} 
\newtheorem{mydef}{Définition} 
\newtheorem{prop}{Proposition} 
\newtheorem{lemma}{Lemme}
\title{Séance 2 - 3ème}
\author{Valentin De Bortoli}
\begin{document}
\maketitle
\section{Exercice 1}
Développer et réduire les expressions suivantes.
\begin{equation}
D=-(2x-\frac{1}{7})(3x^2+\frac{x}{3})
\end{equation}
\begin{equation}
E=\frac{1}{4}(2x-1)^2(2x+1)
\end{equation}
\begin{equation}
F=-(-5x-8)^2
\end{equation}
\section{Exercice 2}
Factoriser le plus possible les expressions suivantes.
\begin{equation}
G=21x^2+49xy+7y^2
\end{equation}
\begin{equation}
H=25x^2+26xy
\end{equation}
\begin{equation}
I=96x+56xy+72x
\end{equation}
\section{Exercice 3}
Exprimer sous forme de puissances les nombres suivants.
\begin{equation}
J=((-3,3^3)^3)^3
\end{equation}
\begin{equation}
K=(-1,7^{-2})^2
\end{equation}
\begin{equation}
L=((-2,99)^{-5})^{-6}
\end{equation}
\end{document}