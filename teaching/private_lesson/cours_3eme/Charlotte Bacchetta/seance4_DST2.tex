\documentclass[10pt,a4paper]{article} 
\usepackage[utf8]{inputenc} 
\usepackage[T1]{fontenc} 
\usepackage[english]{babel} 
\usepackage{supertabular} %Nécessaire pour les longs tableaux
\usepackage[top=2.5cm, bottom=2.5cm, right=2.5cm, left=2.5cm]{geometry} %Mise en page 
\usepackage{amsmath} %Nécessaire pour les maths 
\usepackage{amssymb} %Nécessaire pour les maths 
\usepackage{stmaryrd} %Utilisation des double crochets 
\usepackage{pifont} %Utilisation des chiffres entourés 
\usepackage{graphicx} %Introduction d images 
\usepackage{epstopdf} %Utilisation des images .eps 
\usepackage{amsthm} %Nécessaire pour créer des théorèmes 
\usepackage{algorithmic} %Nécessaire pour écrire des algorithmes 
\usepackage{algorithm} %Idem 
\usepackage{bbold} %Nécessaire pour pouvoir écrire des indicatrices 
\usepackage{hyperref} %Nécessaire pour écrire des liens externes 
\usepackage{array} %Nécessaire pour faire des tableaux 
\usepackage{tabularx} %Nécessaire pour faire de longs tableaux 
\usepackage{caption} %Nécesaire pour mettre des titres aux tableaux (tabular) 
\usepackage{color} %nécessaire pour écrire en couleur 
\newtheorem{thm}{Théorème} 
\newtheorem{mydef}{Définition} 
\newtheorem{prop}{Proposition} 
\newtheorem{lemma}{Lemme}
\title{Séance 4 - 3ème}
\author{Valentin De Bortoli}
\begin{document}
\maketitle
\section{Exercice 1}
Donner sous forme de puissance les nombres suivants.
\begin{equation}
A=\frac{(-1,3^2)^3}{1,3^{-6}}
\end{equation}
\begin{equation}
B=\frac{(\frac{5}{7})^{5}}{(\frac{7}{5})^{-8}}
\end{equation}
\begin{equation}
C=\frac{(8 \times 7)^{20}\times (-7^3)^2 8^{-2}}{7^8}
\end{equation}
\begin{equation}
D=(-(-(-2,7^7)^3)^2)^{-2}
\end{equation}
\begin{equation}
E=\frac{3^8 \times 3^{-8}}{((-3^5)^3)^7}
\end{equation}
\begin{equation}
F=-\frac{(\frac{5}{7})^{-30} \times 5^{15}}{5^{-15}}
\end{equation}
\section{Exercice 2}
Donner l'écriture scientifique des nombres suivants.
\begin{equation}
G=(8\times100)^{2}\times3
\end{equation}
\begin{equation}
H=\frac{(2 \times1000)^3 \times 6 }{(2 \times 10^{-1})^2 \times 3}
\end{equation}
\begin{equation}
I=2,0000800000 \times 10^{-6}
\end{equation}
\begin{equation}
J=\frac{10^8+10^3}{10^5 \times 10^2}
\end{equation}
\begin{equation}
K=(5 \times 2 \times 7 \times 8)^2 \times 10^0
\end{equation}
\begin{equation}
L=5^3
\end{equation}
\section{Exercice 3}
Donner la factorisation en produit de nombres premiers des nombres suivants et donner leur nombre de diviseurs.
\begin{equation}
M=1512
\end{equation}
\begin{equation}
N=3087
\end{equation}
\begin{equation}
O=2310
\end{equation}
\begin{equation}
P=30375
\end{equation}
\section{Exercice 4}
Simplifier les fractions suivantes (sans la calculatrice évidemment !).
\begin{equation}
Q=\frac{167}{79}
\end{equation}
\begin{equation}
R=\frac{543}{54}
\end{equation}
\begin{equation}
S=\frac{986}{261}
\end{equation}
\begin{equation}
T=\frac{321}{478}
\end{equation}
\end{document}