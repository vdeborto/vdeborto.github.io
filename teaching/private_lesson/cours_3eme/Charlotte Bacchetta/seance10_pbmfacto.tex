\documentclass[10pt,a4paper]{article} 
\usepackage[utf8]{inputenc} 
\usepackage[T1]{fontenc} 
\usepackage[english]{babel} 
\usepackage{supertabular} %Nécessaire pour les longs tableaux
\usepackage[top=2.5cm, bottom=2.5cm, right=2.5cm, left=2.5cm]{geometry} %Mise en page 
\usepackage{amsmath} %Nécessaire pour les maths 
\usepackage{amssymb} %Nécessaire pour les maths 
\usepackage{stmaryrd} %Utilisation des double crochets 
\usepackage{pifont} %Utilisation des chiffres entourés 
\usepackage{graphicx} %Introduction d images 
\usepackage{epstopdf} %Utilisation des images .eps 
\usepackage{amsthm} %Nécessaire pour créer des théorèmes 
\usepackage{algorithmic} %Nécessaire pour écrire des algorithmes 
\usepackage{algorithm} %Idem 
\usepackage{bbold} %Nécessaire pour pouvoir écrire des indicatrices 
\usepackage{hyperref} %Nécessaire pour écrire des liens externes 
\usepackage{array} %Nécessaire pour faire des tableaux 
\usepackage{tabularx} %Nécessaire pour faire de longs tableaux 
\usepackage{caption} %Nécesaire pour mettre des titres aux tableaux (tabular) 
\usepackage{color} %nécessaire pour écrire en couleur 
\newtheorem{thm}{Théorème} 
\newtheorem{mydef}{Définition}
\newtheorem{prop}{Proposition} 
\newtheorem{lemma}{Lemme}
\title{Séance 10 - 3ème}
\author{Valentin De Bortoli}
\begin{document}
\maketitle
\section{Exercice 1}
Exprimer sous la forme d'une puissance les expressions suivantes.
\begin{equation}
A=\left( \frac{-8}{3^3}\right)^2 \times (-3^2)^3 +6^2
\end{equation}
\begin{equation}
B=\frac{5}{6^7}\times \left( \frac{6}{5} \right)^8 \times\left( \frac{1}{5^3} \right)^2
\end{equation}
\section{Exercice 2}
Calculer les expressions suivantes \textbf{(à la main...)}.
\begin{equation}
C=PGCD(564,231)
\end{equation}
\begin{equation}
D=PPCM(143,165)
\end{equation}
En profiter également pour calculer tous les diviseurs de 143 et 165. En déduire leur PGCD.
\section{Exercice 3}
Après avoir revu les identités remarquables (qui doivent être sues par cœur !), développer les expressions suivantes.
\begin{equation}
E=(x-3)^2(x+3)^2
\end{equation}
\begin{equation}
F=(x-2+y)^2
\end{equation}
\section{Exercice 4}
De la même manière, factoriser le plus possibles les expressions suivantes.
\begin{equation}
G=x^2y^2-z^2
\end{equation}
\begin{equation}
H=64x^2-14xy^2+49y^4
\end{equation}
\begin{equation}
I=64x^2+14xy^2+49y^4
\end{equation}
\section{Exercice 5}
Un stagiaire est payé 440 euros au mois de janvier, 462 euros au mois de février et 660 euros au mois de mars.
\subparagraph{1}Quel est le salaire journalier du stagiaire ? Sachant que celui-ci est supérieur à 20 euros et est un nombre entier d'euros.
\subparagraph{2}Combien de jours a-t-il travaillé chaque mois ?
\end{document}