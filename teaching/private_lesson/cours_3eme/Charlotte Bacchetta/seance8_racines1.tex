\documentclass[10pt,a4paper]{article} 
\usepackage[utf8]{inputenc} 
\usepackage[T1]{fontenc} 
\usepackage[english]{babel} 
\usepackage{supertabular} %Nécessaire pour les longs tableaux
\usepackage[top=2.5cm, bottom=2.5cm, right=2.5cm, left=2.5cm]{geometry} %Mise en page 
\usepackage{amsmath} %Nécessaire pour les maths 
\usepackage{amssymb} %Nécessaire pour les maths 
\usepackage{stmaryrd} %Utilisation des double crochets 
\usepackage{pifont} %Utilisation des chiffres entourés 
\usepackage{graphicx} %Introduction d images 
\usepackage{epstopdf} %Utilisation des images .eps 
\usepackage{amsthm} %Nécessaire pour créer des théorèmes 
\usepackage{algorithmic} %Nécessaire pour écrire des algorithmes 
\usepackage{algorithm} %Idem 
\usepackage{bbold} %Nécessaire pour pouvoir écrire des indicatrices 
\usepackage{hyperref} %Nécessaire pour écrire des liens externes 
\usepackage{array} %Nécessaire pour faire des tableaux 
\usepackage{tabularx} %Nécessaire pour faire de longs tableaux 
\usepackage{caption} %Nécesaire pour mettre des titres aux tableaux (tabular) 
\usepackage{color} %nécessaire pour écrire en couleur 
\newtheorem{thm}{Théorème} 
\newtheorem{mydef}{Définition}
\newtheorem{prop}{Proposition} 
\newtheorem{lemma}{Lemme}
\title{Séance 8 - 3ème}
\author{Valentin De Bortoli}
\begin{document}
\maketitle
\section{Exercice 1}
Dans cet exercice on présente une notion complémentaire à celle de \textbf{PGCD} (Plus Grand Diviseur Commun), celle de \textbf{PPCM} (Plus Petit Commun Multiple). Si on possède la décomposition en produit de facteurs premiers de deux nombres, calculer le PGCD revient à prendre pour chaque facteur premier \textbf{la puissance la plus basse}. Au contraire, calculer le PPCM revient à prendre pour chaque facteur premier \textbf{la puissance la plus haute}.
\subparagraph{1}Justifier que $PPCM(15,25)=75$ et $PGCD(15,25)=5$.
\subparagraph{2}Calculer $PPCM(560,720)$ ainsi que $PGCD(560,720)$.
\subparagraph{3}Calculer $PGCD(543,765)$.
\section{Exercice 2}
Simplifier les puissances suivantes :
\begin{equation}
A=\frac{\left(\frac{5}{-4}\right)^2 \times (-5)^2}{4^2}\times 4^4
\end{equation}
\begin{equation}
B=\frac{(-5)^3 \times 3^2}{-3^2 \times (-5)^{-6}}
\end{equation}
\section{Exercice 3}
Donner tous les diviseurs des nombres suivants.
\begin{equation}
C=127
\end{equation}
\begin{equation}
D=196
\end{equation}
\section{Exercice 4}
Après avoir \textbf{appris} les règles de calculs mais aussi la définition de la racine carrée (par cœur et comprise) et les subtilités concernant $\sqrt{x^2}$ et $(\sqrt{x})^2$ répondre aux questions suivantes :
\subparagraph{1} A-t-on $\sqrt{25}=\sqrt{-25}$ ? A-t-on $\sqrt{(-25)^2}=\sqrt{25^2}$ ?
\subparagraph{2} Soit $ABCDEFGH$ un cube de côté 1. Quelle est la longueur de la grande diagonale (le segment qui relie un des sommets du cube au sommet qui lui est le plus éloigné) ? \textbf{Faire un dessin}
\subparagraph{3} Exprimer $\sqrt{560}$ comme $a \times \sqrt{b}$ avec $b$ le plus petit possible (on pourra utiliser l'exercice 1, question 2).
\section{Exercice 5}
Simplifier les expressions suivantes :
\begin{equation}
E=(\sqrt{7}-\sqrt{3})(\sqrt{7}+\sqrt{3})
\end{equation}
\begin{equation}
F=(\sqrt{5}-2\sqrt{4})^2
\end{equation}
\begin{equation}
G=\sqrt{64+36}-\sqrt{64}-\sqrt{36}
\end{equation}
\begin{equation}
H=\sqrt{\frac{2}{3}}\times ((\sqrt{3}+1)^2-4)
\end{equation}
\begin{equation}
I=\frac{1}{\sqrt{16}}\times \frac{\sqrt{169} \times \sqrt{2}}{\sqrt{4}}
\end{equation}
\end{document}