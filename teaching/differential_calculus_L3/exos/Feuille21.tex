\documentclass[12pt]{article}
%\usepackage[frenchb]{babel}
\usepackage[T1]{fontenc}                % Accents codés dans la fonte.
\usepackage[utf8x]{inputenc}           % Accents 8 bits dans le fichier.
\usepackage{ifthen}                     % Faire des tests if/then/else.
\usepackage{amsmath}    % Les symboles les plus fréquents.
\usepackage{amsfonts}   % Des fontes, eg pour \mathbb.
\usepackage{vmargin}                    % Régler la taille de la feuille.
\usepackage{amssymb}
\usepackage{lastpage}
\usepackage{accents}
\usepackage{enumitem}

% La taille de la page
\setpapersize{A4}
\setmarginsrb{20mm}{10mm}{14mm}{15mm}{10mm}{6mm}{0mm}{10mm}

% La mise en page
\newcounter{vcenterstest}
\newlength{\leftlength}
\newlength{\rightlength}
\newlength{\vcentersskip}
\newcommand{\leftcentersright}[4][2]{%
        \settowidth{\leftlength}{#2}%
        \settowidth{\rightlength}{#4}%
        \setcounter{vcenterstest}{#1}%
        \ifthenelse{\value{vcenterstest} = 0}
                {\setlength{\vcentersskip}{0pt}}{}%
        \ifthenelse{\value{vcenterstest} = 1}
                {\setlength{\vcentersskip}{\smallskipamount}}{}%
        \ifthenelse{\value{vcenterstest} = 2}
                {\setlength{\vcentersskip}{\medskipamount}}{}%
        \ifthenelse{\value{vcenterstest} = 3}
                {\setlength{\vcentersskip}{\bigskipamount}}{}%
        \ifthenelse{\value{vcenterstest} = 4}
                {\setlength{\vcentersskip}{1cm}}{}%
                % On laisse un espace vertical défini par l'argument
                % optionnel #1
        \vskip\vcentersskip
                % On place #2 et on recule de sa longueur
        \noindent#2\hskip-\leftlength%
                % On centre #3
        \hfill#3\hfill%
                % On va au bout de la ligne, on recule de la longueur de #4 et
                % on place #4
        \mbox{}\hskip-\rightlength#4%
                % On laisse un espace vertical défini par l'argument
                % optionnel #1
        \vskip\vcentersskip%
    }
\newcommand{\centers}[2][2]{\leftcentersright[#1]{}{#2}{}}
\newcommand{\leftcenters}[3][2]{\leftcentersright[#1]{#2}{#3}{}}
\newcommand{\centersright}[3][2]{\leftcentersright[#1]{}{#2}{#3}}
\newcommand{\leftencadreright}[4][2]{\leftcentersright[#1]{#2}{\fbox{#3}}{#4}}

\newcommand{\numtoi}[1]{
  \ifthenelse{ \equal{#1}{1} }{i}{
  \ifthenelse{ \equal{#1}{2} }{ii}{
  \ifthenelse{ \equal{#1}{3} }{iii}{
  \ifthenelse{ \equal{#1}{4} }{iv}{
  \ifthenelse{ \equal{#1}{5} }{v}{
  \ifthenelse{ \equal{#1}{6} }{vi}{
  \ifthenelse{ \equal{#1}{7} }{vii}{
  \ifthenelse{ \equal{#1}{8} }{viii}{
  \ifthenelse{ \equal{#1}{9} }{ix}{
  \ifthenelse{ \equal{#1}{10} }{x}{
  \ifthenelse{ \equal{#1}{11} }{xi}{
  \ifthenelse{ \equal{#1}{12} }{xii}{
  \ifthenelse{ \equal{#1}{13} }{xiii}{
  \ifthenelse{ \equal{#1}{14} }{xiv}{
  \ifthenelse{ \equal{#1}{15} }{xv}{
  \ifthenelse{ \equal{#1}{16} }{xvi}{
  \ifthenelse{ \equal{#1}{17} }{xvii}{
  \ifthenelse{ \equal{#1}{18} }{xviii}{
    ERREUR,~MODIFIER~LA~MACRO~numtoi
  } } } } } } } } } } } } } } } } } }
}

\newcounter{cretraiti}
\newenvironment{ri}[1]{

\begin{list}{($\numtoi{\thecretraiti}$)}{
  \usecounter{cretraiti}
  \topsep=0.5ex
  \itemsep=0.3ex
  \labelsep=0.3em
  \parsep=0ex
  \listparindent=1em
  \settowidth{\labelwidth}{(#1)}
  \leftmargin=\labelwidth
  %\addtolength{\leftmargin}{\labelsep}
}}{\end{list}
}


% Quelques raccourcis pour taper des maths
\newcommand{\Sfr}{\mathfrak{S}}
\newcommand{\Afr}{\mathfrak{A}}
\newcommand{\QQ}{\mathbb{Q}}
\newcommand{\ZZ}{\mathbb{Z}}
\newcommand{\si}{\sigma}
\newcommand{\ins}[1]{{\ensuremath{\textrm{#1}}}}
\newcommand{\tiret}{\rule[0.6ex]{1.3ex}{0.26ex}}
\newcommand{\accolades}[1]{\ensuremath{\left\{#1\right\}}}
\newcommand{\paa}[1]{\ensuremath{\accolades{#1}}}       % Synonyme
\newcommand{\ds}{\displaystyle}
   \newcommand{\der}{
      \mathop{}\mathopen{}\mathrm{d}
   }

% Numerotation des pages / la derniere
\makeatletter
\renewcommand{\@evenfoot}%
        {\thepage \slash \pageref{LastPage}}
\renewcommand{\@oddfoot}{\@evenfoot}
\makeatother


%%%%% Pour faire les exos
\newcommand{\envfont}{\sf\bfseries}
\newcounter{cexo}
\renewcommand{\thecexo}{\arabic{cexo}}

\newenvironment{exo}[1][toto]
{
  \refstepcounter{cexo}
  \ifthenelse{ \equal{#1}{toto} }{
   \noindent {\envfont Exercice \thecexo}
  }{
   \noindent {\envfont Exercice \thecexo{} \tiret\, #1.}
  }
%\newline
%\indent
}
{}

\newcounter{cquestions}
\newenvironment{questions}{
\begin{list}{{\envfont\alph{cquestions})}}{
  \usecounter{cquestions}
  \topsep=0.5ex
  \partopsep=1em
  \labelsep=0.3em
  \itemsep=0.3ex
  \settowidth{\labelwidth}{{\envfont b)}}\listparindent=1em
  \leftmargin=\labelwidth \addtolength{\leftmargin}{\labelsep}
}}{\end{list} }

\newcommand{\poubelle}[1]{}

%%%%%%%%%%%%%%%%%%%%%%%%%%%%%%%%%%%%%%%%%
\newcommand{\spacebeforeenv}{\vspace{1ex}}
\newcommand{\spaceafterenv}{\vspace{1ex}}
\newcommand{\myqedenv}{}


\newcommand{\metaenvironnement}[2]  % 1er argument : nom de l'env, 2e argument : titre
{
  \ifthenelse{ \equal{#2}{toto} }{
    \spacebeforeenv \noindent {\envfont #1 }
  }{
    \spacebeforeenv \noindent {\envfont #1  \tiret\, #2.}
  }
}

%----------------------------------
% Rappel
%----------------------------------
\newenvironment{rappel}[1][toto]
{ \metaenvironnement{Rappel}{#1} }
{\myqedenv\spaceafterenv}


\newcommand{\fonction}[5]{%
        \ensuremath{#1\colon
        \left\{\hskip -1.5 mm
        \begin{array}{c@{\ }c@{\ }l}
        \medskip #2 & \longrightarrow & #3 \\
        #4 & \longmapsto & #5 \\
        \end{array}
        \right .
        }}
\newcommand{\ph}{\varphi}
\newcommand{\funu}[1]{\mathcal{C}^{#1}}
\newcommand{\RR}{\mathbb{R}}
\newcommand{\NN}{\mathbb{N}}
\newcommand{\CC}{\mathbb{C}}

\newenvironment{syst}[1][c]%
    {\ensuremath{\left \{ \hskip -1.5 mm \begin{array}{#1@{\ =\ }l}}}%
    {\end{array}\right.}

\newcommand{\f}[2]{{\ensuremath{\mathchoice%
    {\dfrac{#1}{#2}}
        {\dfrac{#1}{#2}}
    {\frac{#1}{#2}}
    {\frac{#1}{#2}}
    }}}
\renewcommand{\ni}{\noindent}
\newcommand{\ra}{\rightarrow}
\newcommand{\question}[1]{{\sf\bfseries #1}}
\newcommand{\norm}[1]{\ensuremath{\Arrowvert #1 \Arrowvert}}
\newcommand{\Ende}[3]{\ensuremath{\mathrm{End}_{#1}^{#2}(#3)}}
\newcommand{\End}[2]{\Ende{#1}{}{#2}}
\newcommand{\id}{\ensuremath{\mathrm{id}}}
\renewcommand{\d}{\ensuremath{\mathrm{d}}}
\renewcommand{\leq}{\leqslant}
\renewcommand{\geq}{\geqslant}


\usepackage{datetime}
\newdateformat{annscol}{\ifthenelse{\THEMONTH > 7}{\THEYEAR--\refstepcounter{YEAR}\THEYEAR}{\addtocounter{YEAR}{-1}\THEYEAR--\refstepcounter{YEAR}\THEYEAR}}

\begin{document}
 

%------------------------------------------------------------------------------------
\leftcentersright[0]{{\bf\sf\large Formation en Mathématiques Commune Cachan/P7}}{}
    {{\bf\sf\large Année~\annscol\today}}
\leftcentersright[0]{{\bf\sf\large Calcul différentiel}}{}{{\bf\sf Valentin De Bortoli, Frédéric Pascal}}
    \vskip3ex\centers{{\bf\sf \Large
                                      Feuille 2.1 - Différentiabilité, première propriétés}}

%------------------------------------------------------------------------------------------




%------------------------------------------------------------------------------------------
%------------------------------------------------------------------------------------------


\vskip2ex 
\begin{exo}[Exemples et contre-exemples]\label{exo-cdif-exemple-contre-exemple}
  \begin{questions}
\item Soit $f: \mathbb{R}^2 \longrightarrow \mathbb{R} $ l'application donnée par


\begin{displaymath}
f(x,y)=
\left|
  \begin{array}{rc}
  
  
  \frac{y^2}{x} & \mbox{si }  x \neq 0 \\
 y &  \mbox{si } x = 0 \\
  \end{array}
\right.
\end{displaymath}

  
       Montrer que $f$ admet une dérivée suivant toutes les directions mais n'est pas différentiable en $0$ ni même
      continue en $0$.

\poubelle{\item Soit $f: \mathbb{R}^2 \longrightarrow \mathbb{R} $ l'application donnée par
  \begin{displaymath}
f(x,y)=
\left|
  \begin{array}{rc}
  
  
  x y \frac{ x^2 - y^2}{x^2 + y^2} & \mbox{si }  (x,y) \neq (0,0) \\
0 &  \mbox{si }(x,y) = (0,0)\\
  \end{array}
\right.
\end{displaymath}

      Montrer que les dérivées partielles 
      $\frac{\partial^2{f}}{\partial{x}\partial{y}}(0,0)$ et
      $\frac{\partial^2{f}}{\partial{y}\partial{x}}(0,0)$ existent mais ne sont pas égales.
}
\item \'Etudier suivant les valeurs de $\alpha > 0$, la différentiabilité en $(0,0)$ de l'application
      $f : (x,y)\in \RR^2\mapsto |xy|^{\alpha}$.

% 

\item Soit U un ouvert de $\RR ^2$, soit $f: U \ra \RR$ telle que $$\forall (x,y) \in U, \partial_1 f (x,y) = 0$$
A-t-on $f(x,y) = g(y)$ ? ($ie$ que $f$ ne dépend pas de la première variable ? )


\item Soit $f : \RR^2 \ra \RR$ une fonction différentiable. On définit $g : (x,y) \mapsto f(y,x)$,
$g_1 : (x,y) \mapsto f(x, y+x)$ et $h : x \mapsto f(x,-x)$. Calculer
$\frac{\partial g}{\partial x}$, $\frac{\partial g}{\partial y}$, $\frac{\partial g_1}{\partial x}$, $\frac{\partial g_1}{\partial y}$ et $h'(x)$. \\
\textit{Remarque}: Cet exemple illustre bien l'intér\^et d'écrire $\frac{\partial f}{\partial e_i}$ ou $\partial_i$ au lieu de $\frac{\partial f}{\partial x_i}$.

\item Soient $f: \RR^n \longrightarrow \RR^p$, $g: \RR^p \longrightarrow \RR^q$ et $h: \RR^{(n+p)} \longrightarrow \RR^q$. On note $Df(a)$ la matrice telle que $df_a (u) = Df(a) u$ (en identifiant $u$ et son écriture vectorielle).
\begin{enumerate}[label=\roman*]
\item Quel lien y a-t-il entre $D(f\circ g)$, $Df$ et $Dg$? 
\item Soit $(e_1, \cdots, e_{n+p})$ une base de $\RR^{n+p}$. On s'intéresse à la dérivée partielle de $h$ par rapport aux $n$ premières variables, quel est le lien entre sa matrice et la matrice de $d h$ ?
\item On suppose ici $n=p$, soit $a \in \RR^n$ tel que $df$ est un C$^1-$difféomorphisme. Quel est le lien entre $Df (a)$ et $D(f^{-1})(f(a))$.
\end{enumerate}
\item  Soit $f$ définie sur $\mathbb{R}^n\backslash \lbrace 0 \rbrace$ et différentiable sur cet ensemble. Montrer que \[\forall x \in \mathbb{R}^n \backslash \lbrace 0 \rbrace, \ \underset{i = 1}{\overset{n}{\sum}}x_i \frac{\partial f}{\partial x_i}(x) = kf(x)\]
  est équivalent à $\forall x \in \mathbb{R}^n \backslash \lbrace 0 \rbrace, \ \forall t \in \mathbb{R}^*, \ f(tx) = t^kf(x)$ (on dit que la fonction est homogène).

 \end{questions}

\end{exo}


\poubelle{
\begin{exo}\label{exo-cdif-exemple}
\begin{questions}
\item Soit $f : \RR^2 \ra \RR$ une fonction différentiable. On définit $g : (x,y) \mapsto f(y,x)$,
$g_1 : (x,y) \mapsto f(x, y+x)$ et $h : x \mapsto f(x,-x)$. Calculer
$\frac{\partial g}{\partial x}$, $\frac{\partial g}{\partial y}$, $\frac{\partial g_1}{\partial x}$, $\frac{\partial g_1}{\partial y}$ et $h'(x)$.

\item Critiquer la notation $\frac{\partial f}{\partial x_i}$ ? Pourquoi la notation $\frac{\partial f}{\partial e_i}$ ou $\partial_i$ est-elle meilleure ?

\item Soit $a : \RR \ra \RR$ une application $C^1$ et $f : \RR^2 \ra \RR, \ (t,x) \ra f(t,x)$ telle que $f$ est continue et pour tout $t \in \RR$, $f(t,.)$ est de classe $C^{1}$. Calculer la dérivée de $g(x)=\int_{0}^{a(x)} f(t,x) dt$.
 \end{questions}
\end{exo}

}



\vskip2ex
\begin{exo}[Gradient et calcul matriciel]\label{exo-gradient-calcul-matriciel}
 Soit $n \in \NN$.
 \begin{questions}
\item  On se place sur $M_n (\RR)$. Calculer le gradient des applications Trace et Déterminant.


\item Soit $I$ une partie de $[| 1 , n|]$. Pour $A \in M_n (\RR)$, on note $|A_I|$ le mineur principal de $A$
obtenu en considérant les lignes et les colonnes d'indice appartenant à $I$. Déterminer le gradient de $f :  M_n (\RR) \ra \RR$ telle que $f(A) = |A_I|$ .


\item Calculer la différentielle \poubelle{et la différentielle seconde }de l'application $f : \RR^{n} \ra \RR$ définie par :
\begin{equation*}
 f(x)= \frac{1}{2} \langle Ax,x\rangle + \langle d,x \rangle + \delta
\end{equation*}
avec $A \in  M_n (\RR)$, $d \in \RR^{n}$ et $\delta \in \RR$. Cas où $A$ est symétrique ?

\item \'Calculer la différentielle \poubelle{et la différentielle seconde }de $f : GL_n (\RR) \ra GL_n(\RR)$ telle que $f(A) = A^{-1}$.
\ Indication : commencer par déterminer la différentielle en $I_{n}$ en introduisant une norme d'algèbre sur $ M_n (\RR)$. Généraliser ensuite à une matrice inversible quelconque.

\item Calculer la différentielle de l'application $f : \RR ^n - \{ 0 \} \ra \RR$ telle que $$f(x)=\frac{\langle A x , x \rangle }{\Vert x \Vert ^2}.$$
  \end{questions}
\end{exo}
\vskip2ex

\begin{exo}[Suite et densité]

Soit $f : [1, \infty,[ \ra \RR$ une fonction dérivable croissante telle que $f(x)$ croit vers $+\infty$ et $f'(x)$ décroit vers $0$ lorsque $x$ tend vers $ + \infty$. Montrer que les points $\exp (- i f(n))$, pour $n$ entier strictement positif, sont denses sur le cercle unité.
\end{exo}

\begin{exo}[Différentielle et polynôme]\label{exo-cdif-polynome} Soient $E= \RR_n[X]$ ($n \in \NN^\star$) et $I$ un intervalle ouvert de $\RR$. Montrer que l'application $\Phi : E \times I \ra \RR$ telle que $\Phi (P,x) = P(x)$  est différentiable et calculer sa différentielle.
\end{exo}

\vskip2ex

\begin{exo}[Un difféomorphisme ?]\label{exo-cdif-exemple-concret}
On considère l'application $f : \RR ^2 \ra \RR^2$ telle que  $f(x,y)=(x+y , x y)$
 \begin{questions}
\item Montrer que $f$ est de classe $C^{\infty}$. Calculer la différentielle de $f$ en $(x,y)$.
      Déterminer l'ensemble $S$ des points $(x,y) \in \RR^2$ en lesquels $\mathrm{d} f_{(x,y)}$ est inversible.

\item Calculer $f(\RR^2)$. L'application $f$ est-elle un difféomorphisme de $\RR^2$ sur $f(\RR^2)$ ?
      L'application $f$ est-elle un difféomorphisme de $S$ sur $f(S)$ ? Trouver un ouvert connexe maximal $U$
      de $\RR^2$ tel que $f$ soit un difféomorphisme de $U$ sur $f(S)$.
  \end{questions}
\end{exo}

\vskip2ex


\begin{exo}[Différentielle et loi de groupe] Soit $\ast$ une loi de groupe sur $\RR$ dont on note $e$ l'élément neutre. On suppose que la fonction $f$ définie par $f(x,y)=x\ast y$ est de classe $C^{1}$ sur $\RR^{2}$. On note $\partial_{1} f$ et $\partial_{2} f$ les dérivées partielles. 
 \begin{questions}
\item Montrer que pour tous $x,y \in \RR$ : 
\begin{equation*}
 (\partial_{2} f)(x\ast y,e)= (\partial_{2} f)(x,y).(\partial_{2} f)(y,e)
\end{equation*}

\item On cherche maintenant à construire une application $\phi$ de classe $C^{1}$ sur $\RR$ telle que : 
\begin{equation*}
 \phi(x\ast y)=\phi(x)+\phi(y)
\end{equation*}
Montrer que nécessairement $\phi$ est de la forme : $
 \phi(x)=a \int_{e}^{x} \dfrac{dt}{(\partial_{2} f)(t,e)}$
\poubelle{\begin{equation}
\label{eq1}
 \phi(x)=a \int_{e}^{x} \dfrac{dt}{(\partial_{2} f)(t,e)}
\end{equation}
}
\item Réciproquement, montrer que toute application de la forme précédente est  $C^{1}$ et transforme la loi $\ast$ en l'addition.
 \end{questions}
\end{exo}
\vskip2ex
\begin{exo}[Différentielle et mesure] On munit l'espace $\RR^{n}$ d'une norme quelconque et on note $B_{r}$ la boule de centre $0$ et de rayon $r$. Soit $f$ un $C^{1}$ difféomorphisme entre deux ouverts $U$ et $V$ de $\RR^{n}$ contenant l'origine. On supposera pour simplifier que $f(0)=0$. 
 \begin{questions}
\item Soit $\epsilon \in ]0,1[$ fixé. Montrer qu'il existe $R>0$ tel que pour tout $x \in B_{R}$,
\begin{equation*}
 \|\der f(0)^{-1}(f(x))-x\|\leq \epsilon \|x\|
\end{equation*}

\item Montrer qu'il existe $R'>0$ tel que, pour $0\leq r \leq R'$, $(1-\epsilon) \der f(0)(B_{r})\subset f(B_{r}) \subset (1+\epsilon) \der f(0)(B_{r})
$
\poubelle{\begin{equation*}
 (1-\epsilon) \der f(0)(B_{r})\subset f(B_{r}) \subset (1+\epsilon) \der f(0)(B_{r})
\end{equation*}}
Indication : Pour la première inclusion, observer que $f(B_{R})$ est un voisinage de $0$.

\item Soit $\lambda$ la mesure de Lebesgue sur $\RR^{n}$ dont on admet qu'elle satisfait la relation $\lambda(AX)=|\det(A)|\lambda(X)$ pour tout ensemble Lebesgue-mesurable X et toute application linéaire $A \in \mathcal{L}(\RR^{n})$. Montrer alors que :
\begin{equation*}
 |\det \der f(0)|= \lim_{r\rightarrow 0} \dfrac{\lambda(f(B_{r}))}{\lambda(B_{r})}
\end{equation*}
\end{questions}
\end{exo}
\vskip2ex
\begin{exo}[Différenciation sous le signe intégral]
  Soit $E$ et $F$ deux espaces métriques. Soit $K$ un compact de $E$ et $f$ continue de $E$ and $F$
  \begin{questions}
  \item Montrer que
    \[\epsilon \in \RR_+^*, \ \exists \eta \in \RR_+^*, \ \forall (x,y) \in K \times E, \ d_E(x,y)<\eta \ \Rightarrow \ d_F(f(x),f(y)) < \epsilon. \]
  \item Ce théorème est une généralisation d'un autre bien connu. Lequel ? En quoi consiste cette généralisation ?
  \item On suppose que $E$ et $F$ sont des espaces de Banach. Soit $\psi: ]0,1[ \times U \ \rightarrow \ F$ avec $U$ ouvert de $E$. On suppose également que :
    \begin{enumerate}
    \item pour tout $t \in ]0,1[$, $\psi(t,\cdot)$ est différentiable,
      \item $d_2 \psi: \ ]0,1[ \times U \rightarrow F$ est continue.
      \end{enumerate}
      Montrer que $g:U \rightarrow F$ définie par
      \[ \forall x \in U, g(x) = \int_{]0,1[}\psi(t,x)\text{d}t,\]
      est différentiable et donner sa différentielle.\\
      \textit{Remarque :} faire le lien avec le théorème de dérivation sous le signe intégral que vous connaissez.
    \end{questions}
  \end{exo}
\end{document}
%%% Local Variables:
%%% mode: latex
%%% TeX-master: t
%%% End:
