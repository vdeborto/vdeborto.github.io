\documentclass[12pt]{article}
%\usepackage[frenchb]{babel}
\usepackage[T1]{fontenc}                % Accents codés dans la fonte.
\usepackage[utf8x]{inputenc}           % Accents 8 bits dans le fichier.
\usepackage{ifthen}                     % Faire des tests if/then/else.
\usepackage{amsmath}    % Les symboles les plus fréquents.
\usepackage{amsfonts}   % Des fontes, eg pour \mathbb.
\usepackage{vmargin}                    % Régler la taille de la feuille.
\usepackage{amssymb}
\usepackage{lastpage}
\usepackage{accents}
\usepackage{enumitem}

% La taille de la page
\setpapersize{A4}
\setmarginsrb{20mm}{10mm}{14mm}{15mm}{10mm}{6mm}{0mm}{10mm}

% La mise en page
\newcounter{vcenterstest}
\newlength{\leftlength}
\newlength{\rightlength}
\newlength{\vcentersskip}
\newcommand{\leftcentersright}[4][2]{%
        \settowidth{\leftlength}{#2}%
        \settowidth{\rightlength}{#4}%
        \setcounter{vcenterstest}{#1}%
        \ifthenelse{\value{vcenterstest} = 0}
                {\setlength{\vcentersskip}{0pt}}{}%
        \ifthenelse{\value{vcenterstest} = 1}
                {\setlength{\vcentersskip}{\smallskipamount}}{}%
        \ifthenelse{\value{vcenterstest} = 2}
                {\setlength{\vcentersskip}{\medskipamount}}{}%
        \ifthenelse{\value{vcenterstest} = 3}
                {\setlength{\vcentersskip}{\bigskipamount}}{}%
        \ifthenelse{\value{vcenterstest} = 4}
                {\setlength{\vcentersskip}{1cm}}{}%
                % On laisse un espace vertical défini par l'argument
                % optionnel #1
        \vskip\vcentersskip
                % On place #2 et on recule de sa longueur
        \noindent#2\hskip-\leftlength%
                % On centre #3
        \hfill#3\hfill%
                % On va au bout de la ligne, on recule de la longueur de #4 et
                % on place #4
        \mbox{}\hskip-\rightlength#4%
                % On laisse un espace vertical défini par l'argument
                % optionnel #1
        \vskip\vcentersskip%
    }
\newcommand{\centers}[2][2]{\leftcentersright[#1]{}{#2}{}}
\newcommand{\leftcenters}[3][2]{\leftcentersright[#1]{#2}{#3}{}}
\newcommand{\centersright}[3][2]{\leftcentersright[#1]{}{#2}{#3}}
\newcommand{\leftencadreright}[4][2]{\leftcentersright[#1]{#2}{\fbox{#3}}{#4}}

\newcommand{\numtoi}[1]{
  \ifthenelse{ \equal{#1}{1} }{i}{
  \ifthenelse{ \equal{#1}{2} }{ii}{
  \ifthenelse{ \equal{#1}{3} }{iii}{
  \ifthenelse{ \equal{#1}{4} }{iv}{
  \ifthenelse{ \equal{#1}{5} }{v}{
  \ifthenelse{ \equal{#1}{6} }{vi}{
  \ifthenelse{ \equal{#1}{7} }{vii}{
  \ifthenelse{ \equal{#1}{8} }{viii}{
  \ifthenelse{ \equal{#1}{9} }{ix}{
  \ifthenelse{ \equal{#1}{10} }{x}{
  \ifthenelse{ \equal{#1}{11} }{xi}{
  \ifthenelse{ \equal{#1}{12} }{xii}{
  \ifthenelse{ \equal{#1}{13} }{xiii}{
  \ifthenelse{ \equal{#1}{14} }{xiv}{
  \ifthenelse{ \equal{#1}{15} }{xv}{
  \ifthenelse{ \equal{#1}{16} }{xvi}{
  \ifthenelse{ \equal{#1}{17} }{xvii}{
  \ifthenelse{ \equal{#1}{18} }{xviii}{
    ERREUR,~MODIFIER~LA~MACRO~numtoi
  } } } } } } } } } } } } } } } } } }
}

\newcounter{cretraiti}
\newenvironment{ri}[1]{

\begin{list}{($\numtoi{\thecretraiti}$)}{
  \usecounter{cretraiti}
  \topsep=0.5ex
  \itemsep=0.3ex
  \labelsep=0.3em
  \parsep=0ex
  \listparindent=1em
  \settowidth{\labelwidth}{(#1)}
  \leftmargin=\labelwidth
  %\addtolength{\leftmargin}{\labelsep}
}}{\end{list}
}


% Quelques raccourcis pour taper des maths
\newcommand{\Sfr}{\mathfrak{S}}
\newcommand{\Afr}{\mathfrak{A}}
\newcommand{\QQ}{\mathbb{Q}}
\newcommand{\ZZ}{\mathbb{Z}}
\newcommand{\si}{\sigma}
\newcommand{\ins}[1]{{\ensuremath{\textrm{#1}}}}
\newcommand{\tiret}{\rule[0.6ex]{1.3ex}{0.26ex}}
\newcommand{\accolades}[1]{\ensuremath{\left\{#1\right\}}}
\newcommand{\paa}[1]{\ensuremath{\accolades{#1}}}       % Synonyme
\newcommand{\ds}{\displaystyle}
   \newcommand{\der}{
      \mathop{}\mathopen{}\mathrm{d}
   }

% Numerotation des pages / la derniere
\makeatletter
\renewcommand{\@evenfoot}%
        {\thepage \slash \pageref{LastPage}}
\renewcommand{\@oddfoot}{\@evenfoot}
\makeatother


%%%%% Pour faire les exos
\newcommand{\envfont}{\sf\bfseries}
\newcounter{cexo}
\renewcommand{\thecexo}{\arabic{cexo}}

\newenvironment{exo}[1][toto]
{
  \refstepcounter{cexo}
  \ifthenelse{ \equal{#1}{toto} }{
   \noindent {\envfont Exercice \thecexo}
  }{
   \noindent {\envfont Exercice \thecexo{} \tiret\, #1.}
  }
%\newline
%\indent
}
{}

\newcounter{cquestions}
\newenvironment{questions}{
\begin{list}{{\envfont\alph{cquestions})}}{
  \usecounter{cquestions}
  \topsep=0.5ex
  \partopsep=1em
  \labelsep=0.3em
  \itemsep=0.3ex
  \settowidth{\labelwidth}{{\envfont b)}}\listparindent=1em
  \leftmargin=\labelwidth \addtolength{\leftmargin}{\labelsep}
}}{\end{list} }

\newcommand{\poubelle}[1]{}

%%%%%%%%%%%%%%%%%%%%%%%%%%%%%%%%%%%%%%%%%
\newcommand{\spacebeforeenv}{\vspace{1ex}}
\newcommand{\spaceafterenv}{\vspace{1ex}}
\newcommand{\myqedenv}{}


\newcommand{\metaenvironnement}[2]  % 1er argument : nom de l'env, 2e argument : titre
{
  \ifthenelse{ \equal{#2}{toto} }{
    \spacebeforeenv \noindent {\envfont #1 }
  }{
    \spacebeforeenv \noindent {\envfont #1  \tiret\, #2.}
  }
}

%----------------------------------
% Rappel
%----------------------------------
\newenvironment{rappel}[1][toto]
{ \metaenvironnement{Rappel}{#1} }
{\myqedenv\spaceafterenv}


\newcommand{\fonction}[5]{%
        \ensuremath{#1\colon
        \left\{\hskip -1.5 mm
        \begin{array}{c@{\ }c@{\ }l}
        \medskip #2 & \longrightarrow & #3 \\
        #4 & \longmapsto & #5 \\
        \end{array}
        \right .
        }}
\newcommand{\ph}{\varphi}
\newcommand{\funu}[1]{\mathcal{C}^{#1}}
\newcommand{\RR}{\mathbb{R}}
\newcommand{\NN}{\mathbb{N}}
\newcommand{\CC}{\mathbb{C}}

\newenvironment{syst}[1][c]%
    {\ensuremath{\left \{ \hskip -1.5 mm \begin{array}{#1@{\ =\ }l}}}%
    {\end{array}\right.}

\newcommand{\f}[2]{{\ensuremath{\mathchoice%
    {\dfrac{#1}{#2}}
        {\dfrac{#1}{#2}}
    {\frac{#1}{#2}}
    {\frac{#1}{#2}}
    }}}
\renewcommand{\ni}{\noindent}
\newcommand{\ra}{\rightarrow}
\newcommand{\question}[1]{{\sf\bfseries #1}}
\newcommand{\norm}[1]{\ensuremath{\Arrowvert #1 \Arrowvert}}
\newcommand{\Ende}[3]{\ensuremath{\mathrm{End}_{#1}^{#2}(#3)}}
\newcommand{\End}[2]{\Ende{#1}{}{#2}}
\newcommand{\id}{\ensuremath{\mathrm{id}}}
\renewcommand{\d}{\ensuremath{\mathrm{d}}}
\renewcommand{\leq}{\leqslant}
\renewcommand{\geq}{\geqslant}


\usepackage{datetime}
\newdateformat{annscol}{\ifthenelse{\THEMONTH > 7}{\THEYEAR--\refstepcounter{YEAR}\THEYEAR}{\addtocounter{YEAR}{-1}\THEYEAR--\refstepcounter{YEAR}\THEYEAR}}

\begin{document}
 

%------------------------------------------------------------------------------------
\leftcentersright[0]{{\bf\sf\large Formation en Mathématiques Commune Cachan/P7}}{}
    {{\bf\sf\large Année~\annscol\today}}
\leftcentersright[0]{{\bf\sf\large Calcul différentiel}}{}{{\bf\sf Valentin De Bortoli, Frédéric Pascal}}
\vskip3ex\centers{{\bf\sf \Large Feuille 3.1 - Théorèmes d'inversion locale et des fonctions implicites}}

%------------------------------------------------------------------------------------------




%------------------------------------------------------------------------------------------
%------------------------------------------------------------------------------------------

\vskip2ex
\begin{exo}[Exemples et contre-exemples]
\begin{questions}
\item Soit $U$ un ouvert de $\RR ^n$, $f : U \ra \RR ^n$ de classe $C^1$ injective sur $U$ telle que $\der f_x$ soit inversible pour tout $x \in U$. Montrer que $f(U)$ est ouvert et que $f$ est un $C^1$ difféomorphisme de $U$ sur $f(U)$.

\item Soit $U = \{ (r, \theta) \in \RR ^2 , r>0 \}$ et soit $V = \RR ^2 - \{(0,0)\}$, montrer que $f : U \ra V$ définie par $f(r, \theta) = (r \cos \theta , r \sin \theta)$ n'est pas un changement global de coordonnées.


\item Soit $\Phi : \RR \times ]0 , + \infty [ \ra \Omega \subset \RR ^2 $ telle que $\Phi (x,y) = (e^x + \log y , e^x + 2 \log y)$. Déterminer $\Omega$ pour que $\Phi$ définisse un $C^1$ difféomorphisme de $\RR \times ]0 , + \infty [$ sur $\Omega$

\item Soit \begin{equation*}  
f : x \ra  \left\lbrace
\begin{array}{cc}
x + x^2 \sin \frac{\pi}{x} & \text{ si } x \neq 0   \\
0  & \text{ sinon }

\end{array}\right.
\end{equation*}

Montrer que $f$ est dérivable sur $\RR$, de dérivée non nulle en $0$ mais que $f$ n'est inversible sur aucun voisinage de $0$.

\item Montrer qu'il existe une fonction différentiable $G$ définie sur un voisinage $V$ de $I_n$ dans $M_n (\RR)$ telle que pour tout $A \in V$,  on a $G(A)^2 = A$.

\end{questions}
\end{exo}
\vskip2ex
\begin{exo}[Fonctions matricielles]
Soit $n \in \NN ^{\star}$. On note $S_n (\RR)$ l'ensemble des matrices symétriques de $M_n (\RR)$ et $S_n ^{++}$ celui des matrices symétriques définies positives.
\begin{questions}
\item Soit $k \in \NN ^{\star}$. Montrer qu'il existe un voisinage $V$ de $I_n$ tel que pour tout $B \in V$ il existe une matrice $A \in M_n (\RR)$ telle que $A^k =B$.
\item Montrer qu'il existe un voisinage $V$ de $I_n $ tel que pour tout $B \in V$ il existe une matrice $A \in M_n (\RR)$ telle que $\exp A = B$.
\item Montrer $S_n ^{++} (\RR)$ est un ouvert de $S_n (\RR)$. Montrer que pour tout $A \in S_n ^{++} (\RR)$ il existe une unique matrice symétrique positive $B$ telle que $A = B^2$. Montrer que $B$ est définie positive. On note $B = \sqrt{ A}$.
\item Montrer que l'application 
\begin{equation*}  
\left\lbrace
\begin{array}{c}
S_n ^{++} (\RR) \ra S_n ^{++} (\RR)\\
A \ra \sqrt{A}
\end{array}\right.
\end{equation*}

est de classe $C^{\infty}$.
\end{questions}
\end{exo}


\vskip2ex
\begin{exo}[Fonction racine]
Soient $P_0 \in \RR_n [X]$ et $x_0 \in \RR$ une racine simple de $P_0$. Montrer qu'il existe une fonction "racine" de classe $C^{\infty}$ dans un voisinage de $P_0$. De façon précise montrer qu'il existe un voisinage $V$ de $P_0$ dans $\RR _n  [X]$ un voisinage $W$ de $x_0$ et une fonction $rac : V \ra W$ telle que $rac(P)$ soit une racine de $P$ pour tout $P \in V$.
\end{exo}

\vskip2ex
\begin{exo}[Perturbation cubique]
  Soit $P$ une fonction polynomiale de degré deux scindé sur $\mathbb{R}$, $\forall x\in \mathbb{R}, \ P(x) = (x-x_1)(x-x_2)$ avec $x_1 \neq x_2$. On considère $P_{\epsilon}(x) = P(x) + \epsilon x^3$.
  \begin{questions}
    \item    Montrer que $x_1(\cdot)$ définie par $x_1(\epsilon)$ est égal à la racine de $P_{\epsilon}$ la plus proche de $x_1$ est bien définie pour $\epsilon$ assez petit et constitue un $\mathcal{C}^1$ difféomorphisme entre deux ouverts à préciser. Même raisonnement pour $x_2(\cdot)$.
    \item Lorsque $x_1(\cdot)$ et $x_2(\cdot)$ sont bien définies on note $x_3(\cdot)$ la troisième racine de $P_{\epsilon}$. Montrer que $x_3(\cdot)$ est réelle pour $\epsilon$ assez petit.
      \item Donner un développement limité/asymptotique à l'ordre un de chacune de ces fonctions.
    \end{questions}
\end{exo}
\vskip2ex
\begin{exo}[Système non linéaire]
On considère le système d'équations suivant sur $\mathbb{R^3}$
\[
(S)\left\lbrace
\begin{aligned}
& x^3 + y^3 + z^3 + t^2 = 0 \\
& x^2 + y^2 + z^2 + t = 2 \\
& x + y + z + t = 0 \\
\end{aligned}
\right.
\]
\begin{questions}
\item Montrer que le système admet une unique solution $(x,y,z,t)$ pour $t$ assez petit et $(x,y,z)$ proche de $(0,-1,1)$ on note alors $f(t) = (x,y,z)$.
\item Calculer la dérivée de $f$ en $0$.
\end{questions}
\end{exo}

\end{document}
%%% Local Variables:
%%% mode: latex
%%% TeX-master: t
%%% End:
