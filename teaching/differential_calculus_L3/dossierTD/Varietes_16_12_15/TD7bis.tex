\documentclass[12pt]{article}
%\usepackage[frenchb]{babel}
\usepackage[T1]{fontenc}    
%\usepackage[latin1]{inputenc}
\usepackage{graphicx,amssymb,amsmath} % extensions pour maths avancées
\usepackage{graphicx}              % Accents codés dans la fonte.
\usepackage[utf8x]{inputenc}           % Accents 8 bits dans le fichier.
\usepackage{ifthen}                     % Faire des tests if/then/else.
\usepackage{amsmath}    % Les symboles les plus fréquents.
\usepackage{amsfonts}   % Des fontes, eg pour \mathbb.
\usepackage{vmargin}                    % Régler la taille de la feuille.
\usepackage{amssymb}
\usepackage{lastpage}
\usepackage{accents}
\usepackage{enumitem}

% La taille de la page
\setpapersize{A4}
\setmarginsrb{20mm}{10mm}{14mm}{15mm}{10mm}{6mm}{0mm}{10mm}

% La mise en page
\newcounter{vcenterstest}
\newlength{\leftlength}
\newlength{\rightlength}
\newlength{\vcentersskip}
\newcommand{\leftcentersright}[4][2]{%
        \settowidth{\leftlength}{#2}%
        \settowidth{\rightlength}{#4}%
        \setcounter{vcenterstest}{#1}%
        \ifthenelse{\value{vcenterstest} = 0}
                {\setlength{\vcentersskip}{0pt}}{}%
        \ifthenelse{\value{vcenterstest} = 1}
                {\setlength{\vcentersskip}{\smallskipamount}}{}%
        \ifthenelse{\value{vcenterstest} = 2}
                {\setlength{\vcentersskip}{\medskipamount}}{}%
        \ifthenelse{\value{vcenterstest} = 3}
                {\setlength{\vcentersskip}{\bigskipamount}}{}%
        \ifthenelse{\value{vcenterstest} = 4}
                {\setlength{\vcentersskip}{1cm}}{}%
                % On laisse un espace vertical défini par l'argument
                % optionnel #1
        \vskip\vcentersskip
                % On place #2 et on recule de sa longueur
        \noindent#2\hskip-\leftlength%
                % On centre #3
        \hfill#3\hfill%
                % On va au bout de la ligne, on recule de la longueur de #4 et
                % on place #4
        \mbox{}\hskip-\rightlength#4%
                % On laisse un espace vertical défini par l'argument
                % optionnel #1
        \vskip\vcentersskip%
    }
\newcommand{\centers}[2][2]{\leftcentersright[#1]{}{#2}{}}
\newcommand{\leftcenters}[3][2]{\leftcentersright[#1]{#2}{#3}{}}
\newcommand{\centersright}[3][2]{\leftcentersright[#1]{}{#2}{#3}}
\newcommand{\leftencadreright}[4][2]{\leftcentersright[#1]{#2}{\fbox{#3}}{#4}}

\newcommand{\numtoi}[1]{
  \ifthenelse{ \equal{#1}{1} }{i}{
  \ifthenelse{ \equal{#1}{2} }{ii}{
  \ifthenelse{ \equal{#1}{3} }{iii}{
  \ifthenelse{ \equal{#1}{4} }{iv}{
  \ifthenelse{ \equal{#1}{5} }{v}{
  \ifthenelse{ \equal{#1}{6} }{vi}{
  \ifthenelse{ \equal{#1}{7} }{vii}{
  \ifthenelse{ \equal{#1}{8} }{viii}{
  \ifthenelse{ \equal{#1}{9} }{ix}{
  \ifthenelse{ \equal{#1}{10} }{x}{
  \ifthenelse{ \equal{#1}{11} }{xi}{
  \ifthenelse{ \equal{#1}{12} }{xii}{
  \ifthenelse{ \equal{#1}{13} }{xiii}{
  \ifthenelse{ \equal{#1}{14} }{xiv}{
  \ifthenelse{ \equal{#1}{15} }{xv}{
  \ifthenelse{ \equal{#1}{16} }{xvi}{
  \ifthenelse{ \equal{#1}{17} }{xvii}{
  \ifthenelse{ \equal{#1}{18} }{xviii}{
    ERREUR,~MODIFIER~LA~MACRO~numtoi
  } } } } } } } } } } } } } } } } } }
}

\newcounter{cretraiti}
\newenvironment{ri}[1]{

\begin{list}{($\numtoi{\thecretraiti}$)}{
  \usecounter{cretraiti}
  \topsep=0.5ex
  \itemsep=0.3ex
  \labelsep=0.3em
  \parsep=0ex
  \listparindent=1em
  \settowidth{\labelwidth}{(#1)}
  \leftmargin=\labelwidth
  %\addtolength{\leftmargin}{\labelsep}
}}{\end{list}
}


% Quelques raccourcis pour taper des maths
\newcommand{\Sfr}{\mathfrak{S}}
\newcommand{\Afr}{\mathfrak{A}}
\newcommand{\QQ}{\mathbb{Q}}
\newcommand{\ZZ}{\mathbb{Z}}
\newcommand{\si}{\sigma}
\newcommand{\ins}[1]{{\ensuremath{\textrm{#1}}}}
\newcommand{\tiret}{\rule[0.6ex]{1.3ex}{0.26ex}}
\newcommand{\accolades}[1]{\ensuremath{\left\{#1\right\}}}
\newcommand{\paa}[1]{\ensuremath{\accolades{#1}}}       % Synonyme
\newcommand{\ds}{\displaystyle}
   \newcommand{\der}{
      \mathop{}\mathopen{}\mathrm{d}
   }

% Numerotation des pages / la derniere
\makeatletter
\renewcommand{\@evenfoot}%
        {\thepage \slash \pageref{LastPage}}
\renewcommand{\@oddfoot}{\@evenfoot}
\makeatother


%%%%% Pour faire les exos
\newcommand{\envfont}{\sf\bfseries}
\newcounter{cexo}
\renewcommand{\thecexo}{\arabic{cexo}}

\newenvironment{exo}[1][toto]
{
  \refstepcounter{cexo}
  \ifthenelse{ \equal{#1}{toto} }{
   \noindent {\envfont Exercice \thecexo}
  }{
   \noindent {\envfont Exercice \thecexo{} \tiret\, #1.}
  }
%\newline
%\indent
}
{}

\newcounter{cquestions}
\newenvironment{questions}{
\begin{list}{{\envfont\alph{cquestions})}}{
  \usecounter{cquestions}
  \topsep=0.5ex
  \partopsep=1em
  \labelsep=0.3em
  \itemsep=0.3ex
  \settowidth{\labelwidth}{{\envfont b)}}\listparindent=1em
  \leftmargin=\labelwidth \addtolength{\leftmargin}{\labelsep}
}}{\end{list} }

\newcommand{\poubelle}[1]{}

%%%%%%%%%%%%%%%%%%%%%%%%%%%%%%%%%%%%%%%%%
\newcommand{\spacebeforeenv}{\vspace{1ex}}
\newcommand{\spaceafterenv}{\vspace{1ex}}
\newcommand{\myqedenv}{}


\newcommand{\metaenvironnement}[2]  % 1er argument : nom de l'env, 2e argument : titre
{
  \ifthenelse{ \equal{#2}{toto} }{
    \spacebeforeenv \noindent {\envfont #1 }
  }{
    \spacebeforeenv \noindent {\envfont #1  \tiret\, #2.}
  }
}

%----------------------------------
% Rappel
%----------------------------------
\newenvironment{rappel}[1][toto]
{ \metaenvironnement{Rappel}{#1} }
{\myqedenv\spaceafterenv}


\newcommand{\fonction}[5]{%
        \ensuremath{#1\colon
        \left\{\hskip -1.5 mm
        \begin{array}{c@{\ }c@{\ }l}
        \medskip #2 & \longrightarrow & #3 \\
        #4 & \longmapsto & #5 \\
        \end{array}
        \right .
        }}
\newcommand{\ph}{\varphi}
\newcommand{\funu}[1]{\mathcal{C}^{#1}}
\newcommand{\RR}{\mathbb{R}}
\newcommand{\NN}{\mathbb{N}}
\newcommand{\CC}{\mathbb{C}}

\newenvironment{syst}[1][c]%
    {\ensuremath{\left \{ \hskip -1.5 mm \begin{array}{#1@{\ =\ }l}}}%
    {\end{array}\right.}

\newcommand{\f}[2]{{\ensuremath{\mathchoice%
    {\dfrac{#1}{#2}}
        {\dfrac{#1}{#2}}
    {\frac{#1}{#2}}
    {\frac{#1}{#2}}
    }}}
\renewcommand{\ni}{\noindent}
\newcommand{\ra}{\rightarrow}
\newcommand{\question}[1]{{\sf\bfseries #1}}
\newcommand{\norm}[1]{\ensuremath{\Arrowvert #1 \Arrowvert}}
\newcommand{\Ende}[3]{\ensuremath{\mathrm{End}_{#1}^{#2}(#3)}}
\newcommand{\End}[2]{\Ende{#1}{}{#2}}
\newcommand{\id}{\ensuremath{\mathrm{id}}}
\renewcommand{\d}{\ensuremath{\mathrm{d}}}
\renewcommand{\leq}{\leqslant}
\renewcommand{\geq}{\geqslant}


\usepackage{datetime}
\newdateformat{annscol}{\ifthenelse{\THEMONTH > 7}{\THEYEAR--\refstepcounter{YEAR}\THEYEAR}{\addtocounter{YEAR}{-1}\THEYEAR--\refstepcounter{YEAR}\THEYEAR}}

\begin{document}
 

%------------------------------------------------------------------------------------
\leftcentersright[0]{{\bf\sf\large Formation en Mathématiques Commune Cachan / P7}}{}
    {{\bf\sf\large Année~\annscol\today}}
\leftcentersright[0]{{\bf\sf\large Calcul différentiel}}{}{{\bf\sf Barbara Gris, Frédéric Pascal}}

%------------------------------------------------------------------------------------------




%------------------------------------------------------------------------------------------
%------------------------------------------------------------------------------------------


\begin{center}
\huge{TD 7 bis \\ Sous-variétés}
\end{center}



\begin{exo}

Soient deux surfaces $\mathcal{S}_{1}$ et $\mathcal{S}_{2}$ de l'espace $\RR^{3}$ données par les équations implicites $F(x,y,z)=0$ et $G(x,y,z)=0$ où $F$ et $G$ sont deux applications $\RR^{3}\rightarrow \RR$ de classe $C^{1}$. On suppose que ces deux surfaces s'intersectent en un point $a \in \RR^{3}$ tel qu'on ait également $\overrightarrow{grad} \ F(a) \wedge \overrightarrow{grad} \ G(a) \neq 0$. Montrer qu'au voisinage de $a$, l'intersection des deux surfaces est une sous-variété de dimension 1. Interpréter géométriquement l'espace affine tangent de l'intersection. Examiner le cas où $\overrightarrow{grad} \ F(a) \wedge \overrightarrow{grad} \ G(a) = 0$ (avec toujours $\overrightarrow{grad} \ F(a) \neq 0$ et $\overrightarrow{grad} \ G(a) \neq0$).

\end{exo}

\vspace{0.5cm}

\begin{exo}

On pose $M$ l'ensemble des points de $\RR^3$ tel qu'il existe $t \in [-1 , 1]$ et $w \in [0 , 2 \pi]$ tel que 
\begin{equation*}  
 \left\lbrace
\begin{array}{ccl}
x & = & (1 + \frac{t}{2} \cos \frac{w}{2}) \cos w \\
y & = & (1 + \frac{t}{2} \cos \frac{w}{2} ) \sin w \\
z & = & \frac{t}{2} \sin \frac{w}{2} \\
\end{array}
 \right.
\end{equation*}  


Quel est cet ensemble ?  Montrer que c'est une sous-variété.


\end{exo}

\vspace{0.5cm}
\begin{exo}

Montrer que $P = \{  M \in M_2 (\RR) \quad | \quad M \neq 0, \quad M \neq I_n , \quad M^2 = M  \}$ est une sous-variété.

\end{exo}


\vspace{0.5cm}

\begin{exo}

Soit $n \geq 1$ un entier. On identifie $\RR _n [X]$ à $\RR ^{n+1}$.
\begin{questions}
\item Montrer que l'ensemble $E$ des polynômes ayant une unique racine, avec multiplicité $n$, est une sous-variété $C^1$, et indiquer sa dimension.
\item Montrer que, si $n \geq 2$, l'adhérence de $E$ n'est pas une sous-variété.
\end{questions}

\end{exo}


\poubelle{
\vspace{0.5cm}
\begin{exo}
Soit $G$ un sous-ensemble de $\RR^{k}$. On suppose que $G$ est muni d'une structure de groupe dont on note $e$ l'élément neutre et $\times$ la loi. On dira que $G$ est un \textbf{groupe de Lie} lorsque $G$ est en plus une sous-variété de $\RR^{k}$ et que les applications :
\begin{eqnarray*}
 G \times G &\rightarrow& G \hspace{2cm} G \rightarrow G \\
(g,h) &\mapsto& g\times h \hspace{1.5cm}   g \mapsto g^{-1}
\end{eqnarray*}
sont de classe $C^{1}$ au sens où ce sont les restrictions à $G \times G$ (resp $G$) d'applications de classe $C^{1}$ définies sur $\RR^{k} \times \RR^{k}$ (resp $\RR^{k}$). On appelle alors algèbre de Lie l'espace $T_{e}G$. Montrer que les ensembles suivants sont des sous-variétés de $M_{n}(\RR)$ puis des groupes de Lie en prenant pour loi la multiplication matricielle. Préciser les algèbres de Lie.

\begin{questions}
\item $GL_{n}(\RR)$.
\item $SL_{n}(\RR)=\{M \in M_{n}(\RR) / det(M)=1 \}$.
\item $O_{n}(\RR)=\{M \in M_{n}(\RR) / M^{T}M=I_{n} \}$.

\poubelle{
\Question Groupe pseudo-orthogonal : pour une forme quadratique $Q(x)= \sum_{i=1}^{p} x_{i}^{2} - \sum_{i=p+1}^{n} x_{i}^{2}$ avec $p \in \{0,...,n\}$, c'est l'ensemble des matrices $A \in M_{n}(\RR)$ telles que $\forall x \in \RR^{n}$, $Q(Ax)=Q(x)$ (pour $p=n$, on retombe sur $O_{n}(\RR)$). Dans le cas $n=4$ et $p=3$, il s'agit du \textit{groupe de Lorentz} de la relativité restreinte.
\textit{Remarque :} on peut en fait prouver plus généralement que tout sous-groupe fermé et non discret de $GL_{n}(\RR)$ est un groupe de Lie (théorème du à Cartan et Von-Neumann). }

\end{questions}
\end{exo}
}

\vspace{0.5cm}

\begin{exo}

On note $S^2$ la sphère de $\RR ^3$ et $F : \RR ^3 \mapsto \RR$ définie par $F (x_1 , x_2  , x_3 )= (x_1 - \frac{1}{2} ) ^2 + x_2^2$.

\begin{questions}
\item Montrer que pour tout $c \in \RR$, $N_c = F^{-1} (c)$ est soit une sous-variété soit vide.

\item En utilisant les coordonnées cylindriques $: (r,u) \mapsto (\frac{1}{2} + \frac{1}{2}  \cos u , \frac{1}{2}  \sin u , r)$, trouver un arc paramétré $C ^\infty$ régulier $(I, \gamma)$ tel que $\gamma (I) = S^2 \cap N_{\frac{1}{4}}$.

\item Est-ce que $S^2 \cap N_{\frac{1}{4}}$ est une sous-variété ?

\item Quel sous-ensemble maximal de $S^2 \cap N_{\frac{1}{4}}$ est une sous-variété ? 
\end{questions}
\end{exo}

\vspace{0.5cm}
\begin{exo}

Le but de cet exercice est de montrer que la sphère $S^n$ possède des champs de vecteurs $C^\infty$ ne s'annulant jamais $ssi$ $n$ est impair.

\begin{questions}
\item Quel est l'espace tangent à $S^n$ en un point $x \in S^n $ ? 

\item Construire un champ de vecteur $C ^\infty $ sur $S^n$ s'annulant en un seul point de $S^n$.

\item Si $n = 2 p +1$ construire un champ de vecteurs $C^\infty $ sur $S^n$ ne s'annulant jamais.

\item Soient $K$ une partie compacte de $\RR ^{n+1}$, $U$ un ouvert de $\RR ^n{n+1}$ contenant $K$ et $v$ une application $C^\infty$ de $U$ dans $\RR ^{n+1}$. Pour $t \in \RR$ on pose $F_t : U \mapsto \RR^{n+1}$ définie par $F_t (x) = x + t v(x)$. Montrer qu'il existe un ouvert $V$ de $U$ contenant $K$ et $\epsilon > 0 $ tels que pour tout $|t| \leq \epsilon$, $F_t$ est un $C^1$ difféomorphisme de $V$ sur son image. En déduire que la mesure de Lebesgue de $F_t (K)$ est alors un polynôme en $t$.

\item On suppose qu'il existe un champ $v$ de vecteurs unitaires sur $S^n$. On pose à nouveau pour $t \in \RR$ et $x \in S^n \subset \RR^{n+1}$, $F_t (x) = x + t v(x)$. Montrer que pour $t$ suffisamment petit, $F_t$ est un difféomorphisme entre $S^n$ et la sphère de rayon $\sqrt{1+t^2}$. Conclure que $n$ est impaire.
\end{questions}
\end{exo}


\end{document}