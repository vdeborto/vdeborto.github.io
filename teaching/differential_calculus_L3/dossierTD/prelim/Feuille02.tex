\documentclass[12pt]{article}
%\usepackage[frenchb]{babel}
\usepackage[T1]{fontenc}                % Accents codés dans la fonte.
\usepackage[utf8x]{inputenc}           % Accents 8 bits dans le fichier.
\usepackage{ifthen}                     % Faire des tests if/then/else.
\usepackage{amsmath}    % Les symboles les plus fréquents.
\usepackage{amsfonts}   % Des fontes, eg pour \mathbb.
\usepackage{vmargin}                    % Régler la taille de la feuille.
\usepackage{amssymb}
\usepackage{lastpage}
\usepackage{accents}


% La taille de la page
\setpapersize{A4}
\setmarginsrb{20mm}{10mm}{14mm}{15mm}{10mm}{6mm}{0mm}{10mm}

% La mise en page
\newcounter{vcenterstest}
\newlength{\leftlength}
\newlength{\rightlength}
\newlength{\vcentersskip}
\newcommand{\leftcentersright}[4][2]{%
        \settowidth{\leftlength}{#2}%
        \settowidth{\rightlength}{#4}%
        \setcounter{vcenterstest}{#1}%
        \ifthenelse{\value{vcenterstest} = 0}
                {\setlength{\vcentersskip}{0pt}}{}%
        \ifthenelse{\value{vcenterstest} = 1}
                {\setlength{\vcentersskip}{\smallskipamount}}{}%
        \ifthenelse{\value{vcenterstest} = 2}
                {\setlength{\vcentersskip}{\medskipamount}}{}%
        \ifthenelse{\value{vcenterstest} = 3}
                {\setlength{\vcentersskip}{\bigskipamount}}{}%
        \ifthenelse{\value{vcenterstest} = 4}
                {\setlength{\vcentersskip}{1cm}}{}%
                % On laisse un espace vertical défini par l'argument
                % optionnel #1
        \vskip\vcentersskip
                % On place #2 et on recule de sa longueur
        \noindent#2\hskip-\leftlength%
                % On centre #3
        \hfill#3\hfill%
                % On va au bout de la ligne, on recule de la longueur de #4 et
                % on place #4
        \mbox{}\hskip-\rightlength#4%
                % On laisse un espace vertical défini par l'argument
                % optionnel #1
        \vskip\vcentersskip%
    }
\newcommand{\centers}[2][2]{\leftcentersright[#1]{}{#2}{}}
\newcommand{\leftcenters}[3][2]{\leftcentersright[#1]{#2}{#3}{}}
\newcommand{\centersright}[3][2]{\leftcentersright[#1]{}{#2}{#3}}
\newcommand{\leftencadreright}[4][2]{\leftcentersright[#1]{#2}{\fbox{#3}}{#4}}

\newcommand{\poubelle}[1]{}
\newcommand{\numtoi}[1]{
  \ifthenelse{ \equal{#1}{1} }{i}{
  \ifthenelse{ \equal{#1}{2} }{ii}{
  \ifthenelse{ \equal{#1}{3} }{iii}{
  \ifthenelse{ \equal{#1}{4} }{iv}{
  \ifthenelse{ \equal{#1}{5} }{v}{
  \ifthenelse{ \equal{#1}{6} }{vi}{
  \ifthenelse{ \equal{#1}{7} }{vii}{
  \ifthenelse{ \equal{#1}{8} }{viii}{
  \ifthenelse{ \equal{#1}{9} }{ix}{
  \ifthenelse{ \equal{#1}{10} }{x}{
  \ifthenelse{ \equal{#1}{11} }{xi}{
  \ifthenelse{ \equal{#1}{12} }{xii}{
  \ifthenelse{ \equal{#1}{13} }{xiii}{
  \ifthenelse{ \equal{#1}{14} }{xiv}{
  \ifthenelse{ \equal{#1}{15} }{xv}{
  \ifthenelse{ \equal{#1}{16} }{xvi}{
  \ifthenelse{ \equal{#1}{17} }{xvii}{
  \ifthenelse{ \equal{#1}{18} }{xviii}{
    ERREUR,~MODIFIER~LA~MACRO~numtoi
  } } } } } } } } } } } } } } } } } }
}

\newcounter{cretraiti}
\newenvironment{ri}[1]{

\begin{list}{($\numtoi{\thecretraiti}$)}{
  \usecounter{cretraiti}
  \topsep=0.5ex
  \itemsep=0.3ex
  \labelsep=0.3em
  \parsep=0ex
  \listparindent=1em
  \settowidth{\labelwidth}{(#1)}
  \leftmargin=\labelwidth
  %\addtolength{\leftmargin}{\labelsep}
}}{\end{list}
}


% Quelques raccourcis pour taper des maths
\newcommand{\Sfr}{\mathfrak{S}}
\newcommand{\Afr}{\mathfrak{A}}
\newcommand{\QQ}{\mathbb{Q}}
\newcommand{\ZZ}{\mathbb{Z}}
\newcommand{\si}{\sigma}
\newcommand{\ins}[1]{{\ensuremath{\textrm{#1}}}}
\newcommand{\tiret}{\rule[0.6ex]{1.3ex}{0.26ex}}
\newcommand{\accolades}[1]{\ensuremath{\left\{#1\right\}}}
\newcommand{\paa}[1]{\ensuremath{\accolades{#1}}}       % Synonyme
\newcommand{\ds}{\displaystyle}


% Numerotation des pages / la derniere
\makeatletter
\renewcommand{\@evenfoot}%
        {\thepage \slash \pageref{LastPage}}
\renewcommand{\@oddfoot}{\@evenfoot}
\makeatother


%%%%% Pour faire les exos
\newcommand{\envfont}{\sf\bfseries}
\newcounter{cexo}
\renewcommand{\thecexo}{\arabic{cexo}}

\newenvironment{exo}[1][toto]
{
  \refstepcounter{cexo}
  \ifthenelse{ \equal{#1}{toto} }{
   \noindent {\envfont Exercice \thecexo}
  }{
   \noindent {\envfont Exercice \thecexo{} \tiret\, #1.}
  }
%\newline
%\indent
}
{}

\newcounter{cquestions}
\newenvironment{questions}{
\begin{list}{{\envfont\alph{cquestions})}}{
  \usecounter{cquestions}
  \topsep=0.5ex
  \partopsep=1em
  \labelsep=0.3em
  \itemsep=0.3ex
  \settowidth{\labelwidth}{{\envfont b)}}\listparindent=1em
  \leftmargin=\labelwidth \addtolength{\leftmargin}{\labelsep}
}}{\end{list} }


%%%%%%%%%%%%%%%%%%%%%%%%%%%%%%%%%%%%%%%%%
\newcommand{\spacebeforeenv}{\vspace{1ex}}
\newcommand{\spaceafterenv}{\vspace{1ex}}
\newcommand{\myqedenv}{}


\newcommand{\metaenvironnement}[2]  % 1er argument : nom de l'env, 2e argument : titre
{
  \ifthenelse{ \equal{#2}{toto} }{
    \spacebeforeenv \noindent {\envfont #1 }
  }{
    \spacebeforeenv \noindent {\envfont #1  \tiret\, #2.}
  }
}

%----------------------------------
% Rappel
%----------------------------------
\newenvironment{rappel}[1][toto]
{ \metaenvironnement{Rappel}{#1} }
{\myqedenv\spaceafterenv}


\newcommand{\fonction}[5]{%
        \ensuremath{#1\colon
        \left\{\hskip -1.5 mm
        \begin{array}{c@{\ }c@{\ }l}
        \medskip #2 & \longrightarrow & #3 \\
        #4 & \longmapsto & #5 \\
        \end{array}
        \right .
        }}
\newcommand{\ph}{\varphi}
\newcommand{\funu}[1]{\mathcal{C}^{#1}}
\newcommand{\RR}{\mathbb{R}}
\newcommand{\NN}{\mathbb{N}}
\newcommand{\CC}{\mathbb{C}}

\newenvironment{syst}[1][c]%
    {\ensuremath{\left \{ \hskip -1.5 mm \begin{array}{#1@{\ =\ }l}}}%
    {\end{array}\right.}

\newcommand{\f}[2]{{\ensuremath{\mathchoice%
    {\dfrac{#1}{#2}}
        {\dfrac{#1}{#2}}
    {\frac{#1}{#2}}
    {\frac{#1}{#2}}
    }}}
\renewcommand{\ni}{\noindent}
\newcommand{\ra}{\rightarrow}
\newcommand{\question}[1]{{\sf\bfseries #1}}
\newcommand{\norm}[1]{\ensuremath{\Arrowvert #1 \Arrowvert}}
\newcommand{\Ende}[3]{\ensuremath{\mathrm{End}_{#1}^{#2}(#3)}}
\newcommand{\End}[2]{\Ende{#1}{}{#2}}
\newcommand{\id}{\ensuremath{\mathrm{id}}}
\renewcommand{\d}{\ensuremath{\mathrm{d}}}
\renewcommand{\leq}{\leqslant}
\renewcommand{\geq}{\geqslant}


\usepackage{datetime}
\newdateformat{annscol}{\ifthenelse{\THEMONTH > 7}{\THEYEAR--\refstepcounter{YEAR}\THEYEAR}{\addtocounter{YEAR}{-1}\THEYEAR--\refstepcounter{YEAR}\THEYEAR}}

\begin{document}
 

%------------------------------------------------------------------------------------
\leftcentersright[0]{{\bf\sf\large Formation en Mathématiques Commune Cachan / P7}}{}
    {{\bf\sf\large Année~\annscol\today}}
\leftcentersright[0]{{\bf\sf\large Calcul différentiel}}{}{{\bf\sf Mariano Rodríguez, Frédéric Pascal}}
\vskip3ex\centers{{\bf\sf \Large Feuille 0.2}}
%------------------------------------------------------------------------------------------




%------------------------------------------------------------------------------------------
%------------------------------------------------------------------------------------------


\vskip2ex 
\begin{exo}[Ensembles compacts dans les espaces métriques]
Soit $(E,d)$ un espace métrique.
\begin{questions}
\item Soit $A \subset E$ un sous-ensemble de $E$. Montrer que $A$ vérifie la propriété de Bolzano Weiestrass ($ie$ toute suite dans $A$ a au moins 
une valeur d'adhérence dans $A$) $ssi$ il vérifie la propriété de Borel Lebesgue, $ie$ si $(U_i)_{i \in I}$ est une famille d'ouverts de $E$ telle que $\cup_i U_i \cap A = A$ alors il existe $i_1, \cdots i_N \in I$ tels que  $\cup_{k=1}^N U_{i_k} \cap A = A$. Dans ce cas on dit que l'ensemble $A$ est compact.
\item Montrer que toute suite d'un compact $A$ ayant une seule valeur 
d'adhérence converge.
\item Montrer qu'un sous ensemble  $A$ compact de $E$ est fermé et borné. On notera que dans $E=\RR^m$ où $(a_n)_n$ converge vers $a$ si et seulement si les
composantes $(a^i_n)_n$ convergent dans $\RR$ vers les composantes $a^i$, les fermés bornés sont des compacts.  

\end{questions}
\end{exo}




\vskip2ex 
\begin{exo}[Fonctions définies sur un compact]
Soient $(E,d)$ et $(E',d')$ deux espaces métriques, et soit $f : E \longrightarrow E'$.\\
\textit{Définitions} : 
 \begin{itemize}
\item On dit que $f$ est un homéomorphisme de $E$ sur $E'$ si $f$ est inversible continue et $f^{-1}$ est également continue.
\item On dit que $f$ est ouverte si l'image de tout ouvert de $E$ par $f$ est un ouvert de $E'$. 
\item On dit que $f : E \to E'$ est uniformément continue sur $E$
si pour tout $\epsilon >0$, il existe $\eta >0$ tel que $d(x,y) \leq \eta$ entraîne $d'(f(x),f(y))\leq \epsilon$ 
pour tout $x$ et $y$.
\end{itemize}
\begin{questions}
\item Montrer que si $f$ est continue  sur $K$ compact de $E$ alors $f(K)$ est compact et $f$ est uniformément continue sur $K$.
\textit{Remarque:} En particulier une application numérique (à valeur dans $\RR$) définie sur un compact est bornée et atteint ses bornes.
\item Montrer que si $f$ est bijective et continue sur un compact $K$ alors $f$ est une application ouverte de $K$ sur $f(K)$, et donc un homéomorphisme de $K$ sur $f(K)$.
\end{questions}

\end{exo}






\vskip2ex 
\begin{exo}[Espace vectoriel normé]
Soit $(E,\|.\|)$ un evn défini sur $\RR$ ou $\CC$. $E$ est un espace métrique pour la distance associée à la norme.
\begin{questions}
\item Montrer que l'application qui à $x \in E$ associe $\|x\|$ est continue sur $E$.
\item Soit $ A \subset \RR^n$ fermé et non bornée. Montrer que si 
$f : A \to \RR$ est continue et coercive i.e. $\lim_{\|x\|\to + \infty} f(x) = + \infty$ alors $f$ est minorée et atteint sa 
borne inférieure sur $A$.
\item Montrer que $E$ est de dimension finie $ssi$ sa boule unité est compacte (théorème de Riesz).
\end{questions}
\end{exo}


\vskip2ex 
\begin{exo}[Espace de Banach]
Un evn complet est appelé espace de Banach.
\begin{questions}
\item Soit $(a_n)_n$ une suite de $E$ espace de Banach, dont la série est absolument convergente. 
Montrer que cette série converge dans $E$ et que l'on $\|\sum_{n=0}^{+ \infty} a_n \| \leq \sum_{n=0}^{+ \infty} \|a_n\|$.
\item Soient $E$ un espace de Banach et $A$ un ensemble quelconque non vide. Montre que 
${\cal B}=\{ f : A \to E \, / \, f \, \mbox{bornée} \}$ muni de la norme sup (norme de la convergence uniforme) est un espace de Banach.
\item Soient $E$ un espace de Banach et $A$ un ensemble quelconque non vide. Montre que 
${\cal C}_b=\{ f : A \to E \, / \, f \, \mbox{bornée et continue} \}$   est un sous espace de Banach dans ${\cal B}$. On notera que si 
$A$ est compact, il coincide avec ${\cal C}(A,E)= \{ f : A \to E \, / \, f \, \mbox{continue} \}$.
\item Montrer que ${\cal C}([0,1],\RR)$ ensemble des fonctions continues de $[0,1]$ vers $\RR$, 
muni de la norme $\int_0^1 [f(t)|dt$ n'est pas un
espace de Banach en prenant par exemple $f_n(x)= 0$ sur $[0,\frac{1}{2}]$, $f_n(x)=1$ sur $[\frac{1}{2}+\frac{1}{n},1]$ et 
linéaire sur $[\frac{1}{2}, \frac{1}{2}+\frac{1}{n}]$.
\end{questions}
\end{exo}


\vskip2ex 
\begin{exo}[Fonctions linéaires continues dans les evn]
Soient  $(E,\|.\|_E)$ et  $(F,\|.\|_F)$ deux evn. 
\begin{questions}
\item Soit $f : E \to F$ linéaire. Montrer les assertions suivantes sont équivalentes : i) $f$ est continue en 
un point de $E$, ii) $f$ est continue sur $E$, iii) $f$ est bornée sur toute partie bornée de $E$, iv) il existe une constante $M>0$ telle que $\|f(x)\|_F \leq M \|x\|_E$ pour tout 
$x$ dans $E$.
\item Montrer que si $f$ est une forme linéaire non nulle alors $f$ est continue si et seulement si $f^{-1}(0)$ 
est fermé.
\item Montrer que si $F$ est un espace de Banach, l'ensemble ${\cal L}(E,F)$ des applications linéaires continues de $E$ dans $F$ est 
un espace de Banach pour la norme subordonnée.  \\
\textit{Rappel} : Si $f \in {\cal L}(E,F)$, sa norme subordonnée est $$|f| = \sup \{ |f(x)|_F \quad | \quad x\in S_E (0,1) \} = \sup \{ \frac{|f(x)|_F}{|x|_E} \quad | \quad x\in E-\{0\} \}$$
\item Soit $p: E \to \RR^+$ une semi-norme vérifiant donc $p(x+y) \leq p(x)+p(y)$ ; $p(\lambda x) = |\lambda| p(x)$ et $p(0)=0$. Montrer 
que $p$ est continue sur $(E,\|.\|)$ si et seulement si il existe $M>0$ telle que pour tout $x$ de $E$, $$p(x) \leq M \|x\|.$$ 
En déduire que toute 
semi-norme $p$ sur $\RR^n$ muni de la norme euclidienne est continue et par suite que toutes les normes sur $\RR^n$  sont équivalentes et 
que toutes les applications linéaires définies sur $\RR^n$ et à valeur dans 
$F$ sont continues.
\item Soit $E$ un espace vectoriel de dimension finie muni de 2 normes $\|.\|_1$ et $\|.\|_2$. Montrer que les boules unités pour ces 2 
normes sont homéomorphes (on pourra utiliser une fonction qui à $x$ associe $\lambda(x) x$ avec $  \lambda(x) \geq 0$). 
\item En déduire en particulier que dans $\RR^2$ le disque unité est le carré unité sont homéomorphes et définir l'application homéomorphe.
\end{questions}
\end{exo}



\end{document}