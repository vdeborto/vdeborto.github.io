\documentclass[12pt]{article}
%\usepackage[frenchb]{babel}
\usepackage[T1]{fontenc}    
%\usepackage[latin1]{inputenc}
\usepackage{graphicx,amssymb,amsmath} % extensions pour maths avancées
\usepackage{graphicx}              % Accents codés dans la fonte.
\usepackage[utf8x]{inputenc}           % Accents 8 bits dans le fichier.
\usepackage{ifthen}                     % Faire des tests if/then/else.
\usepackage{amsmath}    % Les symboles les plus fréquents.
\usepackage{amsfonts}   % Des fontes, eg pour \mathbb.
\usepackage{vmargin}                    % Régler la taille de la feuille.
\usepackage{amssymb}
\usepackage{lastpage}
\usepackage{accents}
\usepackage{enumitem}

% La taille de la page
\setpapersize{A4}
\setmarginsrb{20mm}{10mm}{14mm}{15mm}{10mm}{6mm}{0mm}{10mm}

% La mise en page
\newcounter{vcenterstest}
\newlength{\leftlength}
\newlength{\rightlength}
\newlength{\vcentersskip}
\newcommand{\leftcentersright}[4][2]{%
        \settowidth{\leftlength}{#2}%
        \settowidth{\rightlength}{#4}%
        \setcounter{vcenterstest}{#1}%
        \ifthenelse{\value{vcenterstest} = 0}
                {\setlength{\vcentersskip}{0pt}}{}%
        \ifthenelse{\value{vcenterstest} = 1}
                {\setlength{\vcentersskip}{\smallskipamount}}{}%
        \ifthenelse{\value{vcenterstest} = 2}
                {\setlength{\vcentersskip}{\medskipamount}}{}%
        \ifthenelse{\value{vcenterstest} = 3}
                {\setlength{\vcentersskip}{\bigskipamount}}{}%
        \ifthenelse{\value{vcenterstest} = 4}
                {\setlength{\vcentersskip}{1cm}}{}%
                % On laisse un espace vertical défini par l'argument
                % optionnel #1
        \vskip\vcentersskip
                % On place #2 et on recule de sa longueur
        \noindent#2\hskip-\leftlength%
                % On centre #3
        \hfill#3\hfill%
                % On va au bout de la ligne, on recule de la longueur de #4 et
                % on place #4
        \mbox{}\hskip-\rightlength#4%
                % On laisse un espace vertical défini par l'argument
                % optionnel #1
        \vskip\vcentersskip%
    }
\newcommand{\centers}[2][2]{\leftcentersright[#1]{}{#2}{}}
\newcommand{\leftcenters}[3][2]{\leftcentersright[#1]{#2}{#3}{}}
\newcommand{\centersright}[3][2]{\leftcentersright[#1]{}{#2}{#3}}
\newcommand{\leftencadreright}[4][2]{\leftcentersright[#1]{#2}{\fbox{#3}}{#4}}

\newcommand{\numtoi}[1]{
  \ifthenelse{ \equal{#1}{1} }{i}{
  \ifthenelse{ \equal{#1}{2} }{ii}{
  \ifthenelse{ \equal{#1}{3} }{iii}{
  \ifthenelse{ \equal{#1}{4} }{iv}{
  \ifthenelse{ \equal{#1}{5} }{v}{
  \ifthenelse{ \equal{#1}{6} }{vi}{
  \ifthenelse{ \equal{#1}{7} }{vii}{
  \ifthenelse{ \equal{#1}{8} }{viii}{
  \ifthenelse{ \equal{#1}{9} }{ix}{
  \ifthenelse{ \equal{#1}{10} }{x}{
  \ifthenelse{ \equal{#1}{11} }{xi}{
  \ifthenelse{ \equal{#1}{12} }{xii}{
  \ifthenelse{ \equal{#1}{13} }{xiii}{
  \ifthenelse{ \equal{#1}{14} }{xiv}{
  \ifthenelse{ \equal{#1}{15} }{xv}{
  \ifthenelse{ \equal{#1}{16} }{xvi}{
  \ifthenelse{ \equal{#1}{17} }{xvii}{
  \ifthenelse{ \equal{#1}{18} }{xviii}{
    ERREUR,~MODIFIER~LA~MACRO~numtoi
  } } } } } } } } } } } } } } } } } }
}

\newcounter{cretraiti}
\newenvironment{ri}[1]{

\begin{list}{($\numtoi{\thecretraiti}$)}{
  \usecounter{cretraiti}
  \topsep=0.5ex
  \itemsep=0.3ex
  \labelsep=0.3em
  \parsep=0ex
  \listparindent=1em
  \settowidth{\labelwidth}{(#1)}
  \leftmargin=\labelwidth
  %\addtolength{\leftmargin}{\labelsep}
}}{\end{list}
}


% Quelques raccourcis pour taper des maths
\newcommand{\Sfr}{\mathfrak{S}}
\newcommand{\Afr}{\mathfrak{A}}
\newcommand{\QQ}{\mathbb{Q}}
\newcommand{\ZZ}{\mathbb{Z}}
\newcommand{\si}{\sigma}
\newcommand{\ins}[1]{{\ensuremath{\textrm{#1}}}}
\newcommand{\tiret}{\rule[0.6ex]{1.3ex}{0.26ex}}
\newcommand{\accolades}[1]{\ensuremath{\left\{#1\right\}}}
\newcommand{\paa}[1]{\ensuremath{\accolades{#1}}}       % Synonyme
\newcommand{\ds}{\displaystyle}
   \newcommand{\der}{
      \mathop{}\mathopen{}\mathrm{d}
   }

% Numerotation des pages / la derniere
\makeatletter
\renewcommand{\@evenfoot}%
        {\thepage \slash \pageref{LastPage}}
\renewcommand{\@oddfoot}{\@evenfoot}
\makeatother


%%%%% Pour faire les exos
\newcommand{\envfont}{\sf\bfseries}
\newcounter{cexo}
\renewcommand{\thecexo}{\arabic{cexo}}

\newenvironment{exo}[1][toto]
{
  \refstepcounter{cexo}
  \ifthenelse{ \equal{#1}{toto} }{
   \noindent {\envfont Exercice \thecexo}
  }{
   \noindent {\envfont Exercice \thecexo{} \tiret\, #1.}
  }
%\newline
%\indent
}
{}

\newcounter{cquestions}
\newenvironment{questions}{
\begin{list}{{\envfont\alph{cquestions})}}{
  \usecounter{cquestions}
  \topsep=0.5ex
  \partopsep=1em
  \labelsep=0.3em
  \itemsep=0.3ex
  \settowidth{\labelwidth}{{\envfont b)}}\listparindent=1em
  \leftmargin=\labelwidth \addtolength{\leftmargin}{\labelsep}
}}{\end{list} }

\newcommand{\poubelle}[1]{}

%%%%%%%%%%%%%%%%%%%%%%%%%%%%%%%%%%%%%%%%%
\newcommand{\spacebeforeenv}{\vspace{1ex}}
\newcommand{\spaceafterenv}{\vspace{1ex}}
\newcommand{\myqedenv}{}


\newcommand{\metaenvironnement}[2]  % 1er argument : nom de l'env, 2e argument : titre
{
  \ifthenelse{ \equal{#2}{toto} }{
    \spacebeforeenv \noindent {\envfont #1 }
  }{
    \spacebeforeenv \noindent {\envfont #1  \tiret\, #2.}
  }
}

%----------------------------------
% Rappel
%----------------------------------
\newenvironment{rappel}[1][toto]
{ \metaenvironnement{Rappel}{#1} }
{\myqedenv\spaceafterenv}


\newcommand{\fonction}[5]{%
        \ensuremath{#1\colon
        \left\{\hskip -1.5 mm
        \begin{array}{c@{\ }c@{\ }l}
        \medskip #2 & \longrightarrow & #3 \\
        #4 & \longmapsto & #5 \\
        \end{array}
        \right .
        }}
\newcommand{\ph}{\varphi}
\newcommand{\funu}[1]{\mathcal{C}^{#1}}
\newcommand{\RR}{\mathbb{R}}
\newcommand{\NN}{\mathbb{N}}
\newcommand{\CC}{\mathbb{C}}

\newenvironment{syst}[1][c]%
    {\ensuremath{\left \{ \hskip -1.5 mm \begin{array}{#1@{\ =\ }l}}}%
    {\end{array}\right.}

\newcommand{\f}[2]{{\ensuremath{\mathchoice%
    {\dfrac{#1}{#2}}
        {\dfrac{#1}{#2}}
    {\frac{#1}{#2}}
    {\frac{#1}{#2}}
    }}}
\renewcommand{\ni}{\noindent}
\newcommand{\ra}{\rightarrow}
\newcommand{\question}[1]{{\sf\bfseries #1}}
\newcommand{\norm}[1]{\ensuremath{\Arrowvert #1 \Arrowvert}}
\newcommand{\Ende}[3]{\ensuremath{\mathrm{End}_{#1}^{#2}(#3)}}
\newcommand{\End}[2]{\Ende{#1}{}{#2}}
\newcommand{\id}{\ensuremath{\mathrm{id}}}
\renewcommand{\d}{\ensuremath{\mathrm{d}}}
\renewcommand{\leq}{\leqslant}
\renewcommand{\geq}{\geqslant}


\usepackage{datetime}
\newdateformat{annscol}{\ifthenelse{\THEMONTH > 7}{\THEYEAR--\refstepcounter{YEAR}\THEYEAR}{\addtocounter{YEAR}{-1}\THEYEAR--\refstepcounter{YEAR}\THEYEAR}}

\begin{document}
 

%------------------------------------------------------------------------------------
\leftcentersright[0]{{\bf\sf\large Formation en Mathématiques Commune Cachan / P7}}{}
    {{\bf\sf\large Année~\annscol\today}}
\leftcentersright[0]{{\bf\sf\large Calcul différentiel}}{}{{\bf\sf Mariano Rodríguez, Frédéric Pascal}}

%------------------------------------------------------------------------------------------




%------------------------------------------------------------------------------------------
%------------------------------------------------------------------------------------------


\begin{center}
\huge{TD 7 \\ Sous-variétés}
\end{center}


\vspace{0.5cm}
\begin{exo}
Soit $G$ un sous-ensemble de $\RR^{k}$. On suppose que $G$ est muni d'une structure de groupe dont on note $e$ l'élément neutre et $\times$ la loi. On dira que $G$ est un \textbf{groupe de Lie} lorsque $G$ est en plus une sous-variété de $\RR^{k}$ et que les applications :
\begin{eqnarray*}
 G \times G &\rightarrow& G \hspace{2cm} G \rightarrow G \\
(g,h) &\mapsto& g\times h \hspace{1.5cm}   g \mapsto g^{-1}
\end{eqnarray*}
sont de classe $C^{1}$ au sens où ce sont les restrictions à $G \times G$ (resp $G$) d'applications de classe $C^{1}$ définies sur $\RR^{k} \times \RR^{k}$ (resp $\RR^{k}$). On appelle alors algèbre de Lie l'espace $T_{e}G$. Montrer que les ensembles suivants sont des sous-variétés de $M_{n}(\RR)$ puis des groupes de Lie en prenant pour loi la multiplication matricielle. Préciser les algèbres de Lie.

\begin{questions}
\item $GL_{n}(\RR)$.
\item $SL_{n}(\RR)=\{M \in M_{n}(\RR) / det(M)=1 \}$.
\item $O_{n}(\RR)=\{M \in M_{n}(\RR) / M^{T}M=I_{n} \}$.

\poubelle{
\Question Groupe pseudo-orthogonal : pour une forme quadratique $Q(x)= \sum_{i=1}^{p} x_{i}^{2} - \sum_{i=p+1}^{n} x_{i}^{2}$ avec $p \in \{0,...,n\}$, c'est l'ensemble des matrices $A \in M_{n}(\RR)$ telles que $\forall x \in \RR^{n}$, $Q(Ax)=Q(x)$ (pour $p=n$, on retombe sur $O_{n}(\RR)$). Dans le cas $n=4$ et $p=3$, il s'agit du \textit{groupe de Lorentz} de la relativité restreinte.
\textit{Remarque :} on peut en fait prouver plus généralement que tout sous-groupe fermé et non discret de $GL_{n}(\RR)$ est un groupe de Lie (théorème du à Cartan et Von-Neumann). }

\end{questions}
\end{exo}




\begin{exo}
Montrer que l'ensemble $N = \{ M \in M_2 (\RR) , \quad M \neq 0 , \quad M^2 = 0 \}$ est une sous-variété. Donner sa dimension.
\textit{Remarque: On pourra commencer par chercher une caractérisation de $N$ à l'aide de la trace et du déterminant}

\end{exo}

\vspace{0.5cm}
\begin{exo}
Soient $0 \leq r \leq n $ des entier (avec $n \geq 2$).

\vspace{0.5cm}
\begin{questions}
\poubelle{
\item Montrer que $SL_n (\RR)$ est une sous-variété $C^\infty$.
\item Montrer que l'ensemble des matrices de $M_n (\RR)$ de rang $n-1$ est une sous-variété.
}
\item Montrer qu'il existe un voisinage $U$ de $0$ dans $M_n (\RR)$ tel que si \begin{equation*}  
 \left(
\begin{array}{cc}
A & C \\ B & D \\
\end{array} \right) \in U
\end{equation*} 
alors la matrice  \begin{equation*}  
 \left(
\begin{array}{cc}
I_r + A & C \\ B & D \\
\end{array} \right) 
\end{equation*} 

est de rang $r$ $ssi$ $D = B (I_r + A)^{-1} C$.

\item Soit $V_r \subset M_n (\RR)$ l'ensemble des matrices de rang $r$. Montrer que c'est une sous-variété et donner sa dimension.

%\item Montrer que l'ensemble des matrices de rang $\leq n-1$ ne forme pas une sous-variété.

\item Montrer que l'ensemble des matrices symétriques de rang $r$ forment une sous-variété de l'espace des matrices symétriques. Donner sa dimension.
\poubelle{
\item Montrer que les projecteurs orthogonaux de rang $r$ forment une sous-variété. Donner sa dimension.}

\end{questions}
\end{exo}


\vspace{0.5cm}
\begin{exo}
Soit $\gamma : \ I \rightarrow \RR^{2}$ de classe $C^{1}$ avec $I$ intervalle ouvert de $\RR$ et $\gamma'(t)\neq 0$ pour tout $t \in I$. $\mathcal{C}=\{ \gamma(t) / \ t \in I \}$ définit une courbe paramétrée du plan. $\mathcal{C}$ est-elle nécessairement une sous-variété de $\RR^{2}$ ? Et si l'on suppose de plus que $\gamma$ est injective ?


\end{exo}



\end{document}