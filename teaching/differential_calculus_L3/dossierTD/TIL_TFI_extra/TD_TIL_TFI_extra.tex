\documentclass[12pt]{article}
%\usepackage[frenchb]{babel}
\usepackage[T1]{fontenc}                % Accents codés dans la fonte.
\usepackage[utf8x]{inputenc}           % Accents 8 bits dans le fichier.
\usepackage{ifthen}                     % Faire des tests if/then/else.
\usepackage{amsmath}    % Les symboles les plus fréquents.
\usepackage{amsfonts}   % Des fontes, eg pour \mathbb.
\usepackage{vmargin}                    % Régler la taille de la feuille.
\usepackage{amssymb}
\usepackage{lastpage}
\usepackage{accents}
\usepackage{enumitem}

% La taille de la page
\setpapersize{A4}
\setmarginsrb{20mm}{10mm}{14mm}{15mm}{10mm}{6mm}{0mm}{10mm}

% La mise en page
\newcounter{vcenterstest}
\newlength{\leftlength}
\newlength{\rightlength}
\newlength{\vcentersskip}
\newcommand{\leftcentersright}[4][2]{%
        \settowidth{\leftlength}{#2}%
        \settowidth{\rightlength}{#4}%
        \setcounter{vcenterstest}{#1}%
        \ifthenelse{\value{vcenterstest} = 0}
                {\setlength{\vcentersskip}{0pt}}{}%
        \ifthenelse{\value{vcenterstest} = 1}
                {\setlength{\vcentersskip}{\smallskipamount}}{}%
        \ifthenelse{\value{vcenterstest} = 2}
                {\setlength{\vcentersskip}{\medskipamount}}{}%
        \ifthenelse{\value{vcenterstest} = 3}
                {\setlength{\vcentersskip}{\bigskipamount}}{}%
        \ifthenelse{\value{vcenterstest} = 4}
                {\setlength{\vcentersskip}{1cm}}{}%
                % On laisse un espace vertical défini par l'argument
                % optionnel #1
        \vskip\vcentersskip
                % On place #2 et on recule de sa longueur
        \noindent#2\hskip-\leftlength%
                % On centre #3
        \hfill#3\hfill%
                % On va au bout de la ligne, on recule de la longueur de #4 et
                % on place #4
        \mbox{}\hskip-\rightlength#4%
                % On laisse un espace vertical défini par l'argument
                % optionnel #1
        \vskip\vcentersskip%
    }
\newcommand{\centers}[2][2]{\leftcentersright[#1]{}{#2}{}}
\newcommand{\leftcenters}[3][2]{\leftcentersright[#1]{#2}{#3}{}}
\newcommand{\centersright}[3][2]{\leftcentersright[#1]{}{#2}{#3}}
\newcommand{\leftencadreright}[4][2]{\leftcentersright[#1]{#2}{\fbox{#3}}{#4}}

\newcommand{\numtoi}[1]{
  \ifthenelse{ \equal{#1}{1} }{i}{
  \ifthenelse{ \equal{#1}{2} }{ii}{
  \ifthenelse{ \equal{#1}{3} }{iii}{
  \ifthenelse{ \equal{#1}{4} }{iv}{
  \ifthenelse{ \equal{#1}{5} }{v}{
  \ifthenelse{ \equal{#1}{6} }{vi}{
  \ifthenelse{ \equal{#1}{7} }{vii}{
  \ifthenelse{ \equal{#1}{8} }{viii}{
  \ifthenelse{ \equal{#1}{9} }{ix}{
  \ifthenelse{ \equal{#1}{10} }{x}{
  \ifthenelse{ \equal{#1}{11} }{xi}{
  \ifthenelse{ \equal{#1}{12} }{xii}{
  \ifthenelse{ \equal{#1}{13} }{xiii}{
  \ifthenelse{ \equal{#1}{14} }{xiv}{
  \ifthenelse{ \equal{#1}{15} }{xv}{
  \ifthenelse{ \equal{#1}{16} }{xvi}{
  \ifthenelse{ \equal{#1}{17} }{xvii}{
  \ifthenelse{ \equal{#1}{18} }{xviii}{
    ERREUR,~MODIFIER~LA~MACRO~numtoi
  } } } } } } } } } } } } } } } } } }
}

\newcounter{cretraiti}
\newenvironment{ri}[1]{

\begin{list}{($\numtoi{\thecretraiti}$)}{
  \usecounter{cretraiti}
  \topsep=0.5ex
  \itemsep=0.3ex
  \labelsep=0.3em
  \parsep=0ex
  \listparindent=1em
  \settowidth{\labelwidth}{(#1)}
  \leftmargin=\labelwidth
  %\addtolength{\leftmargin}{\labelsep}
}}{\end{list}
}


% Quelques raccourcis pour taper des maths
\newcommand{\Sfr}{\mathfrak{S}}
\newcommand{\Afr}{\mathfrak{A}}
\newcommand{\QQ}{\mathbb{Q}}
\newcommand{\ZZ}{\mathbb{Z}}
\newcommand{\si}{\sigma}
\newcommand{\ins}[1]{{\ensuremath{\textrm{#1}}}}
\newcommand{\tiret}{\rule[0.6ex]{1.3ex}{0.26ex}}
\newcommand{\accolades}[1]{\ensuremath{\left\{#1\right\}}}
\newcommand{\paa}[1]{\ensuremath{\accolades{#1}}}       % Synonyme
\newcommand{\ds}{\displaystyle}
   \newcommand{\der}{
      \mathop{}\mathopen{}\mathrm{d}
   }

% Numerotation des pages / la derniere
\makeatletter
\renewcommand{\@evenfoot}%
        {\thepage \slash \pageref{LastPage}}
\renewcommand{\@oddfoot}{\@evenfoot}
\makeatother


%%%%% Pour faire les exos
\newcommand{\envfont}{\sf\bfseries}
\newcounter{cexo}
\renewcommand{\thecexo}{\arabic{cexo}}

\newenvironment{exo}[1][toto]
{
  \refstepcounter{cexo}
  \ifthenelse{ \equal{#1}{toto} }{
   \noindent {\envfont Exercice \thecexo}
  }{
   \noindent {\envfont Exercice \thecexo{} \tiret\, #1.}
  }
%\newline
%\indent
}
{}

\newcounter{cquestions}
\newenvironment{questions}{
\begin{list}{{\envfont\alph{cquestions})}}{
  \usecounter{cquestions}
  \topsep=0.5ex
  \partopsep=1em
  \labelsep=0.3em
  \itemsep=0.3ex
  \settowidth{\labelwidth}{{\envfont b)}}\listparindent=1em
  \leftmargin=\labelwidth \addtolength{\leftmargin}{\labelsep}
}}{\end{list} }

\newcommand{\poubelle}[1]{}

%%%%%%%%%%%%%%%%%%%%%%%%%%%%%%%%%%%%%%%%%
\newcommand{\spacebeforeenv}{\vspace{1ex}}
\newcommand{\spaceafterenv}{\vspace{1ex}}
\newcommand{\myqedenv}{}


\newcommand{\metaenvironnement}[2]  % 1er argument : nom de l'env, 2e argument : titre
{
  \ifthenelse{ \equal{#2}{toto} }{
    \spacebeforeenv \noindent {\envfont #1 }
  }{
    \spacebeforeenv \noindent {\envfont #1  \tiret\, #2.}
  }
}

%----------------------------------
% Rappel
%----------------------------------
\newenvironment{rappel}[1][toto]
{ \metaenvironnement{Rappel}{#1} }
{\myqedenv\spaceafterenv}


\newcommand{\fonction}[5]{%
        \ensuremath{#1\colon
        \left\{\hskip -1.5 mm
        \begin{array}{c@{\ }c@{\ }l}
        \medskip #2 & \longrightarrow & #3 \\
        #4 & \longmapsto & #5 \\
        \end{array}
        \right .
        }}
\newcommand{\ph}{\varphi}
\newcommand{\funu}[1]{\mathcal{C}^{#1}}
\newcommand{\RR}{\mathbb{R}}
\newcommand{\NN}{\mathbb{N}}
\newcommand{\CC}{\mathbb{C}}

\newenvironment{syst}[1][c]%
    {\ensuremath{\left \{ \hskip -1.5 mm \begin{array}{#1@{\ =\ }l}}}%
    {\end{array}\right.}

\newcommand{\f}[2]{{\ensuremath{\mathchoice%
    {\dfrac{#1}{#2}}
        {\dfrac{#1}{#2}}
    {\frac{#1}{#2}}
    {\frac{#1}{#2}}
    }}}
\renewcommand{\ni}{\noindent}
\newcommand{\ra}{\rightarrow}
\newcommand{\question}[1]{{\sf\bfseries #1}}
\newcommand{\norm}[1]{\ensuremath{\Arrowvert #1 \Arrowvert}}
\newcommand{\Ende}[3]{\ensuremath{\mathrm{End}_{#1}^{#2}(#3)}}
\newcommand{\End}[2]{\Ende{#1}{}{#2}}
\newcommand{\id}{\ensuremath{\mathrm{id}}}
\renewcommand{\d}{\ensuremath{\mathrm{d}}}
\renewcommand{\leq}{\leqslant}
\renewcommand{\geq}{\geqslant}


\usepackage{datetime}
\newdateformat{annscol}{\ifthenelse{\THEMONTH > 7}{\THEYEAR--\refstepcounter{YEAR}\THEYEAR}{\addtocounter{YEAR}{-1}\THEYEAR--\refstepcounter{YEAR}\THEYEAR}}

\begin{document}
 

%------------------------------------------------------------------------------------
\leftcentersright[0]{{\bf\sf\large Formation en Mathématiques Commune Cachan / P7}}{}
    {{\bf\sf\large Année~\annscol\today}}
\leftcentersright[0]{{\bf\sf\large Calcul différentiel}}{}{{\bf\sf Mariano Rodríguez, Frédéric Pascal}}

%------------------------------------------------------------------------------------------




%------------------------------------------------------------------------------------------
%------------------------------------------------------------------------------------------


\begin{center}
\huge{TD 5 bis \\ Théorèmes d'inversion locale et des fonctions implicites}
\end{center}


\vskip2ex 
\begin{exo}[TFI => TIL]

On suppose que le théorème des fonctions implicites est vrai. 
Soit $f$ définie sur un ouvert $U$ d'un espace de Banach $E$ et à valeur dans $F$, espace de Banach telle que
$f$ soit de classe ${\cal C}^1$ et telle qu'il existe $a \in U$ tel que $df(a)$ soit un isomorphisme. On note $b=f(a)$. 

\begin{questions}
\item Vérifier que les hypothèses du théorème des fonctions implicites s'appliquent à la fonction $\Phi : U \times F \to F$ 
définie par $\Phi(x,y)=y-f(x)$.
\item En déduire l'existence de voisinages de $a$ et de $b$ et de $f^{-1}$ définie sur un de ces voisinages.
\item Conclure.
\end{questions}
\end{exo}


\vskip2ex 
\begin{exo}[TIL=>TFI]

On suppose que le théorème d'inversion locale est vrai.
On se donne trois espaces de Banach $E,F,G$ et on considère $f$ définie sur un ouvert $U$ de $E\times F$ à valeur dans $G$ et de classe ${\cal C}^1$. On suppose 
qu'il existe $(a,b) \in U$ vérifiant
$f(a,b)=0$ et tel que la différentielle partielle par rapport à $y$ en $(a,b)$ c'est-à-dire $d_yf(a,b)$ soit un isomorphime. 

\begin{questions}
 \item Montrer que la fonction $\Psi : U \to E \times G$ définie par 
$\Psi(x,y) = (x,f(x,y))$ est de classe ${\cal C}^1$ et calculer sa différentielle. 


\item Vérifier que cette différentielle 
est inversible en $(a,b)$, montrer en utilisant 
le théorème d'inversion locale que $\Psi$ est un ${\cal C}^1$-difféomorphisme sur des espaces que l'on précisera. 

\item Reprendre les questions précédentes dans le cas où $E=\RR^n$ et $F=G=\RR^p$

\item En remarquant que l'on peut écrire $\Psi^{-1}(x,z)$ sous la forme$ (x,g(x,z))$, conclure.
\end{questions}
\end{exo}



\vskip2ex 
\begin{exo}[Inversion globale]
Soit $f : \RR^n \to \RR^n$ de classe  ${\cal C}^{1}$ telle que pour tout $(x,y)\in \RR^n \times \RR^n$, on ait
$
 \|f(x)-f(y)\| \geq k \|x-y\|
$
avec $k>0$ constante. 
\begin{questions}
\item Montrer que $f$ est injective.
\item Montre $f(\RR^n)$ est fermé.
\item Montrer que $df(x)$ est inversible pour tout $x$.
\item Conclure.
\item Application à $f : \RR^2 \to \RR^2$ définie par $f(x,y)=(\sin(\frac{y}{2})-x,\sin(\frac{x}{2})-y)$. On prendra la norme $\|\cdot\|_1$.
\end{questions}
\end{exo}


\vskip2ex 
\begin{exo}[Inversion globale]
Soit $f : \RR^3 \to \RR^3$ définie par $f(x,y,z)=(x+y,y/x,z/x)$.
\begin{questions}
\item Trouver un ouvert $U$ de $\RR^3$ sur lequel $f$ est différentiable. Calculer sa différentielle.
\item Soit $V=\{(x,y,z) \in \RR^3 / x>0 \,\,,\,\, x+y>0\}$. $f$ est-elle 
un ${\cal C}^1$-difféomorphisme de $V$ sur $f(V)$ ?
\end{questions}
\end{exo}




\vskip3ex
\begin{exo}[Inversion] 
Soit $H$ un espace de Hilbert de dimension fini, muni d'un produit scalaire euclidien $\langle \cdot,\cdot \rangle$ et 
 de la norme associée $\| \cdot \|$ et soit $f:  H \rightarrow H$ une 
application différentiable. On suppose qu'il existe $0<k<1$ tel que l'application 
$\phi: \ H \rightarrow H$ définie par $\phi(x)=f(x)-x$ soit $k$-lipschitzienne.  
\begin{questions}
 \item Montrer que $f$ est injective.
 \item Montrer que $\forall x \in H, \|d\phi(x)\| \leq k$.
 \item En déduire que $\forall x \in H, \ df(x)$ est inversible.
 \item Montrer que pour tout $x$ de $H$, on a
 \[
(1-k) \|x \| - \|f(0)\| \leq \|f(x)\|  
 \]
 \item On fixe $y\in H$ et on définit $h(x)=\| f(x)-y\|^{2}$. Soit $m=\inf \{ h(x) \ / \ x \in H\}$. 
Déduire de la question précédente qu'il existe $z \in H$ tel que $m=h(z)$. 
 \item Montrer que la différentielle de $h$ en $z$ est nulle.
 \item En déduire que $m=0$.
 \item Que peut-on conclure sur $f$ ?
 \item Quelle est la conclusion dans le cas où $f$ est de classe ${\cal C}^1$
\end{questions}
\end{exo}



\vskip3ex
\poubelle{
\begin{exo}
Soit $\mu \in \RR$. On considère le système différentiel
\begin{equation*}  
S_{\mu}\left\lbrace
\begin{array}{c}
\frac{\der x}{\der t} = y^2 - \cos x\\
\frac{\der y}{\der t} = - y \sin x - \mu y e^y
\end{array}\right.
\end{equation*}

\begin{questions}
\item Montrer que pour tout $0 \leq \mu < 1$ le système $S_{\mu}$ admet un unique point d'équilibre $C(\mu) = ( x_c (\mu , y_c (\mu))$ dans la bande $B = \{ (x,y) \in \RR ^2 | - \frac{\pi}{2} < x \leq 0 , y >0 \}$.
\item Montrer que la fonction $ : \mu \ra C(\mu)$ est de classe $C ^1$ au voisinage de $0$ et donner un développement limité à l'ordre 1 de $x_c (\mu)$ et $y_c (\mu)$.
\item On note $A(\mu)$ la matrice du problème linéarisé au point $C( \mu)$. Écrire le développement limité à l'ordre 1 de $A(\mu)$ en $0$.
\end{questions}
\end{exo}
}


\begin{exo}[Immersion] Soient $U$ un ouvert de $\mathbb{R}^n$ et $f:U\rightarrow\mathbb{R}^p$ une application de classe $C^1$. On suppose qu'en un point $a$ de $U$ la différentielle
$Df\left(a\right)$ est une application linéaire $\it{injective}$ de $\mathbb{R}^n$ dans $\mathbb{R}^p$ (donc $n\leq p$).
\begin{questions}
\item Montrer qu'après permutation des coordonnées de $\mathbb{R}^p$ (si nécessaire), les relations 
\begin{equation*}  
\left\lbrace
\begin{array}{c}
y_i = f_i \left( Y_1,...,Y_n \right) \,\,\,\,\,\,\,\,  si \,\, 1\leq i\leq n \\
y_i = f_i \left( Y_1,...,Y_n \right)+Y_i \,\,\,\,\,  si \,\, n+1\leq i\leq p
\end{array}\right.
\end{equation*}
définissent un changement de coordonnées locales $y\mapsto Y$ sur l'espace d'arrivée $\mathbb{R}^p$.
\item Montrer que $f$ admet (localment) un inverse à gauche, i.e une application $g$ de classe $C^1$ telle que $g\left( f \left( x \right) \right)=x$.
\end{questions}
\end{exo}

\end{document}
%%% Local Variables:
%%% mode: latex
%%% TeX-master: t
%%% End:
