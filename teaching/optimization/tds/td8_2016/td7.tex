\documentclass[11pt, a4paper]{article}
\usepackage{amssymb, amsmath, bm, mathrsfs, amsthm}
\usepackage[french]{babel}
\usepackage[T1]{fontenc}
\usepackage[utf8x]{inputenc}
\usepackage{graphicx}
%
\title{ENS Cachan, DPT Maths \\ [1cm]
Optimisation numérique M1 -- TD8 -- Optimisation sous contraintes}
%
\date{17 novembre 2016}

%
\author{Florian De Vuyst, Adrien Le Coënt - CMLA UMR 8536, ENS Cachan}
%
\newcommand{\mq}{montrer que }
\newcommand{\Mq}{Montrer que }
\newcommand{\bu}{\bm{u}}
\newcommand{\ba}{\bm{a}}
\newcommand{\bn}{\bm{n}}
\newcommand{\bA}{\bm{A}}
\newcommand{\be}{\begin{equation}}
\newcommand{\ee}{\end{equation}}
\newcommand{\bg}{\bm{g}}
\newcommand{\supp}[1]{\mathop{supp}(#1)}
\newtheorem{remark}{Remark}
%
\begin{document}
%
\maketitle
%
\section*{Exercice 1}
%SSSSSSSSSSSSSSSSSS
%
\'Etant donné $s\in \mathbb{R}^n\backslash \{0\}$ et $c\in\mathbb{R}^n$, on considère
le problème de minimisation suivant :
\begin{center}
$(\mathcal{P})\quad$\ Minimiser $\displaystyle{\left\{\frac{1}{2}\|x\|^2-\langle c,x\rangle\right\}}$
sous la contrainte $\langle s,x\rangle\leq 0$.
\end{center}
%
\begin{enumerate}
\item [1°)] Déterminer la fonction duale $\mu\in\mathbb{R}^+\mapsto \psi(\mu)$
associée au lagrangien
\[
\mathcal{L}:(x,\mu) \mapsto \mathcal{L}(x,\mu) = \frac{1}{2}\|x\|^2
-\langle c,x \rangle + \mu \langle s,x\rangle.
\]
%
\item [2°)] Résoudre le problème dual $(\mathcal{D})$ associé à $(\mathcal{P})$,
c'est-à-dire de la maximisation de $\psi$ sur $\mathbb{R}^+$ (on sera amené à 
considérer plusieurs cas, suivant le signe de $\langle c,s \rangle$).
%
\item [3°)] Utiliser le résultat précédent pour résoudre effectivement $(\mathcal{P})$.
%
\end{enumerate}
%
\section*{Exercice 2}
%SSSSSSSSSSSSSSSSSs
%
Dans $\mathbb{R}^n$ muni du produit scalaire usuel $\langle .,. \rangle$, on considère
le problème d'optimisation suivant :
%
\[
(\mathcal{P})\quad \left\{\begin{array}{l}
\mbox{Minimiser } f(x):= \frac{1}{2} \langle Mx,x \rangle +\langle q,x \rangle\\ [1.1ex]
\mbox{sous la contrainte } Ax+b\leq 0,
\end{array}\right.
\]
où $M\in\mathcal{M}_n(\mathbb{R})$ est symétrique définie positive, $q\in\mathbb{R}^n$,
$A\in \mathcal{M}_{m,n}(\mathbb{R})$ et $b\in\mathbb{R}^m$.
%
\begin{itemize}
\item [1°)] Que peut-on dire des solutions de $(\mathcal{P})$ ? Ecrire les relations KKT.
\item [2°)] Déterminer la fonction duale $\Theta$ associée au lagrangien
\[
\mathcal{L}:(x,\mu)\mapsto \mathcal{L}(x,\mu) :=\frac{1}{2}\langle Mx,x\rangle
+\langle q,x \rangle + \langle \mu,Ax+b\rangle.
\]
%
\item [3°)] Formuler le problème dual $(\mathcal{D})$ et $(\mathcal{P})$.
\item [4°)] Comment caractériser la solution $\bar \mu$ de $(\mathcal{D})$ ?
\item [5°)] Donner une condition suffisante pour que $(\mathcal{D})$ ait une solution
unique.
\item [6°)] Comment une solution $\bar \mu$ de $(\mathcal D)$ permettrait-elle
de retrouver la solution de $(\mathcal{P})$ ?
%
\end{itemize}
%
\section*{Exercice 3}
%SSSSSSSSSSSSSSSSSS
%
Soit $\Lambda_n$ le simplexe-unité de $\mathbb{R}^n$ :
\[
\Lambda_n = \left\{ (x_1,\ldots,x_n)\in\mathbb{R}^n\ |\ \forall i,\ x_i\geq 0
\mbox{ et } \sum_{i=1}^n x_i = 1 \right\}.
\]
%
\'Etant donné $r_1,\ldots,r_n$ des réels tous strictement positifs, 
$A\in\mathcal{M}_{m,n}(\mathbb{R})$ et $c\in \mathbb{R}^m$, on considère le
problème de minimisation suivant :
\[
(\mathcal{P})\quad \left\{\begin{array}{l}
\mbox{Minimiser } \displaystyle{\sum_{i=1}^n x_i \log \left(\frac{x_i}{r_i}\right)} \\ [1.1ex]
\mbox{sous les contraintes } x\in\Lambda_n \mbox{ et } Ax\leq c,
\end{array}\right.
\]
où $\alpha\log \alpha$ est prise égale à 0 pour $\alpha=0$ et la notation
$Ax\leq c$ signifie $(Ax)_i\leq c_i$ pour toutes les composantes $i$. 

Soit 
\[
(x,\mu)\in\Lambda_n\times (\mathbb{R}^+)^m \mapsto
\mathcal{L}(x,\mu) = \sum_{i=1}^n x_i \log\left(\frac{x_i}{r_i}\right)
+ \langle \mu,Ax-c \rangle
\]
le lagrangien dans le problème $(\mathcal{P})$ et
\[
\mu \in \left( \mathbb{R}^+ \right)^m \mapsto
\psi(\mu) = \inf_{x\in \Lambda_n} \mathcal{L}(x,\mu)
\]
la fonction duale associée. Montrer que
%
\[
\psi(\mu) = -\log\left( \sum_{i=1}^n r_i\, e^{-(A^T\mu)_i + \langle\mu,c\rangle  } \right)
\]
et formuler le plus simplement possible le problème dual $(\mathcal{D})$ de $(\mathcal{P})$.
%

\end{document}