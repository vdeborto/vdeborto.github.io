\documentclass[11pt, a4paper]{article}
\usepackage{amssymb, amsmath, bm, mathrsfs, amsthm}
\usepackage[french]{babel}
\usepackage[T1]{fontenc}
\usepackage[utf8x]{inputenc}
\usepackage{graphicx}
%
\title{ENS Cachan, DPT Maths \\ [1cm]
Optimisation numérique M1 -- TD5 -- Optimisation sous contraintes} 
%
\author{Florian De Vuyst, Adrien Le Co\"ent - CMLA UMR 8536, ENS Cachan}
%
\date{20 Octobre 2016}
%
\newcommand{\mq}{montrer que }
\newcommand{\Mq}{Montrer que }
\newcommand{\bu}{\bm{u}}
\newcommand{\ba}{\bm{a}}
\newcommand{\bn}{\bm{n}}
\newcommand{\bA}{\bm{A}}
\newcommand{\be}{\begin{equation}}
\newcommand{\ee}{\end{equation}}
\newcommand{\bg}{\bm{g}}
\newcommand{\supp}[1]{\mathop{supp}(#1)}
\newtheorem{remark}{Remark}
\newtheorem{theorem}{Théorème}
%
\begin{document}
%
\maketitle

\section*{Démonstration de cours}

\begin{multline*}
 C(u)= \{0 \} \bigcup \{ w \in V   ~ \text{ pour lesquels il existe au moins} \\ \text{ une suite de points} ~ \{u_k\}_k
 / u_k \in U,~ u_k \neq u, ~ \lim_k u_k = u \\
 \lim_k \dfrac{u_k - u}{\Vert u_k - u \Vert} = \dfrac{w}{\Vert w \Vert}, ~ w \neq 0 \}
\end{multline*}

{\bf Théorème}\\
 1) $\forall u \in U, ~C(u)$ est fermé; \\
 2) $J: \Omega \longrightarrow \mathbb{R}$, $\Omega \subset V$ ouvert. Si $J$ admet en un point 
 un minimum relatif sur $U$, et si $J$ est différentiable en $u$, alors:
 $$ DJ(u)(v-u) \geq 0 \quad \forall v \in u + C(u). $$ \\

Exercice: Montrer que $C(u)$ est fermé.


%
\section*{Théorème de sensibilité}
%SSSSSSSSSSSSSSSSSS
%
Soit $u^*$ et $\lambda^*$ satisfaisant

\begin{equation}
\left\lbrace
\begin{array}{c}
\nabla_u \mathcal{L} (u^*, \lambda^*) = 0\\
\nabla_{\lambda} \mathcal{L} (u^*, \lambda^*) = 0\\
y^T \nabla^2_{uu}\mathcal{L}(u^*, \lambda^*)y >0, ~\forall y\neq0 ~ \text{tel que} ~ \nabla \theta(u^*)^Ty = 0
\end{array}
\right.
\end{equation}
avec $\mathcal{L}$ le lagrangien du problème

\begin{equation*}
(\mathcal{P}_0)
\left\lbrace
\begin{array}{c}
\min J(u)\\
\theta (u) = 0
\end{array}
\right. .
\end{equation*}


On considère la famille de problèmes 

\begin{equation*}
(\mathcal{P})
\left\lbrace
\begin{array}{c}
\min J(u)\\
\theta (u) = \mu
\end{array}
\right.
\end{equation*}

paramétré par le vecteur $\mu \in \mathbb{R}^m$.


\begin{enumerate}
\item On veut montrer qu'il existe $r>0$ tel que $\forall \mu \in \mathcal{B}(0,r)$, il existe $u(\mu) \in \mathbb{R}^n$ et $\lambda(\mu) \in \mathbb{R}^m$ qui forment une paire minimum local-multiplicateur de Lagrange du problème $(\mathcal{P})$. $u(.)$ et $\lambda(.)$ sont $\mu-\mathcal{C}^1$ dans $\mathcal{B}(0,r)$ et $u(0)=u^*$ et $\lambda(0) = \lambda^*$.
\begin{itemize}
 \item \'Ecrire explicitement $\mathcal{L}$ le lagrangien du problème $(\mathcal{P}_0)$ et
 $\mathcal{L}'$ le lagrangien du problème $(\mathcal{P})$, puis écrire le système d'équations 
 avec multiplicateur de Lagrange associé au problème $\mathcal{P}$.
 \item Calculer le jacobien de ce dernier système, et montrer qu'il est inversible.
 \item En utilisant le théorème des fonctions implicites, en déduire la proposition 1.
\end{itemize}

\item Notons $P(\mu) = J(u(\mu))$ le coût optimal, montrer que
$$ \nabla_{\mu} P(\mu) = -\lambda(\mu) , ~ \forall \mu \in \mathcal{B}(0,r).$$
\end{enumerate}

\section*{Exercice}
%SSSSSSSSSSSSSSSSSS
%
\'Etant donné $A\in\mathcal{S}_n(\mathbb{R})$ et $b\in\mathbb{R}$, on considère
le problème d'optimisation suivant :
%
\[
(\mathcal{P})\quad \left\{\begin{array}{l}
\mbox{Minimiser } f(x):= \frac{1}{2} \langle A x,x \rangle +\langle b,x \rangle\\ [1.1ex]
\mbox{sous la contrainte } \|x\|=1.
\end{array}\right.
\]
\begin{itemize}
\item [1°)] On suppose $b=0$. Rappeler alors ce que vaut 
$\bar f = \inf\{ f(x): \|x\|=1 \}$ et quels sont les $\bar x$ de norme 1 pour lesquels
$f(\bar x)=\bar f$.
%
\item [2°)] Soit $\lambda_1$ la plus grande valeur propre de $A$ et $p$ un réel strictement
inférieur à $-\lambda_1$. On pose
\[
A_p := A + p I_n,\quad 
f_p : x\in\mathbb{R}^n \mapsto f_p(x) := \frac{1}{2}\langle A_p x,x \rangle
+\langle b,x \rangle.
\]
	\begin{itemize}
	\item [(a)] Indiquer pourquoi $f_p$ est strictement concave.
	\item [(b)] On considère le problème d'optimisation suivant :
	\[
(\tilde{\mathcal{P}}_p)\quad \left\{\begin{array}{l}
\mbox{Minimiser } f_p(x)\\ [1.1ex]
 \|x\|\leq 1.
\end{array}\right.
\]
Montrer que
\[
\inf\{ f_p(x):\|x\|\leq 1 \} = \inf\{ f(x):\|x\|=1 \}+\frac{1}{2}\, p
\]
et que les solutions de $(\mathcal{P})$ et $(\tilde{\mathcal{P}}_p)$ sont les mêmes.
%	
\end{itemize}
\end{itemize}
%

\end{document}