 \documentclass[a4paper,french,12pt]{article}


\usepackage{../tp}               % package for TP documents
\usepackage{../defs}           % package for shortcuts 
\DeclareMathOperator{\BETA}{BETA}


%%%%%%%%%%%%%%%%%%%%%%%%%%%%%%%%%%%%%%%%%%%%%%%%%%%%%%%%%%%
%%%%%    Headers   			             %%%%%
%%%%%%%%%%%%%%%%%%%%%%%%%%%%%%%%%%%%%%%%%%%%%%%%%%%%%%%%%%%

% \lhead{

% }  




\usepackage[active]{srcltx}
\usepackage{xargs}


\def\betaI{\mathrm{B}}
\def\ie{\textit{i.e.}}
\def\eqsp{\;}
\newcommand{\coint}[1]{\left[#1\right)}
\newcommand{\ocint}[1]{\left(#1\right]}
\newcommand{\ooint}[1]{\left(#1\right)}
\newcommand{\ccint}[1]{\left[#1\right]}

\def\rmd{\mathrm{d}}
\newcommand{\wrt}{w.r.t.}

\def\iid{i.i.d.}
\def\rme{\mathrm{e}}
\def\thetabf{\boldsymbol{\theta}}
\def\thetabfhat{\boldsymbol{\hat{\theta}}}
\def\rset{\mathbb{R}}
\def\nset{\mathbb{N}}
\def\zset{\mathbb{Z}}


\def\restriction#1#2{\mathchoice
              {\setbox1\hbox{${\displaystyle #1}_{\scriptstyle #2}$}
              \restrictionaux{#1}{#2}}
              {\setbox1\hbox{${\textstyle #1}_{\scriptstyle #2}$}
              \restrictionaux{#1}{#2}}
              {\setbox1\hbox{${\scriptstyle #1}_{\scriptscriptstyle #2}$}
              \restrictionaux{#1}{#2}}
              {\setbox1\hbox{${\scriptscriptstyle #1}_{\scriptscriptstyle #2}$}
              \restrictionaux{#1}{#2}}}
\def\restrictionaux#1#2{{#1\,\smash{\vrule height .8\ht1 depth .85\dp1}}_{\,#2}} 



\def\step{h}

\def\transpose{\mathrm{T}}

\newcommand\fracaaa[2]{#1/(#2)}

\newcommand{\ps}[2]{\left\langle#1,#2 \right\rangle}

\newcommandx\lnorm[3][1=]{\lVert #2 \rVert^{#1}_{#3}}
\newcommand{\opernorm}[1]{\Vert\hspace{-1pt}| #1 \Vert\hspace{-1pt}|}
\newcommand{\normop}[1]{\opernorm{#1}}
\newcommand{\normLigne}[1]{\Vert #1 \Vert}
\newcommandx{\norm}[2][1=]{\left\Vert #2 \right\Vert^{#1}}

\newcommand{\abs}[1]{\left\vert #1 \right\vert}
\newcommand{\absLigne}[1]{\vert #1 \vert}
\newcommandx\vectornorm[2][1=]{\left| #2 \right|^{#1}}  %%%% norme de vecteurs pas

\def\prox{\operatorname{prox}}




\newcommand{\defEns}[1]{\left\lbrace #1 \right\rbrace }
\newcommand{\ensemble}[2]{\left\{#1\,:\eqsp #2\right\}}
\newcommand{\ensembleLigne}[2]{\{#1\,:\eqsp #2\}}
\newcommand{\ensembleD}[2]{\left\{#1,\ldots, #2\right\}}
\newcommand{\ensembleDLigne}[2]{\left\{#1,\ldots, #2\right\}}
\newcommand{\set}[2]{\ensemble{#1}{#2}}

\newcommand{\parenthese}[1]{\left( #1 \right)}
\newcommand{\parentheseDeux}[1]{\left[ #1 \right]}

\def\tstar{t^{\star}}
\def\ystar{y^{\star}}


% \newcommand{\ocint}[1]{\left(#1\right]}
% \newcommand{\ooint}[1]{\left(#1\right)}
% \newcommand{\ccint}[1]{\left[#1\right]}
\def\plusinfty{+\infty}

\newcommand{\cointLigne}[1]{[#1)}
\newcommand{\ocintLigne}[1]{(#1]}
\newcommand{\oointLigne}[1]{(#1)}
\newcommand{\ccintLigne}[1]{[#1]}

\def\tr{\operatorname{Tr}}

\begin{document}

\noindent
 \textsc{Valentin De Bortoli} \hfill \url{debortoli@cmla.ens-cachan.fr} \\
    \textsc{Alain Durmus} \hfill \url{alain.durmus@cmla.ens-cachan.fr}
    \\
    TP 2 : optimisation numérique\hfill
    Première année Master Hadamard\\
\hrulefill\ \\
    
 \vspace{-5cm} 


% \sloppy
  %\phantom{am} 
%http://scipy-cookbook.readthedocs.io/items/robust_regression.html 
\titlefront{Minimisation par optimisation proximale}
En préambule de tout module python, on suppose que les paquets suivants ont été chargés
\begin{lstlisting}
  import numpy as np
  import matplotlib.pyplot as plt
\end{lstlisting}

\section{Implémentation de ISTA}
\label{sec:impl-de-ista}
L'algorithme ISTA \cite{beck09_fast_iterat_shrin_thres_algor} est un
algorithme a pour objectif à minimiser une fonction $F : \rset^d \to \rset$ s'écrivant sous la
forme $F = f + \lambda g$ où $\lambda >0$ est une constante, 
\begin{enumerate}
\item $f$ est un fonction convexe, $C^1$, gradient Lipschitz \ie pour
  tout $x,y \in \rset^d$,
  \begin{equation*}
    \norm{\nabla f(x) - \nabla f(y) } \leq L \norm{x-y } \eqsp.
  \end{equation*}
\item $g$ est une fonction continue convexe dont on peut calculer l'opérateur proximal $\prox_g^{\gamma}$ défini pour tout $\gamma >0$ et $x \in \rset^d$ par
  \begin{equation*}
    \prox_g^{\gamma}  = \argmin_{y \in \rset} \defEns{g(y) + (2 \gamma)^{-1} \norm[2]{x-y}} \eqsp.
  \end{equation*}
\end{enumerate}

L'algorithme ISTA consiste alors à construire la suite
$(x_k)_{k \in \nset}$ partant de $x_0 \in \rset^d$ à partir de la récursion suivante :
\begin{equation*}
  x_{k+1} = \prox_{\lambda g}^{\gamma} \defEns{x_k - \gamma \nabla f(x_k)} \eqsp.
\end{equation*}

\begin{enumerate}
\item Montrer que pour tout $\gamma, \lambda >0$, on a
  \begin{equation*}
    \prox_{\lambda g}^{\gamma} = \prox_{ g}^{\lambda \gamma} \eqsp.
  \end{equation*}
\item Implémenter alors l'algorithme ISTA. On pourra par exemple considérer un fonction
  sous la forme
  \begin{lstlisting}
    ista(dim,prox_op_g, grad_f, fun_total, lambda_l, n_it=100):
    Variables d'entree :
    dim : dimension du probleme
    prox_op_g : operateur proximal de g qui prend en entree x et gamma
    grad_f : fonction qui a partir d'un point x retourne le gradient de f
            et la constante de Lipschitz du gradient de f
    fun_total : fonction F = f+lambda*g
    lambda_l : parametre lambda dans F
    n_it : nombre d'iterations
    Variables de sortie : x, fun_iterate
    x : itere final de ISTA
    fun_iterate : suite (f(x_k))
  \end{lstlisting}
\end{enumerate}

\section{Acquisition comprimée}
%-----------------------------------------------------
%
Dans de nombreux problèmes, notamment en traitement du signal,
seulement des observations partielles, regroupées dans un vecteur
$y \in \rset^p$, d'un signal $x \in \rset^d$ sont disponibles. Cela
s'explique par une volonté de compresser le signal par exemple ou
encore parce que les instruments de mesures du signal sont dans
certains cas peu précis. Mathématiquement, cette perte de données se
modélise par une relation $y=Ax$, où $A \in \rset^{p,d}$, et $p <<
d$. Comme $A$ n'est pas inversible, le plus souvent le problème
inverse $y=Ax$ a de nombreuses solutions mais qui ne sont pas
satisfaisantes. En effet, dans de nombreux cas, le signal d'origine
$x$ est le plus souvent parcimonieux, \ie~un grand nombre de ses
composantes sont nulles. Ainsi, pour reconstruire ce type de signal,
il a été proposé de considérer le minimum de la fonction
\begin{equation*}
  F(x) = f(x) + \lambda g(x) \eqsp, \qquad f(x) = \norm[2]{y-Ax} \eqsp, \,  g(x) =  \sum_{i=1}^d \abs{x_i} \eqsp,
\end{equation*}
où $\lambda >0$ est un coefficient qui contrôle le niveau de
parcimonie de $x$.  On peut observer que $F$ peut s'écrire comme $f+g$
où $f$ est une fonction convexe gradient Lipschitz and $g$ est un
fonction convexe continue mais non différentiable. On se propose dans
cette partie d'appliquer l'algorithme ISTA à la minimisation de $F$
avec une application au traitement d'image.
Ici $y$ sera une mesure d'une image et $x$ les coefficients en ondelettes de
la même image. Ainsi la matrice $A$ est simplement l'opérateur linéaire qui à une représentation en ondelettes associe
un signal.
% Ici $y$ sera des
% coefficients d'ondelettes dans une base $(\psi_i)_{i \in I}$ où $I$
% est fini, d'une image $x$, alors $A$ est simplement l'opérateur
% linéaire qui à partir de $x$ donne ses coefficients dans
% $(\psi_i)_{i \in I}$.

Avant de commencer le TP, installer les librairies suivantes de python. Taper dans un terminal:
\begin{lstlisting}
  pip install -U scikit-learn scikit-image
\end{lstlisting}

Vous aurez aussi besoin du fichier \lstinline+tp2_tools.py+ à télécharger sur la page suivante : \url{https://vdeborto.github.io/project/optimization/}

\begin{enumerate}
\item Créer un nouveau module python pour ce problème qui commencera par
\begin{lstlisting}
import numpy as np
import matplotlib.pyplot as plt
from tp2_tools import *
import warnings
warnings.filterwarnings('ignore')
from ista import ista
\end{lstlisting}
\item Générer les données $A$ et $y$ et fixer pour le moment $\lambda$
  \begin{lstlisting}
lambda_l1 = 1
y, A = noisy_observations()
  \end{lstlisting}
\item Calculer le gradient de $f$ pour cet exemple.
\item Dans cette question, on s'intéresse au calcul de l'opérateur
  proximal de $g$.
  \begin{enumerate}
  \item Soit $h$ une fonction convexe sci propre sur $\rset^d$ sous la
    forme $h(x)= \sum_{i=1}^d h_i(x_i)$, pour tout
    $x = (x_i,\ldots,x_d)\in \rset^d$, où pour tout
    $i \in \{1,\ldots,d\}$,
    $h_i : \rset \to \ocint{-\infty,\plusinfty}$, est convexe sci
    propre. Montrer que pour tout $x \in \rset^d$ et $\gamma >0$,
    \begin{equation*}
      \prox_h^{\gamma}(x) = (      \prox_{h_i}^{\gamma}(x_i))_{i \in \{1,\ldots,d\}} \eqsp. 
    \end{equation*}
  \item Considérons $\phi(t) = \abs{t}$ pour tout $t \in \rset$.
    Montrer que pour tout $u \in \rset$ et $\gamma >0$
    \begin{equation*}
      t^{\star} = \argmin_{t \in \rset} \defEns{\abs{t} + (1/2\gamma) \abs{t-u}^2} \eqsp,
    \end{equation*}
    si et seulement si
    \begin{equation*}
      \tstar \in u - \gamma \partial \phi(\tstar) \eqsp. 
    \end{equation*}
  \item Déterminer la sous-différentielle de $\phi$ sur $\rset$.
    
  \item En distinguant les cas $\abs{u} < \gamma$ et $\abs{u} \geq \lambda$, déterminer pour tout $u \in \rset$ et $\gamma >0$, $\prox_{\phi}^{\gamma}(u)$.
    
  \item En déduire pour tout $x \in \rset^d$ et $\gamma >0$, $\prox_{g}^{\gamma}(x)$. 
  \end{enumerate}
  
\item Implémenter le gradient de $f$ et l'opérateur proximal de $g$.
  
\item Appliquer la fonction ista que vous avez précédemment coder pour minimiser $F$ et afficher l'image que vous obtenez à l'aide la fonction \lstinline+plot_image+.
  
\item Vérifier numériquement que l'ordre de convergence de ISTA est de l'ordre $O(k^{-1})$ où $k$ est le nombre d'itérations.
  
\item Changer la valeur du paramètre $\lambda$ et afficher les images que vous obtenez. Discuter de vos résultats. 
\end{enumerate}

\section{Minimisation de la fonction $\max$}
\label{sec:minimisation-de-la}

Dans cette section, on s'intéresse à la minimisation de la fonction $F$ définie sur $\rset^d$ par
\begin{equation*}
  F(x) = f(x) + \lambda g(x) \eqsp, \qquad f(x) = \norm[2]{y-Ax} \eqsp, \, g(x)= \max_{i \in \{1,\ldots,d\}} x_i \eqsp,
\end{equation*}
où $y \in \rset^d$ et $A \in \rset^{d\times d}$. 
\begin{enumerate}
\item Créer un nouveau module python pour ce problème qui commencera par
\begin{lstlisting}
import numpy as np
import matplotlib.pyplot as plt
from tp2_tools import *
import warnings
warnings.filterwarnings('ignore')
from ista import ista
\end{lstlisting}
\item Générer les données $A$ et $y$ et fixer pour le moment $\lambda$
  \begin{lstlisting}
lambda_max = 1
y, A = noisy_observations_inf()
  \end{lstlisting}
\item Calculer le gradient de $f$ pour cet exemple.
\item Dans cette question, on s'intéresse au calcul de l'opérateur proximal de $g$.
  \begin{enumerate}
  \item Montrer que le calcul de $\prox_g^{\gamma}(x)$ pour $\gamma>0$ et $x \in \rset^d$ revient au problème de minimisation sous contrainte
    \begin{equation}
      \label{eq:lagrangian_max_prox}
      \begin{aligned}
      \text{min}_{t \in \rset, y \in \rset^d} &\qquad t + (2\gamma)^{-1} \norm[2]{y-x} \\
      \text{ sous contraintes} &\qquad y_i \leq t \text{ pour tout $i \in \{1,\ldots,d\}$} \eqsp.
      \end{aligned}
    \end{equation}
  \item Donner le Lagrangien associé à ce problème.
  \item En déduire à partir des conditions KKT, une expression
    implicite pour la solution $(\tstar,\ystar)$ du problème \eqref{eq:lagrangian_max_prox}.     
  \end{enumerate}
  
\item Télécharger le module  \lstinline+prox_max.py+ à partir de \url{https://vdeborto.github.io/project/optimization/} et utiliser la fonction ista que vous avez implémenté pour minimiser $F$.
  
\item Illustrer graphiquement la convergence de l'algorithme. 
\end{enumerate}

\section{Complétion de matrice de faible rang}
\label{sec:minimisation-de-la}

Dans cette section, on s'intéresse au problème de complétion de
matrice. Dans certains problèmes statistiques, on a accès à seulement
certaines entrées d'une matrice $X \in \rset^{d \times d}$,
$Y = A \odot X$ où
$A =(\1_{I \times J}((i,j)))_{i,j \in \{1,\ldots,d\}}$, où
$I,J \subset \{1,\ldots,d\}$ et $\odot$ est la multiplication élément
par élément (ou appelé par certains ``matlab'').  Dans ce problème, on
voudrait alors retrouver les composantes manquantes de $X$. Pour cela,
il est commun de chercher une matrice $X$ de faible rang. Cela revient
alors à chercher à minimiser la fonction $F$ définie sur $\rset^{d\times d}$ par
\begin{equation*}
  F(X) = f(X) + \lambda g(X) \eqsp, \qquad f(X) = \norm[2]{Y-A\odot X}_{2} \eqsp, \, g(X )= \norm{X}_{\star}  = \sum_{i=1}^d \abs{\sigma_i(X)}  \eqsp,
\end{equation*}
où $\norm{\cdot}_2$ est la norme de Frobenius et
$(\sigma_i)_{i \in \{1,\ldots,d\}}$ sont les valeurs singulières de
$X$. On rappelle le théorème de décomposition en valeurs propres
singulières:
\begin{theoreme}
  \label{theo:svd}
Soit $A \in \rset^{d \times m}$, $d \leq m$. Il existe alors deux matrices orthogonales $O_1,O_2$ et $\sigma_1(A) \geq \cdots \geq \sigma_d(A) \geq 0$ tel que $A = O_1 \Sigma O_2$ où $\Sigma_A \in \rset^{d \times m}$ est la matrice dont les entrées sont données par $\Sigma_{i,i} = \sigma_i(A)$ pour $i \in\{1,\ldots,d\}$ et $\Sigma_{i,j} = 0$ sinon.  Les réels $(\sigma_1(A), \ldots, \sigma_d(A))$ sont appelés les valeurs singulières de $A$ et sont les valeurs propres de $A A^{\transpose}$.  
\end{theoreme}
\begin{enumerate}
\item Montrer que $g$ est la norme dual de la norme opérateur $\normop{\cdot}$ sur
  $\rset^{d \times d}$, \ie
  \begin{equation*}
    g(X) = \sup_{V \in \rset^{d \times d}, \, \normop{V} \leq 1} \ps{X}{V} = \sup_{V \in \rset^{d \times d}, \, \normop{V} \leq 1}\tr(X^{\transpose}V) \eqsp. 
  \end{equation*}
 En déduire que $g$ est convexe. 
\end{enumerate}
On cherche alors à appliquer ISTA pour minimiser $F$. 
\begin{enumerate}[resume]
\item On pourra dans un premier temps modifier ou créer une nouvelle fonction \lstinline+ista_mat+ qui s'applique directement à des matrices. 
\item Créer un nouveau module python pour ce problème qui commencera par
\begin{lstlisting}
import numpy as np
import matplotlib.pyplot as plt
from tp2_tools import *
import warnings
warnings.filterwarnings('ignore')
from ista import ista_mat
\end{lstlisting}
\item Générer les données $A$ et $y$ et fixer pour le moment $\lambda$
  \begin{lstlisting}
lambda_nuclear = 1
Y, A = noisy_observations_nuclear()
  \end{lstlisting}
\item Calculer le gradient de $f$ pour cet exemple.
\item Dans cette question, on s'intéresse au calcul de l'opérateur proximal de $g$.
  \begin{enumerate}
  \item En utilisant que la norme de Frobenius est invariant par
    rotation, montrer que pour tout $X \in \rset^{d \times d}$  le problème de minimisation
    \begin{equation}
      \label{eq:prox_nuclea_proof_1}
      \min_{Y \in \rset^{d\times d} } \norm{Y}_{\star} +(2\gamma)^{-1} \norm[2]{Y-X}_2 \eqsp,
    \end{equation}
    est équivalent à  
          \begin{equation}
      \label{eq:prox_nuclea_proof_2}
      \min_{D \in \mathsf{E}} \norm{D}_{\star} +(2\gamma)^{-1} \norm[2]{D-\Sigma_X}_2 \eqsp,
    \end{equation}
    où $\mathsf{E}$ est l'ensemble des matrices diagonales de $\rset^{d \times d}$ et $\Sigma_X$ est la matrice diagonale donnée par la décomposition en valeurs singulières de $X$.
    Pour cela on utilisera le résultat suivant \cite{vonNeumann_1937}.
    \begin{theoreme}
      \label{theofr:van_neu_svd}
      Soit $A,B \in \rset^{d \times d}$ deux matrices de dimension $d$. Alors pour toutes matrices orthogonales $O_1,O_2$,
      \begin{equation*}
        \ps{O_1 A O_2}{B} \leq \ps{\Sigma_A}{\Sigma_B} \eqsp,
      \end{equation*}
      où $\Sigma_A$ et $\Sigma_B$ sont les deux matrices diagonales données par la décomposition en valeurs singulières de respectivement $A$ et $B$.
    \end{theoreme}
    Si $D$ est une solution de \eqref{eq:prox_nuclea_proof_2}, comment construire $Y$ solution de \eqref{eq:prox_nuclea_proof_1} ? 
  \item En déduire que pour tout $\gamma >0$ et $X \in \rset^{d \times d}$
    \begin{equation*}
      \prox_g^{\gamma}(X) = U \prox_{\ell_1}^{\gamma}(\Sigma_X) V \eqsp,
    \end{equation*}
    où $X = U \Sigma_X V$ est une décomposition en valeurs singulières
    de $X$ et $\prox_{\ell_1}^{\gamma}$ est l'operateur proximal de la
    norme $\ell_1$, $Y \mapsto \sum_{i,j=1}^d \abs{Y_{i,j}}$. 
  \end{enumerate}
  
\item Implémenter l'opérateur proximal de $g$.
  
\item Tester l'algorithme ISTA et illustrer graphiquement sa convergence.  
\end{enumerate}


\bibliographystyle{plain}
\bibliography{../optim}
\end{document}


%%% Local Variables:
%%% mode: latex
%%% ispell-local-dictionary: "francais"
%%% TeX-master: t
%%% End:
