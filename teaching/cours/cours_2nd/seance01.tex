% cours réccurence (esomme)
% vecteurs équations droites
% produit scalaire

% Charlotte :
% -écriture scientifique/chiffres significatifs
% -problèmes
% -valeur absolue
% -pourcentage

\documentclass[10pt,a4paper]{article} 
\usepackage[utf8]{inputenc} 
\usepackage[T1]{fontenc} 
\usepackage[english]{babel} 
\usepackage{supertabular} %Nécessaire pour les longs tableaux
\usepackage[top=2.5cm, bottom=2.5cm, right=2.5cm, left=2.5cm]{geometry} %Mise en page 
\usepackage{amsmath} %Nécessaire pour les maths 
\usepackage{amssymb} %Nécessaire pour les maths 
\usepackage{stmaryrd} %Utilisation des double crochets 
\usepackage{pifont} %Utilisation des chiffres entourés 
\usepackage{graphicx} %Introduction d images 
\usepackage{epstopdf} %Utilisation des images .eps 
\usepackage{amsthm} %Nécessaire pour créer des théorèmes
\usepackage{algorithmic} %Nécessaire pour écrire des algorithmes 
\usepackage{algorithm} %Idem 
\usepackage{bbold} %Nécessaire pour pouvoir écrire des indicatrices 
\usepackage{hyperref} %Nécessaire pour écrire des liens externes 
\usepackage{array} %Nécessaire pour faire des tableaux 
\usepackage{tabularx} %Nécessaire pour faire de longs tableaux 
\usepackage{caption} %Nécesaire pour mettre des titres aux tableaux (tabular) 
\usepackage{color} %nécessaire pour écrire en couleur 
\newtheorem{thm}{Théorème} 
\newtheorem{mydef}{Définition}
\newtheorem{prop}{Proposition} 
\newtheorem{lemma}{Lemme}
\title{Séance 01 - 1ere}
\author{Valentin De Bortoli}
\begin{document}
\maketitle
\section{Exercice 1}
Donner l'écriture scientifique et le nombre de chiffres significatifs des nombres suivants :
\begin{itemize}
\item $A = 1.34000  \times 10^6 \text{m}$
\item $B = 0.040002 \times \left( 10^{-4} \right)^2 \text{km}$
\item $C = 23.4 \times 10^5 \times 0.0004 \times 10^3\text{J}$
\item $D = 0.400 \times 10^4 - 0.4 \times 10^4\text{kg}$
\item $E = 4 \times 20 \times 300 \times 10^5$
\end{itemize}

\section{Exercice 2}
Quelle est l'unité de poids en physique ? Rappeler la définition de la vitesse de la lumière, sa valeur, ainsi que la définition d'une année lumière. Un peu de culture générale ici : \url{https://www.astronomes.com/divers/lechelle-des-distances-dans-la-galaxie/}.
\subparagraph{} Retrouver par le calcul la valeur annoncée pour le temps que met la lumière à aller du Soleil à Neptune.

\section{Exercice 3}
Résoudre les équations suivantes :
\begin{itemize}
\item $\vert 2x +1 \vert = x+3$
\item $\vert 5x+ 3 \vert \times \vert 2x +1 \vert = 0$
\item $ \vert x + 4 \vert = \vert 2x - 6 \vert$
\end{itemize}

\section{Exercice 4}
Résoudre les inéquations suivantes :
\begin{itemize}
\item $\vert x - 3 \vert \le 2$
\item $\vert 2x +3 \vert > 5$
\item $(2x+1)(x-2) \le 0$
\end{itemize}
\subparagraph{Remarque :} donner les solutions sous la forme d'intervalles.

\section{Exercice 5}
A furniture store buys its furniture from a wholesaler. For a particular table, the store usually charges its costs from the wholesaler plus $75\%$. During a sale the store charged the wholesale costs plus $15\%$. If the sale price of the table was $\$299$, what is the usual price for the table?

\end{document}
