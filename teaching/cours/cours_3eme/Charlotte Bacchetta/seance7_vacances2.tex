\documentclass[10pt,a4paper]{article} 
\usepackage[utf8]{inputenc} 
\usepackage[T1]{fontenc} 
\usepackage[english]{babel} 
\usepackage{supertabular} %Nécessaire pour les longs tableaux
\usepackage[top=2.5cm, bottom=2.5cm, right=2.5cm, left=2.5cm]{geometry} %Mise en page 
\usepackage{amsmath} %Nécessaire pour les maths 
\usepackage{amssymb} %Nécessaire pour les maths 
\usepackage{stmaryrd} %Utilisation des double crochets 
\usepackage{pifont} %Utilisation des chiffres entourés 
\usepackage{graphicx} %Introduction d images 
\usepackage{epstopdf} %Utilisation des images .eps 
\usepackage{amsthm} %Nécessaire pour créer des théorèmes 
\usepackage{algorithmic} %Nécessaire pour écrire des algorithmes 
\usepackage{algorithm} %Idem 
\usepackage{bbold} %Nécessaire pour pouvoir écrire des indicatrices 
\usepackage{hyperref} %Nécessaire pour écrire des liens externes 
\usepackage{array} %Nécessaire pour faire des tableaux 
\usepackage{tabularx} %Nécessaire pour faire de longs tableaux 
\usepackage{caption} %Nécesaire pour mettre des titres aux tableaux (tabular) 
\usepackage{color} %nécessaire pour écrire en couleur 
\newtheorem{thm}{Théorème} 
\newtheorem{mydef}{Définition}
\newtheorem{prop}{Proposition} 
\newtheorem{lemma}{Lemme}
\title{Séance 7 - 3ème}
\author{Valentin De Bortoli}
\begin{document}
\maketitle
\section{Exercice 1}
Calculer le PGCD des nombres suivants.
\begin{equation}
\text{PGCD}(125,899)
\end{equation}
\begin{equation}
\text{PGCD}(900,143)
\end{equation}
\begin{equation}
\text{PGCD}(12,11^7 \times 3)
\end{equation}
\section{Exercice 2}
\textbf{Après} avoir revu le chapitre sur les puissances (et notamment avoir bien revu et compris les règles de calcul), exprimer sous la forme de puissance les expressions suivantes :
\begin{equation}
A=\left(-\frac{99}{7}\right)^5\times (-99)^4 \times (-7^2)^7
\end{equation}
\begin{equation}
B=-(-(-3^4)^5)^6
\end{equation}
\begin{equation}
C=\frac{\left(\frac{8}{6}\right)^2 \times 7^2}{6^2}\times 6^4
\end{equation}
\section{Exercice 3}
Donner tous les diviseurs des nombres suivants.
\begin{equation}
D=143
\end{equation}
\begin{equation}
E=260
\end{equation}
\begin{equation}
F=189
\end{equation}
\section{Exercice 4}
Dans cet exercice on s'intéresse aux \textbf{identités remarquables} qui seront étudiées en classe plus tard. On cherche à simplifier le plus possible les expressions suivantes. Soit $a$ et $b$ deux nombres réels.
\begin{equation}
G=(a+b)^2
\end{equation}
\begin{equation}
H=(a-b)^2
\end{equation}
\begin{equation}
I=(a-b)(a+b)
\end{equation}
\end{document}