\documentclass[10pt,a4paper]{article} 
\usepackage[utf8]{inputenc} 
\usepackage[T1]{fontenc} 
\usepackage[english]{babel} 
\usepackage{supertabular} %Nécessaire pour les longs tableaux
\usepackage[top=2.5cm, bottom=2.5cm, right=2.5cm, left=2.5cm]{geometry} %Mise en page 
\usepackage{amsmath} %Nécessaire pour les maths 
\usepackage{amssymb} %Nécessaire pour les maths 
\usepackage{stmaryrd} %Utilisation des double crochets 
\usepackage{pifont} %Utilisation des chiffres entourés 
\usepackage{graphicx} %Introduction d images 
\usepackage{epstopdf} %Utilisation des images .eps 
\usepackage{amsthm} %Nécessaire pour créer des théorèmes 
\usepackage{algorithmic} %Nécessaire pour écrire des algorithmes 
\usepackage{algorithm} %Idem 
\usepackage{bbold} %Nécessaire pour pouvoir écrire des indicatrices 
\usepackage{hyperref} %Nécessaire pour écrire des liens externes 
\usepackage{array} %Nécessaire pour faire des tableaux 
\usepackage{tabularx} %Nécessaire pour faire de longs tableaux 
\usepackage{caption} %Nécesaire pour mettre des titres aux tableaux (tabular) 
\usepackage{color} %nécessaire pour écrire en couleur 
\newtheorem{thm}{Théorème} 
\newtheorem{mydef}{Définition} 
\newtheorem{prop}{Proposition} 
\newtheorem{lemma}{Lemme}
\title{Séance 1 - 3ème}
\author{Valentin De Bortoli}
\begin{document}
\maketitle
\section{Exercice 1}
Dans chaque cas remplacer $x$ par $-5$ dans un premier temps puis par $\frac{1}{3}$ et enfin par $\frac{-1}{4}$.
\begin{equation}
A=-(3 x -2) \times 4
\end{equation}
\begin{equation}
B=(2 x+1)(-2x-1)
\end{equation}
\begin{equation}
C=\frac{-1}{3}\times (5x+2) 
\end{equation}
\section{Exercice 2}
Développer et réduire les expressions suivantes.
\begin{equation}
D=-(3x+8)(5x^2+\frac{x}{5})
\end{equation}
\begin{equation}
E=\frac{1}{2}(5x-3)^2
\end{equation}
\begin{equation}
F=-(-8x-9)^2
\end{equation}
\section{Exercice 3}
Factoriser le plus possible les expressions suivantes.
\begin{equation}
G=8x+64xy+12y^2
\end{equation}
\begin{equation}
H=\frac{1}{2}x+\frac{1}{8}xy
\end{equation}
\begin{equation}
I=-5x^2+46xy^2-7x^3
\end{equation}
\section{Exercice 4}
\subsection{Question 1}
Démontrer que la somme de deux entiers consécutifs est toujours impaire.
\subsection{Question 2}
Pour cet exercice il est indispensable de faire un dessin...\\
Soit $ABCD$ un rectangle tel que $AB=4\text{cm}$ et $AC=3 \text{cm}$. Soit $E$ un point de $[AB]$ et $F$ un point de $[AD]$ tel que $AE=AF=x \ \text{cm}$. Placer sur le dessin $G$ tel que $AEGF$ soit un carré. Comment calculer l'aire de la zone décrite par le rectangle $ABCD$ privé du carré $AEGF$ ? (donner une expression) \\
~\\
Dans l'expression trouvée pourquoi, alors que l'on calcule une aire, trouve-t-on un résultat négatif pour $x=4$ ?
\end{document}