\documentclass[10pt,a4paper]{article} 
\usepackage[utf8]{inputenc} 
\usepackage[T1]{fontenc} 
\usepackage[english]{babel} 
\usepackage{supertabular} %Nécessaire pour les longs tableaux
\usepackage[top=2.5cm, bottom=2.5cm, right=2.5cm, left=2.5cm]{geometry} %Mise en page 
\usepackage{amsmath} %Nécessaire pour les maths 
\usepackage{amssymb} %Nécessaire pour les maths 
\usepackage{stmaryrd} %Utilisation des double crochets 
\usepackage{pifont} %Utilisation des chiffres entourés 
\usepackage{graphicx} %Introduction d images 
\usepackage{epstopdf} %Utilisation des images .eps 
\usepackage{amsthm} %Nécessaire pour créer des théorèmes 
\usepackage{algorithmic} %Nécessaire pour écrire des algorithmes 
\usepackage{algorithm} %Idem 
\usepackage{bbold} %Nécessaire pour pouvoir écrire des indicatrices 
\usepackage{hyperref} %Nécessaire pour écrire des liens externes 
\usepackage{array} %Nécessaire pour faire des tableaux 
\usepackage{tabularx} %Nécessaire pour faire de longs tableaux 
\usepackage{caption} %Nécesaire pour mettre des titres aux tableaux (tabular) 
\usepackage{color} %nécessaire pour écrire en couleur 
\newtheorem{thm}{Théorème} 
\newtheorem{mydef}{Définition} 
\newtheorem{prop}{Proposition} 
\newtheorem{lemma}{Lemme}
\title{Séance 5 - 3ème}
\author{Valentin De Bortoli}
\begin{document}
\maketitle
\section{Exercice 1}
Calculer les puissances suivantes.
\begin{equation}
A=(\frac{-5}{7})^{-9}(\frac{7}{5})^6
\end{equation}
\begin{equation}
B=(-(-2^4)^5)^3
\end{equation}
\begin{equation}
C=\frac{3^4(\frac{-1}{3})^8}{3^-5}
\end{equation}
\begin{equation}
D=\frac{(\frac{3}{7})^5 7^4}{7^{-4}(\frac{1}{3})^{-2}}
\end{equation}
\section{Exercice 2}
Calculer le PGCD des nombres suivants.
\begin{equation}
\text{PGCD}(960,6048)
\end{equation}
\begin{equation}
\text{PGCD}(42875,23)
\end{equation}
\begin{equation}
\text{PGCD}(352,3969)
\end{equation}
\begin{equation}
\text{PGCD}(13608,59)
\end{equation}
\begin{equation}
\text{PGCD}(n,n+1) \ \text{avec n un nombre entier naturel}
\end{equation}
\section{Exercice 3}
Simplifier les fractions suivantes.
\begin{equation}
E=\frac{345}{27}
\end{equation}
\begin{equation}
F=\frac{924}{900}
\end{equation}
\begin{equation}
G=\frac{770}{594}
\end{equation}
\section{Exercice 4}
Donner l'écriture scientifique des nombres suivants.
\begin{equation}
H=\frac{(3\times 4 \times 10)^4 \times (-10^{-3})^5}{((-10)^2)^5}
\end{equation}
\begin{equation}
I=\frac{10^8+10^4+10^2}{10^6}
\end{equation}
\begin{equation}
J=\frac{(3 \times 4 \times 2)^2 \times 10^{-6}}{2 \times 10^4}
\end{equation}
\end{document}