\documentclass[10pt,a4paper]{article} 
\usepackage[utf8]{inputenc} 
\usepackage[T1]{fontenc} 
\usepackage[english]{babel} 
\usepackage{supertabular} %Nécessaire pour les longs tableaux
\usepackage[top=2.5cm, bottom=2.5cm, right=2.5cm, left=2.5cm]{geometry} %Mise en page 
\usepackage{amsmath} %Nécessaire pour les maths 
\usepackage{amssymb} %Nécessaire pour les maths 
\usepackage{stmaryrd} %Utilisation des double crochets 
\usepackage{pifont} %Utilisation des chiffres entourés 
\usepackage{graphicx} %Introduction d images 
\usepackage{epstopdf} %Utilisation des images .eps 
\usepackage{amsthm} %Nécessaire pour créer des théorèmes 
\usepackage{algorithmic} %Nécessaire pour écrire des algorithmes 
\usepackage{algorithm} %Idem 
\usepackage{bbold} %Nécessaire pour pouvoir écrire des indicatrices 
\usepackage{hyperref} %Nécessaire pour écrire des liens externes 
\usepackage{array} %Nécessaire pour faire des tableaux 
\usepackage{tabularx} %Nécessaire pour faire de longs tableaux 
\usepackage{caption} %Nécesaire pour mettre des titres aux tableaux (tabular) 
\usepackage{color} %nécessaire pour écrire en couleur 
\newtheorem{thm}{Théorème} 
\newtheorem{mydef}{Définition}
\newtheorem{prop}{Proposition} 
\newtheorem{lemma}{Lemme}
\title{Séance 13 - 3ème}
\author{Valentin De Bortoli}
\begin{document}
\maketitle
Factoriser les expressions suivantes :
\begin{equation}
A = x^2+6x+9
\end{equation}
\begin{equation}
B = x^2-21
\end{equation}
\begin{equation}
C = 21 - x^2
\end{equation}
\begin{equation}
D = x^2-3x+(x-3)(x-4)
\end{equation}
\begin{equation}
E = (x+1)^2-36
\end{equation}
\begin{equation}
F = x^2+4xy+4y^2
\end{equation}
\begin{equation}
G = (x+1)^3+x^2+2x+1
\end{equation}
\begin{equation}
H = x(x+5)(x+7) - x^2+10x-25
\end{equation}
\begin{equation}
I = -x(x-4)^2 -x^3+4
\end{equation}
\begin{equation}
J = x(x+7)^2 +x^3+14x^2+49x
\end{equation}
\begin{equation}
K = x^4 -y^2
\end{equation}
\subparagraph{Remarque :} même si la quasi totalité des exercices donnés ici font appel aux identités remarquables, attention à ne pas oublier qu'avant toute chose il faut commencer par regarder si on a un \textbf{facteur commun}.
\end{document}