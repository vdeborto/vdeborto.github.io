\documentclass[10pt,a4paper]{article} 
\usepackage[utf8]{inputenc} 
\usepackage[T1]{fontenc} 
\usepackage[english]{babel} 
\usepackage{supertabular} %Nécessaire pour les longs tableaux
\usepackage[top=2.5cm, bottom=2.5cm, right=2.5cm, left=2.5cm]{geometry} %Mise en page 
\usepackage{amsmath} %Nécessaire pour les maths 
\usepackage{amssymb} %Nécessaire pour les maths 
\usepackage{stmaryrd} %Utilisation des double crochets 
\usepackage{pifont} %Utilisation des chiffres entourés 
\usepackage{graphicx} %Introduction d images 
\usepackage{epstopdf} %Utilisation des images .eps 
\usepackage{amsthm} %Nécessaire pour créer des théorèmes 
\usepackage{algorithmic} %Nécessaire pour écrire des algorithmes 
\usepackage{algorithm} %Idem 
\usepackage{bbold} %Nécessaire pour pouvoir écrire des indicatrices 
\usepackage{hyperref} %Nécessaire pour écrire des liens externes 
\usepackage{array} %Nécessaire pour faire des tableaux 
\usepackage{tabularx} %Nécessaire pour faire de longs tableaux 
\usepackage{caption} %Nécesaire pour mettre des titres aux tableaux (tabular) 
\usepackage{color} %nécessaire pour écrire en couleur 
\newtheorem{thm}{Théorème} 
\newtheorem{mydef}{Définition}
\newtheorem{prop}{Proposition} 
\newtheorem{lemma}{Lemme}
\title{Séance 15 - 3ème}
\author{Valentin De Bortoli}
\begin{document}
\maketitle
\section{Exercice 1}
Exprimer sous la forme d'une puissance les expressions suivantes.
\begin{equation}
A=\left( \frac{-8}{3^3}\right)^2 \times (-3^2)^3 +6^2
\end{equation}
\begin{equation}
B=\frac{5}{6^7}\times \left( \frac{6}{5} \right)^8 \times\left( \frac{1}{5^3} \right)^2
\end{equation}
\section{Exercice 2}
Calculer l'expression suivante
\begin{equation}
D=PPCM(143,165)
\end{equation}
En profiter également pour calculer tous les diviseurs de 144 et 165. En déduire leur PGCD.
\section{Exercice 3}
Développer les expressions suivantes (on utilisera les identités remarquables).
\begin{equation}
E=(x-3)^2(x+3)^2
\end{equation}
\begin{equation}
F=(x-2+y)^2
\end{equation}
\section{Exercice 4}
De la même manière, factoriser le plus possibles les expressions suivantes.
\begin{equation}
G=x^2y^2-z^2
\end{equation}
\begin{equation}
H=64x^2-14xy^2+49y^4
\end{equation}
\begin{equation}
I=\frac{1}{3}x^2+\frac{4}{\sqrt{3}}xy^2+4y^4
\end{equation}
\section{Exercice 5}
Un stagiaire est payé 440 euros au mois de janvier, 462 euros au mois de février et 660 euros au mois de mars.
\subparagraph{1}Quel est le salaire journalier du stagiaire ? Sachant que celui-ci est supérieur à 20 euros et est un nombre entier d'euros.
\subparagraph{2}Combien de jours a-t-il travaillé chaque mois ?
\section{Exercice 6}
\subparagraph{1}Calculer $PGCD(14,16) \times PPCM(14,16)$ et $14 \times 16$. Que remarque-t-on ?
\subparagraph{2}Essayer de démontrer le résultat.
\section{Exercice 7}
Revoir le début du chapitre sur les racines (racine carré d'un nombre au carré, carré d'une racine carrée...) et les grandeurs quotient et produit. Tu dois être capable de résoudre rapidement une conversion du type $15L/h$ en $mm^3/s$.
\end{document}