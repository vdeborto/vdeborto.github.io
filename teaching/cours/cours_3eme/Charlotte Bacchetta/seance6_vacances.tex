\documentclass[10pt,a4paper]{article} 
\usepackage[utf8]{inputenc} 
\usepackage[T1]{fontenc} 
\usepackage[english]{babel} 
\usepackage{supertabular} %Nécessaire pour les longs tableaux
\usepackage[top=2.5cm, bottom=2.5cm, right=2.5cm, left=2.5cm]{geometry} %Mise en page 
\usepackage{amsmath} %Nécessaire pour les maths 
\usepackage{amssymb} %Nécessaire pour les maths 
\usepackage{stmaryrd} %Utilisation des double crochets 
\usepackage{pifont} %Utilisation des chiffres entourés 
\usepackage{graphicx} %Introduction d images 
\usepackage{epstopdf} %Utilisation des images .eps 
\usepackage{amsthm} %Nécessaire pour créer des théorèmes 
\usepackage{algorithmic} %Nécessaire pour écrire des algorithmes 
\usepackage{algorithm} %Idem 
\usepackage{bbold} %Nécessaire pour pouvoir écrire des indicatrices 
\usepackage{hyperref} %Nécessaire pour écrire des liens externes 
\usepackage{array} %Nécessaire pour faire des tableaux 
\usepackage{tabularx} %Nécessaire pour faire de longs tableaux 
\usepackage{caption} %Nécesaire pour mettre des titres aux tableaux (tabular) 
\usepackage{color} %nécessaire pour écrire en couleur 
\newtheorem{thm}{Théorème} 
\newtheorem{mydef}{Définition}
\newtheorem{prop}{Proposition} 
\newtheorem{lemma}{Lemme}
\title{Séance 3 - 3ème}
\author{Valentin De Bortoli}
\begin{document}
\maketitle
\section{Exercice 1}
Calculer le PGCD des nombres suivants.
\begin{equation}
\text{PGCD}(127,806)
\end{equation}
\begin{equation}
\text{PGCD}(978,832)
\end{equation}
\begin{equation}
\text{PGCD}(11,11^7 \times 19)
\end{equation}
\begin{equation}
\text{PGCD}(492,69)
\end{equation}
\section{Exercice 2}
Donner la décomposition en produit de facteurs premiers des nombres suivants.
\begin{equation}
A=(8 \times 35)^4
\end{equation}
\begin{equation}
B=143
\end{equation}
\begin{equation}
C=28512
\end{equation}
\section{Exercice 3}
Factoriser les expressions suivantes :
\begin{equation}
D=156x^2+148x^3y^4z^5t^9
\end{equation}
\begin{equation}
E=88x+114y^3+64xy
\end{equation}
\section{Exercice 4}
Donner l'écriture scientifique des nombres suivants :
\begin{equation}
F=\frac{246 \times 10^{300}}{3 \times 5^2 \times 2^3}
\end{equation}
\begin{equation}
G=\frac{10^2+10^0+10^-4}{2\times 10^2}
\end{equation}
\section{Exercice 5}
Exprimer sous la forme d'une puissance les expressions suivantes :
\begin{equation}
H=\left(\frac{1}{3}\right)^8\left(\frac{1}{3}\right)^{-5}\times (3^{-4})^{-7}
\end{equation}
\begin{equation}
I=\left(\frac{8}{3}\right)^2\times 8^2 \times (-3^2)^2
\end{equation}
\begin{equation}
J=-(-(-5^3)^2)^7
\end{equation}
\end{document}