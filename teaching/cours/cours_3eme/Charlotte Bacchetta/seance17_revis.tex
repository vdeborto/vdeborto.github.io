\documentclass[10pt,a4paper]{article} 
\usepackage[utf8]{inputenc} 
\usepackage[T1]{fontenc} 
\usepackage[english]{babel} 
\usepackage{supertabular} %Nécessaire pour les longs tableaux
\usepackage[top=2.5cm, bottom=2.5cm, right=2.5cm, left=2.5cm]{geometry} %Mise en page 
\usepackage{amsmath} %Nécessaire pour les maths 
\usepackage{amssymb} %Nécessaire pour les maths 
\usepackage{stmaryrd} %Utilisation des double crochets 
\usepackage{pifont} %Utilisation des chiffres entourés 
\usepackage{graphicx} %Introduction d images 
\usepackage{epstopdf} %Utilisation des images .eps 
\usepackage{amsthm} %Nécessaire pour créer des théorèmes 
\usepackage{algorithmic} %Nécessaire pour écrire des algorithmes 
\usepackage{algorithm} %Idem 
\usepackage{bbold} %Nécessaire pour pouvoir écrire des indicatrices 
\usepackage{hyperref} %Nécessaire pour écrire des liens externes 
\usepackage{array} %Nécessaire pour faire des tableaux 
\usepackage{tabularx} %Nécessaire pour faire de longs tableaux 
\usepackage{caption} %Nécesaire pour mettre des titres aux tableaux (tabular) 
\usepackage{color} %nécessaire pour écrire en couleur 
\newtheorem{thm}{Théorème} 
\newtheorem{mydef}{Définition}
\newtheorem{prop}{Proposition} 
\newtheorem{lemma}{Lemme}
\title{Séance 12 - 3ème}
\author{Valentin De Bortoli}
\begin{document}
\maketitle
\section{Exercice 1}
Donner la décomposition en produit de facteurs premiers et les diviseurs des nombres suivants :
\subparagraph{1} $a=143$
\subparagraph{2} $b=528$
\subparagraph{3} $c=455$
\section{Exercice 2}
\subparagraph{1} 187 est-il un nombre premier ?
\subparagraph{2} Simplifier le plus possible $\frac{237}{126}$.
\subparagraph{3} Calculer $PGCD(63,107)$ ainsi que $PPCM(63,107)$.
\section{Exercice 3}
Factoriser les expressions suivantes.
\subparagraph{1} $A= (x-2)^2-x^2+4x-4$
\subparagraph{2} $B=(x-1)(x+1) - x^2 +1$
\subparagraph{3} $C=(x-3)(x-4) +x^3 -6x^2 +9$
\section{Exercice 4}
Revoir la géométrie (théorème de Thalès doit être su ainsi que sa réciproque).
\section{Exercice 5}
\subparagraph{1}Faire un tableau des unités pour les longueurs, les aires et les volumes. Comment passe-t-on d'une colonne à l'autre ? Pourquoi ?
\subparagraph{2} On suppose que le nombre de lapins dans une ferme augmente de $1,5$ pourcent chaque année. En 2017 on en compte $120$. Combien y avait-il de lapins en 2000 ? (On arrondira à l'unité inférieure).
\subparagraph{3}Un pull coûte 120 euros en février 2017, 100 euros en mars 2017, 120 euros en avril 2017, 150 euros en mai 2017. Donner les pourcentages de hausse et de baisse du prix entre chaque mois. Que remarque-t-on ?
\subparagraph{4}Soit un carré de même aire qu'un disque de rayon 1. Quelle est la longueur du côté du carré ? Soit un autre carré de même aire qu'un disque de rayon 2. Quelle est la longueur du côté de ce carré ? Que remarque-t-on ?
\section{Exercice 6}
SImplifier le plus possible les quantités suivantes 
\subparagraph{1} $D = \sqrt{784} - \sqrt{9}$
\subparagraph{2} $E = (\sqrt{8}-\sqrt{\sqrt{2}})^2$
\subparagraph{3} $F = \frac{1}{\sqrt{3}} - \frac{\sqrt{5}}{\sqrt{6}}$
\end{document}