\documentclass[10pt,a4paper]{article} 
\usepackage[utf8]{inputenc} 
\usepackage[T1]{fontenc} 
\usepackage[english]{babel} 
\usepackage{supertabular} %Nécessaire pour les longs tableaux
\usepackage[top=2.5cm, bottom=2.5cm, right=2.5cm, left=2.5cm]{geometry} %Mise en page 
\usepackage{amsmath} %Nécessaire pour les maths 
\usepackage{amssymb} %Nécessaire pour les maths 
\usepackage{stmaryrd} %Utilisation des double crochets 
\usepackage{pifont} %Utilisation des chiffres entourés 
\usepackage{graphicx} %Introduction d images 
\usepackage{epstopdf} %Utilisation des images .eps 
\usepackage{amsthm} %Nécessaire pour créer des théorèmes 
\usepackage{algorithmic} %Nécessaire pour écrire des algorithmes 
\usepackage{algorithm} %Idem 
\usepackage{bbold} %Nécessaire pour pouvoir écrire des indicatrices 
\usepackage{hyperref} %Nécessaire pour écrire des liens externes 
\usepackage{array} %Nécessaire pour faire des tableaux 
\usepackage{tabularx} %Nécessaire pour faire de longs tableaux 
\usepackage{caption} %Nécesaire pour mettre des titres aux tableaux (tabular) 
\usepackage{color} %nécessaire pour écrire en couleur 
\newtheorem{thm}{Théorème} 
\newtheorem{mydef}{Définition}
\newtheorem{prop}{Proposition} 
\newtheorem{lemma}{Lemme}
\title{Séance 9 - 3ème}
\author{Valentin De Bortoli}
\begin{document}
\maketitle
\section{Exercice 1}
\subparagraph{1}Finir les calculs de racine de la feuille précédente. 
\subparagraph{2}Rappeler et écrire les règles régissant le calcul de racines.
\subparagraph{3}Calculer $(3+\sqrt{5})^2$.
\subparagraph{4}Montrer que $\sqrt{14+6\sqrt{5}}=3+\sqrt{5}$.
\section{Exercice 2}
\subparagraph{1}Soit $x$ un nombre positif (non nul). Montrer que $(x+\frac{1}{x})^2-4 \ge 0$.
\subparagraph{2}En déduire que $x+\frac{1}{x} \ge 2$.
\subparagraph{3}Soit $ABCD$ un rectangle tel que $AB=CD=x \ \text{cm}$ et $AC=BD=y \ \text{cm}$. Quelle est l'aire de ce rectangle ? Quel est son périmètre ?
\subparagraph{4}Supposons que l'aire du rectangle soit égale à $1 \ \text{cm}^2$. Exprimer $y$ en fonction de $x$. Exprimer le périmètre en fonction de $x$.
\subparagraph{5}Montrer que le rectangle d'aire fixée à $1 \ \text{cm}^2$ qui a le plus petit périmètre est le carré de côté $1 \ \text{cm}$.
\section{Exercice 3}
Factoriser les expressions suivantes :
\begin{equation}
A=9x^2-121y^2
\end{equation}
\begin{equation}
B=x^2-12xy+36y^2
\end{equation}
\begin{equation}
C=3x^2-75
\end{equation}
\section{Exercice 4}
Développer les expressions suivantes :
\begin{equation}
D=(x+9y)^2 \times 5
\end{equation}
\begin{equation}
E=(x-5z^2)^2
\end{equation}
\begin{equation}
F=(x-6)(x+6)^2
\end{equation}
\section{Exercice 5}
\subparagraph{1}A Chatelet, un RER A passe toutes les 22 minutes et un RER B toutes les 26 minutes. Combien de temps faut-il attendre au minimum pour voir un RER A \textbf{et} un RER B à quai ? (Le résultat sera donné sous la forme heure/minute).
\subparagraph{2}Charlotte organise une soirée. Elle prépare 39 tartes et 156 cocktails. Combien d'invités peut-elle convier sachant que chacun des invités doit avoir autant de tartes et de cocktails que les autres et que tous les cocktails et tartes doivent être utilisés ?
\subparagraph{3}Dans ce cas combien chaque invité aura de tartes et de cocktails ?
\end{document}