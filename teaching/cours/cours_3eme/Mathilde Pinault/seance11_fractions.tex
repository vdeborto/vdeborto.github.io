\documentclass[10pt,a4paper]{article} 
\usepackage[utf8]{inputenc} 
\usepackage[T1]{fontenc} 
\usepackage[english]{babel} 
\usepackage{supertabular} %Nécessaire pour les longs tableaux
\usepackage[top=2.5cm, bottom=2.5cm, right=2.5cm, left=2.5cm]{geometry} %Mise en page 
\usepackage{amsmath} %Nécessaire pour les maths 
\usepackage{amssymb} %Nécessaire pour les maths 
\usepackage{stmaryrd} %Utilisation des double crochets 
\usepackage{pifont} %Utilisation des chiffres entourés 
\usepackage{graphicx} %Introduction d images 
\usepackage{epstopdf} %Utilisation des images .eps 
\usepackage{amsthm} %Nécessaire pour créer des théorèmes 
\usepackage{algorithmic} %Nécessaire pour écrire des algorithmes 
\usepackage{algorithm} %Idem 
\usepackage{bbold} %Nécessaire pour pouvoir écrire des indicatrices 
\usepackage{hyperref} %Nécessaire pour écrire des liens externes 
\usepackage{array} %Nécessaire pour faire des tableaux 
\usepackage{tabularx} %Nécessaire pour faire de longs tableaux 
\usepackage{caption} %Nécesaire pour mettre des titres aux tableaux (tabular) 
\usepackage{color} %nécessaire pour écrire en couleur 
\newtheorem{thm}{Théorème} 
\newtheorem{mydef}{Définition}
\newtheorem{prop}{Proposition} 
\newtheorem{lemma}{Lemme}
\title{Séance 11 - 3ème}
\author{Valentin De Bortoli}
\begin{document}
\maketitle
\section{Fractions}
Simplifier les fractions suivantes :
\begin{equation}
A = \frac{1}{27} - \frac{1}{9}
\end{equation}
\begin{equation}
A = \frac{3}{2} - \frac{7}{3} \times \frac{3}{14}
\end{equation}
\begin{equation}
A = \left(\frac{6}{3}\right)^2 - \frac{1}{100}
\end{equation}
\begin{equation}
A = \left(\frac{1}{7} - \frac{4}{14}\right)^2
\end{equation}
\begin{equation}
A = \frac{1}{\frac{1}{9}} - 8
\end{equation}
\begin{equation}
A = \frac{2}{3} \times \frac{4}{6} \times \frac{8}{7}
\end{equation}
\begin{equation}
A = \left(\frac{1}{2}\times \frac{1}{2}\right)^3 - \frac{1}{80}
\end{equation}
\begin{equation}
A = \frac{1}{5}\times 9 - 4
\end{equation}
\begin{equation}
A = \frac{1}{27} + \frac{27}{1}
\end{equation}
\end{document}