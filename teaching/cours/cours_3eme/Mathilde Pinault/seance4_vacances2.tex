\documentclass[10pt,a4paper]{article} 
\usepackage[utf8]{inputenc} 
\usepackage[T1]{fontenc} 
\usepackage[english]{babel} 
\usepackage{supertabular} %Nécessaire pour les longs tableaux
\usepackage[top=2.5cm, bottom=2.5cm, right=2.5cm, left=2.5cm]{geometry} %Mise en page 
\usepackage{amsmath} %Nécessaire pour les maths 
\usepackage{amssymb} %Nécessaire pour les maths 
\usepackage{stmaryrd} %Utilisation des double crochets 
\usepackage{pifont} %Utilisation des chiffres entourés 
\usepackage{graphicx} %Introduction d images 
\usepackage{epstopdf} %Utilisation des images .eps 
\usepackage{amsthm} %Nécessaire pour créer des théorèmes 
\usepackage{algorithmic} %Nécessaire pour écrire des algorithmes 
\usepackage{algorithm} %Idem 
\usepackage{bbold} %Nécessaire pour pouvoir écrire des indicatrices 
\usepackage{hyperref} %Nécessaire pour écrire des liens externes 
\usepackage{array} %Nécessaire pour faire des tableaux 
\usepackage{tabularx} %Nécessaire pour faire de longs tableaux 
\usepackage{caption} %Nécesaire pour mettre des titres aux tableaux (tabular) 
\usepackage{color} %nécessaire pour écrire en couleur 
\newtheorem{thm}{Théorème} 
\newtheorem{mydef}{Définition}
\newtheorem{prop}{Proposition} 
\newtheorem{lemma}{Lemme}
\title{Séance 4 - 3ème}
\author{Valentin De Bortoli}
\begin{document}
\maketitle
\section{Exercice 1}
Revoir la table de $7$, de $8$ et de $9$
\section{Exercice 2}
Simplifier les expressions suivantes :
\begin{equation}
A=\sqrt{64}-\sqrt{25}
\end{equation}
\begin{equation}
B=\sqrt{36}(\sqrt{5}-1)
\end{equation}
\begin{equation}
C=(\sqrt{7}+\sqrt{3})(\sqrt{7}-2\sqrt{3})^2
\end{equation}
\section{Exercice 3}
Déterminer si les nombres suivants sont premiers :
\begin{equation}
D=143
\end{equation}
\begin{equation}
E=59
\end{equation}
\section{Exercice 4}
Donner la décomposition en produit de facteurs premiers de :
\begin{equation}
F=2160
\end{equation}
\begin{equation}
G=378
\end{equation}
\section{Exercice 5}
Exprimer sous la forme d'une puissance les expressions suivantes :
\begin{equation}
H=\left(-\frac{8}{11}\right)^4\times 11^2 \times (-11^2)^5
\end{equation}
\begin{equation}
I=-(-(-6^2)^3)^7
\end{equation}
\end{document}