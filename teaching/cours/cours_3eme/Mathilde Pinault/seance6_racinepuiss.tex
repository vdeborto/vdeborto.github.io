\documentclass[10pt,a4paper]{article} 
\usepackage[utf8]{inputenc} 
\usepackage[T1]{fontenc} 
\usepackage[english]{babel} 
\usepackage{supertabular} %Nécessaire pour les longs tableaux
\usepackage[top=2.5cm, bottom=2.5cm, right=2.5cm, left=2.5cm]{geometry} %Mise en page 
\usepackage{amsmath} %Nécessaire pour les maths 
\usepackage{amssymb} %Nécessaire pour les maths 
\usepackage{stmaryrd} %Utilisation des double crochets 
\usepackage{pifont} %Utilisation des chiffres entourés 
\usepackage{graphicx} %Introduction d images 
\usepackage{epstopdf} %Utilisation des images .eps 
\usepackage{amsthm} %Nécessaire pour créer des théorèmes 
\usepackage{algorithmic} %Nécessaire pour écrire des algorithmes 
\usepackage{algorithm} %Idem 
\usepackage{bbold} %Nécessaire pour pouvoir écrire des indicatrices 
\usepackage{hyperref} %Nécessaire pour écrire des liens externes 
\usepackage{array} %Nécessaire pour faire des tableaux 
\usepackage{tabularx} %Nécessaire pour faire de longs tableaux 
\usepackage{caption} %Nécesaire pour mettre des titres aux tableaux (tabular) 
\usepackage{color} %nécessaire pour écrire en couleur 
\newtheorem{thm}{Théorème} 
\newtheorem{mydef}{Définition}
\newtheorem{prop}{Proposition} 
\newtheorem{lemma}{Lemme}
\title{Séance 5 - 3ème}
\author{Valentin De Bortoli}
\begin{document}
\maketitle
\section{Exercice 1}
\subparagraph{1} Donner les nombres premiers compris entre 1 et 20. Pour chacun de ces nombres expliquer pourquoi celui-ci est premier.
\subparagraph{2} Donner les carrés des nombres de 1 à 13.
\subparagraph{3} Donner la table de 7, de 8 et de 9 (et les \textbf{connaître}).
\section{Exercice 2}
Simplifier le plus possible les expressions suivantes :
\begin{equation}
A=\sqrt{3}(\sqrt{5}+\sqrt{25})^2
\end{equation}
\begin{equation}
B=(\sqrt{49}\sqrt{7} +1)^2
\end{equation}
\begin{equation}
C=\left( \sqrt{\frac{2}{3}} \right)^2 \frac{1}{\sqrt{5}}
\end{equation}
\section{Exercice 3}
Calculer les expressions suivantes :
\begin{equation}
D=PGCD(457,342)
\end{equation}
\begin{equation}
E=PGCD(211,143)
\end{equation}
\begin{equation}
F=PPCM(720,396)
\end{equation}
\section{Exercice 4}
Exprimer sous la forme d'une puissance les expressions suivantes :
\begin{equation}
G=\left( \frac{-2}{5} \right)^4 \times \left( \frac{5}{2} \right)^4 \times (5^{-3})^{-2}
\end{equation}
\begin{equation}
H=\frac{5^3 \times 4^{-2}}{4^{-4}}\times 5
\end{equation}
\end{document}
